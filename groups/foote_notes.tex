\chapter{Foote Notes}
\section{Supporting results}
{\bf Definition 1:}  Suppose $A$ is abelian and
$A= {\mathbb Z}_{p^{\alpha_1}} \times {\mathbb Z}_{p^{\alpha_2}} \times \ldots {\mathbb Z}_{p^{\alpha_r}}$, define
$m(A)=r$.
\\
\\
{\bf Theorem 1} If $F$ is a $q$ group normalized by an elementary abelian $p$-group, $A$,
$F= \langle C_F(B), B \leq A, B$ a hyperplane of $A \rangle$.
\begin{quote}
\emph{Proof:}  
Let $A, F$ be a counterexample wiht $|F|$ minimal and, subject to that, $|A|$ minimal.
$F/\Phi(F)$ is also a minimal counterexample so we can assume $F$ is also elementary abelian.
Put $V=F$.
$A$ acts faithfully on $V$, $\langle C_V(B): B \leq A, |A:B|=p \rangle = 0$ and
not proper non-trivial subspace of $V$ is notmalized by $A$. So $M=F_q(A)$ is an irreducible
module and $End(V)$ is a division ring. The multiplicative group of $End(V)$ is cyclic and $A$ cyclic
which is a contradiction.
\end{quote}
{\bf Theorem 2} If $B$ and $C$ are subspaces of $A$, $dim(B+C)= dim(B) + dim(C)- dim(B \cap C)$.
\begin{quote}
\emph{Proof:}  This follows from the homomorphism theorem on $B \times C \rightarrow B+C$.
\end{quote}
\section{Thompson}
{\bf Definition 2:}  Suppose $P$ is a $p$-group, $d(P) = max_{A} m(A)$, $A$ is an abelian subgroup of $P$. 
${\cal A}(P)= \{ A: m(A)=d(P) \}$.  $J(P) = \langle {\cal A}(P) \rangle$.
\\
\\
{\bf Theorem 3} (1) $J(P) \thinspace char \thinspace P$.
(2) $H \leq P$ and $J(P) \subseteq H$ then $J(P)= J(H)$.
(3) $H \leq P$, $K \lhd G$ and $J(H) \leq K$ then $J(H) \lhd G$.
\begin{quote}
\emph{Proof:}  (1) follows from the definition of $J(P)$.  (2) ${\cal A}(H) \subseteq {\cal A}(P)$.  Since
$J(P) \subseteq H$, $\langle {\cal A}(P) \rangle \subseteq \langle {\cal A}(H) \rangle$ and the result follows.
(3) $J(K) \thinspace char \thinspace K \lhd G$ so $J(K) \lhd G$.  Since $J(H) \subseteq J(K)$, the result follows from (2).
\end{quote}
{\bf Thompson factorization (restricted)} If $K$ is solvable, $P \in S_p(K)$, $Z= \Omega_1({\mathbb Z}(P))$ and 
(1) $O_{p'}(K) = 1$ and (2) $C_K(Z)=P$, then one of the following holds: (1) $p=2$ and $ 2^3 3 \mid |K|$, (2)
$p=3$ and $2^2 3^3 \mid |K|$ or (3) $J(P) \lhd K$.
\begin{quote}
\emph{Proof:}  Let $K$ be a minimal counterexample.  Put $V= \Omega_1({\mathbb Z}(O_p(K)))$. $K$ acts on $V$ by
conjugation. From now on, put ${\overline K} = K/O_p(K)$.
\\
\\
1. ${\overline K}$ acts faithfully on $V$.\\
Let $\phi: K \rightarrow Aut(V)$, $ker(\phi) = O_p(K)$.
$Z \leq V$ so $C_K(V) \leq C_K(Z)$ and $C_K(V) \lhd K$ so $C_K(V) \subseteq O_p(G)$.  $ ker(\phi) \leq C_K(V) \leq O_p(K)$,
so ${\overline K}$ acts faithfully.
\\
\\
2. $\exists A: A \in {\cal A}(P)$, $A \nsubseteq O_p(K)$, ${\overline Q}$ is a $q$-group, ${\overline A}$ acts on
${\overline Q} \lhd {\overline K}$, non-trivially. \\
$J(P) \nsubseteq O_p(K)$ otherwise $J(P) \lhd K$ by the theorems above.  So $\exists A \in {\cal A}(P): A \nsubseteq O_p(K)$.
$O_p({\overline K}) = 1$, $F({\overline K})= \prod_{q \ne p} O_q({\overline K})$.
Since $A$ is a $p$-group, ${\overline A}$ acts faithfully on $F({\overline K})$ so it must act
nontrivially on $O_q({\overline K})$ for some $q$.  Fix $q$.
Let $Q$ be a Sylow $q$ subgroup on the preimage of $O_q({\overline K})$.
Put $K_0 = A O_p(K)Q$.  $P_0 := A O_p(K) \in S_p(K_0)$.  $K_0$ is a $\{p, q \}$ group.
\\
\\
3. $K_0=K$, so $K= A O_p(K)Q$. \\
Since $A$ acts nontrivially on ${\overline Q}$, $A \nsubseteq O_p(K_0)$.
So, $J(P_0)$ is not normal in $K_0$.
$m(A)=d(P_0)$. $A \leq J(P_0)$.
$K_0$ satisfies the hypothesis of the theorem:
$Z \leq O_p(K) \leq P_0$ so $Z \leq \Omega_1({\mathbb Z}(P_0))$ and
$C_{K_0}(\Omega_1({\mathbb Z}(P_0))) \leq C_{K_0}(Z)$. Thus hypothesis 2 holds.  $Z \leq O_p(K) \leq O_p(K_0)$.  $O_p(K_0)$ centralizes $O_q(K_0)$
and $O_p(K_0)=1$ by hypothesis 2.  $m(A) =d(P) \geq d(P_0)$ so $A \in {\cal A}(K_0)$ so $J(P_0)$ is not normal in $K_0$.  By minimality, $K_0 = K$
\\
\\
4. $|{\overline A}| = |A: A \cap O_p(K)|$,
${\overline Q} = \langle C_{\overline Q}({\overline B}), {\overline B}$ is a hyperplane of $A \rangle$.
$B= O_p(K) \cap A$, $\langle {\overline a} \rangle \in S_p({\overline K})$.
There is another Sylow $p$ group,
$\langle a^g \rangle \leq {\overline K}$.  Put ${\overline H}= \langle {\overline a}, {\overline {a^g}} \rangle$. ${\overline H}$
is not a $p$-group.\\
${\overline Q} =C_{\overline Q}({\overline B})$  where ${\overline B}$ is a hyperplane of $A$.
Since ${\overline A}$ acts nontrivially on ${\overline Q}$, ${\overline A}$ acts non-trivially on some hyperplane ${\overline B_1}$
Put ${\overline Q_1} = C_{\overline Q}({\overline {B_1}})$ and put $K_1= A Q_1 O_p(K)$.  As above, $K_1$ satisfies the hypothesis of the theorem
so $K = K_1$ and thus $Q= Q_1$.  ${\overline {B_1}}$ centralizes ${\overline Q} = O_q({\overline K})$. Apply the result: If $K$ is solvable
with no non-trivial normal subgroup of order prime to $p$ then $C_K(O_p(K)) = {\mathbb Z}(O_p(K))$ and so $C_K(O_p(K)) \leq O_p(K)$.  This
yields ${\overline {B_1}} = 1$ and thus $|{\overline A}| = |A:A \cap O_p(K)|= p$. Put $B = A \cap O_p(K)$.  $B$  is a hyperplane of $A$.
Choose $a \in A \setminus B$.  Then $A = \langle a \rangle \times B$ and
$\langle {\overline a} \rangle \in S_p({\overline K})$.
Since $O_p({\overline K})=1$, there is another $p$ Sylow subgroup $\langle {\overline {a^g}} \rangle$. Fix $g$.
${\overline H} = \langle {\overline a}, {\overline {a^g}} \rangle$ is not a $p$-group.
\\
\\
5. $a$ centralizes a hyperplane of $V$.\\
$m(A)=r=d(P)$ and $m(B)=r-1$.  $B \leq O_p(K)$ and $V \leq {\mathbb Z}(O_p(K))$.  $\langle B, V \rangle$ is abelian and
$B+V$ is elementary abelian.  $r-1 = m(B) \leq m(B+V) \leq r$.  Since$A$ does not centralize $V$ but $B$ does, $B+V \ne B$ and thus
$m(B+V)=r$.  $B \cap V$ is a hyperplane in $V$, so $a$ centralizes a hyperplane.
\\
\\
6. ${\overline H}$ centralizes $W = C_V(a) \cap C_V(a^g)$.  ${\tilde V} = V /W$ has dimension $2$ and all $q$-elements of
${\overline H}$ centralize ${\tilde V}$.\\
If $H$ centralizes ${\tilde V}$, all elements of order $q$ in $H$ centralize ${\tilde V}$.  Contradiction.
\\
\\
7. $a$ acts non trivially on ${\tilde V}$ and ${\tilde V}$ is $2$-dimensional. 
$a$ permutes the $p+1$ one dimensional subspaces in ${\tilde V}$ (the set of lines is denoted by ${\cal L}$).
If $a$  acts trivially, so does $a^g$.  Let $C$ be a hyperplane: $W < C < V$.  Then $C/W$ is fixed by $a$ and $a^g$
and quotients in the chain.  Contradiction!  So $a$ acts non-trivially on ${\cal L}$.  The orbits of the action
have size $1$ and $p$.  Let $W < C < V$ and $C/W$ is the 1-orbit of $a$.  If $a^g$ also acts trivially $H$ acts
trivially on successive quotients, a contradiction.  So $a^g$ must move $C/W$ 
$H$ acts transitively on ${\cal L}$.
\\
\\
8. Let $C$ be a hyperplane of $V$, $W \leq C \leq V$. $C/W$ is fixed by $a$ but $a^g$ moves $C/W$.  $H$ acts transitively and
on ${\cal L}$, $p+1 \mid |H|$, $|{\hat V} |a| (p+1) \mid p^3 (p+1)$.\\
By the above, $H$ acts transitively on lines so $p+1 \mid |H|$, and
$|{\tilde V}| |a| |(p+1) = p^3 (p+1)$.\\
\\
\\
9. If $p=2$, case (1) holds.  If $p=3$, case (2) holds.
\\
\\
10. If $p\geq 5$, $J(P) \lhd K$.\\
Let $H_0$ be the kernal of tha action of $H$ on ${\cal L}$. Put ${\hat H} = H/H_0$. ${\hat H}$ acts faithfully on ${\cal L}$.
Since $SL_2(p)$ is generated by 2 non commuting $p$ elements, ${\hat H} = PSL_2(p)$.   $p+1 \mid |{\hat H}|$ so $p+1=q^c$, $c \geq 2$,
and $q=2$.
${\hat H}$ is 2-transitive on ${\cal L}$.  Let ${\hat N}$ be a minimal normal subgroup of ${\hat H}$.  ${\hat N}$ is elementary
abelian $2^c=p+1 \mid |{\hat N}|$. ${\hat N}$ is a 2-group. $m({\hat N}) \geq c$ and
$5 \leq p= 2^c-1$, so $c \geq 3$.  But $m({\hat A}) \leq 2$ for every elementary abelian subgroup ${\hat A}$ of $PSL_2(p)$.  This contradiction
establishes the result.
\end{quote}
\section{Notes}
{\bf Thompson. maximal subgroups and uniqueness groups:}
$P \in S_p(G)$, $t \in {\mathbb Z}(P), |t|=p$.  Let $M$ be a maximal subgroups of $G$ containing $M$, $M$, solvable.
$O_q(M)= 1$ [Because No non-trivial $p$-central element normalizes a non-trivial $q$-subgroup of $G$.
\\
\\
$C_G(t)=P$ and $M = N_G(J(P))$.  $M$ is uniquely determined by $J(P)$. $J(P_1) \thinspace P_1 \lhd P_2$.
$J(P_1)$ allows us to "push-up" from $P_1$ to $P_2$.
