\chapter{Foote Notes}
\section{Notation}
{\bf Definition 1:}  Suppose $A$ is abelian and
$A= {\mathbb Z}_{p^{\alpha_1}} \times {\mathbb Z}_{p^{\alpha_2}} \times \ldots {\mathbb Z}_{p^{\alpha_r}}$, define
$m(A)=r$.
\\
\\
{\bf Theorem 1} If $F$ is a $q$ group normalized by a $p$-group, $A$, $F= \langle C_F(B), B \leq A, B$ a hyperplane of $A \rangle$.
\begin{quote}
\emph{Proof:}  
\end{quote}
{\bf Theorem 2} If $B$ and $C$ are subspaces of $A$, $dim(B+C)= dim(B) + dim(C)- dim(B \cap C)$.
\begin{quote}
\emph{Proof:}  
\end{quote}
{\bf Definition 2:}  Suppose $P$ is a $p$-group, $d(P) = max_{A} m(A)$, $A$ is an abelian subgroup of $P$. 
${\cal A}(P)= \{ A: m(A)=d(P) \}$.  $J(P) = \langle {\cal A}(P) \rangle$.
\\
\\
{\bf Theorem 3} (1) $J(P) \thinspace char \thinspace P$.
(2) $H \leq P$ and $J(P) \subseteq H$ then $J(P)= J(H)$.
(3) $H \leq P$ and $J(H) \subseteq K \lhd G$ then $J(H) \lhd P$.
\begin{quote}
\emph{Proof:}  
\end{quote}
{\bf Thompson factorization (restricted)} If $K$ is solvable, $P \in S_p(K)$, $Z= \Omega_1({\mathbb Z}(P))$ and 
(1) $O_{p'}(K) = 1$ and (2) $C_K(Z)=P$, then one of the following holds: (1) $p=2$ and $ 2^3 3 \mid |K|$, (2)
$p=3$ and $2^2 3^3 \mid |K|$ or (3) $J(P) \lhd K$.
\begin{quote}
\emph{Proof:}  Let $K$ be a minimal counterexample.  Put $V= \Omega_1({\mathbb Z}(O_p(K)))$. $K$ acts on $V$ by
conjugation. Let $\phi: K \rightarrow Aut(V)$, $ker(\phi) = O_p(K)$.  From now on, put ${\overline K} = K/O_p(K)$.
$Z \leq V$ so $C_K(V) \leq C_K(Z)$ and $C_K(V) \lhd K$ so $C_K(V) \subseteq O_p(G)$.
\\
\\
1. ${\overline K}$ acts faithfully on $V$.
\\
\\
2. $K= O_p(K) A Q$, $A \in {\cal A}(P)$, $A \nsubseteq O_p(K)$, ${\overline Q}$ is a $q$-group, ${\overline A}$ acts on
${\overline Q} \lhd {\overline K}$, non-trivially.
\\
\\
3. $O_p({\overline K}) = 1$, $F({\overline K})= \prod_{q \ne p} O_q({\overline K})$.  Put ${\overline Q} \in S_q({\overline K})$
and let $O_p(K) Q$ be its preimage.  Put $K_0 = A O_p(K)Q$.  $K_0=K$
\\
\\
4. ${\overline Q} = \langle C_{\overline Q}({\overline B}) {\overline B}$ is a hyperplane of $A \rangle$.
$B= O_p(K) \cap A$, $\langle {\overline a} \rangle \in S_p({\overline K})$.  There is another sylow group,
$\langle a^g \rangle \leq {\overline K}$.  Put ${\overline H}= \langle {overline a}, {\overline {a^g}} \rangle$. ${\overline H}$
is not a $p$-group.
\\
\\
5. $a$ centralizes a hyperplane of $V$.
\\
\\
6. ${\overline H}$ centralizes $W = C_V(a) \cap C_V(a^g)$.  ${\hat V} = V /W$ has dimension $2$ and all $q$-elements of
${\overline H}$ cnetralize ${\hat V}$.
\\
\\
7. $a$ acts non trivially on ${\hat V}$
\\
\\
8. $a$ permutes the $p+1$ lines in ${\hat V}$, denoted ${\cal L}$ and so does $a^g$.
\\
\\
9. Let $C$ be a hyperplane of $V$, $W \leq C \leq V$. $C/W$ is fixed by $a$ but $a^g$ moves $C/W$.  Thus $H$ acts transitively
on ${\cal L}$, $p+1 \mid |H|$, $|{\hat V} |a| (p+1) \mid p^3 (p+1)$.
\\
\\
10. If $p=2$, case (1) holds.
\\
\\
11. If $p=3$, case (2) holds.
\\
\\
12. If $p\geq 5$, 
there are $2$ p-elements in $H/H_0$ which generate $SL_2(p)$.
This contradicts the solvability of $K$ and $J(P) \lhd K$.


\end{quote}
