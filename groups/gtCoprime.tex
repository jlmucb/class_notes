\chapter{Coprime action and $p$-constraint}
\section {Basic Results}
{\bf Definition 1:} A group of automorphisms $A$ of a group $P$ stabilizes a chain
$1=P_0 \subseteq P_{1} \subseteq \ldots \subseteq P_n = P$ if
$[A,P_{i+1}] \subseteq P_{i}$.  \\
\\
{\bf Theorem 1:} If $P$ is a $\pi$ group with chain stabilized by
$A$ then $A$ is a $\pi$ group.
\begin{quote}
\emph{Proof:} Suppose $a \in A$ is a $\pi'$ automorphism.  Proof is by induction on the
length of the chain and so we can assume $[a,P_1]=1$.
$x^a =xy, y \in P_1$.  So, $x^{a^{|a|}}= x y^{|a|}=x$, so $y=1$ and
$[a,P_2]=1$.
\end{quote}
{\bf Theorem 2:} If $A$ is a $\pi'$ group of automorphisms on a $\pi$ group $P$ with
$[P,A,A]=1$ then $[P,A]=1$.
\begin{quote}
\emph{Proof:}  $A$ stabilizes
$[P,A,A] \subseteq [P,A] \subseteq P$.
\end{quote}
{\bf Theorem 3:}
Let $A$ be a $\pi'$ group of automorphisms of a $\pi$ group $P$.  Let $Q$ be
an $A-$invariant normal subgroup of $P$.  Then $C_{P/Q}(A)= (C_P(A) Q)/Q$.
\begin{quote}
\emph{Proof:}   $C_P(A)Q/Q \subseteq C_{P/Q}(A)$.  Suppose $xQ$ is a subset in $P$ fixed by
$A$.  $QA$ acts transitively on the set $xQ$.  Let $A_1$ be the point stabilizer.
$|A_1|=|QA|/|xQ|=|A|$.  By Schur-Zassenhaus, $A$ and $A_1$ are conjugate, so $\exists y \in xQ:
y^A=y$.
\end{quote}
{\bf Theorem 4:} If $P$ is a $\pi$ group, $A$ is a $\pi'$ group, $P= [P,A] C_P(A)$.
\begin{quote}
\emph{Proof:}
$[P,A] \subseteq P$ and $A$ centralizes $P/[P,A]$.
\end{quote}
{\bf Lemma A:} If the action of $A$ on $G$ is co-prime and $U$ is an $A$-invariant subgroup of $G$ with
$(Ug)^A=Ug$, then $\exists c \in C_G(A)$: $Ug = Uc$.
\begin{quote}
\emph{Proof:}  $[a, g^{-1}] \in U$ so $A^{g^{-1}} \leq AU$.  Both $A$ and $A^{g^{-1}}$ are $U$ complements in
$AU$ so by S-Z, $\exists u \in U$: $A^{g^{-1}} = A^u$.  Put $c= ug$.  $c \in N_{AG}(A) \cap Ug$, so
$[A,c] \subseteq A \cap G = 1$.
\end{quote}
{\bf  Theorem:} Let $N$ be an $A$-invariant normal subgroup of $G$ and
let $A$ act co-primely on $G$ then (i)
$C_{G/N}(A)= C_G(A)N/N$ and
(ii) if $A$ acts trivially on $N$ and $G/N$ then $A$ acts trivially on $G$.
\begin{quote}
\emph{Proof:}  $C_G(A)N/N \subseteq C_{G/N}(A) $ is obvious.
If $Ng$ is in $C_G(A)N/N$, $(Ng)^A=Ng$ and by
the lemma, $\exists c \in C_G(A): Ng=Nc$, proving (i).  ii follows directly from the Lemma.
\end{quote}
{\bf Theorem:} Let $Y \lhd X \in \pi$ and $[Y,A]=1$ $(|A|, |X|) = 1$.  If $C_X(Y) \subseteq Y$ then $[X,A]=1$.
\begin{quote}
\emph{Proof:}
$[X, Y] \subseteq Y$, so $[X, Y, A] = 1$.  $[Y, A, X] =1$ so $[A, X, Y] = 1$.
$[X, A] \subseteq Y$.  Thus $[X,A,A]=1$ and $[X,A]=1$.
\end{quote}
{\bf Alternate proof of 2, 4:}
\begin{quote}
\emph{Proof:}
$G/[G,A] = C_{G/[G,A]}(A) = C_G(A)[G,A]/[G,A]$, proving (4).  Let $g \in G$ then $g = hk, h \in [G,A], k \in C_G(A)$ by the above. Since $[xy, z]= [x,z]^y [y,z]$, for $a \in A$,
$[g,a]= [hk,a] = [h,a]^k [k,a]$.  $[k,a]=1$ since $k \in C_G(A)$.  Thus
$[g,a] = [h,a]^k \in [G,A,A]$.
\end{quote}
{\bf Alternate proof of Thompson:}  Let $A= P \times Q$ act on a $p$-group $G$ and suppose $C_G(P) \leq C_G(Q)$.  Then
$Q$ acts trivially on $G$.
\begin{quote}
\emph{Proof:}  $C_U(P) \leq C_U(Q)$ for all $A$-invariant subgroups, $U$.  By induction, we can assume
$[U,Q]=1$ for all proper subgroups.  $[G,P,Q]=1$ and $[P, Q, G] =1$ so $[Q, G, P]=1$.  Thus
$[Q,G] \leq C_G(P) \leq C_G(Q)$ and $[G,Q,Q] = 1$.  By the previous result, $[G,Q]=[G,Q,Q]=1$.
$G/[G,A] = C_{G/[G,A]}(A) = C_G(A)[G,A]/[G,A]$.  \end{quote}
{\bf Theorem 5:} $P$ is an abelian $\pi$ group, $A$ is a $\pi'$ group.  $P= [P,A] \oplus C_P(A)$.
\begin{quote}
\emph{Proof:}
$\theta= {\frac 1 {|A|}} \sum_a a$.  For $a \in A$, $\theta a = a \theta = \theta$ and
$\theta^2= \theta$. $\theta P \oplus ker(\theta)= P$.  $C_P(A) \subseteq \theta P$ and
$\theta([x,a]) =0$, so $[P,A] \subseteq ker(\theta)$.  This shows $C_P(A) \cap [P,A] = \emptyset$
and the result follows from the earliler result that $P= [P,A] C_P(A)$.
\end{quote}
{\bf Theorem 6:}
If $\phi$ is a $p'$-automorphism of a $p$-group, $P$, which induces the identity on
$P/ \Phi(P)$ then $\phi= 1$.
\begin{quote}
\emph{Proof:}  Let $H= \Phi(P), {\overline P}= P/H, |{\overline P}|= p^r, |H|= p^m$.
$|P|= p^{m+r}$ and for any subset $Y= \{ y_i, 1 \le i \le n \} \subseteq P$,
$P= \langle P, Y \rangle$ iff $P= \langle Y \rangle$.  Note that ${\overline P}$ and
hence $P$ cannot be generated by $<r$ elements.  Let $\{ x_i \}$ be a minimal generating
set for $P$ and $x_i'= h_i x_i$ be another such generating set.  Let $\phi$ be as set forth in the
statement of the theorem.  Finally, let ${\cal M}$ be the collection of all such minimal generating
sets.  $\phi$ fixes each coset  $H x_i'= H x_i$ and $\phi(x_i')= h_i x_i$ so $\phi$
indusces a permutation representation on ${\cal M}$.  The cycle length, $s$, of any cycle in this
permutation representation of $\phi$ must divide $t= | \phi |$.  Suppose some such cycle
had length $s<t$.  $\phi_1= \phi^s$ fixes the elements of this cycle and so fixes some
minimal generating set.  But then $\phi_1$ fixes evely element of $P$ contrary
to the fact that $t>s$.  Since ${\cal M}$ decomposes as a product of disjoint cycles,
$t \mid |{\cal M}| = p^{mr}$.  Since $(t,p)= 1$, $t=1$ and $\phi= 1$ on $P$.
\end{quote}
{\bf $P \times Q$ lemma:} Let $A = P \times Q$, $P$ a $p-$group, $Q$ a $p'$-group, act
on a $p$-group $G$.  If $C_G(P) \le C_G(Q)$ then $Q$ acts trivially on $G$.
\begin{quote}
\emph{Proof:}
$C_U(P) \le C_U(Q)$ for all $A$-invariant subgroups $U$.  By induction, $[U,A]=1$ if
$U < G$.  Now $[G,P] < G$ so $[G, P, Q] = 1$ and since $[P, Q]=1$,
$[P, Q, G]=1$ so by the three subgroups lemma $[Q, G, P]=1$ so
$[Q,G] \le C_G(P) \le C_G(Q)$ and $[G, Q, Q]=1$ hence $[G,Q]=1$.
\end{quote}
{\bf Theorem 7:} Suppose $M$ is a $p$-group and $C_M(P) \le C_M(Q)$.  Then $Q$ acts trivially on $M$.
\begin{quote}
\emph{Proof:}  This is just a restatement of the $P \times Q$ Lemma.
\end{quote}
{\bf Application of $P \times Q$:}  Let $P \in p(X)$, $M=O_p(X)$, $Q= O_{p'}(N_X(P))$.  $[P, Q]=1$ so
$P \times Q$ acts on $M$.  Next, $[C_M(P), Q] = 1$ because $[C_M(P), Q] \subseteq M$ and $[C_M(P), Q] \subseteq Q$, since
$C_M(P)$ normalizes $P$ so $C_M(P)$ normalizes $Q=O_{p'}(N_X(P))$; as a result, we have,
$[C_M(P), Q] = M \cap Q = 1$.
Let $x \in C_M(P)$ then $[x, Q] = 1$ so $x \in C(Q)$ and $x \in M$ so
$x \in C_M(Q)$.  Thus $C_M(P) \subseteq C_M(Q)$ and the $P \times Q$ lemma applies, that is, $[M,Q]=1$.
If $X$ be solvable, we can use this to show $O_{p'}(N_X(P)) \subseteq O_{p'}(X)$, since $C_X(F(X)) \subseteq F(X)$.
\\
\\
{\bf Theorem 8:}
Suppose $X$ acts faithfully on an Abelian group $A$ and $X= \bigcup_{i=0}^t X_i$ with $X_i \cap X_j =1$ for $i \ne j$ then $C_A(X_i) \ne 1$ for some $i$ or
$a^t=1, \forall a \in A$.
\begin{quote}
\emph{Proof:} Let $\sigma_i(a)= \prod_{x \in X_i} a^x$ and
$\sigma(a)= \prod_{x \in X} a^x$.  If the first condition doesn't hold for $i$, $(\sigma_i(a))^x = \sigma_i(a)$ for
all $x \in X_i$, so $\sigma_i(a) = 1$.  Similarly, $\sigma(a)$ is fixed
by all elements of $X$ and since $X$ acts faithfully, $\sigma(a) = 1$.
If $\sigma_i(a)=\sigma(a)= 1, \forall a, i$ then
$\prod_{i=0}^t \sigma_i(a)= \sigma(a) a^t$, so $a^t=1$.
\end{quote}
{\bf Application:} $X= {\mathbb Z}_p \times {\mathbb Z}_p$, $|A| \ne 0 \jmod{p}$.
\\
\\
{\bf Theorem 9:}
If $A$ is abelian and $A$ acts on $G$, $(|A|, |G|)=1$ then $G= \langle C_G(A_0): A/A_0 \textnormal{ is cyclic} \rangle $.
\begin{quote}
\emph{Proof:} By induction on $|AG|$.  We may assume $A$ acts faithfully otherwise $C_A(G) \ne 1$ and
the result holds.  \\
\\
\emph{Claim:} Under the hypothesis, $\exists P \in S_p(G): P$ is
$A$-invariant. \\
\emph{Proof of claim:}  By Frattini, if $P \in S_p(G) \rightarrow AG=GN_{AG}(P)$.
$\exists A_1 \subset N_{AG}(P): A_1 \cap G=1$ and $AG = A_1 N_{AG}(P)$ so $\exists g \in N_{AG}(P):
A_1^g=A$ so $A \subseteq N_G(P)$.   This proves the claim.
\\
\\
If $G$ is not a $p$-group, $\exists P \in S_p(G)$ for each $p$ which are $A$-invariant.  In
this case, $|AP|<|AG|$ so by induction,
so $ \langle C_P(A_0): A/A_0 \textnormal{ is cyclic}  \rangle =P$ for each $p$ and we're done.
So $G$ is a $p$-group.  If $A$ is cyclic, the result holds trivially.  If not, $\exists A_0
\leq  A$ with $A_0= {\mathbb Z}_q \times {\mathbb Z}_q$. Put ${\overline P}= P/\Phi(P)$ then
$A_0$ acts on ${\overline P}$.  By the previous result,
$\exists a \in A_0^{\#}: C_{\overline P}(a) \ne 1$.  Put
${\overline P}_0= C_{\overline P}(a)$.
If ${\overline P}_0={\overline P}$, $\forall a: [P, a] \subseteq \Phi (P)$ which contradicts
the faithfulness hypothesis.  So ${\overline P}_0 < {\overline P}$, ${\overline P}_0$ is
$A$-invariant.  Since ${\overline P}$ is elementary abelian, $\exists {\overline P}_1$
also $A$-invariant: ${\overline P}= {\overline P}_0 \times {\overline P}_1$.  Thus
$P= P_0 P_1$, $P_0 \cap P_1 \subseteq \Phi (P)$ and by induction,
$P_i= \langle C_{P_i}(A_0): A/A_0 \textnormal{ is cyclic } \rangle $ and again we're done by induction.
\end{quote}
{\bf Theorem 10:}  Suppose $N \lhd G$, $H < G$, ${\overline G}= G/N$.  If $(|N|, |H|)=1$
and either is solvable then (a) $N_{{\overline G}}({\overline H}) = {\overline {N_G(H)}}$ and
(b) $C_{{\overline G}}({\overline H}) = {\overline {C_G(H)}}$.
\begin{quote}
\emph{Proof:} Note that ${\overline {N_G(H)}} \subseteq N_{\overline G}({\overline H})$ which is the
inverse image of $N_G(HN)$.  We want to show $N_G(HN)= N_G(H)N$.
Let $g \in N_G(HN)$.  $H, H^g$ are complements to $N$ in $HN$ so by Schur-Zassenhaus,
$H^{gn}=H, n \in N$ and $gn=m \in N_G(H)$ and thus $g \in N_G(H)N$.  Let $K$ be the
inverse image in $G$ of $C_{\overline G}({\overline H})$ and $K \supseteq N$.
We have $K \subseteq KN \subseteq N_G(H)N$ and so by Dedekind, $K=N N_N(H)$.
Since $[ {\overline K}, {\overline H}]= 1$, $[K, H] \subseteq N$ and
$[N_K(H), H] \subseteq N \cap H = 1$.  So $N_K(H) \subseteq C_G(H)$ and
$K \subseteq C_G(H)N$ and therefore,
$C_{\overline G}({\overline H})= {\overline K} \subseteq {\overline {C_G(H)}}$.  The
opposite inclusion obviously holds.
\end{quote}
\section {$p$-constraint}
{\bf Definition:} $G$ is $p$-constrained if $O_{p'}(G)=1$ and $C_G(O_p(G)) \leq O_p(G)$ (So
$O_p(G)=F(G)$).
\\
\\
{\bf Theorem 11:}
Suppose $G$ is solvable and $O_{p'}(G)=1$ then $C_G(O_p(G))) \subseteq O_p(G)$.
\begin{quote}
\emph{Proof:}
$O_p(G)$ is nilpotent so $O_p(G) \subseteq F(G)$.  If $Q \in S_q(F(G))$ then
$Q \; char \; F(G)$ so $Q \lhd G$.  Thus $Q=1$ for $q \ne p$ by hypothesis and
$F(G)$ is a $p$-group and $F(G)=O_p(G)$.  Now the result follows since $C_G(F(G)) \subseteq F(G)$.
\end{quote}
{\bf Note:} $G$ acts on $F(G)$ by conjugation so $G/{\mathbb Z}(F(G)) \rightarrow Aut(F(G))$.
\\
\\
{\bf Theorem 12:}  If $G$ is $p$-constrained, $P \in p(G)$ and $P \subseteq G$ then
(1) $O_{p'}(N_G(P)) \subseteq O_{p'}(G)$ and (2) $N_G(P)$ is $p$-constrained.
\begin{quote}
\emph{Proof:}  (1) Let ${\overline G}= G/O_{p'}(G)$, then, as above,
$N_{{\overline G}}({\overline P}) = {\overline {N_G(P)}}$.
${\overline {O_{p'}(N_G(P))}} \subseteq O_{p'}({\overline {N_G(P)}}) = O_{p'}(N_{\overline G}({\overline P}))$, so to prove (1), we can assume $O_{p'}(G) = 1$.
Let $Q= O_{p'}(N_G(P))$ and $M= O_p(G)$.  $[P, Q] \subseteq P \cap Q = 1$ and so
$PQ= P \times Q$ acts on $M$.  $[C_M(P),Q] \subseteq M \cap [N_G(P), Q] = 1$ so
by the $p \times q$ lemma, $[M, Q] = 1$.   Hence
$Q \subseteq C_G(M) \subseteq M$ (by $p$-constraint) and $Q = 1$.
\\
(2)
$N_{{\overline G}}({\overline P}) = {\overline {N_G(P)}}= N_G(P)/(N_G(P) \cap O_{p'}(G))$.
By (1), $O_{p'}(N_G(P)) = N_G(P) \cap O_{p'}(G)$ and so
$N_{\overline G}({\overline P})= N_G(P)/O_{p'}(N_G(P))$ so we
may assume $O_{p'}(G) = 1$.
Put $M_1 = O_p(N_G(P))$ and $M=O_p(G)$.  It suffices to show that $C_G(M_1 )$ is a $p$-group.
Let $Q \in S_q(C_G(M_1 )), q \ne p$.  Since $P \lhd N_G(P)$ and $P \subseteq M_1$ and
$P \times Q$ acts on M.  $C_M(P)= O_p(G) \cap C_G(P) \subseteq C_G(P)$ and therefore
$C_M(P) \subseteq O_p (N_G(P))$.  $O_p (C_G(P)) \lhd N_G(P)$ so
$O_p(C_G(P)) \subseteq O_p(N_G(P))= M_1$ and $C_M(P) \subseteq M_1$.  Since
$Q \subseteq C_G(M)$, we have $[C_M(P), Q] = 1$, thus $[M, Q]= 1 $, $Q = 1$, $C_G(M)$ is
a $p$-group.
\end{quote}
{\bf Theorem 13:}  Suppose $G$ is $p$-constrained, $P \in S_p(G)$ and $Q$ is a $P$-invariant
$p'$-subgroup of $G$ then $Q \subseteq O_{p'}(G)$.
\begin{quote}
\emph{Proof:}  Reduce to $O_{p'}(G) = 1$ and let $M= O_p(G)$ so $M \subseteq P$.
$[M, Q] \subseteq M \cap [P, Q] \subseteq M \cap Q = 1$ and $Q = 1$.
\end{quote}
{\bf Theorem 14:}
If $P \in p(G)$ with $N_G(P)$ $p-$constrained then $C_G(P)$ is also $p-$constrained.
\begin{quote}
\emph{Proof:}  $O_{p'}(N_G(P)) = O_{p'}(C_G(P))$ so by considering factor groups, we may assume
$O_{p'}(N_G(P))= 1$.  Let $N= N_G(P)$, $C= C_G(P)$, $Q= O_p(C)$, $R= O_p(N)$.
$Q \; char \; C \lhd N$ and $Q \subseteq R \cap C$, so $Q= R \cap C$.
$R \cap C \lhd C$ and $ R \cap C \subseteq Q$ so $Q = R \cap C$.
Let $x \in C_C(Q)$ be a $p'$ element.   $[x, R] \subseteq R \cap C = Q$,
so $X$ stabilizes the chain $R \supseteq Q \supset 1$ and so $x \in C_N(R) \subseteq R$ as
$N$ is $p$-constrained.  Thus $C_C(Q)$ is a $p$-group and normal in $C$ hence $C_C(R) \subset Q$.
\end{quote}
{\bf Example:} $G$ solvable and $P_1, \ldots, P_n$ a Sylow system.  Let $Q_i= O_{p'}(P_1P_2)$
then $Q_3, \ldots, Q_n$ is a Sylow system for $O_{p'}(G)$.
\\
\\
{\bf Theorem 15:}  Suppose $V$ is a non-cyclic abelian $r$-group of operators on a $p$-constrained
$r'$ group $X$.  For $r \ne p$ then $\bigcap_{v \in V^{\#}} O_{p'}(C_X(v)) \subseteq O_{p'}(X)$.
$p'$-subgroup of $G$ then $Q \subseteq O_{p'}(G)$.
\begin{quote}
\emph{Proof:}  Let ${\overline X}= X/O_{p'}(X)$ be $V$-invariant.  $C_{\overline X}({\overline V}) = {\overline {C_{X}( V)}}$.  We may asssume
$O_{p'}(X) = 1$.  Let $M = \langle C_M (V_0 ): V/V_0 \rangle = \langle C_M(v), v \in V^{\#} \rangle $
where $V/V_0$ is cyclic.
The last equality holds since $V$ is non-cyclic.
$C_M(v) = M \cap C_X(v) \lhd C_X(v)$ so $C_M(v) \subseteq O_p(C_V(X))$.
Now, $Q \subseteq O_{p'}(C_V(X))$ and $[C_M(V), Q] =1, \forall v \in V^{\#}$ therefore
$[M, Q] = 1$ and so $Q = 1$.
\end{quote}
{\bf Theorem 16:}
If $G$ is solvable, $C_G(F(G)) \subseteq F(G)$.
\begin{quote}
\emph{Proof:}
Let $Z = {\mathbb Z}(F(G))$ and $N/Z$ be a minimal normal subgroup of $C_G(F(G))/Z$ to obtain a contradiction.
\end{quote}
{\bf Theorem 17 (Another version of the $P \times Q$ Lemma):}
If $P$ is a $p$-group and $Q$ is a $p'$ group, and $P \times Q$ acts on a $p$-group,
$M$ then $[M,Q] = 1$.
\begin{quote}
\emph{Proof:}
For the semi-direct product $(P \times Q)M$ then $P_1= PM$ is $Q$-invariant.  Put
$P_0=P C_M(P)$.  $P_0 \lhd \lhd P_1$, $[P_0, Q]= 1$ so $[M, Q] = 1$.
\end{quote}
{\bf Theorem 18:}
Suppose $X$ acts on $A$, $(|X|, |A|)= 1$, $A$ abelian.  Then
(1) If $A_0 \subseteq A$ is an $X$-invariant direct factor of $A$, $\exists A_1 \subseteq A$:
$A_1$ is $X$ invariant and $A= A_0 \times A_1$; and,
(b) $A= C_A(X) \times [A,X]$.
\begin{quote}
\emph{Proof:}
Let $f: A \rightarrow A_0$ be any projection.  $\exists i: |X|i= 1 \jmod{|A|}$.
Put $f_1(a)= \prod_{x \in X} [f(a^x)^{x^{-1}}]^i$.
$f_1$ is an endomorphism, $im(f_1) \subseteq A_0$, if $a_0 \in A_0, f(a_0) =a_0$ and
$f_1^2= f_1$
so $A= A_0 \times ker(f_1)$.
$f_1(a^y)= (f_1(a))^y, \forall a \in A, y \in X$ so $ker(f_1)$ is $A$-invariant which proves (a).
For (b), let $f(a)= (\prod_{x \in X} a^x)^i$ then
$f(a^y)= f(a)^y$ so $im(f) \subseteq C_A(X)$.   If $a \in C_A(X)$ then
$f(a)=a$ hence $C_A(X)$ is a direct factor of $A$.
$A= C_A(X) \times A_1$, $A_1$, $X$-invariant.  $[A, X] \subseteq A_1$ and
$A_1= [A, X] C_{A_1}(X)$  so $[A, X] \subseteq A_1 \subseteq [A_1 , X] \subseteq [A, X]$.
\end{quote}
{\bf Theorem 19:}   Suppose $G$ is an $X$-group and $(|[G, X]|, |X|) = 1$.
(1) $[G,X]= [G, X, X]$; (2) if either $X$ or $[G, X]$ is solvable, $G= [G,X]C_G(X)$;
(3) if $[G, X] \subseteq \Phi(G)$ then
$[G, X] = 1$.
\begin{quote}
\emph{Proof:}
\\
\emph{Claim:} $[g, x^i] = [g, x]^i \jmod{[G, X, X]}$.
\\
\emph{Proof of claim:} $[g, x^{i+1}] = [g, x] [g, x^i]^x = [g, x] [g, x^i]^x [g, x^i, x]$ and this proves the claim.
\\
In particular, $1 = [g,1] = [g, x^{|X|}]=
[g, x]^{|X|} \jmod{[G, X, X]}$ so $[G,X]/[G,X,X] =1$ which proves (1).
\\
\\
For (2), form the semi-direct product, $GX$.  $[G,X] \lhd GX$.  Put $G_0 = [G, X]X$.
\\
\emph{Claim:} $G_0 \lhd G$.
\\
\emph{Proof of claim:}
Suffices to show $[G, G_0] \subseteq G_0$.  Let $g \in G, g_0 \in G_0$ and put
$g_0 = g_1 x$ then $[g, g_0] [g, g_1]^x \in G_0$.  $G_0$ satisfies the Schur-Zassenhaus theorem.
Let $g \in G$ then $X^g \subseteq G_0^g = G_0$ and $X^g$ and $X$ are complements so
$X^{g g_1} = X, g_1 \in [G, X]$ and $[X, g g_1] \in  X \cap G = 1$.
$g g_1 \in C_G(X)$ and $g= c g_1^{-1}, c \in C_G(X) \in C_G(X)[G,X]$.
\\
\\
For (3), $G= [G,X] C_G(X)$.  If $\Phi(G) \supseteq [G,X]$, $G= \langle [G, X], C_G(X) \rangle \subseteq
\langle \Phi(G), C_G(X) \rangle =  C_G(X)$, so $[G, X]= 1$.
\end{quote}
{\bf Theorem 20:}
If $\phi$ is an irreducible representation of an abelian group $G$ with kernel $K$, then
$G/K$ is cyclic.
\begin{quote}
\emph{Proof:}  \emph{Claim:} If $G$ is abelian and acts irreducibly on $V$ then $gv= \lambda v$.\\
\emph{Proof of claim:}  $W= \{ w \in V: gw= \lambda w \} \ne 0$ is $G$-invariant,
so it is all of $V$.  This proves the claim.\\
Now $G/K$ acts faithfully and irreducibly its letting $\lambda$ be the
$|G|$-th root of unity in the claim, $g \mapsto \Lambda(g)$.  This is an injection into
a subgroup of a multiplicative subgroup of $F$ and these subgroups are cyclic.
\end{quote}
{\bf Corollary:}  If a $G$ possesses a faithful irreducible representation, then
${\mathbb Z}(G)$ is cyclic.
\begin{quote}
\emph{Proof:}  Simple application of previous theorem.
\end{quote}
{\bf Lemma:} Let $P$ be an elementary abelian $p$-group and $Q$ be a non-cyclic
abelian $q$-group of $Aut(P)$, $p \ne q$, then $P= \prod_{x \in Q^{\#}} C_P(x)$.
\begin{quote}
\emph{Proof:}  Regard $P$ as a vector space over $F_p$.  $Q$ acts on $P$ as a linear transformation
and denote $\varphi: x \mapsto \phi_x$ be the map between $Q$ and the linear transformations
under this correspondence.
By Maschke, $P= P_1 \oplus \ldots \oplus P_n$, each $P_i$ irreducible.  Since
$Q$ is abelian and let $Q_i= ker(\varphi_{|P_i})$ under the restricted map $\varphi_{|P_i}: Q \rightarrow P_i$.  $Q/Q_i$ must be cyclic and
$Q$ is non-cyclic so $Q_i \ne 1$.  If $x \in Q_i$ then $P_i \subseteq C_P(x)$.
\end{quote}
{\bf Theorem 21:}
Let $G$ be a $p$-group and
$f: Aut(G) \rightarrow Aut(G/\Phi(G))$.  $ker(f)= \langle \{ X: [G, X] \subseteq \Phi(G) \} \rangle$ and so $ker(f) = 1$.
\begin{quote}
\emph{Proof:}  This is practically the definition.
\end{quote}
{\bf Theorem 22:}
If $X$ acts on $G$, $(|X|, |G|)= 1$ and $G_0 \lhd \lhd G$, $C_G(G_0) \subseteq G_0$ and
$[G_0 , X] =1$ then $X$ acts trivially on $G$.
\begin{quote}
\emph{Proof:}  Suppose $G_0 \lhd G_1 \lhd \ldots \lhd G_n = G$.  Choose $i$ maximal such that
$[G_i, X]= 1$ (we want $i=n$).  $N_G(G_i)$ is $X$-invariant.  If
$N_G(G_i) \nsubseteq G$, we get $[N_G(G_i ), X]= 1$; but $G_{i+1} \subseteq N_G(G_i )$ which
implies $[G_i, X]= 1$ --- a contradiction.
$C_G(G_i ) \subseteq C_G(G_0) \subseteq G_0 \subseteq G_i$.
So $[G_i, X, G]=1= [G, G_i, X]$ and by the three subgroups lemma,
$[X,G,G_i]= 1$ and $[X,G] \subseteq C_G(G_i) \subseteq G_i$.
Therefore, $[G,X,X]=1$ and hence $[G,X]= 1$.
\end{quote}
{\bf Definitions:} $G$ acts \emph{irreducibly} on $V$  if there are no proper $G$-invariant
subgroups of $V$.  $G$ acts \emph{faithfully} on $V$ if $C_G(V) = 1$.
$G$ acts \emph{non-trivially} on $V$ if $C_G(V) \ne G$.
For $\alpha \in Aut(G)$, define $\sigma(G)= \sum_{g \in G} \alpha(g)$.
\\
\\
{\bf Theorem 23:}
If $G$ has a faithful, irreducible representation on a vector space $V$ over $F$,
$char(F) = p$ then $G$ has no non-trivial normal $p$-subgroup.
\begin{quote}
\emph{Proof:}  Let $\varphi$ be such a representation and $P \lhd G$.  $\varphi(P)$ fixes some
$v \ne 0$.  Set $W= C_V(\varphi(P))$.  $W$ is $G$-invariant and by irreducibility,
$W=G$ and $\varphi$ acts trivially on $G$ but $\varphi$ is faithful so
$P= 1$.
\end{quote}
{\bf Theorem 24:}
Let $G$ act on $V$, $V$, abelian with $(|G|, |V|)= 1$ then $C = V_1 \oplus V_2$ is the
direct sum of $G$-invariant subgroups of $V$ with $V_1= C_V(G)$.
\begin{quote}
\emph{Proof:}  Let $\sigma= \sigma(G)$.  Set $V_1 = C_V(\sigma)$ and $V_2= im(\sigma)$. $V_1$ and $V_2$ are $G$-invariant and $|V_1| \cdot |V_2| = |V|$.  Set $Z= C_V(G)$, $Z$ is $G$-invariant
and $Z \subseteq V_2$.   \emph{Claim:} $Z \cap V_1 = 0$.  \emph{Proof of claim:}
Let $v \in Z \cap V_1$ then $\sigma(v)= |G|v$ and $\sigma(v)= 0$ hence $v= 0$.
\end{quote}
{\bf Theorem 25:}
Let $G$ act non-trivially on $V$, abelian, $(|G|, |V|)=1$, then $\exists W \subseteq V$,
$G$-invariant, on which $G$ acts non-trivially and irreducibly.
\begin{quote}
\emph{Proof:}
$V= [G,V] \oplus C_V(G)$ and $[G,V] \ne 1$.  By Maschke, $[G,V]$ decomposes into
irreducible modules.
\end{quote}
{\bf Theorem 26:}
If $A$ is a $p'$-group of automorphisms of the abelian $p$-group $P$ which
acts trivially on $\Omega_1(P)$ then $A= 1$.
\begin{quote}
\emph{Proof:}  $P= C \times H$ where $C= C_P(A)$ and $H= [P, A]$.  By hypothesis,
$\Omega_1(P) \subseteq C$ so $\Omega_1(P) \subseteq \Omega_1(C)$ and
$\Omega_1(P) = \Omega_1(C) \times \Omega_1(H)$ so $\Omega_1(H) = 1$.
So $H = 1$ and $P= C$  thus $A$ acts trivially on $P$ and $A \subseteq Aut(P)$.
and $P = 1$
\end{quote}
{\bf Theorem 27:}
If $A$ is a $p'$-subgroup of $Aut(P)$ and $1 \ne \phi A$, $\phi$ acts trivially
on every proper $A$-invariant normal subgroup of $P$, then
(1) $P' \subset {\mathbb Z}(P)$;
(2) $P/P'$ is an elementary abelian subgroup and $A$ acts irreducibly on $P/P'$;
(3) Either $P$ is elementary abelian or $P$ has class $2$, $P' = {\mathbb Z}(P) = \Phi(P)$
is elementary abelian and $\phi$ acts trivially on $P'$.
\begin{quote}
\emph{Proof:}  First, $\phi$ acts trivially on $P'$ since it is a proper $A$-invariant normal
subgroup of $P$.  $\phi$ does not act trivially on $P/P'$ because if it were,
$P \supset P' \supseteq 1$ of $P$ and $\phi= 1$ which it is not.
Suppose ${\overline P}= P/P' = {\overline P}_1 \times {\overline P}_2$ where
${\overline P}_1 \ne 1$ and
${\overline P}_i$ is $A$-invariant.  Then $P= P_1 P_2$ and $P_i$ is a proper
$A$-invariant normal subgroup of $P$ where $P_i$ is the inverse image of ${\overline P}_i$.
If $\phi$ is trivial on $P_1$ and $P_2$ then $\phi$ is trivial on $P$.
Thus we can assume $A$ acts indecomposiblyon ${\overline P}$.
If $\Omega_1({\overline P}) \subset {\overline P}$ then $\phi$ acts trivially on the
inverse image of ${\overline P}$ and $A$ acts trivially on
$\Omega_1({\overline P})$ but then $\phi$ acts trivially on ${\overline P}$.
Thus ${\overline P} = \Omega_1({\overline P})$.  From Maschke, $A$ acts irreducibly on
${\overline P}$.  Now, put $B= \langle \phi^A \rangle$.  Since $\phi$ acts trivially
on $P'$, $B$ acts trivially on $P'$.  Set $H= [P, B]$.  If $H \subseteq P'$, we have
$1 = [H,B]= [H, B, B]$ and so $B=1$.  Thus $H \nsubseteq P'$.
Moreover, since $B$ and $P$ are $A$-invariant and $H \lhd P$ and thus $H$ is $A$-invariant.
If $H \subset P$ then $\phi$ acts trivially on $H$ and hence on ${\overline H}$ but ${\overline H}$ is a non-trivially of ${\overline P}$
hence ${\overline H} = {\overline P}$ by irreducible action of $A$ on ${\overline P}$ and
thus $H = P$.  Since $B$ centralizes $P'$ and $P' \lhd P$, $[P',P,B] = 1 = [P, P', B]$
so $[P, B, P'] = 1$.  Since $[P, B] = P$ and $P'$ centralizes $P$ and $P' \subseteq {\mathbb Z}(P)$ which proves 1.
Assume $P$ is elementary abelian.  But $P' \ne 1$ and so $P' \subseteq {\mathbb Z}(P) \subset P$.
Now if $Z= {\mathbb Z}(P)$, ${\overline Z}$ is a proper $A$-invariant subgroup.  So
${\overline Z} = 1$ by the $A$-irreducibility.  Hence ${\mathbb Z}(P) = P'$ and
$cl(P)=2$.  Since $P' \subseteq \Phi(P)$, $P' = \Phi(P)$.
If $x, y \in P$ then $[x, y]= z \in  P'= {\mathbb Z}(P)$  so $[x,y^p]= z^p$.  But
$y^p \in {\mathbb Z}(P)$ and then $z^p= 1$ and $[x,y]^p= 1$.  Since $P'$ is abelian,
it follows that $P'$ is elementary abelian and the proof is complete.
\end{quote}
{\bf Theorem 28:}
Let $G$ be a $p'$-group acting non-trivially on an abelian $p$-group $V$ then $G$ acts
non-trivially on $V_0 \subseteq V$ with $V_0 = \{ v: pv= 0 \} = \Omega_1(V)$.
\begin{quote}
\emph{Proof:}  By induction on $|P|$.  If $Q \le P, \Omega_1(Q) \le \Omega_1(P)$.  So if
$Q$ is any proper $A$-invariant subgroup, the action is trivial on $\Omega_1(Q)$
and $P$ is a special $p$-group (see chapter 15) and $A$ acts trivially on $P'$ and
irreducibly on $P/P'$ so $cl(P) \le 2$.  Hence,
$x \in P$ and $\varphi \in  A$, implies $ (\varphi(x)x^{-1})^p = (\varphi(x))^px^{-1)^p}) $
but then $x^p \in P'$ and $\varphi$ fixes $x^p$.  Since $[P,A] = \langle \varphi(x) x^{-1} \rangle$,
$[P, A] \subseteq \Omega_1(P)$ and $A$ would stabilize the series
$P \supseteq \Omega_1(P) \supseteq 1$ and thus $A=1$.
\end{quote}
{\bf Theorem 29:}
Let $G$ act faithfully and irreducibly on an abelian group of $V$, $(|G|, |V|)= 1$ then
${\mathbb Z}(G)$ is cyclic.
\begin{quote}
\emph{Proof:}  If ${\mathbb Z}(G)$ is not cyclic it contains a subgroup $H$ of type $(p, p)$ hence
by an earlier result, $H= \bigcup_{i} H_1, H_i \cap H_j = \{ 0 \}$ since $p \nmid |V|$ then $pV \ne \{ 0 \}$, $\exists x_i: C_V(x_i ) \ne 0$.  Now $G$
acts faithfully so $C_V(x_i) \ne V$ and $C_V(x_i)$ is $G$-invariant.  This is a contradiction.
\end{quote}
{\bf Theorem 30:}
Let $P$ be a $p$-group and $Q$ be a non-cyclic abelian $q$-group of automorphisms acting on a $P$,
$q \ne p$
then $P= \prod_{x \in Q^{\#}} C_P(x)$.
\begin{quote}
\emph{Proof:}  By induction on $|P|$.
Let $x_j$ be a fixed ordering of $Q^{\#}$.
Put $Z= \Omega_1({\mathbb Z}(P))$ and $Z_j= C_Z(x_j)$, some $j$,
then $Z \ne 1$ by the previous result on abelian $P$.
The theorem holds by induction on ${\overline P}= P/Z_j$ and
${\overline {C_P(x_i)}}= C_{\overline P}(x_i)$, so $P= Z_j \prod_{i=1}^n C_P(x_i)$ but
$Z_j \subseteq C_P(x_j) \subseteq P$.  This proves the result.
\end{quote}
{\bf Theorem 31:}
Let $\varphi$ be a representation of $G$ on a vector space $V$ with $char(F)=0$ or
$(char(F), |G|) = 1$. If $ V= V_1 \supset V_2 \supset \ldots \supset V_n \supset V_{n+1}= 0$ a sequence of $G$-invariant
subspaces and $\varphi(G)$ acts trivially on each $V_i/V_{i+1}$ then $\varphi$ is the trivial
representation.
\begin{quote}
\emph{Proof:}  $V= V_n \oplus W$ and $W$ is $G$-invariant.  $\varphi_{|W}$ is equivalent to the
quotient representation, ${\overline {\varphi}}$ of $G$ on $V/V_n$ but then
induction on $dim_F(V)$ yields $\varphi_{W}$ is trivial and since $\varphi_{V_n}$ is also
trivial, $\varphi$ is trivial.
\end{quote}
{\bf Theorem 32:}
Let $A$ be a regular group of automorphisms of a $p$-group $P$ then
(1) $A$ is a $p'$-group;
(2) $A$ possesses no non-cyclic abelian subgroups;
(3) A subgroup of $A$ of order $qr$ is cyclic.
\begin{quote}
\emph{Proof:}  Let $B \in S_p(A)$.  $B$ acts on $V= \Omega_1({\mathbb Z}(P))$ which is elementary abelian
and so $C_V(B) \ne 1$.  Since elements of $A^{\#}$ only fix $1 \in G$, this forces
$B= 1$ and $A$ is a $p'$-group. (2) follows from the fact that $V= \langle C_V( \varphi): \varphi \in Q^{\#} \rangle$ and this forces $C_V(\varphi) \ne 1$
contradicting regularity.  For (3), let $D$ be such a subgroup. $q \ne r$ is obvious so
assume $q > r$ and $Q \in S_q(D), R \in S_r(D)$.  $D= QR$ and $Q \lhd D$.  If
$C_D(Q) \supset Q$ then $C_Q(D)= D$ and $D$ is abelian and hence cyclic.  Hence we
can also assume $C_D(Q)= Q$.  Applying Thompson's $p \times q$ theorem to the
action of $D$ on $V$ gives $C_V(R) \ne 1$ which again contradicts regularity.
\end{quote}
\section {$A$-invariance}
{\bf Lemma:}
If $G$ is a group acting on a group $V$ and $H \lhd G$ then $C_V(H)$ is $G$-invariant.
\begin{quote}
\emph{Proof:}  Put $W= C_V(H)$.  For $w \in W, h \in H$ and any $x \in G$, $w^{xhx^{-1}} = w$ since
$xhx^{-1} \in H$.  Thus  $(w^x)^h = (w^x)$ and $w^x$ is fixed by $h$, that is $w^x \in W$.
\end{quote}
{\bf Theorem 33:}
Let $G$ be a $p$-group acting on an elementary abelian $p$-group $V$ (Hence $G$ is a set of
invertible transformations in $GL_n(p)$.)  Then $\exists v \in V, v \ne 0: v \in C_V(G)$.
\begin{quote}
\emph{Proof:}  By induction on $|G|$.  Let $M$ be a maximal subgroup of $G$.  $|G:M|= p$ and $M \lhd G$.
Put $W= C_V(M)$.  By induction, $W > \{1 \}$ and by the lemma, $W$ is $G$-invariant.
Choose $y \in G \setminus M$.  The minimal polynomial for $y$, $min(y) \mid (x^p-1)$ since
$y^p \in M$.  So $\exists w_1 \in W, w_1 \ne 0: w_1^y = w_1$ and we already know
$w_1^m = w_1, m \in M$ so $w_1 \in C_V( \langle y, M \rangle)= C_V(G)$ and we're done.
\end{quote}
{\bf Note:} In this section, $A$ acts on $G$ and $(|A|, |G|)=1$ with either
$A$ or $G$ solvable.  Note that $A/C_A(G)$ acts faithfully on $G$.
\\
\\
{\bf Theorem 34:}
Suppose that $A$ is an elementary abelian $p$-group such that $r(A) \ge 3$.  If $P, Q$ are
$A$-invariant $p'$-groups, $\exists a \in A$ such that
$C_P(a) \ne 1$ and $C_Q(a) \ne 1$.
\begin{quote}
\emph{Proof:}  Let $V$ be a subgroup of type $(p, p)$.  Since
$P= \langle C_P(v): v \in V^{\#} \rangle$, $\exists v \in V: C_P(v) \ne 1$.  Let
$W \subseteq A$ such that $W$ is of type $(p, p)$ and $W \cap \langle v \rangle = 1$.
$\exists w \in W: C_P(w) \cap C_P(v) \ne 1$ since $C_P(v)$ is $W$-invariant.
Then $\langle v, w \rangle$ is of type $(p, p)$ and acts on $Q$.  Thus $\exists a \in \langle v, w \rangle^{\#}$ such that  $C_Q(a) \ne 1$.
Since $C_P(a) \supseteq C_P(w) \cap C_P(v) \ne 1$, we are done.
\end{quote}
{\bf Theorem 35:}
If $U \leq G$ is $A-$invariant and $g$ satisfies $(Ug)^A=Ug$ then
$\exists c \in C_G(A)$: $Ug=Uc$. If $N$ is an $A-$invariant normal subgroup of $G$ then
(1) $C_{G/N}(A)= C_G(A)N/N$ (This shows $G=[G,A]C_G(A)$.) and (2) if $A$ acts
trivially on $N$ and $G/N$ then $G$ acts trivially on $G$. If $p \mid |G|$ (the analogous
results hold for $\pi$) then (1) $\exists S \in S_p(G): S^A=S$, (2) all such $A-$invariant
Sylow $p-$groups are conjugate under $C_G(A)$, (3) every $A-$invariant $p$-group of
$G$ is contained in an $A-$invariant Sylow $p-$group. \begin{quote}
\emph{Proof:}  Suppose $g \in G$ then $(UG)^A = Ug$ so $[g,a] \in U, \forall a \in A$.
$A^{g^{-1}}a \subseteq AU$ and $A$ and $A^{g^{-1}}$ are complements $U$ in $AU$
so $\exists n \in U: A= A^{gn}$ and the result holds.
\end{quote}
{\bf Theorem 36:}
Let $N$ be an $A$-invariant normal subgroup of $G$, $(|A|, |G|)=1$ then
(a) $C_{G/N}(A)= C_G(A)N/N$ and  (b) if $A$ acts trivially on $N$ and $G/N$ then
$A$ acts trivially on $G$.
\begin{quote}
\emph{Proof:}  By the previous result, $\exists c \in C_G(A): Ng= Nc$.  This proves (a).(b) follows from (a).
\end{quote}
{\bf Theorem 37:}
Let $N$ be an $A$-invariant normal subgroup of $G$, $(|A|, |G|)=1$ then
if $A$ acts trivially on $N$, it acts trivially on $G/C_G(A)$.
\begin{quote}
\emph{Proof:}  Since $[N,A]=1$, $[N,A,G]= 1= [G,N,A]$ and so $[A,G,N] = 1$.
\end{quote}
{\bf Theorem 38:}
Let $G$ act on $\Omega$ and $K \lhd G$ with (1) $(|K|, |G/K|)=1$, (2) $K$ or $G/K$ solvable,
(3) $K$ acts transitively on $\Omega$.  Then for evey complement, $H$ of $K$, (1) $C_{\Omega}(H) \ne \emptyset$ and
(2) $C_K(H)$ acts transitively on $C_{\Omega}(H)$.
\begin{quote}
\emph{Proof:}  For (a), let $\beta \in \Omega$.  Since $K$ is transitive, $|\Omega| \mid |K|$ and $G= K G_{\beta}$.
$G/K \approx G_{\beta}/(G_{\beta} \cap K)$.  Apply Schur-Zassenhaus to $G_{\beta} \cap K$ to get a complment
$H_1$ with $\beta \in C_{\Omega}(H_1)$, $H_1(G_{\beta} \cap K)= G_{\beta}$ and $H_1K=G$.  Since all complements
conjugate to $H_1$, (a) is established.  For (b), let $\alpha, \beta \in C_{\Omega}(H), k \in K$ with
$\alpha^k= \beta$.  $H$ and $H^k$ are two complements of $K \cap G_{\beta}$ in $G_{\beta}$ so
$\exists k' \in G_{\beta}$ such that $\alpha^{k k'}= \beta$.  $[k k', H] \leq H \cap K =1$ so
$k k' \in C_K(H)$.
\end{quote}
{\bf Theorem 39:}
Let $p \mid |G|$ and suppose the action of $A$ on $G$ is co-prime.  (1) There is an $A$-invariant Sylow $p$-group of
$G$. (2) The $A$-invariant Sylow $p$-groups are conjugate under $C_G(A)$.  (3) Every $A$-invariant $p$-subgroup is
contained in an $A$-invariant Sylow $p$-subgroup of $G$.
\begin{quote}
\emph{Proof:}  The semi-direct product $AG$ acts on $S_p(G)$ and (1) and (2) follow from previous result.
For (3), let $U$ be a maximal $A$-invariant $p$-subgroup of $G$, we show $U \in S_p(G)$. Assume not.  Then $U \notin S_p(G_1)$, $G_1 = N_G(U)$.  $G_1$ is $A$-invariant and $\exists T \in S_p(G_1)$
by (1).  $U<T$ which contradicts maximality.
\end{quote}
{\bf Theorem 40:}
If $T=\bigcap_{S \in S_p(G), S^A=S} S$,
then $T$ is the largest $A-$invariant $p-$subgroup of $G$ normalized by $C_G(A)$.
\begin{quote}
\emph{Proof:}  By previous result, $\exists S \in S_p(G): S^A=A$ and
$\{P \in S_p(G): P^A=P \} = \{S^c: c \in C_G(A) \}$, so the intersection of the Sylow groups is
$A$-invariant. By previous result, any $A$-invariant $p$-subgroup is contained in an $A$-invariant
Sylow $p$-group of $G$.  If $U$ is normalized by $C_G(A)$, then $U$ is contained in every $A$-invariant
Sylow $p$ group and hence their intersection.
\end{quote}
{\bf Theorem 41:}
If $P$ is an $A-$invariant Sylow $p-$group of $G$ and $H \le G$ with $H^A=H, H^{C_P(G)}=H$ then
$P \cap H \in S_p(H)$.  If $A= P \times Q$ acts on $M$ and $P, M$ are $p-$groups and
$Q$ is a $p'-$group with $C_M(P) \le C_M(Q)$ then $[M,Q]=1$. If $A$ acts trivially on
$G/\Phi(G)$ then $A$ acts trivially on $G$ and if $\Phi(G)$ is a $p-$group then
so is $A/C_A(G)$. \begin{quote}
\emph{Proof:}  $G= \Phi(G) C_G(A)$ so $G= C_G(A)$.
\end{quote}
{\bf Applying $P \times Q$:} If $p \in \pi(G)$ and ${\overline G}= G/O_{p'}(G)$
with $C_{\overline G}(O_p({\overline G})) \le O_p({\overline G})$ then $\forall P \in p(G), O_{p'}(N_G(P))=O_{p'}(G) \cap N_G(P)$.
\\
\\
{\bf Thompson:} Let $a$ be a $p'$ automorphism of a $p$ group $G$ and suppose
$X$ is a $p$-group of automorphisms of $G$ and
$[a,X]=1=[a,C_G(X)]$ then $a=1$.
\begin{quote}
\emph{Proof:}  Let $N \subseteq G$ be and $X$-invariant subgroup such that $[a,N] \ne 1$ but
$[a, K]= 1$ for all $X$-invariant proper subgroups $K$ of $N$.  Since
$[N,X,a]=1$ and $[X,a,N]=1$,  $[a, N, X] = 1$, $[N,a] \subseteq C_G(X)$,
$[N,a,a]= 1$ and thus $[N,a]= 1$.
\end{quote}
{\bf Theorem 42:} Let $a$ be a $\pi'$ automorphism of a $\pi$ group $G$ and suppose
$X \lhd \lhd G$ such that $[a,X]=1=[a,C_G(X)]$ then $a=1$.
\begin{quote}
\emph{Proof:}  Let $X \lhd X_1 \lhd \ldots \lhd X_n=G$ and choose $i$ so that $[a, X_{i+1}] \ne 1$ but
$[a, X_{i}] = 1$.  Let $N= N_G(X_i )$.  Since $X_{i+1} \subseteq N$, $[a,N] \ne 1$, yet
$[X_i, N, a] =1$ and
$[X_i, a, N] =1$ so
$[a, N, X_i] =1$ and $[N,a] \subseteq C_G(X_i ) \subseteq C_G(X)$.  Hence
$[N,a,a] = 1$ and $[N,a] = 1$.
\end{quote}
{\bf Theorem 43:}
If $P$ is a $p-$group and $Q$ a $p'-$ group with $Q \mapsto Aut(P)$ then
$Q$ is faithful on $P/\Phi(P)$.
\begin{quote}
\emph{Proof:}  Suppose $b \in A$ centralizes $G/\Phi(G)$.  If $b \ne 1$, there is a non-trivial
power of $b$ which is a $q$-element and which centralizes $G/ \Phi(G)$.  Let
$B= \langle b \rangle $ and $g \in G$.  $B$ acts on the coset $Xg$.  $|Fix_B(Xg)| = |X| \jmod{q}$.
$X= \Phi(G)$ has order that is a power of $p \ne 0 \jmod{q}$, so $b$ centralizes some
element $x \in X$.  So $G= \langle Y \rangle \le C_G(B)$ where $Y$ is a set of coset representatives,
thus $B=1$.
\end{quote}
{\bf Theorem 44:}
If $A$ acts on $G$, $(|A|,|G|)= 1$ and $G$ be abelian, $A$ is faithful on $\Omega_1(G)$.
\begin{quote}
\emph{Proof:}  WLOG, $[A, \Omega_1(G)]=1$.  Let $X$ have order $p$, ${\overline G}= G/X$.  $A$ is
faithful on $\Omega_1(G)$ so $A$ is faithful on $\Omega_1({\overline G})$ so WLOG
${\overline G}= \Omega_1({\overline G})$.  By maximality, $C_{\overline G}(A)=1$ so
$X=C_G(A)$, $X= \Omega_1(G)$ and so $G$ is cyclic.  This is not possible given the
automorphism group of a cyclic group.
\end{quote}
\section {More Results}
{\bf Definition:} $G$ is $p$-solvable if $G$ has a normal series whose quotients consist of $p$-groups
and $p'$-groups.
\\
\\
{\bf Definition:} $G$ is an $A_p$-group if $\forall S, T \in S_p(G)$, $\exists x \in C(S \cap T)$ such that
$S^x = T$.  For $X \in p(G)$, define $A_G(X) = N_G(X)/C_G(X)$.
\\
\\
{\bf Theorem 45:} If $G$ is a solvable group and $H$ is a minimal normal subgroup then $H$ is elementary abelian.
\begin{quote}
\emph{Proof:}
$H$ is characteristically simple so $H' = 1$ and $\Omega_1(H) = H$ so $H$ is elementary abelian.
\end{quote}
{\bf Theorem 46:} If $\alpha \in Aut(G)$ and $\alpha$ acts trivially on $G/\Phi(G)$ then $G$ acts trivially on $G$.
\begin{quote}
\emph{Proof:}
Suffices to assume that $\alpha$ has order $q \neq p$.  Let $x_1 , x_2 , \ldots x_r$ be a minimal generating set.
Since $\alpha$ fixes each coset, $\alpha$ acts on each coset.  Since $q \neq p$, $\alpha$ must fix an element
of each coset. Say $y_i$ is fixed in $\Phi(G) x_i$.  $\langle y_1, \ldots y_r \rangle = G$ and the result holds.
\end{quote}
{\bf 1.2.3 Theorem:} If $G$ is $p$-solvable with $O_{p'}(G) = 1$ then $C_G(O_p(G)) \subseteq O_p(G)$.
\begin{quote}
\emph{Proof:}
Put $H = O_p(G) C_G(O_p(G)) \lhd G$.  If $O_{p'}(H) = 1$: if not, $O_{p'}(H) \subseteq O_{p'}(G) = 1$
which is a contradiction.  Suppose $H > O_p(G)$.  $H/O_p(G)$ is $p$-solvable.  $H/O_p(G)$ cannot have
a normal $p$-group since it would be in $O_p(G)$.  Let $K$ be the inverse image of $O_{p'}(H/O_p(G))$.
$K \lhd G$ and $K > O_p(G)$.  $O_p(G) \in S_p(K)$ so $\exists L \in p'(G)$: $K = LO_p(G)$.  Let $x \in L$,
$x = yz, y \in O_p(G), z \in C(O_p(G))$.  $[x, O_p(G)] = [y, O_p(G)] =1$ so $K \subseteq O_{p'}(G)$, contradiction.
\end{quote}
{\bf Theorem 47:} If $N \subseteq G$ has a normal $p$-complement ($G = P N, N \lhd G$) then $G$ is an
$A_p$ group.  If $X \in p(G)$ then $A(X)$ is a $p$-group.
\begin{quote}
\emph{Proof:}
Suppose $S, T \in S_p(G)$, $G = SN = TN$.  $\exists y \in N: S^x = T$.  Suppose $s \in S \cap T$, $S^y \in T$
so $[s, y] \in T$,  so $[s, y] = 1$ and $y \in C(S \cap T)$.  Let $X \in p(G)$.  $M = N_G(X)$, then $M \cap N$
is a normal $p$-complement of $M$.  If $x \in X$ and $y \in M \cap N$ then $[x,y] \in X \cap (M \cap N)$, so
$y \in C(X)$ and $M \cap N \subseteq C(X)$.  Thus $y$'s action is as a $p$-group and $A(X)$ is a $p$-group.
\end{quote}
{\bf Frobenius Theorem:} $G$ be group. If $\forall X \in p(G)$, $A_G(X)$ is a $p$-group then $G$ has a normal
$p$-complement.
\begin{quote}
\emph{Proof:}
Let $G$ be a minimal counterexample.
\\
\\
1. If $H < G$ then $H$ has a normal $p$-complement.
\\
Let $X \in p(H)$, $A_H(X) = N_H(X)/C_H(X) \rightarrow A_G(X)$, so $H$ satisfies the hypothesis of theorem and
$H$ has a normal $p$-complement.
\\
\\
2. $O_p(G) = 1$.
\\
Let $K = O_p(G)$.  $X/K \in p(G/K)$.  So $A_G(X) \rightarrow A_{G/K}(X/K)$.  $G/K$ has a normal $p$-complement
$(M/K) (X/K) = G/K$.  $M \lhd G$ and $K \in S_p(M)$, $K \lhd M$, $LK = M$ and $[L,K] \subseteq K$.
$M = L \times K$,
$L = O_{p'}(M)$,  $M \lhd G$ and $L \lhd G$.  $L$ is a normal $p$-complement.  Contradiction.
\\
\\
3. $G$ is an $A_p$-group
\\
Let $P, P_1 \in S_p(G)$.  Proof is by induction on $|P: P \cap P_1|$.  $P \neq P_1$ and $P \cap P_1 \neq 1$.
Put $H = N_G(P \cap P_1 )$.  Since $O_p(G) = 1$, $H \neq G$ and $H$ has a normal $p$-complement.  If $P > Q$.
$N_P(Q) > Q$.  Pick $R, R_1 \in S_p(H)$ then $H \cap P \subseteq R$ and
$H \cap P_1 \subseteq R_1$.  $\exists y \in C_H(R \cap R_1): R_1 = R^y$.  Let $S \in S_p(G)$ with $R \subseteq S$.  Set $S_1 = S^y$ so
$R_1 \subseteq S_1$.   $P \cap S > P \cap P_1$ and $P_1 \cap S_1 > P \cap P_1$
By induction, $\exists u \in C_G(S \cap P),
\exists v \in C_G(S_1 \cap P_1):$ $P^u=S$ and $S_1^v=P_1$ .  $P^{uyv} = P_1$ and $uyv \in C(P \cap P_1)$.
Since $R \cap R_1$, $S \cap P$ and $S_1 \cap P_1$ all contain $P \cap P_1$, $uyv \in C_G(P \cap P_1 )$.
\\
\\
4. Contradiction
\\
Let $P \in S_p(G), g,h \in P$, $h= g^x$ and $h \in P \cap P_1$.  Since $G$ is a $A$-group,
$\exists y \in C(P \cap P^x)$ with $P^{xy}=P$, $h^y=h$.  Now $xy \in N(P)= C(P)P$.  Since $G$ is an $A_G(P)$ is a $p$-group, $xy=uv$, wtih $u \in C(P)$ and $ v \in P$.
$g^v = g^{uv} = g^{xy} = h^y = h$.  Thus any two elements conjugate in $G$ are conjugate in $P$ so $G$ has a normal
$p$-complement.
\end{quote}
{\bf Theorem 48:} Let $G$ be a group with no normal $2$-complement and $Q \in S_2(G)$.
Suppose $Q$ has a cyclic subgroup,
$A$ of index $2$.  Then $ 3 \mid |G|$.
\begin{quote}
\emph{Proof:}
Since $G$ does not have a normal $p$-complement, $\exists X \subseteq Q$ such that $N(X)/C(X)$ is not a
$2$-group. $X \cap A$ is cyclic and $|X: A \cap X|$ is 1 or 2 and $X$ has two generators.
$\exists g \in N(X)/C(X)$ of order
$q, q \neq 2$ that acts faithfully on $X$ and hence $X/\Phi(X)$.  Since $X$ has 2 generators, the Burnside
Basis Theorem yields $|X/\Phi(X)| \leq 4$ and $q = 3$.
\end{quote}
{\bf Definition:} Let $S$ be a  $p$-group.  For $A: A' = 1$, $m(A)$ is the minimal number of generators of $A$.
$d(P)$ is the maximum value of $m(A)$ for $A \subseteq S$, $A' = 1$. $J(S) = \langle A: m(A) = d(S) \rangle$.
${\cal H}(G)$ be a set of $p$-groups, $H \subseteq G$: $N_G(H)$ does not have a normal $p$-complement.
\emph{Thompson ordering:} For $H, K \in G$, define $H \leq \leq K$ iff (a) $|N_G(H)|_p < |N_G(K)|_p$, or
(b) $|N_G(H)|_p = |N_G(K)|_p$ and $H < K$, or
(c) $H = K$.
\\
\\
{\bf Theorem 49:}   Let $G$ be a $p'$ group acting non-trivially on an abelian $p$-group, $V$ with
$(|G|, |V|) =1$.  $\exists W \subseteq V$ on which $G$ acts non-trivially and irreducibly.
\begin{quote}
\emph{Proof:}
$V = V_1 \times V_2$, $V_1 = ker(\sigma)$, $V_2 = im(\sigma)$, where
$\sigma(v)= \sum_{g \in G} v^g$.
Let $V_1$ and $V_2 \neq 0$ be as above.  Let $W \neq 0$ be a minimal $G$-invariant
subgroup of $V_2$.  Since $W \cap C_G(G) = 0$, $G$ acts nontrivially on $W$.
\end{quote}
{\bf Corollary 1:}   Let $G$ be a $p'$ group acting non-trivially on an abelian $p$-group, $V$.
Then $G$ acts non-trivially on $V_0 \subseteq V: V_0 = \{ v \in V: vp = 0 \}$
\\
\\
{\bf Corollary 2:}   Let $G$ act faithfully and irreducibly on an abelian group, $V$.
If $(|G|, |V|) =1$, then ${\mathbb Z}(G)$ is cyclic.
\begin{quote}
\emph{Proof:}
If not, $G$  has a central subgroup $H$ which is abelian of type $(p, p)$, some $p \mid |G|$.
$H$ is a disjoint union of $p+1$ subgroups $J_0, \ldots, J_p$ of order $p$. Since $p \nmid |V|$,
$vp \neq 0$ and $\exists i: C_V(J_i) \neq 0$.  $G$ acts faithfully on $V$ so $C_V(J_i) \neq V$.
Since $J_i \lhd G$, $C_V(J_i)$ is a proper $G$ invariant subgroup of $V$ and $G$ does not act irreducibly,
contradiction.
\end{quote}
{\bf Theorem (Thompson):}  Let $p$ be an odd prime.  If $S \in S_p(G)$ and both $C_G({\mathbb Z}(S))$ and $N_G(J(S))$ have normal $p$-complements, so does $G$.  There is another proof of this in the
stability section.
\begin{quote}
\emph{Proof:}
Let $G$ be a minimal counterexample of minimal order.
\\
\\
1. ${\cal H}(G) \neq \emptyset$ by Frobenius. Let $H$ be maximal in the Thompson ordering of ${\cal H}(G)$.
Put $N = N_G(H)$.  $P \in S_p(N)$ with $H \subseteq P \subseteq S$.
\\
\\
2. $H \neq S$.
\\
If $H = S$ then $N \subseteq N_G(J(S))$ which has a normal $p$-complement.
So $N$ has a normal $p$-complement.
Contradiction.
\\
\\
3. ${\overline N} = N/H$ has a normal $p$-complement.
\\
$P \in S_p(N)$.  Since $H \neq S$, $N_S(H) > H$ and $P > H$.
Let ${\overline P}$ be the image of $P$ in ${\overline N}$ then ${\overline P} \in S_p({\overline N})$.
Suppose ${\overline N}$ does not have a normal $p$-complement.  Since $|{\overline N}| < |G|$, either
$C_{\overline N}({\mathbb Z}({\overline P}))$ or $N_{\overline N}(J({\overline P}))$ does not have a normal $p$-complement.
Let $K$ be the complete inverse image of either ${\mathbb Z}({\overline P})$ or $J({\overline P})$, which ever one fails.
$K$ is a $p$-group and $N_G(K)$ does not have a normal $p$-complement.
$P \subseteq N_G(K)$ so either $|N_G(K)|_p > |N_G(K)|_p$ or
$|N_G(K)|_p = |N_G(K)|_p$ and $|K| > |H|$.  So $K \in {\cal H}$, $K \geq \geq H$ but $K \neq H$, contradiction.
\\
\\
4. $N = G$.
\\
$N$ satisfies the hypothesis. $H \subseteq P \subseteq S$, ${\mathbb Z}(S) \subseteq N_G(H) = N$. $P {\mathbb Z}(S) \subseteq N \cap S$
and $P {\mathbb Z}(S) = P$.  ${\mathbb Z}(S) \subseteq P$ so ${\mathbb Z}(S) \subseteq {\mathbb Z}(P)$ and
$C_N({\mathbb Z}(P)) \subseteq C_G({\mathbb Z}(P)) \subseteq C_G({\mathbb Z}(P))$, so $C_N({\mathbb Z}(P))$ has
a normal $p$-complement and so does $C_N({\mathbb Z}(P))$.
If $P = S$, $N_N(J(P) \subseteq N_G(J(S))$ has a normal $p$-complement.
Now suppose $P < S$.  Then $N_S(P) > P$ so $|N_G(J(P)|_p > |N|_p$.
By maximality of $H$, $J(P) \notin {\cal H}$
and so $N_G(J(P)$ has a normal $p$-complement and so does $N_N(J(P))$.
If $N \neq G$, since $G$ is minimal,
$N$ has a normal $p$-complement, contradiction.  So $N = G$.
\\
\\
5. $O_{p'}(G) = 1$
\\
Put $L = O_{p'}(G)$.  We show ${\overline G} = G/L$ satisfies the assumptions of the theorem.
Let $K \subseteq S$.  We show ${\overline {N_G(K)}} = N_{\overline G}({\overline K})$.  Let
${\overline x} \in N_{\overline G}({\overline K})$.  Since $L \lhd KL$, $K \in S_p(KL)$ so
$\exists a \in K, b \in L: K^x = K^{ab}= K^b$.
Hence $xb^{-1} \in N_G(K)$ and ${\overline {xb^{-1}}}= {\overline x}$.
${\overline {N_G(J(S)})} = N_{\overline G}(J({\overline S}))$ so ${\overline {J(S)}} = J({\overline S})$
and it has a normal $p$-complement.
${\overline {C_G(K)}} = C_{\overline G}({\overline K})$.  Conversely, let
${\overline x} \in C_{\overline G}({\overline K}) \subseteq N_{\overline G}({\overline K})$,
so we can assume $x \in N_G(K)$. Since $[x, K] \subseteq  K \cap L = 1$, $x \in C_G(K)$.
${\overline {{\mathbb Z}(S)}} = {\mathbb Z}({\overline S})$ so
${\overline {C_G({\mathbb Z}(S)}} = C_{\overline G(}{\mathbb Z}({\overline S})$
and it has a normal $p$-complement.
If $|L| > 1$, $|{\overline G}| < |G|$ so ${\overline G}$ has a normal $p$-complement
and its complete inverse image is a normal $p$-complement of $G$.  Contradiction.
\\
\\
6. $H = O_p(G)$ and $G$ is $p$-solvable of $p$-length $2$.
\\
Set $K = O_p(G)$ so $H \subseteq K$.  Since $G = N_G(K)$has no normal $p$-complement,
$K \in {\cal H}$.  By the
maximality of $H$ in ${\cal H}$ and the fact that $G=N_G(K)= N_G(H)$, $H=K$.
By 3 and 4, $G/H$ has a normal $p$-complement, so $G$ is $p$-solvable of length at most 2.
\\
\\
7.  ${\overline G} = G/H = {\overline {SM}}$ then ${\overline M}$ has a normal $p$-complement and
${\overline M}$ contains no proper ${\overline S}$-invariant subgroup.
\\
Let ${\overline M} > {\overline {M_0}} > 1$ and suppose ${\overline {M_0}}$ is
$S$-invariant.  Let $M_0$ be the complete inverse image of ${\overline {M_0}}$.  Put $G_0 = S M_0$,
so $G > G_0$.
Since
$C_{\overline {G_0}}({\mathbb Z}({\overline P}))$ or $N_{\overline {G_0}}(J({\overline P}))$ have normal $p$-complements,
so does $G_0 = S K_0$.  $K_0 \lhd G_0, S \cap K_0 = 1, [H, K_0] \subseteq K_0, [H, K_0] \subseteq H$.  Hence
$[H, K_0] = 1$ and $K_0$ centralizes $H$, this contradicts 5, 6, and 1.2.3.
\\
\\
8. ${\overline M} = {\overline Q}$ is an elementary abelian $q \neq p$ group.  ${\overline S}$
acts irreducibly on ${\overline Q}$ and $S$ is a maximal subgroup of $G$.
\\
Let $q \mid |{\overline M}|$.  ${\overline S}$ permutes $S_q$ subgroups of ${\overline M}$ and the
number of such subgroups divides $|{\overline M}|$.  Some such orbit has size $1$, say it contains
${\overline Q}$.  By step 7, ${\overline M} = {\overline Q}$.
${\overline Q}$ has no proper $G$-invariant
subgroup so ${\overline S}$ acts irreducibly on ${\overline Q}$.  Finally, $G > L >S$ and
${\overline Q} > {\overline Q} \cap {\overline L} > 1$ and ${\overline Q} \cap {\overline L}$ is
a proper ${\overline S}$ invariant subgroup of ${\overline Q}$, which is a contradiction.  So
$S$ is maximal in $G$.
\\
\\
9. $\exists A$, abelian: $A \subseteq S$ with $m(a) = d(S)$ and with $A \nsubseteq H$.  Let $A$ be
fixed such group of minimal order.  Let $A_0 = A \cap H$ then $A/A_0$ is elementary abelian.
\\
If $J(S) \subseteq H$, $J(S) \; char \; H \lhd G$ so $G = N(J(S))$
has a normal $p$-complement, contradiction.
So $J(S) \nsubseteq H$.  Since $J(S) = \langle A \rangle, S \subseteq S$, $A' =1$ with $m(A)=d(p)$,
at least one such $A \nsubseteq H$.  Let $A$ be one such of minimal order.  Set $A_0 = A \cap H$.  $A_1/A_0 = \Omega_1(A/A_0)$,
$A_1'=1$.  So $A_1 \nsubseteq H$. Also, $m(A)= m(A_1)$.
By minimality, $A = A_1$ and $A/A_0$ is elementary abelian.
\\
\\
10. Let ${\overline A} = AH/H$ then ${\overline G} = {\overline A} {\overline Q}$ and $|{\overline A}| = p$.
\\
first, we show ${\overline G} = {\overline A} {\overline Q}$.
Since ${\overline G} = {\overline S} {\overline Q}$ and
${\overline S} \subseteq N({\overline Q})$.  This is failthful since $H = O_p(G)$ and hence a normal subgroup
of ${\overline G}$.  ${\overline A} = AH/H \cong A/A_0 \neq 1$.
\\
Now, ${\overline A}$ acts non-trivially and faithfully on some
${\overline {Q_1}} \subseteq {\overline Q}$ by theorem 49. Let $G_1$ be the complete inverse image in ${\overline G}$ of ${\overline A} {\overline {Q_1}}$.
$P_1 \in S_p(G_1)$, $A \subseteq P_1$. $P_1 \subseteq S$.  Since $H \subseteq P_1$, and $C_G(H) \subseteq H$, by
1.2.3, ${\mathbb Z}(S) \subseteq C_G(H) \subseteq H \subseteq P_1$ and ${\mathbb Z}(S) \subseteq {\mathbb Z}(P_1)$.  Hence
$C_G({\mathbb Z}(P_1)) \subseteq C_G({\mathbb Z}(S))$ and the latter has a normal $p$-complement.
Since $A \subseteq P_1$ and $m(A) = d(S)$, $D(P_1)=d(S)$.
Thus $A \subseteq J(P_1)$.  Let $Q_2 \in S_q(N_{G_1}(J(P_1))$.
$[A, Q_2] \subseteq [J(P_1), Q_2] \subseteq J(P_1)$ and $[A, Q_2]$ is a $p$-group.
${\overline {Q_2}} \subseteq {\overline {Q_1}}$ so
$[{\overline A}, {\overline {Q_2}}] \subseteq {\overline {Q_1}}$
is a $q$-group and $[{\overline A}, {\overline {Q_2}}] =1$.
${\overline {Q_2}}$ is an ${\overline A}$- invariant
of $Q_1$ centralized by ${\overline A}$.
So ${\overline {Q_2}} = 1$  and $Q_2 = 1$.  Thus $N_G(J(P)$ is
a $p$-group and has a normal $p$-complement.  If $G_1 < G$, $G_1$ has a normal $p$-complement which centralizes
$H$, contradiction.  So $G = G_1$, ${\overline G} = {\overline A} {\overline Q}$.  By step 8,
${\overline A}$ acts failthfully and irreducibly on ${\overline Q}$ and ${\overline A}$ is elementary abelian.
It is cyclic by the corollary above, so $|{\overline A}|=p$.
\\
\\
11.  Set $W = {\mathbb Z}(H)$, $Z \subseteq W: \Omega_1(W) = Z$.  If $Q \in S_q(G)$ then $Q \subseteq N(Z)$ but
$Q \nsubseteq C(Z)$.
\\
${\mathbb Z}(S) \subseteq C_G(H) \subseteq H$ so ${\mathbb Z}(S) \subseteq W$.
If $[Q, W] =1$, then it centralizes ${\mathbb Z}(S)$ so ${\mathbb Z}(S)$
is central in $G$.
But then $C_G({\mathbb Z}(S))$ has a normal $p$-complement, which is a contradiction.
By the corollary above, $Q$ acts non-trivially on $W$.
\\
\\
12.  Contradiction
\\
$Z \lhd G$ and $G$ acts on $Z$ by conjugation.  The kernel of the the action acts on
$C_G(Z)$ and $H \subseteq C_G(Z)$.
${\overline G}$ acts on $Z$.  Since ${\overline Q}$ is the unique maximal normal subgroup of ${\overline G}$
and ${\overline Q}$ acts non-trivially on $Z$ by 11, ${\overline G}$ acts faithfully on $Z$.
$Z = C \times V$ where $C= C_Z({\overline Q})$ and $V$ is ${\overline Q}$-invariant, ${\overline Q} \lhd G$.
${\overline G}$ acts faithfully on $V$.  Put $d=d(S)$.  Since $|{\overline A}| = p$.  $m(A_0) \geq d - 1$.
Now put
$V_0 = V \cap A_0$, $m(V_0) = t, m(V/V_0) = r, V \subseteq {\mathbb Z}(H)$ and $\langle V, A_0 \rangle$ is abelian.
$m(V) = t+r$  $d \geq m( \langle V, A_0 \rangle )= m(V) + m(A_0) = m( V \cap A_0 )= t+r+m(A_0 ) -t \geq d-1+r$.
So $r=0$ or $r = 1$.  Either $V = V_0$ or $V_0$ is maximal.  Choose $a \in A \setminus A_0$ and
$1 \neq {\overline b} \in {\overline Q}$, so $[{\overline a}, {\overline b}] \neq 1$ and
$[{\overline a}, V_0] = 1 = [{\overline b}, V_0]$.  $ \langle {\overline a}, {\overline b} \rangle$ centralizes
$V_0 \cap V_0^{\overline b}$.  $V: V \cap V_0| \leq p^2$ since ${\overline A}$ is maximal in $G$ and
${\overline a}^{\overline b} \notin {\overline A}$. $[{\overline G}, V_0 \cap V_0^{\overline b}] = 1$ so
$V_0 \cap V_0^{\overline b} = 1$, $|V| \leq p^2$.  ${\overline G}$ is generated by two elements so
${\overline G} \subseteq SL_2(p)$.  ${\overline Q}$ is normalized but not centralized by ${\overline A} \subseteq SL_2(p)$, which is a contradiction (Let $N \in p'(SL_2(p)), |P|= p$ if
$P$ normalizes but does not centralizer $N$, then $p=3$ and $N'=1$.
\end{quote}
{\bf Theorem 50a:} Let $G$ be a primitive permutation group on $\Omega$ and $N$ a normal Hall subgroup.  Let $H$ be a transitive complement for $N$.
If either $N$ or $H$ is solvable, $\exists a \in \Omega: N \subseteq G_a$.
\begin{quote}
\emph{Proof:}
Let $b \in \Omega$.  $G= NG_b$ and $N_b = N \cap G_b \lhd G_b$ and that $N_b$ is a Hall subgroup.  $G_b/N_b \cong H_b$.
Let $T$ be a complement for $N_b$ in $G_b$.  $N \cap T = N \cap G_b \cap T =1$ and $NT= N N_H T = N G_b = G$ and
$T$ is a complement for $N$ is $G$.  By SZ, $\exists g: T^g=H$, so $H \subseteq {G_b}^g = G_a, gb=a$.
\end{quote}
{\bf Theorem 50b:}  Let $G$ be a primitive group and $G_a$ a normal Hall subgroup.  Let $N$ be a regular normal subgroup
and either $N$ or $G_a$ is solvable.  Then $N$ is an elementary abelian $p$ group.
\begin{quote}
\emph{Proof:}
$G= G_aN$ and $N \cap G_a = 1$ so $N$ is also a Hall subgroup of $G$.  Let $P \in S_p(N)$.  $G$ permutes the conjugates
of $P$ and $N$ acts transitively.  By 50a, $G_a$, $G_a$ normalizes one such $S_p(N)$ group, say $P$.  Let $Q$ be a characteristic
subgroup of $P$.  $Q \lhd G_a$ so $G_a \subseteq G_aQ$, which is a subgroup.  Since $G_a$ is maximal, $G_aQ = G$ and $N=Q$.
\end{quote}
{\bf Theorem 50c:} Let $G$ be primitive and $G_a$ is nilpotent but not a $2$-group then $G$ has a regular
normal elementary abelian subgroup, $N$.
\begin{quote}
\emph{Proof:}
If $K \lhd G$ and $K \subseteq G_a$, $K = 1$.  If $1 < K \lhd G_a$, since $G_a$ is maximal, $N_G(K)=G_a$.  There
are two cases:\\
\\
Case 1: $|G_a|$ has at least two prime factors and $1 \ne P \in S_p(G)$.\\
$N_G(P)=H=G_a$ and $P \in S_p(G)$.  $H$ is a Hall subgroup.  Let $x, y \in P, x_z=y$ and let $1 \ne Q \in S_q(H)$,
$q \ne p$.  $x, y \in C_G(Q)$ so $y \in C_G(Q) \cap C_G(Q^z)$, $Q=Q^zu, u \in N_G(Q)$.  Put $h =zu$.
$x^h=y^u=y$.  If $h=h_1 h_2$ with $h_1 \in P, h_2 \in C_G(P)$ (Possible since $H$ is nilpotent.), $x^h = x^{h_2}= y$ and $x, y$ are $P$ conjugate.  If $H = \prod P_i$, where $P_i$ is a Sylow system, and
$x^z=y$, $\exists h_i \in P_i: y_i = {x_i}^{h_i}$.  Put $h= \prod h_i$ and $x^h=y$. So $H$ has a n.p.c.
\\
\\
Case 2: $G_a$ is a $p$-group.\\
$p \ne 2$.  If $K = {\mathbb Z}(H)$, $1 \ne K \lhd H$ and $N_G(K)$ has a npc.  By Thompson, so does $G$, cll it $N$.
$N$ acts semi-regularly since $N \cap G_a = 1$.  $G_a < G$ so $N \ne 1$. Since $G$ is primitive, $N$ is regular
and by 50c, $N$ is elementary abelian.
\end{quote}
{\bf Result:} If $G= PSL_{2}(17)$, $S \in S_2(G)$ is maximal.
\\
\\
{\bf Theorem 50:} Let $H$ be a maximal subgroup of $G$ and $H$ is nilpotent of odd order then
$G$ is solvable.
\begin{quote}
\emph{Proof:}
By induction on $|G|$. Suppose there is a $K$: $1 < K < H$ and $K \lhd G$.  $H/K$ is maximal in $G/K$
so $G/K$ is solvable.  Since $K$ is nilpotent, $G$ is solvable. If no such $K$ exists, $G$ permutes cosets
of $H$ in $G$.  Let $a = H$, then $G_a = H$.  Let $K$ be the subgroup of $G$ fixing all points.  Then
$K \subseteq G_a = H$ and $K \lhd G$ so $K = 1$.  Since $G$ is primitive, the result follows.
\end{quote}
{\bf Theorem 51:} $\Phi(G) \subseteq F(G)$.
\begin{quote}
\emph{Proof:} $\Phi(G)$ is nilpotent (Apply the Frattini argument to $\Phi(G)N_G(P), P \in S_p(\Phi(G))$.)
and characteristic in $G$ so $F(G)\Phi(G)$ is nilpotent so
$\Phi(G) \subseteq F(G)$.
\end{quote}
{\bf Definition:} $H$ is hyperfocal in $G$ if $H_{n+1} = [G, H_n]$, $H_0 = H$ is a series that terminates in
$1$.
\\
\\
{\bf Theorem 52:} Let $H$ is hyperfocal $S_{\pi}$ subgroup of $G$ then $H$ has a normal $\pi$-complement.
\begin{quote}
\emph{Proof:}
Induction on $|G|$.  $H \neq 1$.  Let $L = Foc_G(H)$ so $H' \subseteq L < H$.  Let $x \in H \setminus L$.  If
$(g^r)^x \in H$ then $[g^r, x] \in H \subseteq L$ and $(g^r)^x = g^r \jmod{L}$ and $G$ has a normal subgroup $K$
such that $G/K$ is a $\pi$ group.  $H \cap K \in S_{\pi}(G)$ which is hyperfocal in $G$ so, by induction, $K$ has
a normal $\pi$ complement, $N$.  $N \;  char  \; K$ so $N \lhd G$.  Since $G/K$ is a $\pi$ number, $N$ is a normal $p$-complement
in $G$.
\end{quote}
{\bf Theorem 53:} Let $H$ be a nilpotent $\pi$ - subgroup of $G$.  Suppose any two elements $x, y$ elements of
$H$ that are $G$-conjugate are $H$-conjugate.  Then $G$ has a normal $\pi$ complement.
\begin{quote}
\emph{Proof:}
$H$ is hyperfocal in $G$ because it is nilpotent and the result follows.
\end{quote}
{\bf Theorem 54:} Let $P \in S_p(G)$ and suppose $P \subseteq Z(N_G(P))$ then $G$ has a normal $p$-complement.
\begin{quote}
\emph{Proof:}
$P$ is nilpotent and the result follows from the previous theorem.
\end{quote}
{\bf Theorem 55:} If every Sylow subgroup of $G$ is cyclic, $G$ is solvable.
\begin{quote}
\emph{Proof:}
Induction on $|G|$.  Let $P \in S_p(G)$ for the smallest prime, $p \mid |G|$.  Set $N= N(P)$.  $N$ acts
on $P$ by conjugation so $\phi: N \rightarrow Aut(P)$.  Since $P$ is abelian, $Im( \phi )$ is a $p'$ group.
But $Aut(P)= p^t (p-1)$, $|P|= p^t$.  This is impossible since $p$ is the smallest prime dividing $|G|$.
\end{quote}
{\bf Theorem 56:} If $P \in S_p(G)$ and $P$ is cyclic with $P \nsubseteq G'$, then $G$ has a normal $p$-complement.
\begin{quote}
\emph{Proof:}
Put $N = N(P)$ and, again, let N act on $P$ by conjugation.  The image is a $p'$ group in $Aut(P)$.
$[P, x] \subseteq P \cap G' <P$.  If $S$ is a subgroup of $P$ of index $p$, $g^{-1} g^x = 1 \jmod{S}$
and $g^x=g$.  $P$ is in the center of its normalizer so it has a normal $p$-complement.
\end{quote}
{\bf Theorem 57:} There are $p+1$ $p$-subgroups of $G = SL_2(p)$.  A $2$-Sylow subgroup of $G$ is quaternion.
If $P$, $P_1$ are groups of order $p$ in $G = SL_2(p)$ then $\langle P, P_1 \rangle = G$.
\begin{quote}
\emph{Proof:}
Since there are $p + 1$ Sylow subgroups in two orbits of size 1 and $p$, $\langle P_1, P \rangle =
\langle P_1, \ldots P_{p+1} \rangle \lhd SL_2(p)$, so $\langle P_1, P \rangle = SL_2(p)$.
\end{quote}
{\bf Theorem 58:}
Suppose $A$ acts trivially on $G/\Phi(G)$.  (a) If the action of $A$ on $G$ is coprime, $A$
then $A$ acts trivially on $G$. (b) If $\Phi(G)$ is a $p$-group, then so is $A/C_A(G)$.
\begin{quote}
\emph{Proof:} (i)$G= \Phi(G)C_G(A)$ so  $G= C_G(A)$.
(ii) By (i), if $\Phi(G)$ is a $p$-group, $A$ acts trivially on $G$.
\end{quote}
{\bf Theorem 59:}
Let $G$ be a $p$-group and ${\cal K}$ the set of $A$-composition factors of $G$.  Suppose the action on
of $A$ on $G$ is coprime then $\bigcap_{K \in {\cal K}} C_A(K)/C_G(A) = O_p(A/C_A(G))$.
\begin{quote}
\emph{Proof:}  We may assume, by induction, that $A$ acts failthfully on $G$.
$[K,O_p(G)] = 1$ for all such factor, $K$.
By Theorem 3, every $p'$ subgroup $B$ acts trivially on $K$ and the result follows.
\end{quote}
{\bf Theorem 60:}  (a) If $G/\Phi(G)$ is cyclic, $G$ is cyclic.
(b) If $P$ is a $p$-group, $P/\Phi(P)$ is elementary abelian.
\begin{quote}
\emph{Proof:}  If $G/\Phi(G)$ is cyclic, $\langle g\Phi(P) \rangle = G/\Phi(G)$, so $\langle g, \Phi(P) \rangle = G$ and so
$\langle g \rangle = G$.  If $M$ is maximal in $P$, $|P:M| = p$ and $M \lhd P$.  Since $P/M$ is abelian,
$G' \subseteq M$ for any such $M$ and for $x \in G$, $x^p \in M$ for any such $M$.  The second result follows.
\end{quote}
