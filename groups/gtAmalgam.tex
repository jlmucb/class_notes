\chapter{Primitive groups, pairs and amalgams}
\section {Primitive Groups}
{\bf Definition 1:} $M<G$ is \emph{primitive} if $M=N_G(A), \forall A \lhd M$. $1 \ne M <G$ is
\emph{strongly $p$-embedded} if $|M \cap M^g|_p = 1$ for $g \in G \setminus M$.
$H<G$ is a $p$-local subgroup of $G$ if $M= N_G(P), P \in p(G)$.
\\
\\
{\bf Comment:} Bender classified groups with strongly embedded $2$-subgroups.
\\
\\
{\bf Theorem 1:}
Let $N \lhd G$, be abelian.  If $G/N$ is perfect then $G'$ is perfect.
\begin{quote}
\emph{Proof:}  
$(G/N)= (G/N)'$ so $G= G'N$ and 
$G/N \approx G'/(G' \cap N)$.  So
$G'/(G' \cap N)=
(G'/(G' \cap N))'$ and thus $G=G''N$ so $G/G'' \approx N/(G'' \cap N)$.  Since this is abelian,
$G' \subseteq G''$.
\end{quote}
{\bf Theorem 2:}
$C_M(O_p(M)) \subseteq O_p(M)$ is equivalent to $F^*(M)=O_p(M)$ and $O_{p'}(M)=1$.
\begin{quote}
\emph{Proof:}  
Clear.
\end{quote}
{\bf Theorem 3:}
If $M<G$ is maximal and $N \lhd G$ then $M/N$ is a maximal subgroup of $G/N$.
\begin{quote}
\emph{Proof:}  Clear.
\end{quote}
{\bf Comment:} The forgoing theorem let's us assume a maximal subgroup contains no normal subgroup of
$G$.
\\
\\
{\bf Theorem 4:}  Let $L \lhd \lhd G$.  If $L \leq F^*(G)$ then (a)
$L= (L \cap F(G))(L \cap E(G))$; (b) $F^*(L) = F^*(G) \cap L$; (c)
$E(L) C_{E(G)}(L) = E(G)$.  $E(L) \lhd E(G)$.
\begin{quote}
\emph{Proof:}  
Every component of $L$ is a component of $G$ and $F(L) \leq F(G)$.  Since $[F(G), E(G)]=1$.
\end{quote}
{\bf Theorem 5:}  The action of $G$ by right conjugation on cosets of primitive groups is primitive.
\begin{quote}
\emph{Proof:}  The action is obviously transitive and 
$Mg=M \rightarrow g \in M$ which is maximal so the action is primitive.
\end{quote}
{\bf Theorem 6:}
If $M<G$ is primitive, $N \lhd G$ and $M \cap N \neq 1$ then $C_G(N) =1$.
\begin{quote}
\emph{Proof:}  
$C_G(N) \lhd G$ and $C_G(N) < M \cap N$ then $C_G(N)=1$.
\end{quote}
{\bf Theorem 7:}
Let $M$ be a primitive subgroup of $G$.  No non-trivial subnormal subgroup of $G$ is contained in $M$.
$F(G) \cap M = 1$.
\begin{quote}
\emph{Proof:}  
Suppose $L \lhd \lhd G, 1 \neq L \leq M$.  Pick a minimal one, $L \leq F^*(G)$.  By the previous result,
${\mathbb Z}(F^*(G)) = 1$.  So $F(G)=1$ and $L$ is a component of $G$.
$\langle L^M \rangle \lhd E(G)$ and by primitivity, $E(G) \leq M$.  So $E(G)=1$, which is a contradiction.
\end{quote}
{\bf Theorem 8:}
If $M<G$ is a primitive subgroup, $p \in \pi(M)$, $N \lhd G$.  Suppose $M \cap N =1$ and $O_p(M) \neq 1$.
(a) $p \notin \pi(N)$, (b) $\forall q \in \pi(N), \exists$ an $M$-invariant subgroup Sylow $q$-subgroup
of $N$; (c) if $|\pi(N)| \geq 2$ then $M$ is a maximal subgroup of $G$.
\begin{quote}
\emph{Proof:}  
(a) Put $P= O_p(M)$ then $M= N_G(P)$ and $P \in S_p(NP)$ since $N \cap P = 1$ thus $p \notin \pi(N)$.
For (b), $PN$ acts on $\Omega= S_q(N)$. by conjugation and $N$ is a transitive normal subgroup of $PN$ so
$C_{\Omega}(P) \neq \emptyset$ and $C_N(P)$ is transitive on $C_{\Omega}(P)$.  Now $C_N(P) \leq M \cap N =1$ gives
$|C_{\Omega}(P)|=1$ and $C_{\Omega}(P)= C_{\Omega}(M)$ since $P \lhd M$.
For (c), $\exists M$-invariant $Q \in S_q(N)$.  Since $Q < N$, $M < QM < NM \leq G$.
\end{quote}
{\bf Theorem 9:}
Let $M<G$ be a primitive subgroup, $N \lhd G$: $M \cap F^*(N) \neq 1$ then
$F(G)=1$, $F^*(N)=F^*(G)= E(G)$. Every minimal normal subgroup of $G$ is contained in $M$.
\begin{quote}
\emph{Proof:}  
$F^*(N) \leq F^*(G)$ and by a previous result, ${\mathbb Z}(R(G)) =1$ and so $F(G)=1$.$F^*(N)=E(N)$ and
$F^*(G)= C_{F^*(G)}(E(N))E(N)$.  Applying the result again,
$F^*(G)= E(N)= F^*(N)$.
\end{quote}
{\bf Theorem 10:}
Suppose $G$ contains a primitive maximal subgroup $M$ then one of the following holds:
(F1) $F(G)=F^*(G)$ and $M$ is a complement of $F(G)$ in $G$;
(F2) $G$ contains exactly two minimal normal subgroups $N_1, N_2$ which are non-abelian
$F^*(G)= N_1 \times N_2= E(G)$; (F3) $F^*(G)$ is a minimal non-abelian subgroup of $G$.
\begin{quote}
\emph{Proof:}  
See Stellmacher, 6.6.4.
\end{quote}
{\bf Theorem 11:}
If F1 holds and $p \in \pi(M), O_p(M) \neq 1$ then all primitive maximal subgroups of $G$ are
conjugate.
\begin{quote}
\emph{Proof:}  
Put $P= O_p(M), F= F^*(G)$ then $M= N_G(P)$, $FP \lhd G$ and $S_p(M) \subseteq S_p(G)$.  Let
$H$ be another primitive maximal subgroup.  $H$ is also a complement of $F$ so $|H|=|M|$.
$\exists g: P \leq P^g$.  $P= H^g \cap FP \lhd H^g$ so $H^g= N_G(P)= M$.
\end{quote}
{\bf Theorem 12:}
Suppose F2 holds.   There is an $M$-isomorphism $\alpha: N_1 \rightarrow N_2$ such that
$M \cap F^*G) = \{ x x^{\alpha}: x \in N_1 \}$.
\begin{quote}
\emph{Proof:}  
Let $D= M \cap F^*(G)$ then by a previous result ($C_G(N)=1$), $ D \cap N_1 = 1 = D \cap N_2$. 
Since $G= N_iM$, $F^*(G)= N_iD$ and $\forall x_1 \in N_1, \exists ! x_2 \in N_2: x_1 x_2 \in D$.
The mapping $\alpha: N_1 \rightarrow N_2, x_1 \mapsto x_2$ is an isomorphism.  The isomorphism
commutes with  elements of $M$ since $N_1, N_2, D$ are all $M$-invariant.
\end{quote}
{\bf Theorem 13:}
Let $F$ be a minimal normal subgroup of $G$, $N<G$, $G= FM$.
(a) Suppose $U \subseteq F$ and $U^M=U$, then $UM<G$.
\begin{quote}
\emph{Proof:}  
(b) follows from (a).  For (a), assume the contrary, $G=UM$.  Then $U \lhd G$ and $F$ is not a minimal
normal subgroups of $G$.
\end{quote}
\section {Primitive pairs}
{\bf Definition 2:} $M<G$ is \emph{primitive} if $M=N_G(A), \forall A \lhd M$. $1 \ne M <G$ is
\emph{strongly $p$-embedded} if $|M \cap M^g|_p = 1$ for $g \in G \setminus M$.
A group $M$ is of characteristic $p$ if $C_M(O_p(M)) \leq O_p(M)$ (If
$M$ is $p$-separable then $O_{p'}(M)=1$).
\\
\\
{\bf Theorem 14:}
Let $M$ be a group of characteristic $p$.
Suppose $U \le M$ and $U \lhd \lhd M$ or $O_p(M) \le U$ then
$U$ has characteristic $p$.
\begin{quote}
\emph{Proof:}  
The case $O_p(M) \leq $ is clear.  In the other case, apply $F^*(L) =F^*(G) \cap L$.
\end{quote}
{\bf Definition 3:}  $M_1, M_2$ is called a \emph{primitive pair} if $1 \ne A \lhd M_i$,
$A \le M_1 \cap M_2 \rightarrow N_{M_j}(A)= M_1 \cap M_2$.  The pair is respectively
(solvable or characteristic $p$ if each are.  The property ${\cal P}(M_1 , M_2, A)$ is
$1 \neq A \lhd M_i$, $A \leq M_1 \cap M_2$.
\\
\\
{\bf Theorem 15:}
Let $M_1, M_2$ be two different maximal $p$-local subgroups of $G$ that both have characteristic $p$.
Suppose $M_1$ and $M_2$ have a common Sylow $p$-subgroup, then $(M_1 , M_2)$ is a primitive pair of 
characteristic $p$.
\begin{quote}
\emph{Proof:}  
Let $1 \neq A \lhd M_i, A \leq M_i \cap M_j, i \neq j$.  By previous result,
$1 \neq O_p(A) \lhd M_i$ and the maximality of $M_i$ gives $M_i= N_G(O_p(A))$.  So
$N_{M_j}(A)= M_i \cap M_j$ has property ${\cal P}$.  Let $S$ be a common Sylow subgroup of
$M_1$ and $M_2$ then $O_p(M_1 ) O_p(M_2 ) \leq S \leq M_1 \cap M_2$.
\end{quote}
{\bf Theorem 16:}
Suppose $p \in \pi(G)$, every $p$-local has characteristic $p$ and $O_p(G)=1$ then either
(a) there is a primitive pair of characteristic $p$ or (b)
every maximal $p$-local subgroup of $G$ is strongly embedded in $G$.
\begin{quote}
\emph{Proof:}  
Let $M$ be a maximal $p$-local of $G$, then $O_p(M) \neq 1$ and
$N_G(M) \leq N_G(O_p(M))=M < G$ so $M^g \neq M, g \in G-M$.  $M^g$ is a maximal $p$-local.
Among all such, choose one, $L \neq M$ with $|M \cap L|_p$ maximal.
If $|M \cap L|_p>1$ then $T \in S_p(M \cap L)$ and $U= N_G(T)$.
Since $U$ is a $p$-local, there is a maximal $p$-local $H \subseteq G$, maximal with $U \subseteq H$.
If $H \neq M$, let $T \in S_p(M)$.  If $T<S$, $T<N_S(T) \leq H \cap M$ which contradicts the maximality of
$|M \cap L|_p$.  This $T \in S_p(M)$.  $\exists S_1 \in S_p(L), g \in S_1 - T$ such that $T^g=T$ and
$M \neq M^g$.  By the previous result, $(M, M^g)$ is a primitive pair of characteristic $p$.  We've shown (a)
holds if $|M \cap L|_p = 1$ when $M, L$ are two different $p$-locals.   If $|M \cap M^g|_p=1, \forall g \in G - M$,
$M$ is strongly $p$-embedded.
\end{quote}
{\bf Bender's Little Theorem:}
Let $(M_1, M_2)$ be a primitive pair of $G$.  Suppose 
$F^*(M_1) \le M_2$ and
$F^*(M_2) \le M_1$ then $\exists p: (M_1, M_2)$ has characteristic $p$.
\begin{quote}
\emph{Proof:}  
See Stellmacher, 10.1.4.
\end{quote}
{\bf Theorem 17:}
Let $(M_1, M_2)$ be a primitive pair of $G$ of characteristic $p$ then $\exists i \in \{1 , 2 \}$
such that either 
(1) The action of $M_i$ on $O_p(M_i/ \Phi(M_i))$ is not $p$-solvable or
(2) $W_i$ is elementary abelian and the action of $M_i$ on $W_i$ is not $p$-stable.
\begin{quote}
\emph{Proof:}  
See Stellmacher, 10.1.5.
\end{quote}
{\bf Theorem 18:}
Let $(M_1, M_2)$ be a primitive pair of characteristic $p$, then $M_1$ or
$M_2$ has a non-abelian Sylow $2$-subgroup.
\begin{quote}
\emph{Proof:}  
If $p \neq 2$, it follows from the previous result and $p$-stability for groups with abelian $2$-syslow subgroups.
For $p= 2$, if the Sylow $2$-groups are abelian, then $O_2(M_1)= O_2(M_2)$ and $(M_1, M_2)$ is not primitive.
\end{quote}
{\bf Theorem 19:}
Let $M$ be $p$-seperable and $A$ a $p$-subgroup of $M$ satisfying
$Phi(A) \leq O_p(M)$ and $A \nleq O_p(M)$ then $\exists x \in O_{p,p'}(M)$
such that for $L= \langle A, A^x \rangle$, (a) $x \in O^p(L) \leq O_{p,p'}(M)$,
(b) $[O^P(L),A]= O^p(L)$, and (c) $|A/(A \cap O_p(L))|=p$ and $[A \cap O_p(L), L] \leq O_p(M)$.
\begin{quote}
\emph{Proof:}  
$\exists L$ with property (a).  Choose $L$ maximal among all such groups then (b) follows.
${\overline L}= L/O_p(L)$, ${\overline Q}= O_{p'}({\overline L})$ so ${\overline L} = {\overline {AQ}}$.
$A$ is an elementary abelian $p$-group and
$\Phi(A) \leq O_p(M) \cap L \leq O_p(L)$.  Let ${\cal B}$ be the set of maximal subgroups of $A$.
${\overline Q}=  \langle C_{\overline Q}({\overline U}): U \in {\cal B} \rangle$ so
$[C_{\overline Q}({\overline U}), A] \neq 1$ for some $U \in {\cal B}$ since $A$ acts non-trivially
on ${\overline Q}$.  This implies $U= A \cap O_p(L)$ and
$[U, O^p(L)] \leq O_p(L) \cap O_{pp'}(M) \leq O_p(M)$ and (c) follows.
\end{quote}
{\bf Theorem 20:}
Let $M$ be a group of characteristic $2$ that 
possesses a section isomorphic to $S_4$  then $M$
possesses a section isomorphic to $S_4$ .
\begin{quote}
\emph{Proof:}  
Let $M$ be a minimal counterexample.  $O_2(S_3)=1$ and $M/O_2(M) \leq N \lhd X < M$ and $X/N \approx S_3$.
$X=M$ by minimality.  Let ${\overline M}= M/N$ and $D \in S_3(M)$, ${\overline D} \approx C_3 \lhd M$.
By Frattini, $M= N_M(D) N$.  There are $2$-elements that act nontrivially on the $3$ group $D$.  Let $t \in N_M(D)$
of minimal order.  $\exists d \in D: |d|=3, d^t= d^{-1}$, $\langle d, t \rangle/\langle t^2 \rangle \approx S_3$.
Minimality of $M$ shows  $M= O_2(M) \langle d,t \rangle$ and $t^2 \in _2(M)$.
$\Phi(O_2(M))= 1$ and $C_{O_2(M)}(d)=1$.  Thus  $t^2=1$ and $\exists 1 \neq Z \in C_{O_2(M)}(t)$.
$V= \langle z, z^d, Z^{d^2} \rangle$ and $|V| \leq 8$ and $V \lhd M$.  Hence $V= C_2 \times C_2$ and
$V \langle d, t \rangle \approx S_4$ so $M$ is not a minimal counterexample.
\end{quote}
{\bf Notation:}
${\cal Q}(Z, X)= \{ A \leq X: [Z,A,A]=1 \neq [Z,A] \}$.
$q(Z,X)= 0$ if ${\cal Q}(Z,X)= \emptyset$;
$q(Z,X)= min \thinspace \{e \in {\mathbb R}: |A/C_A(Z)|^e= |Z/C_Z(A)|, A \in {\cal Q}(Z,X) \}$, otherwise.
\\
\\
{\bf Theorem 21:}
Let $M$ act faithfully on 
an elementary abelian $2$-group, $V$ and let
$A$ be an elementary abelian $2$-subgroup of $M$.  Suppose
$C_M(O_{2'}(M)) \leq O_{2'}(M)$ and $|V/C_V(A)| < |A|^2$.  Then
$M$ posseses a section isomorphic to $S_3$.
\begin{quote}
\emph{Proof:}  
Among all the elementary abelian $2$-subgroups that satisfy the condition, choose $A$ of minimal order.
Assume $|A|=2$, then $A \in {\cal A}_V(M)$ and the result follows from Glauberman's theorem.
If $|A|>2$,$C_M(O_{2'}(M) \leq O_{2'}(M)$ means $A$ acts non-trivially on $O_{2'}(M)$.  Let $Q \subseteq O_{2'}(M)$
be minimal.  $Q^A=Q$ and $[Q,A] \neq 1$ so $A_0= C_A(Q)$ is a maximal subgroup of $A$ and
$QA/A_0$ acts faithfully on $C_V(A_0)$.  The conclusion follows from the $|A|=2$ case if
$|C_V(A_0)/C_V(A)| \leq |A/A_0|^2=4$ which in turn follows from the minimality of $A$ since
$|V/C_V(A)| < |A|^2 \leq 4 |V/C_V(A_0)|$.
\end{quote}
{\bf Theorem 22:}
Let $M$ be a group and $V$ an elementary abelian normal $p$-subgroup of $M$; let $Z \leq V$ with
$V= \langle Z^M \rangle$ and $Z \lhd O_p(M)$.  Suppose $\exists A \leq O_p(M)$ with $[V,A,A]=1$.
Then $|A/C_A(V)|^q \leq |V/C_V(A)|$ where $q= q(Z,O_p(M))$.
\begin{quote}
\emph{Proof:}  
See Stellmacher, 10.1.10.
\end{quote}
{\bf Theorem 23:}
Let $(M_1, M_2)$ be a solvable primitive pair of characteristic $2$, then $M_1$ or
$M_2$ has a section isomorphic to $S_4$.
\begin{quote}
\emph{Proof:}  
See Stellmacher, 10.1.11.
\end{quote}
{\bf Theorem 24:}
Let $G$ be a group of even order, $O_2(G)=1$.  Suppose that for every $2$-local $M$ of $G$,
(1) $M$ has characteristic $2$ and is solvable and (2) $M$ does not possess a section
isomorphic to $S_4$ then every maximal $2$ local of $G$ is strongly $2$-embedded in $G$.
\begin{quote}
\emph{Proof:}  
This is a direct consequence of the result preceeding Bender's theorem and the previous result.
\end{quote}
\section {Amalgam Graphs}
{\bf Definition 4:}
$P_1, P_2 \le G$, $|P_i|< \infty$.  Construct a graph $\Gamma(G, P_1, P_2)=\Gamma$
as follows: 
$\Gamma$ has verticies consisting of right cosets of $P_1$ and $P_2$; the verticies
$P_i g_j$ and $P_n g_m$ are
joined by an edge if 
$P_i g_j \ne P_n g_m$ and
$P_i g_j \cap P_n g_m \ne \emptyset$.  $\Delta(\alpha)$ denotes the verticies
adjacent to $\alpha$.  $G$ acts on graph by right multiplication on cosets.  
$\Delta(\alpha)$ is a set of vertices adjacent to $\alpha$.
$G \rightarrow Aut(\Gamma)$.
\\
\\
{\bf Theorem 25:}
Suppose $G$ has two orbits and $P_1, P_2$ are representatives.  Every vertex stabilizer
$G_{\alpha}$ is a $G$-conjugate of $P_1$ or $P_2$ then
(a) $G$ acts transitively on $\Gamma$ and every edge stabilizer in
$G$ is a $G$-conjugate of $P_1 \cap P_2$, (b) $G_{\alpha}$ acts transitively
on $\Delta(\alpha)$, (c) $|\Delta(\alpha)|= |G_{\alpha}:G_{\alpha, \beta}|$,
(d) $P_1 \cap P_2$ is the kernel of the action on $\Gamma$.
\begin{quote}
\emph{Proof:}  
(a) For $P_ix \in \Gamma$ and $g \in G$, $P_ixg=P_ix$ is equivalent to
$P_i g^{x^{-1}} = P_i$ is equivalent $g \in P_i^x$.
(b) Let $\langle P_1x, P_2y \rangle$ be an edge.  $\exists z \in P_1x \cap P_2y$.  Hence
$P_1x = P_1z$ and $P_1x = P_1z$ so $z^{-1}$ conjugates $(P_1x, P_2y)$ to $(P_1, P_2)$.  The
stabilizer of $(P_1z, P_2z)$ is in $P_1^z \cap P_2^z= (P_1 \cap P_2)^z$.
(c) By (a), we can assume $\alpha= P_1$ then $\Delta( \alpha ) = \{ P_2y : P_2 \cap P_1 \neq \emptyset \}$
$= \{ P_2y : y \in P_1 \}$ thus $P_1$ is transitive on $\Delta( \alpha )$.
(d) By (a), any normal subgroup of $G$ is contained in $P_1 \cap P_2$ fixes every vertex of $\Gamma$.
\end{quote}
{\bf Theorem 26:}
$\Gamma$ is connected iff $G= \langle P_1 , P_2 \rangle $.
\begin{quote}
\emph{Proof:}  
Assume $G= \langle P_1 , P_2 \rangle$ and let $\Delta$ be a connected component of $\Gamma$ that contains
$P_1$.  Since $P_1$ and $P_2$ are adjacent, $P_2 \in \Delta$.  Since different components are disjoint,
$\Delta = \Delta^{\langle P_1, P_2 \rangle}= \Delta^G$ and thus $\Delta= \Gamma$ by (a) above.
Now assume $\Gamma$ is connected and let $G_0= \langle P_1, P_2 \rangle$ and $\Gamma_0= 
\{ P_1x: x \in G_0 \} \cup \{ P_xx: x \in G_0 \} $ be the coset graph of $G_0$ with respect to
$P_1, P_2$.  $\Gamma_0$ is connected.  If $\Gamma= \Gamma_0$ then $G= G_0$.  Assume $\Gamma \neq \Gamma_0$.
Since $\Gamma$ is connected, $\exists \alpha, \beta \in \Gamma$ such that $\alpha \in \Gamma_0, \beta \in \Gamma
- \Gamma_0$.  By (c) above, $\beta$ and all other elements of $\Delta( \alpha )$ are in $\Gamma - \Gamma_0$.
Hence $\Gamma$ is not connected.  Contradiction.
\end{quote}
{\bf Theorem 27:}
Let $G= \langle P_1, P_2 \rangle $, $U \le G_{\alpha} \cap G_{\beta}$ and
$(\alpha, \beta)$ is an edge.  The (1) $N_{G_{\delta}}(U)$ acts transitively on
$\Delta(\delta), \delta \in \{ \alpha , \beta \}$ and (2) if $U \le G_{\alpha} \cap G_{\beta}$ then
$U$ acts trivially on $\Gamma$.
\begin{quote}
\emph{Proof:}  
(2) together with (c) implies (1) so we may assume (1) holds.
Let $\Gamma_0= \alpha^{N_G(U)} \cup \beta^{N_G(U)}$.  $U$ fixes $\Gamma_0$.
If $\gamma \in \Gamma_0$, $\exists g \in N_G(U)$ and $\delta \in \{ \alpha, \beta \}$ such that
$\gamma= \delta^g$.  Then $\Delta( \delta^g ) = \Delta ( \gamma )$ and 
$N_{G_{\gamma}}(U)= N_{G_{\delta}}(U)^g $.  By (1), $N_G(U)$ is transitive on $\Delta( \delta^g ) = \Delta( \gamma )$.
One of $\alpha^g, \beta^g$ is adjacent to $\gamma$ and $\{ \alpha^g , \beta^g \} \subseteq \Gamma_0$.
It follows that $\Delta( \gamma ) \subseteq \Gamma_0$.  By the previous result, $\Gamma$ is connected
and we must have $\Gamma= \Gamma_0$ so $U$ stabilizes every vertex in $\Gamma$.
\end{quote}
{\bf Condition} ${\cal A}$:  Let $G$ be a
finite group generated by $P_1, P_2$,
$T= P_1 \cap P_2$ satisfying: $C_{P_i}(O_2(P_i)) \le O_2(P_i)$, $T \in S_2(P_i)$,
$T_G=1$, $P_i/O_2(P_i) \approx S_3$ and $[\Omega(Z(T)), P_i] \ne 1$.
\\
\\
{\bf Theorem 28:}
Let ${\cal A}$ holds and $(\alpha, \beta)$ be an edge of $\Gamma$.
(a) $|G_{\alpha}:G_{\alpha, \beta}|=3$ and is a Sylow $2$ subgroup of $G_{\alpha}$.
$G_{\alpha} = \langle t, G_{\alpha, \beta} \rangle$ for $t \in G_{\alpha} - G_{\beta}$.
(b) $|\Delta( \alpha )| =3$ and $O_2(G_{\alpha})= \bigcap_{\delta \in \Delta( \alpha )} (G_{\alpha} \cap G_{\beta})$,
(c) $G_{\alpha}$ acts $2$-transitively on $\Delta( \alpha )$.
\begin{quote}
\emph{Proof:}  
(a) follows from $P_1/O_2(P_i) \approx S_3$ and (b) and (c) follows from a previous result.
\end{quote}
{\bf Notation:}
For the remainder of the section, $Q_{\alpha}= O_2(G_{\alpha})$ and
$Z_{\alpha}= \langle \Omega(Z(T)): T \in S_2(G_{\alpha} \rangle$.
\\
\\
{\bf Theorem 29:}
Suppose ${\cal A}$ holds and $\alpha \in \Gamma$, $V \lhd G_{\alpha}$, $T \in S_2(G_{\alpha})$ and
$\Omega(Z(T)) \leq V \leq \Omega(Z(Q_{\alpha}))$ and $|V:\Omega(Z(T))|= 2$
then
$V= C_V(G_{\alpha}) \times W$, $W= [V, G_{\alpha}]$. $W= C_2 \times C_2$,
$C_{G_{\alpha}}(W)= Q_{\alpha}$.  
\begin{quote}
\emph{Proof:}  
Let $D \in S_3(G_{\alpha})$.  $V= C_V(D) \times W$, $W= [V,D]$.  By ${\cal A}$, since
$G_{\alpha}= DT$, $W \neq 1$ and thus $|W| \geq 4$.   Let $d \in D^{\#}$,
$|V/ \Omega(Z(T))|=2=|V/ \Omega(Z(T^d))|$.  $G_{\alpha}= \langle T, T^d \rangle$ means
$|V/C_V(G_{\alpha})| \leq 4$ so $C_V( G_{\alpha}) = C_V(D)$ and $|W|=4$.  The remainder follows from
${\cal A}$.
\end{quote}
{\bf Theorem 30:}
Let ${\cal A}$ hold and $(\alpha, \beta)$ be an edge of $\Gamma$;
(a) $Z_{\alpha} \leq \Omega(Z(Q_{\alpha}))$;
(b) $Q_{\alpha} Q_{\beta} = G_{\alpha} \cap G_{\beta} \in S_2(G)$;
(c) $C_{G_{\alpha}}(Z_{\alpha}) = Q_{\alpha}$;
(d) $Z_{\alpha}Z_{\eta} \lhd G_{\alpha}$ iff $\exists \gamma \in \Delta( \alpha) - \{ \beta \}$ such that
$Z_{\alpha}Z_{\beta} = Z_{\alpha}Z_{\gamma} $.
\begin{quote}
\emph{Proof:}  
(a) Let $T \in S_2(G_{\alpha})$ then $Q_{\alpha} \leq T$ and ${\cal A}$ implies 
$\Omega(Z(Y)) \leq Z(Q_{\alpha})$.
(b) By ${\cal A}$ and a previous result, 
$Q_{\alpha}$ and $Q_{\beta}$ have index $2$ in $G_{\alpha} \cap G_{\beta}$ so it STS $
Q_{\alpha} \neq Q_{\beta} $.
If $Q_{\alpha} = Q_{\beta} $, 
now two previous results show
$G$ acts failthfully on $\Gamma$, and this contradicts ${\cal A}$.
\end{quote}
{\bf Theorem 31:}
Let ${\cal A}$ hold and $(\alpha, \beta)$ be an edge of $\Gamma$.  The following are equivalent:
(1) the conclusion of Goldschmidt's Theorem holds; (2) $Z \leq Q_{\beta}$.
\begin{quote}
\emph{Proof:}  
Assume the conclusion of Goldschmidt's theorem holds.  For $\delta \in \{ \alpha \beta \}$,
either 
(i) $G_{\delta} = S_4$ and $Q_{\delta} \approx S_4 \times C_2$, or 
(ii) $G_{\delta} = S_4 \times C_2$ and $Q_{\delta} \approx C_2 \times C_2 \times C_2$.  In either
case, $Z_{\delta} - Q_{\delta}$ and by the previous result $Z_{\alpha} \nleq Q_{\beta}$.
Set $T= Q_{\alpha} Q_{\beta}$ and $E= Q_{\alpha} \cap Q_{\beta}$.  The previous result gives
$T \in S_2(G_{\delta})$ and $|T/Q_{\delta}|=2$.  Thus 
$ |Q_{\alpha}:E|=2= |Q_{\beta}:E| $ and
$T= Q_{\beta}Z_{\alpha}$ and
$Q_{\alpha}= E Z_{\alpha}$.
Todo, more.
\end{quote}
{\bf Definition 5:} 
$\alpha, \alpha'$ is a \emph{critical pair} if $Z_{\alpha} \nleq Q_{\alpha'}$ and
$b=d(\alpha, \alpha')$.  Let $b$ be minimal.
\\
\\
{\bf Theorem 32:}
Suppose ${\cal A}$ holds and (a) ($\alpha, \alpha')$ is a critical pair;
(b) $G_{\alpha} \cap G_{\alpha+1} = Z_{\alpha'}Q_{\alpha}$, $G_{\alpha'-1} \cap G_{\alpha'} = Z_{\alpha}Q_{\alpha'}$;
(c) $R \leq Z(G_{\alpha}) \cap Z(G_{\alpha+1}) \cap Z(G_{\alpha'}) \cap Z(G_{\alpha'-1})$ and
$R = [Z(G_{\alpha}), G_{\alpha+1}) \cap G_{\alpha}]= [Z(G_{\alpha'}), G_{\alpha'-1}) \cap G_{\alpha'}] $
(d) $ Z_{\alpha} = [ Z_{\alpha} , G_{\alpha} ] \times \Omega( Z(G_{\alpha} ))] $ and
$[ Z_{\alpha} , G_{\alpha} ] = C_2 \times C_2$;
(e) $|Z_{\alpha}:\Omega(Z(Y))|=2$ for $Y \in S_2(G_{\alpha})$.
\begin{quote}
\emph{Proof:}  
Minimality of $(b)$ implies 
$Z_{\alpha} \leq Q_{\alpha'-1} \leq G_{\alpha'-1} \cap G_{\alpha'} $ and
$Z_{\alpha'} \leq Q_{\alpha+1} \leq G_{\alpha+1} \cap G_{\alpha} $.
$Z_{\alpha} \nleq Q_{\alpha'}$ shows that
$G_{\alpha'-1} \cap G_{\alpha'} = Z_{\alpha} Q_{\alpha} $ since $Q_{\alpha'}$ has index $2$ in
$G_{\alpha'-1} \cap G_{\alpha'}$.
$Z_{\alpha} \lhd G_{\alpha}$ and
$Z_{\alpha'} \lhd G_{\alpha'}$  so
$R \leq Z_{\alpha} \cap Z_{\alpha'} $.  By a previous result, $R \neq 1$ and so $Z_{\alpha'} \nleq Q_{\alpha}$
and $G_{\alpha+1} \cap G_{\alpha}= Z_{\alpha'} Q_{\alpha}$.  (a) and (b) follow and (c) follows from
$R \leq Z_{\alpha} \cap Z_{\alpha'} $ and a previous result.   These also show
$ |Z_{\alpha} / C_{\alpha}(Z_{\alpha'}(Z_{\alpha})= |Z_{\alpha'} / Z_{\alpha'}Z(Z_{\alpha}) =2 $
and
$C_{Z_{\alpha}}(Z_{\alpha'})= \Omega( Z(
G_{\alpha+1} \cap G_{\alpha}
)$ which gives (d) and (f) and, with a previous result (e).
\end{quote}
{\bf Theorem 33:}
Let $\alpha - 1 \in \Delta(\alpha) - \{ \alpha+1 \}$.  Suppose $(\alpha-1, \alpha'-1)$ is not a critical pair.
Then
(1) $ Z_{\alpha} Z_{\alpha+1} = Z_{\alpha} Z_{\alpha-1} $;
(2) $Q_{\alpha} \cap Q_{\beta} \lhd G_{\alpha}, \beta \in \Delta( \alpha)$;
(3) $\alpha$ and
$\alpha'$ are conjugate and $b$ is even.
\begin{quote}
\emph{Proof:}   Since $(\alpha-1, \alpha'-1)$ is not critical, $Z_{\alpha-1}, Z_{\alpha'}] \leq R \leq Z_{\alpha}$,
so $\langle Z_{\alpha'}, G_{\alpha} \cap G_{\alpha-1} \rangle \subseteq N(Z_{\alpha-1}Z_{\alpha})$.  The previous result
gives (a). (b) follows by a previous result and the fact that $G_{\alpha}$ is transisitve on $\Delta( \alpha )$.
Either 
$\alpha \in (\alpha')^G$ or
$\alpha \in (\alpha'-1)^G$ , so the former holds and $b$ is even.
For (c), note $\alpha$ and $\alpha'-1$ are conjugate so
$G_{\alpha}$ and $G_{\alpha'-1}$ are conjugate.  Then (b) gives $Z_{\alpha} \leq
Q_{\alpha'-2} \cap Q_{\alpha'-1} =
Q_{\alpha'-1} \cap Q_{\alpha'} $.  This contradicts $Z_{\alpha} \nleq Q_{\alpha'}$.
\end{quote}
{\bf Theorem 34:}
Suppose $\alpha - 1 \in \Delta(\alpha) - \{ \alpha+1 \}$ such that  $(\alpha-1, \alpha'-1)$ is a critical pair
then $b=1$.
\begin{quote}
\emph{Proof:}  
Put $R_1= [Z_{\alpha - 1}, Z_{\alpha' - 1}]$.  Assume $b > 1$.
$Z_{\alpha} \leq Q_{\alpha+1}$ and $Z_{\alpha'} \leq Q_{\alpha'-1}$.  $(\alpha - 1, \alpha' - 1)$ is a 
critical pair so $|R_1'| =2$.  
$R_1 = [Z_{\alpha - 1}, G_{\alpha - 1} \cap G_{\alpha}] \leq (Z(G_{\alpha - 1} \cap  Z(G_{\alpha})) 
\cap (Z(G_{\alpha' - 2} \cap  Z(G_{\alpha' - 1}))$.  $R_1 \leq [Z_{\alpha - 1}, Z_{\alpha' - 1}]$, so $[R_1, Z_{\alpha'}] = 1$.
$\langle Z_{\alpha'} , G_{\alpha - 1} \cap G_{\alpha} \rangle = G_{\alpha}$.
\\
\\
(1) $R_1 \leq Z(G_{\alpha})$.
\\
Let $\alpha - 2 \in \Delta(\alpha - 1) \setminus \{ \alpha \}$.  Now we show \\
\\
(2) $(\alpha - 2, \alpha' - 2)$ is a critical pair.\\
If not, $Z_{\alpha + 1} Z_{\alpha} = Z_{\alpha - 1} Z_{\delta}, \delta \in \Delta(\alpha - 1)$
$Z_{\alpha + 1} Z_{\alpha} = Z_{\alpha + 1} Z_{\alpha + 2}$.  Minimality of $b$ gives
$Z_{\alpha + 1} Z_{\alpha + 2} \leq Q_{\alpha'}$ and $Z_{\alpha} \leq Q_{\alpha'}$, so $(\alpha, \alpha')$ is not a critical pair.
Now put $R_2= [Z_{\alpha - 2}, Z_{\alpha' - 2}]$.  $\alpha - 2$ and $(\alpha, \alpha')$ satisfy the hypothesis,
hence $|R_2| = 2$.\\
\\
(3) $R_2 = [Z_{\alpha - 2},  G_{\alpha - 2} \cap G_{\alpha - 1}] \leq Z(G_{\alpha - 1})$.\\
$\exists y \in G_{\alpha - 1}, x \in G_{\alpha} : (\alpha - 2)^y = \alpha$ and $(\alpha + 1)^x = \alpha - 1$.
$[Z_{\alpha}, G_{\alpha - 1} \cap G_{\alpha}] = [Z_{\alpha - 2}, G_{\alpha - 2} \cap G_{\alpha - 1}] = (R_1)^y \leq Z_{\alpha - 1}$.
$R^x = [Z_{\alpha}, G_{\alpha + 1} \cap G_{\alpha}]^x = [Z_{\alpha}, G_{\alpha} \cap G_{\alpha - 1}] = (R_2)^y \leq Z(G_{\alpha - 1})$.
It follows that\\
\\
(4) $ R \leq Z(G_{\alpha + 1})$, so \\
\\
(5) $ R \cap R_1 =1$.\\
If not, $Z_{\alpha'} \leq Q_{\alpha' - 2}$, $[R_2, Z_{\alpha'}] = 1= [R_2, G_{\alpha - 1}]$, so $R_2 =1$.
This contradicts $|R_2|=2$.\\
\\
(6) $b = 2$\\
If $b>2$, 
$V_{\alpha} = \langle Z_{\beta}, \beta \in \Delta(\alpha) \rangle \lhd G_{\alpha}$.
$V_{\alpha + 1} = \langle Z_{\beta}, \beta \in \Delta(\alpha + 1) \rangle \lhd G_{\alpha + 1}$.
$V_{\alpha} \leq Q_{\alpha}$,
$V_{\alpha + 1} \leq Q_{\alpha + 1}$, and
$Z_{\alpha} = \langle \Omega(Z(G_{\alpha + 1} \cap G_{\alpha})^{G_{\alpha}} \rangle \leq V_{\alpha}$.
$Z_{\alpha + 1} \leq Q_{\alpha + 1}$, so \\
\\
(7) $Z_{\alpha} Z_{\alpha + 1} \leq V_{\alpha} \cap V_{\alpha + 1}$.\\
Since $R_1 \leq Z(G_{\alpha})$ is a $2$-transitive action of $G_{\alpha}$ on $\Delta(\alpha )$.
${V_{\alpha}}' = R_1 \leq Z(G_{\alpha})$.
We show $V_{\alpha}$ is abelian and $V_{\alpha}$ is generated by involutions.
$V_{\alpha} / R_1$ is elementary abelian so $R_1 = \Phi(V_{\alpha})$ and $R = \Phi(V_{\alpha + 1})$, put
${\overline {V_{\alpha}}} = V_{\alpha} / Z_{\alpha}$.  We get $Z_{\beta}/(Z_{\alpha} \cap Z_{\beta})= 2, \forall \beta \in \Delta(\alpha)$,
so $|{\overline {Z_{\beta}}}| = 2$.  ${\overline {V_{\alpha}}} = \langle Z_{\beta}, \beta \in \Delta(\alpha)\rangle$, so 
$|{\overline {V_{\alpha}}}| \leq 8$.  Set $W= V_{\alpha} \cap V_{\alpha + 1} \lhd G_{\alpha} \cap G_{\alpha + 1}$ then
$Z_{\alpha}Z_{\alpha + 1} \leq W$ and\\
\\
(8)  $V_{\alpha} = \langle W^{G_{\alpha}} \rangle$.\\
$\Phi(W) \leq \Phi(V_{\alpha}) \cap \Phi(V_{\alpha + 1}) =R_1 \cap R =1$ and $W$ is elementary abelian
${V_{\alpha}}' \ne 1$ and $|V_{\alpha} / W| \geq 2$.
The kernel of the action of $G_{\alpha}$ on ${\overline {V_{\alpha}}}$ contains $Q_{\alpha}$ since
$[Z_{\alpha - 1} , G_{\alpha} \cap G_{\alpha - 1}] \leq R_1 \leq Z_{\alpha}$.
${\overline {V_0}} = [{\overline {V_{\alpha}}}, O^2(G_{\alpha})]$.
If ${\overline {V_0}} = 1$ $W \lhd G_{\alpha}$ and ${V_{\alpha}}' =1$ but this contradicts ${V_{\alpha}}' = R_1$ and
$|R_1| = 2$.
Now suppose ${\overline {V_0}} \ne 1$ since $|{\overline {V_{\alpha}}}| \leq 8$, \\
\\
(9) $|{\overline {V_0}}| = 4$.\\
Assume $|V_{\alpha} / W| = 2$.  Let $x \in G_{\alpha}$ st $W^x \ne W$
then $V_{\alpha} = W W^x$.
$W  \cap W^x = Z(V_{\alpha})$ and $|V_{\alpha} (W \cap W^x)| = 4$.  Let $D \in S_3(G_{\alpha})$, $D$ acts non-trivially on
$V_{\alpha} / (W \cap W^x)$ so all maximal subgroups of $V_{\alpha}$ that contain $W \cap W^x$ are $D$-conjugate.
Every $x \in V^{\#}$ is an involution, $V_{\alpha}$ is elementary abelian contradicts ${V_{\alpha}}' = R_1$.
We have shown $|V_{\alpha}/ W| \geq 4$ so  $|{\overline {V_{\alpha}}}| \leq 8$. \\
\\
(10) $|V_{\alpha}|=8$, $W = Z_{\alpha}Z_{\alpha + 1}$ and $|{\overline W}|=2$.\\
$Z_{\alpha'} \leq G_{\alpha}$, $Z_{\alpha'} \nleq Q_{\alpha}$,
we get $[{\overline {V_0}}, Z_{\alpha}'] \ne 1$.  But $b = 2$,
so $[V_{\alpha}, Z_{\alpha}] \leq [V_{\alpha}, V_{\alpha + 1}] \leq W$.
${\overline W} = [{\overline {V_0}}, Z_{\alpha'}], \langle {\overline W}^{G_{\alpha}} \rangle = {\overline {V_0}}$, which
contradicts (8), (9) and (10).
\end{quote}
{\bf Goldschmidt's Theorem:}
If ${\cal A}$ holds either (i) $P_1 \approx P_2 \approx S_4$ or 
(ii) $P_1 \approx P_2 \approx C_2 \times S_4$.
\begin{quote}
\emph{Proof:}  
Let $G$ be a counter example and choose $(G, P_1, P_2, T)$ with $|T|$ minimal.
$b > 1$ and $(\alpha - 1, \alpha' - 1)$ is not a critical pair $\forall \alpha - 1 \in \Delta(\alpha) \setminus \{\alpha + 1 \}$.
\\
\\
(1) $b = 0 \jmod{2}$, $X = Q_{\alpha} \cap Q_{\alpha + 1} \lhd G_{\alpha}$.
\\
$|Q_{\alpha} : X| = |Q_{\alpha + 1}| = 2$.  Let $D \in S_3(G_{\alpha})$, ${\overline {G_{\alpha}}} = G_{\alpha} / X$.
$|{\overline {G_{\alpha}}}|= 12$, ${\overline {Q_{\alpha}}} \leq {\overline {G_{\alpha}}}$ and $|{\overline {Q_{\alpha}}}|=2$ so
${\overline D} \lhd {\overline {G_{\alpha}}}$.\\
\\
We get (2a) $L \lhd G_{\alpha}$, $|G_{\alpha}:L| = 2$,\\
(2b) ${\overline L} = S_3$, \\
(2c) $S_2(L) = \langle Q_{\beta}: \beta \in \Delta(\alpha)\rangle$,\\
(2d) $O_2(L) = X = Q_{\alpha} \cap Q_{\beta}, \forall \beta \in \Delta(\alpha)$,\\
(2e) $Q_{\alpha +1} = Z_{\alpha'} O_2(L))$, (f) $C_L(O_2(L)) \leq O_2(L)$.\\
$Z_{\alpha} \leq G_{\alpha}$, $Z_{\alpha} \leq Q_{\alpha + 1} \leq O_2(L)$.
$C_L(O_2(L)) \leq C_L(Z_{\alpha}) \leq Q_{\alpha} \cap L \leq O_2(L)$.
By $A2$, $\exists t \in G_{\alpha + 1} \setminus Q_{\alpha + 1}, \alpha^t = \alpha + 2$ and $t^2 \in Q_{\alpha + 1}$
$Q_{\alpha + 1}, Q_{\alpha}^t \in S_2(L) \leq G_{\alpha}$, $L^t \leq G_{\alpha + 2}$. \\
\\
(3) $O_2(L)$ is not elementary abelian.
\\
$A_1 = O_2(L)$, $A_2=O_2(L^t)$ are elementary abelian of index $2$ in $Q_{\alpha + 1}$.
If $A_1 = A_2$, $A_1 \lhd \langle G_{\alpha}, G_{\alpha + 2}\rangle$.
$\langle G_{\alpha}, G_{\alpha} \cap  G_{\alpha + 1}, G_{\alpha + 1} \cap  G_{\alpha + 2}\rangle = \langle G_{\alpha}, G_{\alpha + 1}\rangle=G$
such that $A_1 \ne A_2$, $Q_{\alpha + 1}$ is non-abelian.
$A = A_1 \cap A_2 = Z(Q_{\alpha + 1})$, $|(Q_{\alpha + 1}/A)| = 4$.
$\langle G_{\alpha}, O^2(G_{\alpha + 1} \rangle = N_G(A_1)$, which is a contradiction.
$O^2(G_{\alpha + 1})$ acts transitively on $(Q_{\alpha + 1}/A)^{\#}$, its elements are involutions and $Q_{\alpha + 1}$ is elementary
abelian.
$G_0 = \langle L, L^t \rangle$.  Denote the largest normal subgroup of $G_0$ in $Q_{\alpha +1}$ by $Q$.
${G_0}^t=G_0, Q^t=Q$.  \\
\\
(4) We show $[Q, D] \ne 1$.\\
Suppose not put ${\tilde {G_0}} = G_0/Q$.
$Q_{\alpha + 1} \in S_2(L) \cap S_2(L^t)$.
$({\tilde {G_0}}, {\tilde L}, {\tilde {L^t}}, {\tilde {Q_{\alpha + 1}}})$ satisfy ${\cal {A_2}}$, ${\cal {A_3}}$, ${\cal {A_4}}$.
${\tilde W}= [{\tilde {Z_{\alpha}}}, {\tilde D}] \ne 1$.
$C_{\tilde L}(O_2({\tilde L})) \leq O_2({\tilde L})$, so  ${\cal {A_1}}$ and ${\cal {A_5}}$ hold.
Because of the minimality of $|T|$ and $|Q_{\alpha + 1}| < |T|$, ${\tilde L} = S_4$ or ${\tilde L} = C_2 \times S_4$.
${\tilde W} = [O_2(({\tilde L}), O^2({\tilde L})]$ is not contained in $O_2({\tilde L}^t)$ and ${\tilde W} \leq {\tilde {Z_{\alpha}}}$
and so $O_2(L) =(O_2(L) \cap O_2(L^t))Z_{\alpha}$.
Since $Z_{\alpha} \leq \Omega(Z(O_2(L)))$, $\Phi(O_2(L))= \Phi(O_2(L) \cap O_2(L^t))$ and thus
$\Phi(O_2(L)) = \Phi(O_2(L^t))$. $\Phi(O_2(L)) \lhd \langle G_{\alpha}, G_{\alpha + 2} \rangle = G$ and ${\cal {A_3}}$ gives
$\Phi(O_2(L))=1$ which contradicts (3) and (4).\\
\\
(5) Let $\beta \in \Delta(\alpha) \setminus \{\alpha \}$, then $\langle Z_{\alpha}, Z_{\gamma} \rangle$ is not normal in $L$.
\\
Put $\Delta(\beta) = \{ \alpha, \delta, \gamma \}$ and $V_{\beta}= \langle Z_{\alpha}, Z_{\beta}, Z_{\gamma} \rangle \lhd G_{\beta}$.
If $x \in Q_{\alpha} \setminus Q_{\beta}$ interchanges $\gamma$ and $\delta$ and normalized $L$.
If $\langle Z_{\alpha} , Z_{\gamma} \rangle$ is normal in $L$ and also 
$\langle Z_{\alpha}, Z_{\delta} \rangle = \langle Z_{\alpha}, Z_{\gamma}^x \rangle$ is normal in $L$.  Thus $V_{\beta}$ is normal
in $L$ (which is not contained in $G_{\alpha} \cap G_{\beta}$) which contradiction!\\
\\
(6) Let $b \geq 4, \alpha - 1 \in \Delta(\alpha) \setminus \{ \alpha + 1 \}$ and
$\alpha - 2 \in \Delta(\alpha - 1) \setminus \{\alpha \}$ then $(\alpha - 2, \alpha' - 2)$ is a critical pair.
\\
Assume not.  $Z_{\alpha - 2} \leq Q_{\alpha' - 3} \cap Q_{\alpha' - 2}$.  Since
$\alpha' - 2$ is conjugate to $\alpha$,
$Z_{\alpha -2} \leq Q_{\alpha' -3} \cap Q_{\alpha' - 2} = Q_{\alpha' - 2} \cap Q_{\alpha' - 1} \leq
G_{\alpha' - 1} \cap G_{\alpha'} = Z_{\alpha}Q_{\alpha'}$.  So
$[Z_{\alpha - 2}, Z_{\alpha'}] \leq [Z_{\alpha}, Z_{\alpha'}] \leq Z_{\alpha}$ and so $Z_{\alpha - 2} Z_{\alpha} \leq Q_{\alpha} \cap Q_{\alpha - 1}$ is normalized by $Z_{\alpha'}$ is normal in $L$ which contradicts (5).\\
\\
$\alpha - 1 \in \Delta(\alpha) \setminus \{ \alpha + 1 \}, x \in L \leq G_{\alpha}$ with $(\alpha - 1 = (\alpha + 1)^x$.  Thus
$\alpha - 2 = (\alpha + 2)^x$, which is adjacent to $\alpha - 1$.  If $b \geq 4$, $(\alpha - 2, \alpha' - 2)$ is critical.  Hence
$R_2= [Z_{\alpha - 2}, Z_{\alpha' - 2}] \leq Z(G_{\alpha - 2} \cap G_{\alpha - 1}) \cap Z_{\alpha' - 2}$.  Also,
$b \geq 4$ also implies $Z_{\alpha'} \leq Q_{\alpha' - 2}$ and $[R_2, Z_{\alpha'}] = 1$ and so $[R_2 , L] = 1$ and
$R_2 \leq Z(G_{\alpha + 2} \cap G_{\alpha + 1}$ since $x \in L$.\\
Now $(\alpha', \alpha)$ is also a critical pair so $\exists \alpha' +2$ such that $d(\alpha', \alpha' + 2) = 2$ and
$(\alpha' +2, \alpha + 2)$ is also critical. So $Z_{\alpha' + 2} \leq Q_{\alpha' - 2}$ and so
$[R_2, Z_{\alpha' +2}] = 1$ since $R_2 \leq Z_{\alpha' - 2}$.  Hence $G_{\alpha + 2} \cap G_{\alpha + 3} = Q_{\alpha +2}Z_{\alpha'+2}$
is centralized by $R_2$.  But then $R_2 \leq Z(G_{\alpha + 2})$ and  $R_2 \leq Z(G_{\alpha - 2})$ after conjugation.  This contradicts
the action of $Z_{\alpha' - 2}$ on  $Z_{\alpha + 2}$.
This proves:\\
(7) $b \leq 4$
\\
Finally, $Q \leq O_2(L^t) \leq Q_{\alpha +2}] =  1$.\\
\\
Case a: $Z_{\alpha + 2}$ is not contained in $O_2(L)$.\\
$Q_{\alpha + 1} \leq O_2(L) Z_{\alpha + 2}$ and $L= \langle (Z_{\alpha + 2})^L \rangle O_2(L) = C_L(O_2(L)) O_2(L)$.
So $O^2(L) \leq C_L(Q)$ since $Q \lhd L$ but then $[Q, D] = 1$ which contradicts (4).
\\
\\
Case b: $Z_{\alpha + 2} \leq O_2(L)$
\\
$Z_{\alpha + 2} \leq Q_{\alpha}$ and (7) implies $b=4$.  (6) then implies
$Z_{\alpha + 2}$ is not contained in $Q_{\alpha - 2}= Q_{(\alpha + 2)^x}$ and $L^{tx}$ is normal of index $2$ in $G_{\alpha - 2}$.
$\langle (Z_{\alpha + 2})^{L^{tx}} \rangle \leq G_0$ has a Sylow $3$ subgroup, $D_2$ of $G_{\alpha - 2}$. But then
$Q \lhd G_0$ shows $[Q, D] = 1$ which contradicts (4).
This contradicts (4) since $D_2$ is a $G_0$ conjugate of $D \in S_3(G_{\alpha})$.
Assume $(\alpha - 2, \alpha' - 2)$ is not normal in $L$.
\end{quote}
{\bf Amalgam example 1:} $G= S_6$, $a=(12)$, $b=(12)(34)(56)$.  $P_1 = C_G(a)$, $P_2 = C_G(b)$.  
$P_1 = \langle a \rangle \times \langle (34)(56) \rangle \times \langle (35)(46) \rangle$.
\\
\\
{\bf Amalgam example 2:} $G= SL_3(2)$, $|G|=168$.
$P_1 = \{\left(
\begin{array}{ccc}
a & b & c \\
0 & d & e \\
0 & 0 & f \\
\end{array}
\right) \}$.
$P_2 = \{\left(
\begin{array}{ccc}
a & b & c \\
d & e & f \\
0 & 0 & g \\
\end{array}
\right) \}$. $G=\langle P_1, P_2\rangle$. $P_1 \cong P_2 \cong S_4$.
$P_1 \cap P_2 = \{\left(
\begin{array}{ccc}
1 & a & b \\
0 & 1 & c \\
0 & 0 & 1 \\
\end{array}
\right) \}$.  If $M_1, M_2$ are a solvable primitive pair oc characteristic $2$, either $M_1$ or $M_2$ possesses a section isomorphic
to $S_4$.
\section {ONan}
{\bf Theorem:} Let $M$ be a primitive maximal subgroup of $G$ then either (1) $F^*(G)=F(G)$ [Example: $G=S_4$, $M=S_3$], (2)
$F(G)= 1$ and $F^*(G) = N_1 \times N_2$ where $N_1$ and $N_2$ are the only minimal normal subgroups
[$G$ involves $A_5$] or (3) $F(G)=1$ and $F^*(G)$ is the unique minimal subgroup pf $G$.

