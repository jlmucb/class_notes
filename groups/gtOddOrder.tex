\chapter{Notes on odd order techniques}
\section {An idea} 
In CA groups of odd order, the maximal abelian subgroups are TI (Reason: from character theory every such
subgroup, $A$, is of the form $C_G(x), 1 \ne x \in A$).   For each such $A$, $M=N_G(A)$ is maximal
and $M$ is Frobenius.  This gives characters of $G$ which lead to a contradiction.
\\
\\
In the general
odd order case, a minimal counterexample, $G$, has the property that every elementary abelian subgroup
of order $p^3$ is in a unique maximal subgroup (by the Uniqueness Theorem).  This allows
us to determine the structure of such maximal subgroups of $G$ and their embeddings in $G$.
This is the origin of
$SCN_k(P), P \in S_p(G)$.  For these groups, $A=C_G(A)$ and $C_G(A) = A \times D$ with $D=O_{p'}(N(A))$.
This in turn takes advantage of the fact that for $m(A) \geq 3$, $C_G(A)$ acts transitively on the set of
maximal $p'$ groups normalized by $A$.  If $G$ is a group of odd order thet doesn't contain any
elementary abelian subgroup of order $p^3$ for any $p$, it is solvable and $G'$ is nilpotent.
\\
\\
If $SCN_3(P)$ is empty, $P$ is generated by $3$ elements.  Then $P \subseteq M'$ where $M$ is a maximal subgroup
of $N_G(P)$.  These $p^2$ groups restrict the structure and embedding of the maximal subgroups which, coupled
with the Uniqueness result provides contradictory evidence for the existance of $G$.
Finally, an argument using generators and relations leads to a contradiction.  
\\
\\
$PSL_3(7)$ has an elementary abelian $p^2$ groups as an example.  $PSL_2(11)$ has a single conjugacy
class of involutions and the centralizer of an involution is dihedral of order $12$; it has an elementary
abelian group of order $4$ but the transitivity theorem doesn't hold.
$|PSL_2(11)| = 1320= 2^3 \cdot 3 \cdot 5 \cdot 11$.
\\
\\
{\bf Lemma:} If $q= p^n$, $x=
\left(
\begin{array}{cc}
a &  b \\
c &  d \\
\end{array}
\right) \in
PSL_2(q)$ and $|x| = 4$, $a=-d$ and $a^2 + bc = -1 \jmod{p}$.
\begin{quote}
If $x$ has order $4$,
$x^2 =
\left(
\begin{array}{cc}
a &  b \\
c &  d \\
\end{array}
\right) ^ 2
=
\left(
\begin{array}{cc}
a^2 +bc &  b(a+d) \\
c(a+d) &  d^2 + bc \\
\end{array}
\right)
=
\left(
\begin{array}{cc}
-1 &  0 \\
0 &  -1 \\
\end{array}
\right)$.
So either $a=d=0$ and $bc = -1$ or $d= -a \ne 0$ and $a^2+bc = -1 \jmod{p}$.
\end{quote}
{\bf Lemma:} Let $G=SL_2(11)$.  $S \in S_2(G)$ is quaternion.
$G$ has one element of order $2$, and $110$ elements of order $4$.
If $x$ has order $4$, $C_G(x)$ is dihedral of order $12$.
$PSL_2(11)$ has elementary abelian Sylow $2$ subgroups.  There are $55$ involutions in $PSL_2(11)$;
all involutions are conjugate in $PSL_2(11)$.
$A_1 =
\left(
\begin{array}{cc}
0 &  4 \\
8 &  0 \\
\end{array}
\right)$
has order $4$ and so does
$A_2 =
\left(
\begin{array}{cc}
1 &  3 \\
3 &  -1 \\
\end{array}
\right)$.
$B =
\left(
\begin{array}{cc}
5 &  1 \\
2 &  5 \\
\end{array}
\right)$ has order $3$. $[A_1, B]=1$ and $C_G(A_1)= \langle A_1, B \rangle$. $|C_G(A_1)| = 12$.
$C =
\left(
\begin{array}{cc}
2 &  0 \\
0 &  6 \\
\end{array}
\right)$ has order $10$.
\begin{quote}
There are $110$ solutions to $a^2 + bc = -1 \jmod{11}$.
Let $H=G'=PSL_2(11)$.  $S \in S_2(H)$ is elementary abelian and since $H$ is simple,
$N_H(S) > C_H(S)$ and all the involutions of $S$ are conjugate. Most of the remaining
results are direct calculations.  Note that $PSL_2(11) \subseteq PSL_2(11^2)$. Let
$\phi: SL_2(11^2) \rightarrow PSL_2(11^2)$ be the usual homomorphism.  If $\phi(A)$ is an
involution in $PSL_2(11^2)$, $\phi(A)^2 = \mu I$.  $\exists B: A = B^{-1}
\left(
\begin{array}{cc}
\lambda_1 &  0 \\
0 &  \lambda_2 \\
\end{array}
\right)
B$, $ {\lambda_1}^2 = {\lambda_2}^2 = \mu$ so $\mu = \pm 1$.  If $\mu =1$, $\phi(A)$ is not an involution,
so $\mu = -1$.  $A$ has order $4$ in $SL_2(11^2)$ so $A = C^{-1}
\pm \left(
\begin{array}{cc}
i &  0 \\
0 &  -i \\
\end{array}
\right)
C$ for some $C$.  These two possibilities are conjugate in $SL_2(11^2)$.
\end{quote}
\section {Guide to Feit-Thompson} 
If $G$ is a minimal counterexample to the Odd Order conjecture, we'd like to prove something like
the following (which is nor quite true).
\\
\\
{\bf Goal:} For every maximal subgroup, $M$ of $G$, $\exists M_0 \lhd M$ such that:
\\ (1) $C_{M_0}(a) =1$ for $a \in M \setminus M_0$;
\\ (2) $M \cap M^g =1$ for $g \in G \setminus M$;
\\ (3) $M_0$ is nilpotent;
\\ (4) $M/M0$ is cyclic;
\\ (5) $\forall 1 \ne x \in G$, $x$ lies in exactly one sucb $M_0$.
\\
\\
If all that happened, the $M_0$ would form a partition of $G$.
In the real argument, such $M$ are either almost Frobenius or $3$-step groups (see section on CN-groups).
In the real case, if $E$ is a complement of $M_0$ in $M$, $r(E) \leq 2$ and two elements of a nilpotent Hall
subgroup $H$ are $G-$conjugate iff they are $N_G(H)$-conjugate ($N_G(H)$ controls fusion).



