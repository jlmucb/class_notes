\chapter{Notes on odd order techniques}
\section {An idea} 
In CN groups of odd order, the maximal abelian subgroups are TI (Reason: from character theory every such
subgroup, $A$, is of the form $C_G(x), 1 \ne x \in A$).   For each such $A$, $M=N_G(A)$ is maximal
and $M$ is Frobenius.  This gives characters of $G$ which lead to a contradiction.
\\
\\
In the general
odd order case, a minimal counterexample, $G$, has the property that every elementary abelian subgroup
of order $p^3$ is in a unique maximal subgroup (by the Uniqueness Theorem).  This allows
us to determine the structure of such maximal subgroups of $G$ and their embeddings in $G$.
This is the origin of
$SCN_k(P), P \in S_p(G)$.  For these groups, $A=C_G(A)$ and $C_G(A) = A \times D$ with $D=O_{p'}(N(A))$.
This in turn takes advantage of the fact that for $m(A) \geq 3$, $C_G(A)$ acts transitively on the set of
maximal $p'$ groups normalized by $A$.  If $G$ is a group of odd order thet doesn't contain any
elementary abelian subgroup of order $p^3$ for any $p$, it is solvable and $G'$ is nilpotent.
\\
\\
If $SCN_3(P)$ is empty, $P$ is generated by $3$ elements.  Then $P \subseteq M'$ where $M$ is a maximal subgroup
of $N_G(P)$.  These $p^2$ groups restrict the structure and embedding of the maximal subgroups which, coupled
with the Uniqueness result provides contradictory evidence for the existance of $G$.
Finally, an argument using generators and relations leads to a contradiction.  
\\
\\
$PSL_3(7)$ has ann elementary abelian $p^2$ groups as an example.  $PSL_2(11)$ has a single conjugacy
class of involutions and the centralizer of an involution is dihedral of order $12$; it has an elementary
abelian group of order $4$ but the transitivity theorem doesn't hold.
