\chapter{Extensions}
\section {Transversals} 
{\bf Definition 1:}
As before, $G$ is an \emph{extension} of $K$ by $Q$ if $G \triangleright K$ and $G/K \cong Q$
or equivalently $1 \rightarrow K \rightarrow G \rightarrow Q \rightarrow 1$.
\\
\\
{\bf Extending a group:}
Suppose $G$ is an extension of $N$ by $H$ and let $\phi: H \rightarrow G/N$.  Pick
$s:H \rightarrow G$ such that $s(1)=1$ and $\phi(h) = N s(h)$, then 
$\exists f: H \times H \rightarrow N: s(h_1) s(h_2)= f(h_1, h_2) s(h_1 h_2)$
and $f(h_1, h_2) f(h_1 h_2, h_3)= f(h_2, h_3)^{s(h_1)} f(h_1 , h_2 h_3)$.  Note
that $\theta_h: n \mapsto s(h) n s(h)^{-1}$ is in $Aut(N)$ and
$\theta_{h_1}(\theta_{h_2}(n))= \theta_{h_1 h_2}(n)^{f(h_1, h_2)}$.  Note, here $a^b = b a b^{-1}$, usually,
we write $a^b = b ^ {-1} a b$, oh well.
\\
\\
{\bf Theorem:}
Given $N,H$ with $\theta_h \in Aut(N)$ and $\theta_1 = 1$ and a map
$f: H \times H \rightarrow N$ with $f(1,h)=f(h,1)=1$ and
$f(h_1, h_2) f(h_1 h_2 , h_3)= \theta_{h_1}(f(h_2, h_3)) f(h_1, h_2 h_3)$, 
suppose $f$ is compatible in the sense that 
$\theta_{h_1}(\theta_{h_2}(n))= \theta_{h_1h_2}(n)^{f(h_1, h_2)}$ then the
operation $(n_1, h_1) \cdot (n_2, h_2) = (n_1 \theta_{h_1}(n_2) f(h_1, h_2), h_1 h_2)$
defines a group $G$ which is an extension of $N$ by $H$.  
$1 \rightarrow N \rightarrow G \rightarrow H \rightarrow 1$ holds with the obvious
embedding $n \mapsto (n,1)$ and $h \mapsto (1,h)$.
\begin{quote}
\emph{Proof:}  Put $(n,h)^{-1}= \theta^{-1}[n^{-1} f(h,h^{-1})^{-1}], h^{-1})$.
$(n,h)^{-1} \cdot (n,h)= (1,1)$.   $(n,h) \cdot (1,1)= (n,h)$.  Associativity is a long
calculation but works.
\end{quote}
{\bf Definition 2:}
A subset ${\cal T}$ consisting of a representative of each coset
in $G/K$ is called a \emph{transversal}.
\\
\\
{\bf Definition 3:}
If $\pi: G \rightarrow Q$ is a surjective homomorphism with kernel
$K$, $l: Q \rightarrow G$ is a \emph{lifting} if $\pi(l(x))=x$.
If $\pi: Q \rightarrow G$ is a surjective homomorphism with
kernel $K$ and $l:Q \rightarrow G$ is a transversal with $l(1)=0$ then
$f: Q \times Q \rightarrow K$ defined by $l(x)+l(y)= f(x,y) + l(xy)$ is
called a \emph{factor set}.  An ordered triple, $(Q, K, \theta)$ is called \emph{data}
if $K$ is an abelian group, $\theta: Q \rightarrow Aut(K)$; a group $G$ is said to
\emph{realize} the data if $G$ is and extension of $K$ by $Q$ and for every transversal,
$l: Q \rightarrow G$ satisfies $xa = \theta_x(a)
= l(x) +a-l(x)$.  Note additive notation for non-abelian operation.
\\
\\
{\bf Theorem 2:}
Let $G$ be an extension of $K$ by $Q$ and
$l: Q \rightarrow G$ a transversal.  If $K$ is abelian there is a homomorphism
$\theta: Q \rightarrow Aut(K)$ with $\theta(a)= l(x)+a-l(x), \forall a \in K$;
If $l_1: Q \rightarrow G$ is another transversal then
$ l(x)+a-l(x)= l_1(x)+a-l_1(x)$.
\begin{quote}
\emph{Proof:}
$K \lhd G$ so $\gamma_{g|K}$ is an automorphism of $K$ ($\gamma_g$ is conjugation by $g$).
$\mu: G \rightarrow Aut(K)$ given by $\mu(g)= \gamma_g$ is a homomorphism with
$K \leq ker(\mu)$.  $\mu$ induces a homomorphism $\mu_{\#}: G/K \rightarrow Aut(K)$ given by
$\mu_{\#}(Kg)= \mu(g)$.
The first isomorphism theorem gives the isomorphism $\lambda: Q \rightarrow G/K$ and if
$l:Q \rightarrow G$ is a transversal $\lambda(x)=K+l(x)$.
If $l_1: Q \rightarrow G$ is another transversal then $l(x) - l_1(x) \in K$ so
$ K+l(x)= K+l_1(x), \forall x \in Q $.  Thus $\lambda$ does not depend on the choice of transversal.
Put $\theta= \mu_{\#} \lambda$.  $\theta_x = \mu_{\#}(K+l(x))= \mu(l(x)) \in Aut(K)$ so for $a \in K:
\theta_x(a)= \mu(l(x))(a)= l(x)+a-l(x)$.
\end{quote}
{\bf Theorem 3:}
Let $\pi: G \rightarrow Q$ be a surjective homomorphism with kernel $K$ and
$l:Q \rightarrow G$ be a transversal with $l(1)=0$ and $f: Q \times Q \rightarrow K$ the corresponding
factor set.  $f(1,y)=0=f(x,1), \forall x,y \in Q$ and
the \emph{cocycle identity} $xf(y,z)-f(xy,z)+f(x,yz)-f(x,y)=0$ holds $\forall x,y,z \in Q$.
\begin{quote}
\emph{Proof:}  
Definition gives $l(x)+l(y)= f(x,y)+l(xy)$.
So $l(1)+l(y)= f(1,y)+l(y)$ and since $l(1)=0$, $f(1,y)= 0$.  Similarly, $f(x,1)=0$.
$[l(x)+l(y)]+l(z)= f(x,y)+f(xy,z)+l(xyz)$ and
$l(x)+[l(y)+l(z)]= xf(y,z)+f(x,yz)+l(xyz)$.
\end{quote}
{\bf Theorem 4:}
Let $G$ realize $(Q, K, \theta )$ and $l$ and $l'$ be transversals with
$l(1)=l'(1)=0$ giving rise to factor sets $f$ and $f'$ then there is an
$h:Q \rightarrow K$ with $h(1)=0$ such that 
$f'(x,y)-f(x,y)= xh(y)-h(xy)+h(x), \forall x,h \in Q$.
The \emph{cocycle identity} $xf(y,z)-f(xy,z)+f(x,yz)-f(x,y)=0$ holds $\forall x,y,z \in Q$.
\begin{quote}
\emph{Proof:}  
By definition, 
$l(x) + l(y)= f(x,y) + l(xy)$
and
$l(1) + l(y)= f(1,y) + l(xy)$; since $l(1)=0$, this gives $f(1,y)=0$ similarly $f(x,1)=0$.
$[l(x)+l(y)]+l(z) = f(x,y)+f(xy,z)+l(xyz)$ and
$l(x)+[l(y)+l(z)] = xf(y,z)+l(xy)+l(z)= xf(y,z)+f(xy,z)+l(xyz)$ so the result follows from associativity.
\end{quote}
{\bf Definition:}  Given data $(Q, K, \theta)$ a \emph{coboundary} is a function
$g: Q \times Q \rightarrow K$ for which $\exists h: Q \rightarrow K$ such that
$h(1)=0$ and $g(x,y)= xh(x)-h(xy)+h(x)$.
The set of all coboundaries is $B^2 (Q,K,\theta)$.
$Z^2 (Q, K, \theta)$ is the set of all \emph{ factor sets}.
$H^2 (Q, K, \theta) \cong Z^2 (Q, K, \theta) / B^2 (Q, K, \theta)$.
\\
\\
{\bf Theorem 5:} 
Given data $(Q, K, \theta)$ a function $f: Q \times Q \rightarrow K$ is a factor set iff it satisfies the
cocycle identity.
\begin{quote}
\emph{Proof:}  
The $\rightarrow$ direction is the previous theorem.  For $\leftarrow$, let $G$ be the set of all ordered pairs,
$(a,x) \in K \times Q$ with the operation
$(a,x)+(b,y)= (a+xb+f(x,y), xy)$.  This is a group with the cocycle identity required for associativity.
\end{quote}
{\bf Notation:} Denote the $G$ constructed in the previous proof as $G_f$ realizing
$(Q, K, \theta )$ with factor set $f$ arising from $l(x)= (0,x)$.
\\
\\
{\bf Definition:}
Two extensions $G, G'$ realizing
$(Q, K, \theta )$ with factor sets $f, f'$ are \emph{equivalent} if $f'-f \in B^2(Q, K, \theta)$.
\\
\\
{\bf Theorem 6:}
Two extensions are equivalent if the difference of
their two factor sets is in $B^2 (Q, K, \theta)$.
\begin{quote}
\emph{Proof:}  
For each $x \in Q$, $l(x)$ and $l'(x)$ are representatives of the same coset
of $K$ in $G$.  Thus $\exists h(x) \in K$ with $l'(x)= h(x) + l(x)$.  Since
$l'(1)= 0 = l(1)$, $h(1) = 0$.  We have
$l'(x) + l'(y) =
(h(x) + l(x)) +
(h(y) + l(y)) =
h(x) + x h(y) + f(x, y) + l(xy) =
h(x) + x h(y) + f(x, y) - h(xy) + l'(xy)$.  Therefore
$f'(, y)= h(x) + x h(y) + f(x, y) - h(xy)$.  The conclusion follows since each term
is in an abelian group.
\end{quote}
{\bf Theorem 7:}
There is a bijection from $H^2(Q,K,\theta)$ to the set, $E$, of 
all equivalence classes realizing the data
$(Q, K, \theta )$ taking the identity to the class of the semidirect product.
\begin{quote}
\emph{Proof:}  
Let  the
equivalence classes of the extensions realizing the data
$(Q, K, \theta )$ be denoted by 
$[G]$.  Define $\varphi: H^2(Q, K, \theta ) \rightarrow E $ by $\varphi(f+B^2(Q, K, \theta))= G_f$.
$\varphi$ is well defined since if $f,g$ are factor sets, $f-g \in
B^2(Q, K, \theta )$ and $[G_f]=[G_g]$. $\varphi$ is a surjection since if
$[G] \in E$ and $f$ is a factor set, $[G]=[G_f]$ and $[G]= \varphi(f+B^2)$.  Finally, an extension is a
semidirect product off its factor set is in $B^2(Q,K,\theta)$.
\end{quote}
{\bf Definition:}
A \emph{projective representation} of a group $Q$ is a homomorphism
$\tau: Q \rightarrow PGL_n({\mathbb C})$.
We say $U$ has the \emph{projective lifting property} if every projective representation
of $Q$ can be lifted to $U$.
If $Q$ is a group then a \emph{cover} (or \emph{representation group}) of $Q$ is a
central extension of $U$ of $K$ by $Q$ (for some abelian $K$) with the projective lifting property
and $K \le U'$.
\\
\\
{\bf Definition:}
The \emph {Schur multiplier} is
$M(Q)=H^2 (Q, {\mathbb C}^{\times})$ ($\theta$ is trivial).
Here $f(1,y)=f(x,1)=1$, $f(x,y) f(xy,z)^{-1} f(x,yz) f(x,y)^{-1}=1$,
$g: Q \times Q \rightarrow {\mathbb C}^{\times}$ is a coboundary
iff $\exists h: Q \rightarrow {\mathbb C}^{\times}$ with $h(1)=1$ such that
$g(x,y)= h(y)(h(xy))^{-1}h(x)$.
\\
\\
{\bf Schur's Theorem:} Every finite group, $Q$, has a cover $U$ which is a central extension
of $M(Q)$ by $Q$.
\begin{quote}
\emph{Proof:} See Rotman.
\end{quote}
{\bf Definition:}
$exp(G)= min \{e: x^e = 1, \forall x \in G \}$.
\\
\\
{\bf Theorem 8:} If $Q$ is finite then $M(Q)$ is a finite abelian group and
$exp(M(Q)) \mid |Q|$.
\begin{quote}
\emph{Proof:}  
See Rotman, p 201-211
\end{quote}
{\bf Theorem 9:} If $Q$ is finite $p$-group then $M(Q)$ is a finite abelian $p$-group.
\begin{quote}
\emph{Proof:}  
See, Rotman, p 202
\end{quote}
{\bf Theorem 10:} Let $G$ be a group with $G/{\mathbb Z}(G)$ finite, then $G^{(1)}$ is finite.
\begin{quote}
\emph{Proof:}
Let $n= |G/{\mathbb Z}(G)|$.  For $z \in {\mathbb Z}(G)$ and $g,h \in G$: $[g,hz]=[g,h]=[gz,h]$ so the 
set of commutators, $\Delta$, is of order at most $n^2$.
Claim: $g \in G^{(1)}$ then $g= x_1 x_2 \ldots x_m$, $x_i \in \Delta$ and
$m \le n^3$.
\end{quote}
{\bf Observation:}
A cyclic extension $G$ of $N$ is
one where $G/N$ is cyclic.  Solvable groups are built from cyclic extensions.
\\
\\
{\bf Definition:} 
$G$, an extension of $K$ by $Q$, is a \emph{central extension} if $K<{\mathbb Z}(G)$.  Functorially,
a central extension $G$ is a pair $(H, \pi)$ satisfying
$\pi: H \rightarrow G, ker(\pi) \subseteq {\mathbb Z}(H)$.  
$\alpha: (H_1 , \pi_1) \rightarrow (H_2, \pi_2)$ is a morphism in this category. 
The universal object in this category (if it exists) is called a 
\emph{universal central extension}.
It follows that $(\tilde{G}, \tilde{\pi})$ is universal if 
$\forall (H, \sigma), \exists ! \alpha : (\tilde{G}, \tilde{\pi}) \rightarrow
(H, \sigma)$.
\\
\\
{\bf Theorem 11:} Up to isomorphism, there is at most one universal central extension.
\begin{quote}
\emph{Proof:} 
If $(G_1, \pi_i), i= 1, 2$ are universal central extensions of $G$, 
$\exists \alpha_1:(G_i, \pi_i) \rightarrow (G_{3-i}, \pi_{3-i})$.
$\alpha \alpha_{3-i} = 1$ and by the uniqueness $\alpha_1 \alpha_2 = 1 = \alpha_2 \alpha_1$.
Thus the $\alpha_i$ is an isomorphism.
\end{quote}
{\bf Theorem 12:}   If $(\tilde{G}, \pi)$ is a universal central extension of $G$ then
both $G$ and $\tilde{G}$ are perfect.
\begin{quote}
\emph{Proof:} 
Let $H = \tilde{G} \times (\tilde{G}/\tilde{G}')$ and define $\alpha: H \rightarrow G$ by
$\alpha(x,y)= \pi(x)$.   Then $(H, \alpha)$ is a central extension of $G$ and
$\alpha_i: (\tilde{G}, \pi) \rightarrow (H, \alpha)$ are morphisms
where $\alpha_1(x)= (x,1)$ and $\alpha_2(a)= (x, x \tilde{G}')$.  By uniqueness,
$\alpha_1= \alpha_2$, hence $\tilde{G} = \tilde{G}'$.  Thus $\tilde{G}$ is perfect and so
$G= \pi(\tilde{G})$.
\end{quote}
{\bf Theorem 13:}  Let $G$ be perfect and $H, \pi$ be a central extension of $G$ then
$H=ker(\pi)H'$ with $H'$ perfect.
\begin{quote}
\emph{Proof:} $\pi: H \rightarrow G$, $\pi(H')= \pi(H)'=G'=G$ and $ker(\pi) \le {\mathbb Z}(H)$.
$H/H^2 = {\mathbb Z}(H/H^2)= H'/H^2$ is abelian and $H'=H^2$ so $H'$ is perfect.
\end{quote}
{\bf Theorem 14:}
$G$ possesses a universal central extension iff $G$ is perfect.
If $(\tilde{G}, \pi)$ is a universal central extension then $ker(\pi)$ is called
the \emph{Schur multiplier}.
\begin{quote}
\emph{Proof:}  $\rightarrow$ is easy.  For the converse, suppose $G$ is perfect
$g \mapsto {\overline g}$ is a bijection, $F$ the free group on the symbols of $G$,
$\Gamma= \{ {\overline x} {\overline y} {\overline x}^{-1} {\overline y}^{-1}$.
$M= \langle \Gamma \rangle \lhd F$, $\Delta= \{ [w,z], w \in \Gamma, z \in G \}$, 
$N= \langle \Delta \rangle \lhd F$.
$N=[M,F] \lhd M$, $M/N \le {\mathbb Z}(F/N)$.. $\exists \pi: F/N \rightarrow G$
$\pi({\overline x}N)= x$, $ker(\pi) \le {\mathbb Z}(F/N)$.  Now let $H, \sigma$ be any
central extension.  $w= h(x) h(y) H(xy)^{-1} \in ker(\sigma)$.  $ker(\sigma) \le C_H(h(z))$
and $[w,h(z)]=1$.  $\tilde{G} =(F/N)'$ and $F/N = ker(\pi) \tilde{G}$; further,
$\tilde{G}' = \tilde{G}$.  $(\tilde{G}, \pi)$ is a universal central extension.
\end{quote}
{\bf Note:} Quasisimple groups are exactly the central extensions of simple groups.  
$H_2(G,{\mathbb Z}) = {\frac {(R \cap F')} {[F,R]}}$, further, if $Q$ is
perfect then  $F'/[F,R]$ is a cover of $Q$. $SL_k(p)$ is a central extension of
$PSL_k(p)$.
\\
\\
{\bf Theorem 15:}
Let $(H, \alpha)$ be a central extension of a group $G$ and $(K, \beta)$ is a perfect
central extension of $H$ then $(K, \alpha \beta )$ is a perfect central extension of
$G$.
\begin{quote}
\emph{Proof:}  
$\alpha \beta : K \rightarrow G$ is surjective.  Let $x \in ker(\alpha \beta)$ and
$y \in K$.  $\beta(x) \in ker( \alpha ) {\mathbb Z}(H)$, so 
$\beta([x,y]) = [\beta(x), \beta(y)] =1$ and $[x,y] \in ker ( \beta ) \le {\mathbb Z}(K)$.
Thus $[ker( \alpha \beta ), K, K] = 1$ so $ker( \alpha \beta ) \le  {\mathbb Z}(K)$.
\end{quote}
{\bf Theorem 16:}
Let $(H, \alpha)$ and $(K, \beta)$ be central extensions of $G$ with $K$ with
$K$ perfect and $\gamma: (H, \alpha) \rightarrow (K, \beta)$ a morphism of central
extensions, then $(H, \alpha)$ is a central extension of $K$.
central extension of $H$ then $(K, \beta \alpha)$ is a perfect central extension of
\begin{quote}
\emph{Proof:}  
$\gamma: H \rightarrow K$ is a homomorphism with $\alpha = \beta \gamma$.
$ker( \gamma ) \le {\mathbb Z}(H)$ so all we have to show is that $\alpha$ is surjective.
$K= \gamma (H) ker( \beta )$.  $ker( \beta ) \le {\mathbb Z}(H)$,
$\gamma(H) \lhd K$ and $K/ {\mathbb Z}(H)$ is abelian and thus $K= \gamma(H)$ is perfect.
\end{quote}
{\bf Theorem 17:}
Let $\tilde{G}$ be the covering group of a perfect group $G$ and let $(H, \alpha)$ be
a perfect central extension of $\tilde{G}$, then $\alpha$ is an isomorphism.
\begin{quote}
\emph{Proof:}  
$\pi : \tilde{G} \rightarrow G$ be the universal covering.  By previous result,
$H( \pi \alpha)$ is a perfect central extension of $G$.  By universality,
$\exists \beta: ( \tilde{G}, \pi) \rightarrow (H, \pi \alpha )$.  By uniqueness,
$\pi \alpha \beta = \pi$, $\alpha \beta = 1$ and $\beta: \tilde{G} \rightarrow H$ is
an injection.  By previous result, $\beta$ is surjective.  Thus $\beta$ is an isomorphism,
$\alpha \beta = 1$, $\alpha = \beta^{-1}$ is an isomorphism too.
\end{quote}
{\bf Theorem 18:}
Let $G$ be perfect and $(\tilde{G}, \pi)$ the universal central extension of $G$ and
$(H, \sigma)$ a perfect central extension of $G$.  Then:
(1) There exists a covering $\alpha: \tilde{G} \rightarrow H$ with $\pi = \alpha \sigma$;
(2) $(\tilde{G}, \alpha)$ is the universal central extension of $H$;
(3) The Schur multiplier of $H$ is a subgroup of the Schur multiplier of $G$;
(4) if ${\mathbb Z}(G)=1$ then ${\mathbb Z}(\tilde{G})$ is the Schur multiplier of $G$
and ${\mathbb Z}(H) \cong ker(\pi)/ker(\alpha)$ is the quotient of the Schur multiplpier of
$G$ by the Schur multiplier of $H$.
\begin{quote}
\emph{Proof:}  
By the universal property, $\exists \alpha: (\tilde{G}, \pi) \rightarrow (H, \sigma)$.
$\alpha$ is a covering by previous result.  Let $(\tilde{H}, \beta )$ be the universal covering
of $H$.  By the universal property, 
$\exists \gamma: (\tilde{H}, \beta ) \rightarrow (\tilde{G}, \alpha)$.  By previous result,
$\gamma$ is an isomorphism so (2) holds.  (3) and (4) are routine.
\end{quote}
{\bf Theorem 19:}
Let $G$ be a group with $G/{\mathbb Z}(G)$ finite then $G'$ is finite.
\begin{quote}
\emph{Proof:}  
Let $n=|G/{\mathbb Z}(G)|$, $z \in {\mathbb Z}(G)$, $g, h \in G$.
$[g, hz]= [g,h]= [gz, h]$ so the set $\Delta$ of commutators has order $\le n^2$.
\\
\emph{Claim:} If $g \in G'$ then $g= x_1 x_2 \ldots x_m$, $x_i \in \Delta$ then $m \le n^3$. 
This and the previous statement proves the theorem.
\\
\emph{Proof of Claim:} Pick expression of minimal length, $m$.  If $m > n^3$ then,
since $|\Delta| \le n^2$, $\exists d \in \Delta$ with $\Gamma= \{i: x_i=d \}$
of order $k > n$.
$x_i x_{i+1}= x_{i+1} x_i^{x_{i+1}}$
$x_i^{x_{i+1}} \in \Delta$ and $\Gamma= \{ 1 \le i \le n \}$.
Now it STS, $d^{n+1}$ is  a product of $n$ commutators which contradicts the minimality of
$m$.  Let $d= [x,y]$, $|G/{\mathbb Z}(G)|=n$, $d^n \in {\mathbb Z}(G)$ so
$d^{n+1}= (d^n)^x d= (d^{n-1})^x d^x d = (d^x)^{n-1} [x^2 , y]$ so $d$ is a product of $n$
 commutators.
\end{quote}
{\bf Theorem 20:}
Let $G$ be a perfect finite group then the universal covering group of $G$ and the Schur
multiplier are both finite.
\begin{quote}
\emph{Proof:}  
Follows from previous result.
\end{quote}
{\bf Theorem 21:}
Let $(H, \sigma)$ be a perfect central extension of a finite group, $G$,
and $M$ the Schur multiplier of $G$, $p$,
a prime and $P \in S_p(H)$ then $P \cap ker(\sigma) \le \Phi(P)$.
\begin{quote}
\emph{Proof:}  
By looking at $H / (\Phi(P) \cap ker( \sigma )$, we can assume $\Phi(P) ker( \sigma )=1$
and show $X= P \cap ker( \sigma ) = 1$.  ${\overline P}= P/\Phi(P)$ is elementary
abelian so $\exists {\overline Y}: {\overline Y} {\overline X}= {\overline P}$,
$P = X \times Y$ and $P$ splits by Gaschutz, $H$ splits over $X$ hence, $H$ is perfect,
$X \le {\mathbb Z}(H)$, $X=1$.
\end{quote}
{\bf Theorem 22:}
Let $(H, \sigma)$ be a perfect finite group and $M$ the Schur multiplier of $G$,
then $\pi(M) \subseteq \pi(G)$.
\begin{quote}
\emph{Proof:}  
Follows from previous result.
\end{quote}
{\bf Homological version:} If $G>N$
and $H>K$ are normal subgroups isomorphic under $\phi$, the pullback
is $(g, h)$ where $gN= \phi(hK)$.
$(Q, K, \theta)$ is trivial iff every extension realizing $(Q, K, \theta)$
is a central extension.  There's a bijection between $H^2 (Q, K, \theta)$
and central extensions. 
\\
\\
{\bf Theorem 23:}
Assume $G$ is perfect then a central extension
$(E, \phi)$ of $G$ is universal iff (a) $E$ is perfect and (b) all 
central extensions of $E$ are trivial. In that case,
$1 \rightarrow R \rightarrow F \rightarrow G \rightarrow 1$, $F$, free and
$E= [F,F][F,R] \rightarrow [F,F]/R=G$.
\begin{quote}
\end{quote}
\section {Miscellaneous Central Extensions}
{\bf Definition:} If $K$ and $Q$ are groups, $G$ is an \emph{extension} of $K$ by $Q$ if $\exists K_1 \cong K$,
$K_1 \lhd G$ and $G/K_1 \cong Q$.  Let $\pi: G \rightarrow Q$ be a surjection. A \emph{lifting}
of $Q$ is a map $l:Q \rightarrow G: \pi(l(x))=x$.
\emph{Data:} $(K,Q,\theta)$, $K$, abelian, $\theta: Q \rightarrow Aut(K)$, for every transversal $l: Q \rightarrow G$.
$G$ realizes $(K,Q,\theta)$ if $G$ is an extension of $K$ by $Q$, if $\forall: Q \rightarrow G$ with
$xa=\theta(x)(a)= l(x) + a -l(x)$.
\\
\\
{\bf Definition:} Let $1 \rightarrow K \rightarrow G \rightarrow Q \rightarrow 1$, $l(x)$ are
transversals. $f$ is a cocycle if $l(x) + l(y) = f(x, y) +l(xy)$, $f(y,1) = 0 = f(1,x)$ and
$xf(y,z)+f(x,yz) = f(x,y)+f(xy,z)$. $f$ is a \emph{factor} set.
\\
\\
{\bf Theorem:} Let $G$ be an extension if $K$ by $Q$, $l:Q \rightarrow G$, a transversal.  $K$, abelian,
$\theta:Q \rightarrow Aut(K)$ then $\theta(x): a \rightarrow l(x) + a - l(x)$.  If $l_1$ is another
transversal $l(x) + a - l(x) = l_1(x) + a - l_1(x)$.  This gives rise to the a cocycle
$xf(y,z) + f(x, yz) = f(xy,z) + f(x,y)$.
\begin{quote}
\emph{Proof:}
Define $[G_f]$ by $(a,x)*(b,y) = (a + xb + f(x,y), xy)$.  We can prove identity, inverse and, using the cocycle
identity we get associativity.
\end{quote}
{\bf Theorem:} Let $G$ be an extension, $l, l'$ are transversals, $l(1)=0=l'(1)$ giving rise
to $f, f'$, $h:Q \rightarrow Q$ with $h(1)=0$ such that $f'(x,y) - f(x,y) = xh(y) = h(xy) +h(x)$.
\begin{quote}
\emph{Proof:} If $l'(x)= h(x) + l(x)$,
$l'(x) + l'(h) = h(x) + l(x) + h(y) + l(y) = h(x) +xh(y) + f(x,y) + l(xy) = h(x) + xh(y)+f(x,y) -h(xy) +l'(xy)$
and $f'(x,y) = h(x) +xh(y) +f(x,y) -h(xy)$.
\end{quote}
{\bf Definition:} Under the conditions of the previous theorem, the extensions $G_{f}$ and $G_[f']$ are called
\emph{equivalent}.  Define $Z^2(K, Q, \theta)$ as the set of cocycles under addition.  Given
$(K,Q,\theta)$, $g: Q \times Q \rightarrow K$ is a \emph{co-boundary}, if $g(x,y) =xh(y)-h(xy)+h(x)$,
the set of co-boundaries is denoted $B^2(K,Q, \theta)$.  $H^2(K, Q, \theta) = Z^2(K, Q, \theta) / B^2(K,Q, \theta)$.
If $G$ and $G'$ realize $(K, Q, \theta)$, they are equivalent iff $f' - f \in B^2(K, Q, \theta)$ by the theorem
above.
\\
\\
{\bf Theorem:} Let $G$ and $G'$ be extensions realizing$(K,Q, \theta)$ are equivalent if
$\exists \gamma: G \rightarrow G'$ making the diagrams
$1 \rightarrow K \rightarrow G \rightarrow Q \rightarrow 1$ and
$1 \rightarrow K \rightarrow G' \rightarrow Q \rightarrow 1$ commute.
\\
\\
{\bf Definition:} An extension of $K$ by $Q$ is central iff $ K \subseteq {\mathbb Z}(G)$.
$Q$ is a finite group $U$ is a \emph{cover} of $Q$
is an extension of $K$ by $Q$ and $K \leq U'$.
A \emph{projective representation} of a group $Q$ is a homomorphism
$\tau: Q \rightarrow PGL_n({\mathbb C})$.
We say $U$ has the \emph{projective lifting property} if every projective representation
of $Q$ can be lifted to $U$.hi (i.e.- $\tau:Q \rightarrow PGL_n({\mathbb C})$ lifts to
${\overline {\tau}}: Q \rightarrow GL_n({\mathbb C})$.)
If $Q$ is a group then a \emph{cover} (or \emph{representation group}) of $Q$ is a
central extension of $U$ of $K$ by $Q$ (for some abelian $K$) with the projective lifting property
and $K \le U'$.
\\
\\
\\
{\bf Schrier} There is a bijection from $H^2(K, Q, \theta)$ to $E$, the set of all equivalence classes
of extensions realizing $(K,Q,\theta)$ taking $0$ to the semi-direect product.  $H^2(K, Q, \theta) = 0$
iff every extension realizing $(K, Q, \theta)$ is a semi-direct product.  $\phi: H^2(K, Q, \theta) \rightarrow E$
by $\phi(f +  B^2(K,Q, \theta)) = G_{f}$.
\\
\\
{\bf Definition:} $M(Q)= H^2(K,C^{\times}, \theta = 1)$.  Note under these circumstances,
$f(1,y)= 0 = f(x,1)$ and $f(x,y)f(xt,z)^{-1}f(x,yz) f(x,y)^{-1}=1$.
\\
\\
{\bf Theorem:} If $Q$ is finite, $M(Q)$ is a finite abelian group, $exp(M(Q)) \mid |Q|$.
\begin{quote}
\emph{Proof:}
Define $sigma(x) = \prod_z f(x,z)$.  From cocycle identity,
$\sigma(y)\sigma(xy)^{-1}\sigma(x) = f(x,y)^n$, $n = |Q|$.
For $x \in Q$, define $h:Q \rightarrow {\mathbb C}$ by $h(1)=1$ and $h(x)$ an $n$th root of $\sigma(x)^{-1}$.
$g(x,y)=f(x,y)h(y)h(xy)^{-1}h(x), f \sim g$ and $g(x,y)^n =1$, $f] \in M(Q)$ determines
$g: q \times Q \rightarrow {\mathbb Z}_n$ and there are only finitely many such $g$.
\end{quote}
{\bf Theorem:} Every finite group $Q$ has a cover $U$ which is a central extension of $M(Q)$ by $Q$.
\\
\\
{\bf Theorem:} If $Q=F/R$ where $F$ is free, $M(Q)= (R \cap F')/[F,R]$.
\\
\\
{\bf Theorem:} If $\nu: U \rightarrow Q$ is a central extension $ker(\nu)=K$ and $Q$ is perfect then
$U'$ is perfect and $|nu_{U'}: U' \rightarrow Q$ is surjective.
\\
\\
{\bf Theorem:} If $Q$ is perfect, $Q=F/R$, $M(Q)= (R \cap F')/[F,R]$ and $F'/[F,R]$ is a cover.
\\
\\
{\bf Theorem:} $U = F'/[F,R]$ is a universal central extension.
