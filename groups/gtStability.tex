\chapter{Stability}
\section {Quadratic Action}
{\bf Definition 1:} 
$G$ is \emph{$\pi$-separable} iff every composition factor of $G$ is
either a $\pi$-group or a $\pi'$ group.
$G$ is \emph{$\pi$-solvable} iff every composition factor of $G$ is
either a solvable $\pi$-group or a $\pi'$ group.
\\
\\
{\bf Theorem 1:}
(1) If $G$ is $\pi$ separable iff the upper and lower $\pi$ series of $G$ terminate at
$G$.\\
(2) If $G$ is $\pi$-separable ($\pi$-solvable) so are subgroups and homomorphic images.
$G$.\\
(3) If $G$ is $\pi$-separable a minimal normal subgroup of $G$ is either a 
$\pi$-group or a
$\pi'$-group.
\begin{quote}
\emph{Proof:}  
For 3,
let $K$ be a minimal normal subgroup of the $\pi$ separable group $G$.
$K$ is characteristically simple an is thus a direct product of simple groups.
There is a composition series in which one of the isomophic subgroups in the
direct product is the last term and this must be a $\pi$ or $\pi'$ group.
For 2, homomorphic and normal subgroups are sutomatically $\pi$-separable (or $\pi$-solvable).
Let $H \le G$ and $K$ be a minimal normal subgroup of $G$, ${\overline G}= G/K$.
By induction, ${\overline H}$ is $\pi$-separable so we only need to show $H \cap K$ is;
this follows from 3 since $K$ is either a $\pi'$ group or a solvable $\pi$ group.
For 1, if the upper or lower series terminate, they can be refined to a composition series and
each of the factors is either a $\pi$ group or a $\pi'$ group. 
In either case, $G$ is $\pi$ separable.
Conversely, if $G$ is $\pi$-separable and the upper $\pi$ series of $G$ terminates in
the proper subgroup $H$ of $G$, putting ${\overline G}= G/H$, we have 
$O_{\pi}(G) = O_{\pi'}(G)= 1 $. But ${\overline G}$ is $\pi$ separable by 2 and so a minimal
normal subgroup ${\overline K}$ is either a $\pi$ or $\pi'$ group by 3 and thus either
${\overline K} \subseteq O_{\pi}({\overline G})$ or
${\overline K} \subseteq O_{\pi'}({\overline G})$ and either is a contradiction establishing 1.
\end{quote}
{\bf Theorem 2:}
If $G$ is $\pi$ separable and ${\overline G}= G/O_{\pi'}(G)$ then
$C_{\overline G}(O_{\pi}({\overline G})) \subseteq O_{\pi}({\overline G})$.
\begin{quote}
\emph{Proof:}  
STS this when
$O_{\pi'}( G ) = 1$.  Set $H= O_{\pi}(G)$ and $C= C_G(H)$ so $C \cap Z = {\mathbb Z}(H)$ and
we must show $C= {\mathbb Z}(H)$.  $O_{\pi}(C) \thinspace char \thinspace \lhd G$ so
$O_{\pi}(C) \lhd G$ and hence
$O_{\pi}(C) \subseteq H$.  Thus $O_{\pi}(C)= C \cap H = {\mathbb Z}(H)$.  On the other hand,
${\mathbb Z}(H) \lhd G$ and
${\mathbb Z}(H) \subseteq C$ so
${\mathbb Z}(H) \subseteq O_{\pi}(C)$ and thus
${\mathbb Z}(H) = O_{\pi}(C)$.
Assume, by way of contradiction, that $C \supset {\mathbb Z}(H)$ then
$C \supset O_{\pi}(C)$.  $C$ is $\pi$-separable so
$L= O_{\pi, \pi'}(C) \subset O_{\pi}(C)$.  $L/O_{\pi}(C)$ is a $\pi'$ group, 
${\mathbb Z}(H)= O_{\pi}(C)$ is a normal $S_p$ subgroup of $L$ and by Schur-Zassenhaus,
${\mathbb Z}(H)$ has a normal complement, $K \ne 1$ in $L$ which is a normal $S_{\pi'}$
subgroup of $L$.  But $K \subseteq C$ and $[C, {\mathbb Z}(H)]=1$ so
$L= {\mathbb Z}(H) \times K$ and since $K$ is a $\pi'$ group, $K \lhd G$ and
$K \subseteq O_{\pi'}(G)= 1$, a contradiction.
\end{quote}
{\bf Theorem 3:}
If $G$ is $\pi$-solvable 
$C_G(P \cap O_{p', p}(G)) \subseteq O_{p', p}(G)$.
\begin{quote}
\emph{Proof:}  
Let ${\overline G} = G/O_{p'}(G)$.  $C_{\overline G}({\overline P}) \subseteq  O_{p}({\overline G})$.
By coprime action, $C_{\overline G}({\overline P}) = C_G(P) O_{p'}(G)/O_{p'}(G)$.  By Theorem 2,
$C_G(O_{p',p}(G)) \subseteq O_{p',p}(G)$ and the result follows.
\end{quote}
{\bf Theorem 4:}
Let $G$ be a $p$-solvable group in which $O_{p'}(G)= 1$ and $H= O_p(G)$, then
$G/H$ is faithfully represented on $H/\Phi(H)$.
\begin{quote}
\emph{Proof:}  
Let $g$ act by conjugation on $H$. 
If $[g, H]=1$, $g \in p(G)$ because $O_{p'}(G)=1$.  So $gH$ acts on $H$ and is a $p'$-element.
By the result in the critical subgroups section, if $gH$ acts non-trivially on $H$ then
$gH$ acts non-trivially on $H/\Phi(H)$.
\end{quote}
{\bf Definitions 2:} If $V$ is an elementary abelian $p$-group then $a$ 
acts quadratically on $V$ if $[V,a,a]=1$, in which  case, $v^{(a-1)^2}=0$.
$G$ is said to be 
\emph{$p$-separable} if two non conjugate elements of $G$ remain non-conjugate in 
some finite $p$-group endomorphic image of $G$.
The \emph{upper $\pi$ series}  is
$\{ 1 \} \subseteq O_{\pi}(G) \subseteq O_{\pi, \pi'}(G) \subseteq O_{\pi, \pi', \pi}(G) \ldots$
The \emph{lower $\pi$ series} is
$\{ 1 \} \subseteq O_{\pi'}(G) \subseteq O_{\pi', \pi}(G) \subseteq O_{\pi', \pi, \pi'}(G) \ldots$.
\\
\\
{\bf Examples of quadratic action:} \\
(a) $G$ a $p$-group acting on the elementary abelian $p$-group $V$: $|V/C_V(G)|= p$;\\
(b) $G \in S_p(SL(V))$ acting on $V$ the vector space over $F_{p^m}$; in this case, note $x \in S_p(SL_2(p)) \rightarrow (x-1)^2 = 0$.\\
(c) $V, G \lhd H$, $G$-abelian, since $[V,G] \subseteq V \cap G$ and $[V \cap G, G] = 1$.
\\
\\
{\bf Theorem 5:}
If $G$ acts quadratically on $V$ then (a) $[v^n,a]=[v,a^n]=[v,a]^n$,
(b) $|V| \le |C_V(a)|^2$, (c) $G/C_G(V)$ is an elementary abelian $p-$group.
\begin{quote}
\emph{Proof:}  
$[v,a^2]= [v,a] [v,a]^a$ but quadratic action gives $[v,a]^a= [v,a]$ so
$[v,a^2]= [v,a]^2$.  $[V,G']= 1$ so $G' \subseteq C_G(V)$ and so $G/C_G(V))$ is abelian.
Since $[v, a^p] = 1$, it is elementary abelian.  For (b), note that
$V/C_V(G) \cong [V, a] \le C_V(a)$.
\end{quote}
{\bf Theorem 6:}
Let $G$ act on an $F_q$ vector space $W \ne 0$, $q=p^m$.  Suppose $G= \langle a,b \rangle $ and
$a, b$ act quadratically on $W$, $G/C_G(W)$ is not a $p-$group, $|ab|=p^ek, k \mid (p-1)$
then $\exists \varphi: G \rightarrow SL_2(q)$.  
\begin{quote}
\emph{Proof:}  
By induction on $|G|+ dim(W)$.  If action is not faithful, we're done by induction.
Let $W_1$ be a maximal $G$-invariant subspace of $W$.  If $G/C_G(W_1)$ is not a
$p$-group,then $C_G(W/W_1)$ is not a $p$-group; if $W_1 \ne 0$, again we're done by induction.
So $G$ acts faithfully and irreducibly on $W$ and if $a, b$ 
are $p$-elements, they act quadratically and $G$ is not abelian.
By Schur, $< \langle ab \rangle $ acts as a scalar on a minimal 
$ \langle ab \rangle $-invariant subspace
of $W$.  $\exists 0 \ne W \in W, \lambda \in F_q^*: w^{ab}= \lambda w$.
If $w^a \in F_q w$, it is $G$-invariant and $W= F_qw$ by irreducibility and $G$ is
abelian.  Contradiction.  Thus $W_1 = F_q w + F_q w^a$ is $2$-dimensional and we
have:
$[w,a] \in C_{W_1}(a)$ and
$w^{b^{-1}} -w \in C_{W_1}(b)$.  So $(w^a)^a \in W_1$, $w^b \in W_1$ and $w^{ab} \in W_1$.
so $W_1$  is $G$-invariant and $W= W_1$.  $G \le SL(W)$ since $SL_2(q)$ is generated by
$p$-elements $a, b$.
\end{quote}
{\bf Definition 3:} $G$ is \emph {$p$-stable} if $\forall a \in G, [V,a,a]=1$ implies 
$a C_G(V) \in O_p ( G/C_G(V))$.
\\
\\
{\bf Question:}  If $V/F_q$, $q= p^n$, is a faithful $G$-module, when does a $p$ element of
$G$ have a quadratic minimal polynomial?  This is trivial for $p=1$ and is true for
all elements of $SL_2(q)$.  Let $G$ be a group and $O_p(G)=1, p \ne 2$
A faithful representation of $G$ on $V$, $\varphi$ is \emph{$p$-stable} if
no $p$-element of $\varphi(G)$ has a quadratic minimal polynomial.  $G$ is $p$-stable
if all such faithful representations of $G$ are $p$-stable.
\\
\\
{\bf Lemma:} If $G$ is $p$-stable and $a$ acts trivially on $[V,a]$ then $ \langle A \rangle 
C_G(V)/C_G(V)$ 
is a $p$-group.
\begin{quote}
\emph{Proof:}  
$[V,a,a]=1$ so $a C_G(v) 
\in O_p(G/C_G(V))$ and $\langle a C_G(V) \rangle = \langle A \rangle
\subseteq  O_p(G/C_G(V))$.
\end{quote}
{\bf Theorem 7:} Let $p \ne 2$ and $G$ be faithful on $V$.
Suppose (1) $G= \langle a,b \rangle $ where $a$ and $b$ act quadratically on $V$ and
(2) $G$ is not a $p-$group then (1) the Sylow $2$ subgroups of $G$ are not abelian and
(2) If $Q$ is a normal $p'$-subgroup of $G$ and $[Q,a] \ne 1$ then $p=3$ and there
is a section of $G$ isomorphic to $SL_2(3)$.  If $p \ne 2$.
\begin{quote}
\emph{Proof:}  
Let $|ab|= p^e k$ and $q$ be a power of $p$ with $k \mid (q-1)$.  Write $V$ additively
as a vector space over $F_p$ and choose a basis $ \langle v_1 , v_2 , \ldots , v_n \rangle $; 
the action
can be extended to an action of $G$ over $W$.  By the previous result,
$\exists \varphi: G \rightarrow SL_2(q)$ with 
$G^{\varphi}= \langle a^{\varphi}, b^{\varphi} \rangle $ and
$G$ is not a $p$-group so it is not $p$-closed.  This gives (a).  (b) follows since
if $Q^{\varphi}$ is an $a^{\varphi}$-invariant $p'$-subgroup
such that $[Q^{\varphi}, a^{\varphi}] \ne 1$.
\end{quote}
{\bf Theorem 8:}
Suppose $p \ne 2$ and the action of
$G$ on $V$ is faithful and not $p-$stable then (1) the Sylow $2$-subgroups of
$G$ are non-Abelian and (2) if $G$ is $p-$separable 
then $p=3$ and there is a section of
$G$ isomorphic to $SL_2(3)$.
\begin{quote}
\emph{Proof:}  
$\exists a \in G \setminus O_p(G)$ such that $[V,a,a] = 1$.  Let ${\cal K}$ be the
$G$-composition factors of $V$.  $O_p(G)= \bigcap_{W \in {\cal K}} C_G(W)$.  Hence,
$\exists W \in {\cal K}, a \notin C_G(W)$ so a $C_G(W) \notin O_p(G/C_G(W)) = 1$.
Thus $G/C_G(W)$ and $W$ satisfy the hypothesis and we can assume $W=V, O_p(G)=1$ by
induction.  By Baer, $\exists b \in a^G: G_1 = \langle a,b \rangle $ is not a $p$-group.  Now
(a) follows from the previous result.  If $G$ is $p$-seperable, put $Q= O_{p'}(G)$ then
$[Q, a] \ne 1$ and $b$ can be chosen in $a^Q$ so we get (b).
\end{quote}
{\bf Theorem 9:} Suppose $G$ acts faithfully on $V$ and $E_1 , E_2$ are
two subnormal subgroups of $G$ 
such that $[V,E_1 , E_2 ]=1$ then $[E_1 , E_2 ] \le O_p (G)$.
\begin{quote}
\emph{Proof:}  
By hypothesis, $V_1= [V, E_1]$ is invariant under $E= \langle E_1 , E_2 \rangle $ so
$E_0 = C_E(V_1)$ and $E^0= C_E(V/V_1)$ are normal in $E$.
$E_0 \cap E^0$ acts quadratically on $V$ and is a $p$-group by the earlier result.
Thus $E_0 \cap E^0 \le O_p(E)$.
$E_1 \le E^0$ and
$E_2 \le E_0$ so $[E_1 , E_2 ] \le [E^0 , E_0] \le E^0 \cap E_0 \le O_p(E)$ so
$E$ and $O_p(E)$ are subnormal in $G$ and hence, $O_p(E) \le O_p(G)$.
\end{quote}
$Q_8= \langle
\left(
\begin{array}{cc}
i & 0 \\
0 & -i \\
\end{array}
\right),
\left( 
\begin{array}{cc}
0 & -1 \\
-1 & 0 \\
\end{array}
\right) \rangle $.
\section {Replacement Results}
{\bf Observation:} $A^* = C_A([V,A])$ acts quadratically on $V$.
\\
\\
{\bf Condition ${\cal Q}_1$:}
$|A| |C_V(A)| \ge |A^*| C_V(A^*)|, \forall A, A^*$.
\\
\\
{\bf Condition ${\cal Q}_2$:}
$A/C_A(V)$ is an elementary abelian $p$-group.
\\
\\
{\bf Definition 4:}  ${\cal A}_V(G) = \{ A \le G: A$ satisfies ${\cal Q}_1$ and ${\cal Q}_2 \}$.
\\
\\
{\bf Theorem 10:}  
Suppose $A$ acts on an elementary abelian $p$-group, $V$, with $A/C_A(V)$ abelian.
Let $U \le V$.  Then $\exists A^* \le A$ such that one of the following holds:
(a) $|A| |C_V(A)| \l |A^*| |C_V(A^*)|$, or 
(b)  $A^* = C_A([U,A]), C_V(A^*)= [U,A] C_V(A), |A| |C_V(A)| = |A^*| |C_V(A^*)|$.
\begin{quote}
\emph{Proof:}  
Assume ${\cal Q}_1$ does not hold.   $\forall B \le A$, so $|A| |C_V(A)| \ge |B| |C_V(B)|$.
Put $A^* = C_A([U,A])$.  $[U, A, A^*]=1$ and $A/C_A(V)$ is abelian so
$[A, A^*, U] =1  \rightarrow [U,A^* A] = 1$ so $[U, A^*] \le C_V(A)$.
\\
\\
\emph{Claim:} $|A| |C_V(A)| \le |A^*| |[U,A] C_V(A)|$.
\\
\\
Assuming claim, $|A^*| |C_V(A^*)| \le |A| |C_V(A)| \le |A^*| |[U,A] C_V(A)| \le |A^*| |C_V(A^*)|$
and we're done.
\\
\\
\emph{Proof of claim:}
Let $Y= C_V(A), X= [U,A]$.
We can assume $|U|= p, U= \langle u \rangle $ and $[U, A]= [u,a]$.
Let $\varphi: A/A^* \rightarrow (XY)/Y$ by $a A^* \mapsto [u,a]Y$ is well defined.
$\forall a^* \in A^*$.
$[u, a^*, a]= [u,a] [u, a^*]^a \in [u,a]Y$.
If $\varphi$ is injective, $|A/A^*| \le (XY)/Y|$ and the result follows.
Let $a_1, a_2 \in A$ such that $[u, a_1]Y= [u, a_2 ]Y$.
$[u, a_1] [u, a_2 ]^{-1} \in Y$ then $[u, a_1 a_2^{-1}] \in Y$ so
$[u, A, A_1 s_2^{-1}]= 1 $
and $a_1 a_2^{-1} \in C_A([U,A])=A^*$.  Thus $\varphi$ is injective.
Now assume $|U| > p$, $|U:U_1|= p, U= U_1 \langle u \rangle $.
Put $X_1= [U_1, A]$, $A_1 = C_A(X_1)$ and
$X_2= [U_2, A]$, $A_2 = C_A(X_2)$.
Note $X_1 X_2 C_V(A)= X C_V(A), A^* = A_1 \cap A_2$ and
$X_1 C_V(A) \cap X_2 C_V(A) \le C_V(A_1 A_2 )$.
By induction on $|U|$,
$|A| |C_V(A)| = |A_i | |X_i:C_V(A)|$.  Hence,
$|A| |C_V(A)| \ge
|A_1 A_2| |C_V(A_1 A_2 ) | \ge
{\frac {|A_1| |A_2| |X_1 C_V(A)| |X_2 C_V(A)|}
{|A_1 \cap A_2| |X_1 C_V(A)| |X_2 C_V(A)|} }
= {\frac {|A|^2 |C_V(A)|^2} {|A^*| |X C_V(A)|}}$ and this proves the claim.
\end{quote}
{\bf Theorem 11:} 
$A \in {\cal A}_V(G)$ and $A^* = C_A([V,A])$ then 
$|A/A^*| = |C_V(A^*)/C_V(A)|$ and $C_V(A^*)= [V,A] C_V(A)$.
\begin{quote}
\emph{Proof:}  
By the previous result, every quadratically acting subgroup, $A$, satisifes
${\cal Q}_2$ and the result follows from the second conclusion of that theorem, with $U=V$.
\end{quote}
{\bf Timmesfeld Replacement Theorem:}  
Let $A \in {\cal A}_V(G)$ and $U \le V$ then $C_A([U,A]) \in {\cal A}_V(G)$ and
$C_V(C_A([U,A]))= [U,A] C_V(A)$.  Moreover,
$[V,C_A(([U,A])] \ne 1$ if $[V,A] \ne 1$.
\begin{quote}
\emph{Proof:}  
Let  $A^* C_A([U,A])$.  Since $A \in {\cal A}_V(G)$, we can apply the previous
result so
$|A^*| |C_V(A^*)| = |A| |C_V(A)|$ and $C_V(A^*)= [U,A] C_V(A)$.
$\forall A_0 \le A, {\cal Q}_1$ gives
$|A_0| |C_V(A_0)| \le |A^*| |C_V(A^*)|$.
Hence 
$A^* \in {\cal A}_V(G)$, we may assume $[V, A^*] =1$ then $V= [U,A] C_V(A)= [V,A] C_V(A)$.
In particular, $[V,A,A]=[V,A]$ but then $[V,A] = 1$ since $A/C_A(V)$ is a $p$-group.
\end{quote}
{\bf Definition 5:} ${\cal A}_V(G)_{min}= \{ A \in {\cal A}(G), [V,A] \ne 1 \}$.
\\
\\
{\bf Theorem 12:}  Every element of ${\cal A}_V(G)_{min}$ acts quadratically and non-trivially on
$V$.
\begin{quote}
\emph{Proof:}  
Let $A \in 
{\cal A}_V(G)_{min}$.  By previous result, $A^*= C_A([V,A])$ is also in
${\cal A}_V(G)$ and $[V, A^*] \ne 1$.  Minimality forces $A=A^*$ and thus
$[V,A,A]=1$.
\end{quote}
{\bf Theorem 13:}  Suppose $G$ is $p$-stable on $V$ and $O_p(G/C_G(V))=1$ then every element of
${\cal A}_V(G)_{min}$ acts trivially on $V$.
\begin{quote}
\emph{Proof:}  
Follows from previous result.
\end{quote}
{\bf Theorem 14:}
Let $V= \langle C_V(S), S \in S_p(G) \rangle $ then $O_p(G/C_G(V))=1$.
\begin{quote}
\emph{Proof:}  
Let $S \in S_p(G)$, $Z= C_V(S), C= C_G(V)$.  $V= \langle Z^G \rangle $.  Let $C \le D \le G$:
$D/C \cong O_p(G/C)$ then $D \cap S \in S_p(D)$ and $D= C(D \cap S)$ and by
Frattini: $G=C N_G(D \cap S)$.  So $V= \langle Z^{N_G(D \cap S)} \rangle $, so $[V, D \cap S]= 1$
and $D=C$.
\end{quote}
{\bf Theorem 15:}
Let $C_G(O_p(G)) \le O_p(G)$ then $V= \langle \Omega({\mathbb Z}(S)), S \in S_p(G) \rangle $ 
is an elementary
abelian normal subgroup of $G$ and $O_p(G/C_G(V))=1$.
\begin{quote}
\emph{Proof:}  
Let $S \in S_p(G)$ then $\Omega ({\mathbb Z}(S)) \le C_G(O_p(G)) \le O_p(G) \le S$ so
$V$ is contained in $\Omega({\mathbb Z}(O_p(G)))$ and $\Omega({\mathbb Z}(S))= C_V(S)$.  
Now the result follows from the previous result.
\end{quote}
{\bf Definition 6:} Let ${\cal E}(G)$ be the set of elementary elementary subgroups of $G$ and
$m$ the the size of the element of ${\cal E}(G)$ of maximal order.
$J(G)= \langle A \in {\cal A}(G): |A|=m \rangle $.
\\
\\
{\bf Theorem 16:}
Let $A \in {\cal A}(G)$ act quadratically
on $V$ and $A_0 = [V,A] C_A([V,A])$ then $A_0$ is in ${\cal A}(G)$ and acts quadratically
on $V$ and if $[V,A] \ne 1$ then $[V, A_0 ] \ne 1$.
\begin{quote}
\emph{Proof:}  
Let $X= [V,A]$ and $A^*= C_A(X)$ with $A_0= A^*X$.  $A_0$ is elementary abelian
and $[V,A_0, A_0] \le [V, A, A_0 ] =1$.  It suffices to show $|A|= |A_0|$ to establish
$A_0 \in {\cal A}(G)$.  The maximality of $A$ gives $C_V(A)= V \cap A= V \cap A^*$ and since 
$X \cap A = X \cap A^*$, $|A| |A \cap V| = |A| |C_V(A)| = |A^*| |XC_V(A)|$.
and so
$|A| = {\frac {|A^*| |XC_V(A)|} {|C_V(A)|}} =
{\frac {|A^*|} {|X \cap C_V(A)|}} = {\frac {|A^*| |X|} {|X \cap A^*|}} = |A_0|$.
\end{quote}
{\bf Theorem 17:} 
(a) ${\cal A}(G) \subseteq {\cal A}_V(G)$ and (b) $V \nleq {\mathbb Z}(J(G))$ then
$\exists A \in {\cal A}(G): [V,A] \ne 1$.
\begin{quote}
\emph{Proof:}  
Let $ A^* \in {\cal A}(G)$ then $A^* C_V(A^*) \in {\cal E}(G)$ so
$|A| \ge |A^* C_V(A^*)|=
{\frac {|A^*| |C_V(A^*)|} {|A^* \cap V|}} \ge
{\frac {|A^*| |C_V(A^*)|} {|C_V(A)|}} $ and (a) follows.  (b) is clear.
\end{quote}
{\bf Condition ${\cal Q}_1'$:} $|A/C_A(V)| \ge |V/C_V(A)|$.
\\
\\
{\bf Theorem 18:} 
Let ${\cal B}$ be the set of subgroups $A \le G$ satisfying 
${\cal Q}_1'$ and
${\cal Q}_2$.  Let $A \in {\cal B}$ and suppose that $\forall A^* \le A, A^* \in {\cal B}$,
$|A^*/C_{A^*}(V)| |C_V(A^*)| \le |A/C_A(V)| |C_V(A)|$ then $A \in {\cal A}_V(G)$.
\begin{quote}
\emph{Proof:}  
We need to verify ${\cal Q}_1$ for $A$.  Let $A^* \le A$.  If $A^*$ does not satisfy
${\cal Q}_1'$ then $A^* \notin {\cal B}$ and
$|A^*/C_{A^*}(V)| |C_V(A^*)| <
|V| \le |A/C_A(V)| |C_V(A)| $.  So 
$|A/C_A(V)| |C_V(A)| \ge |A^*/C_{A^*}(V)| |C_V(A^*)|=
|(A^*/C_{A^*}(V))/C_A(V)| |C_V(A^*)|$.  The inequality also holds for $A^* \in {\cal B}$ since
the inequality holds.
Thus $\forall A^* \le A$: 
$|A^*| |C_V(A^*)| \le |A^* C_{A}(V)| |C_V(A^*)| \le |A| |C_V(A)|$ and
$A$ satisfies ${\cal Q}_1$.
Assume $\exists A \in {\cal B}$ that act non-trivially on $V$.
Among all such choose it with the property that $|A/C_A(V)| |C_V(A)|$ is
maximal. Then the inequality in the theorem holds for $A$.  Thus
$A \in {\cal A}_V(G)$ and ${\cal A}(G)_{min} \ne \emptyset$.  A previous result insures
the existence that act quadratically and non-trivially on $V$.
\end{quote}
{\bf Theorem 19:} If ${\cal K} \in  \{N, S, \Pi\}$ then 
$\forall G: O_{\cal K}(G)= \langle A: A \in {\cal K}, A \lhd \lhd G \rangle $.
\begin{quote}
\emph{Proof:}  
Tor $A \lhd G$, this is clear.  May assume $A$ is not normal in $G$ so $\exists N \lhd G$ with
$A \lhd \lhd N<G$.  By induction, 
$A \leq O_{\cal K}(N)$.  
$O_{\cal K}(N) \lhd G$ and  ${\cal K} \in \{N , S , \Pi \}$.  Hence $A \leq O_{\cal K}(N) \leq O_{\cal K}(G)$ which
proves the result.
\end{quote}
\section{Stability and $SL_2(p)$}
{\bf Hall's remark on Thompson:}  If $P \in S_p(K)$ and $K \ne XP$ for any $X \in O_{p'}(K)$, then
there is a characteristic subgroup $D$ of $P$ of nilpotence class at most $2$ such that
$N_K(D)/C_K(D)$ is not a $p$-group.
\\
\\
{\bf Definition 7:}  Let $G$ be a $p$-constrained group with $O_{p'}(G)=1$.
$G$ is $p$\emph{-stable} if $\forall H \lhd G, H \in p(G)$,
$[H,x,x]=1 \rightarrow {\overline x} \in O_p(G/C_G(H))$.
\\
\\
{\bf Theorem 20:}
A group is $p$-separable iff
$H<G$ has a non-trivial $\pi$-closed factor group or, equivalently
$G$ has a normal series $1= A_0 < A_1 < \ldots <A_n=G$ of characteristic
subgroups and $A_i/A_{i-1}$ is a $\pi$ or $\pi'$ group.
\begin{quote}
\emph{Proof:}  Sort of a definition.
\end{quote}
{\bf Theorem 21:}
If $p \ne 2$, $O_{p'}(G)=1$ and $G$ is $p$-constrained or $p$-solvable and $SL_2(p)$ is
not involved in $G$, then $G$ is $p$-stable.
\begin{quote}
\emph{Proof:}
Suppose $H \lhd G$ is a $p$-group.  $G/C_G(H)$ acts on $H/\Phi(H)$.  Let $K=ker(\varphi)$
where $\varphi: G/C_G(H) \rightarrow Aut(H/\Phi(H))$.
\\
\\
\emph{Claim:} $K$ is a $p$-group.
\\
\emph{Proof of claim:} It suffices to show that if
${\mathbb Z}_q < G/C_G(H)$ acts non-trivially on $H$, it acts nontrivially on
$H/\Phi(H)$.  If ${\mathbb Z}_q$ acts trivially on $H/\Phi(H)$.  For each coset,
$\Phi(H)x$, by counting, at least one element of the coset is fixed, say $x_i$.
By the Burnside Basis Theorem, $ \langle x_i \rangle_i = H$.  
Thus there is no $q$-element in $K$ and
$|K|=p^m$.
\\
\\
Put $L= G/C_G(H)/K$.  $L$ act faithfully on $H/\Phi(H)$.  Since $SL_2(p)$ is not involved,
$K$ is $p$ stable.  Now let $x \in G$ with $[H,x,x]=1$ and let the canonical map
$G/C_G(H) \rightarrow L$ be denoted by $\tilde {}$.  
$[H,x,x]=1 \rightarrow [H,x,x] \in \Phi(H) \rightarrow
(\tilde {\overline x}-1)^2=0$, so
$\tilde {\overline x} \in O_p(\tilde{G/C_G(H)})$ we have
$\tilde {\overline x} \in O_p(G/C_G(H))$.
\end{quote}
{\bf Theorem 21:}
Let $G$ be a group with no non-trivial normal $p$-subgroup, $p \ne 2$ which satisfies one of
the following:
(1) $G$ has odd order;
(2) $G$ has an abelian Sylow $2$-subgroup;
(3) $G$ has a dihedral Sylow $2$-subgroup;
(4) $G \cong PSL_2(q) = L_2(q)$;
(5) $G$ is solvable and $p \ge 5$ or $p=3$ and $SL_2(3)$ is not involved in $G$
then $G$ is $p$-stable.
\begin{quote}
\emph{Proof:}  
Generalization of earlier result.
\end{quote}
\section{The Thompson Subgroup}
{\bf Definition} Let ${\cal E}(G)$ be the elementary abelian subgroups of $G$ of maximal order.
$J(G)= \langle {\cal E}(G) \rangle)$.
\\
\\
{\bf $GL$ Lemma:} Let $G= GL_2(p)$, $p \neq 2$, $P \in S_p(G)$.  Suppose $L \in p'(G)$ and $P \subseteq N_G(L)$ and
if $S \in S_2(G)$, $S'=1$.  Then $P \subseteq C_G(L)$.
\begin{quote}
\emph{Proof:}   Sublemma: If $q$ is odd, $-I$ is the unique involution in $SL_2(q)$.  Let $P$ be
a $p$-group with at most one group of order $p$.  Either $P$ is cyclic or $p=2$ and
$P$ is generalized quaternion.
\\
By induction on $|L|$, we can assume $P$ centralizes every proper subgroup that it stabilizes.
Choose $q \mid |L:C_L(P)|$.  We can find a $P$ invariant Sylow $q$ subgroup $Q \subseteq L$.
Since $Q \neq C_L(P)$, $Q=L$.  So $L$ is a $q$-group.  $[L, P] \subseteq P$.
If $[L,P] < L$, $[L,P,P]=1$ and we're done.  So $[L,P]=P$, $L \subseteq G' \subseteq SL_2(p)$ since
$GL_2(p)/SL_2(p)$ is abelian.  If $q = 2$, $L$ is abelian and has a unique involution.
$L$ is a cyclic $2$-group and so is $Aut(L)$  $P$ cannot act non-trivially on $L$ so $q \neq 2$.
$|L| \mid (p-1)p(p+1)$ so $q \mid p-1$ or $q \mid p+1$. This $|L| \leq p+1$.  If $P$ acts non-trivially on $L$
then there must be a $p$-orbit of $L$ of size at least $p$. $|L|=p+1$ and $|L|$ is even but
$|L|$ is a power of $q$.  Contradiction.
\end{quote}
{\bf Normal P-Theorem:} Let $P \in S_p(G)$ and suppose (1) $G$ is $p$-solvable, (2) $p \ne 2$, (3) if
$R \in S_2(G)$, $R' = 1$, (4) $O_{p'}(G) = 1$, and (5) $P = C_G({\mathbb Z}(P))$ then $P \lhd G$.
\begin{quote}
\emph{Proof:} 
Suppose $G$ is a minimal counter-example. $\exists Q \in S_p(G), P \neq Q: \langle P, Q \rangle G$.
$Q= P^g$ and $C_V(Q) = C_V(P)^g$.  Put $U= C_V(Q) \cap C_V(P)$. $|V:U| \leq |V:C_V(P)| |V:C_V(Q)| = p^2$ and
$U \lhd V$.  $G$ acts trivially on $U$: $[U, G] = 1$.  $G$ acts on $V/U$; let $K$ be the kernel of this map.
$[V,U] \subseteq U$ and $[V, K, K] =1$ so $K$ is a $p$-group.  Note $K \subseteq O_p(G)$ so 
$K \subseteq P$ and $K \subseteq Q$. ${\overline G} = G/K$ has Sylow subgroups ${\overline P}, {\overline Q}$
and ${\overline G}$ acts faithfully on $V/U$ so $[{\overline P}, C_V(P)/U] = 1$.  ${\overline G}$ satisfies
all the hypothesis of the theorem so $K=1$ and $G$ acts faithfully on $V/U$.
Replace $V$ by $V/U$. $|V| \leq p^2$. $G \rightarrow Aut(V)$.  If $V$ is cyclic, $Aut(V)$ is abelian and
so is $G$ therefore $P \lhd G$.  So $V$ is elementary abelian and $Aut(V) = GL_2(p)$ and $O_p(G) = 1$ since
$|P| \leq p$ and $P$ is not normal.  Let $L = O_{p'}(G)$ and apply the previous lemma.  $[P, L]=1$ so by
Hall-Higman, $P \subseteq C_G(L) \subseteq L$ and $P=1$.
\end{quote}
{\bf Normal J-Theorem:} Let $P \in S_p(G)$ and suppose (1) $G$ is $p$-solvable, (2) $p \ne 2$, (3) if
$R \in S_2(G)$, $R' = 1$, (4) $G$ acts faithfully on some $p$-group, $V$ and (5) $|V:C_P(V)| \leq p$,
then $J(P) \lhd G$.
\begin{quote}
\emph{Proof:} 
Let $G$ be a minimal counterexample.  $U=O_{p}(G) > 1$, ${\overline G} = G/U$,
${\overline L} = O_{p'}({\overline G})$, where $U \subseteq L$. \\
\\
\emph{Step 1:} (a) ${\mathbb Z}(P) \subseteq U$, (b) $U \subseteq H \subseteq G$ implies $O_{p'}(H)=1$ and
(c) $C_{{\overline G}}({\overline L}) \subseteq {\overline L}$.\\
\emph{Proof:} 
Since $G$ is $p$-solvable and $O_{p'}(G)=1$,
by 1.2.3, $C_G(U) \subseteq O_p(G) = U$ but $U \subseteq P$ so
${\mathbb Z}(P) \subseteq C_G(U)$, proving (a).
Since $U = O_p(G)$ and $O_p({\overline G}) = 1$, so 
$C_{{\overline G}}({\overline L}) \subseteq {\overline L}$ by 1.2.3,
proving (c).  For (b), let
$U \subseteq H \subseteq G$, and put $M= O_{p'}(H)$.  $M, U \lhd H$. $M \cap U =1$, since
$U$ is a $p$-group and $M$ is a $p'$-group, so $M \subseteq C_G(U) \subseteq U$ and thus $O_{p'}(H)=1$
and $M= M \cap U =1$ proving (b).
\\
\\
\emph{Step 2:} $\exists A \in {\cal E}(P): A \not\subseteq U$.\\
\emph{Proof:} If not, all members of ${\cal E}(P)$ are contained in $U$ and so $J(P) \subseteq U$.
By the $GL$ Lemma, $J(U) = J(P)$ is characteristic in $U$ and since $U \lhd G$, $J(P) \lhd G$,
which contradicts the fact that $G$ is a counterexample.
\\
\\
\emph{Step 3:} Let $UA \subseteq H \subset G$ and $H \cap P \in S_p(H)$ then ${\overline A}$ centralizes
${\overline {H \cap L}}$.\\
\emph{Proof:}  $H$ satisfies the first four hypothesis of the theorem.  A Sylow $2$-group of $H$ is abelian so
it meets condition (3).  $O_{p'}(H) = 1$ by 1(b).
Put $S = H \cap P \in S_p(H)$.  Since ${\mathbb Z}(P) \subseteq U \subseteq S \subseteq P$ by 1(a).
${\mathbb Z}(P) \subseteq {\mathbb Z}(S)$ and thus $C_H({\mathbb Z}(S)) \subseteq C_G({\mathbb Z}(P)) = P$.
So $C_H({\mathbb Z}(S)) = S$ is a $p$-group of $H$ containing the Sylow $p$ group $S$.
So $C_H({\mathbb Z}(S)) = S$ and $H$ satisfies the fifth hypothesis.  Since $H < G$, the theorem holds
for $H$ and so $J(S) \lhd H$.  Since $A \in {\cal E}(P)$ and $A \subseteq S \subseteq P$ and $A \in {\cal E}(S)$ so
$A \subseteq J(S)$.  Thus $[H \cap L, A] \subseteq [H \cap L, J(S)] \subseteq (H \cap L) \cap J(S) = L \cap J(S) \subseteq U$
(Reason: $H \cap L$ and $J(S)$ are normal in $H$ and $U$ is the unique Sylow $p$-subgroup of $L$).
$1 = [{\overline {H \cap L}}, {\overline {A}}]$.
\\
\\
\emph{Step 4:} $G = LA$, $P=UA$.\\
\emph{Proof:}  $H=LA$ and $UA \in p(H)$.  Further,  $|H:UA|= |L(UA):UA| = |L:L \cap UA|$ which divides the $p'$-number
$|L:Y|$.  So, $UA \in S_p(H)$, $UA = H \cap P$.  
If $H < G$
then since $L \subseteq H$, step 3 gives 
${\overline A} \subseteq C_{\overline G}({\overline L}) \subseteq {\overline L}$ (by step 1(c)).  Since ${\overline A}$ is
a $p$-group and ${\overline L}$ is a $p'$-group, ${\overline A} = 1$ and $A \subseteq U$.  This contradicts the choice of $A$
so $H = G$.  Finally, $UA = H \cap P = G \cap P = P$.
\\
\\
\emph{Proof:} Put $H=LA$, $UA \in p(H)$.  $|H:UA| = |LUA:UA|= |L:L \cap UA|$ so $UA \in S_p(H)$ and
$UA = H \cap P$.
so $UA \in S_p(H)$.  If $H \subseteq G$, since $L \subseteq H$, step 3 gives
${\overline A} \subseteq C_{{\overline G}}({\overline L}) \subseteq {\overline L}$.  Since
$A$ is a $p$-group and ${\overline L}$ is a $p'$-group, ${\overline A} =1$ and $A \subseteq U$.
This contradicts the choice of $A$ so $H=G$.  Finally, $UA = H \cap P = G \cap P = P$.
\\
\\
\emph{Step 5:} $|{\overline A}| = p$.\\
\emph{Proof:}  
${\overline A} \ne 1$ since $A \not\subseteq U$.  ${\overline A}$ is elementary abelian, it STS ${\overline A}$ is cyclic.
${\overline A}$ acts coprimely on ${\overline L}$ and the action is faithful since $C_{\overline L}({\overline L}) \subseteq {\overline L}$
and ${\overline L} \cap {\overline A} = 1$.  A previous result shows ${\overline A}$ is cyclic so STS ${\overline A}$ acts trivially
on every ${\overline A}$-invariant proper subgroup of ${\overline L}$.
Suppose
${\overline M}$ is ${\overline A}$-invariant, ${\overline M} < {\overline L}$, we can assume
$U \subseteq M$. $A \subseteq N_G(M)$ so $MA$ is a group $P = UA \subseteq MA$.  Since $A$ is a $p$-group,
the $p'$ part of $|MA|$ is equal to the
$p'$ part of $|M|$ which is less than the $p'$-part of $|L|$ since
the $p'$ part of$|L:M| >1$.
It follows $MA < G$.  Apply step 3 to show ${\overline A}$ centralizes
${\overline {MA \cap L}} \supseteq M$, proving 5.
\\
\\
\emph{Step 6:}  Let $V= \{ z: z \in {\mathbb Z}(U) | z^p = 1\}$.  $V$ is an elementary abelian normal subgroup of $G$ so $G$ acts by
conjugation on $V$.  Since $V \subseteq {\mathbb Z} (U)$, the action by $U$ is trivial, so ${\overline G} = G/U$ on $V$.
Now we prove:
The action of ${\overline G}$ on $V$ is faithful. \\
\emph{Proof:}  
Let $K = C_G(V)$ so ${\overline K}$ is the kernel of the action of ${\overline G}$ on $V$.
We argue $K$ is a $p$-group.
Let $Q \in S_q(K)$, $q \ne p$.  $Q$ acts coprimely on ${\mathbb Z}(U)$, so
and $Q$ fixes all elements of order $p$ in ${\mathbb Z}(U)$, these make up $V$. 
$Q \subseteq K =C_G(V)$.   By Fitting, $Q$ acts trivially on ${\mathbb Z}(U)$ but
${\mathbb Z}(P) \subseteq U$ so ${\mathbb Z}(P) \subseteq {\mathbb Z}(U)$ and so $Q \subseteq C_G({\mathbb Z}(P)) = P$.
Thus $Q = 1$ and $K$ is a $p$-group, as claimed.   But $K \lhd G$.  So $K \subseteq O_p(G)=U$ and ${\overline K} = 1$ as needed.
\\
\\
\emph{Step 7:} $|V: V \cap A| \leq p$.\\
\emph{Proof:}  Put $D = U \cap A$ and $E = V \cap A$.  $|V:E| = |V:V \cap D| = |VD:D|$.  $D$ is an elementary abelian
subgroup of $U$ and $V$ is a central elementary abelian subgroup of $U$ and so $VD$ is elementary abelian.
Since ${\cal E}(P)$, $|VD| \leq |A|$ and so $|VD:D| \leq |A:D| = |{\overline A}| = p$.  We get $|V:E|= |VD:D| \leq p$
as required.
\\
\\
\emph{Step 8:} Contradiction. \\
\emph{Proof:}
We apply the Normal-P theorem to the action of ${\overline G}$ on $V$.
$|V: V \cap A| \leq p$. Now, $P=UA$ so ${\overline P} = {\overline A}$.
$[{\overline A}, V \cap A] = 1$ since $A' = 1$.  
So $|V:C_V({\overline P})| \leq | V: V \cap A| \leq p$.  Now we can apply the Normal P theorem,
${\overline P} \lhd {\overline G}$ so $P \lhd G$ and $A \subseteq P \subseteq O_p(G)=U$, which is not
the case.
\end{quote}
{\bf Theorem 22:} 
(1) $J(G) \; char \; G$ and $J(G) > 1$ if $p \in \pi(G)$;
(2) If $J(G) \le U \le G$ then $J(G) = J(U)$; 
(3) $J(G)= \langle J(S): S \in S_p(G) \rangle $;
(4) If $x \in C_G(J(G))$ and $|x|=p$, then $x \in {\mathbb Z}(J(G))$.
(5) If ${\cal B} \subseteq {\cal A}(G)$ then 
$J( \langle {\cal B} \rangle) =  \langle {\cal B} \rangle $.\\
\begin{quote}
\emph{Proof:}   This is straightforward.
\end{quote}
{\bf Definition 8:}
$G$ is Thompson factorizable
with respect
to $p$ if $G=O_{p'}(G) C_G( \Omega(Z(S))) N_G(J(S))$.  
Note that $G$ is Thompson factorizable iff $G/O_{p'}(G)$ is.
\\
\\
{\bf Thompson Factorization:} 
Let $O_{p'}(G)=1$ and $V= \langle \Omega({\mathbb Z}(S)): S \in S_p(G) \rangle $.
Then $G$ is Thompson factorizable iff $J(G) \le C_G(V)$.
\begin{quote}
\emph{Proof:}  
Let $S \in S_p(G)$ and $C= C_G(V)$.  Assume that $G$ is Thompson factorizable.
$\Omega({\mathbb Z}(S)) \le {\mathbb Z}(J(S))$ and so $V \le {\mathbb Z}(J(S))$ and
$C \le J(G)$.
Assume $C \le J(G)$.  The $J(G) \le C \cap S$.  Since $J(S) \; char \; C \cap S \in S_p(C)$
and
$\Omega({\mathbb Z}(S)) \le {\mathbb Z}(J(S))$, Frattini yields
$G= C N_G(C \cap S) =
C_G( \Omega(Z(S))) N_G(J(S))$.
\end{quote}
{\bf Alternate Thompson subgroup:} 
$P$ a $p$-group and set $d(P)= sup\{ |A| : A \le P, A'=1 \}$.  Let
${\cal A}(P)= \{ A: |A|=d(P), A'=1 \}$ then $J(P)= \langle A : A \in {\cal A}(P) \rangle $.
\\
\\
{\bf Lemma:}
(1) $A \in {\cal A}(P) \rightarrow A=C_P(A)$;
(2) $C_P(J(P))= {\mathbb Z}(J(P)) = \bigcap_{A \in {\cal A}(P)} A$;
(3) $H \le P$ and $d(H)=d(P) \rightarrow J(H) \le J(P)$ and
${\mathbb Z}(J(H)) \le {\mathbb Z}(J(P))$;
(4) Suppose $J(P) \subseteq H \subseteq P$ then $J(H)=J(P)$.
\begin{quote}
\emph{Proof:}
If $x$ centralizes $A$, $ \langle x, A \rangle $ is abelian but then 
$| \langle x,A \rangle | > |A|$ so $x \in A$; this proves
(1).  $C_P(J(P))= \bigcap_{A \in {\cal A}(P)} C_P(A)$ since $ \langle A \rangle =J(P)$.  Thus
$C_P(J(P)) = \bigcap_{A \in {\cal A}(P)} A \subseteq J(P)$ so
$C_P(J(P))= {\mathbb Z}(J(P))$; this proves (2).  (3) and (4) are clear.
\end{quote}
{\bf Theorem 23:}
Suppose $O_{p'}(G)=1$, $G$ is $p$-stable and $P \in S_p(G)$.  If $H \in {\cal SCN}(P)$ then
$H \subseteq O_p(G)$.
\begin{quote}
\emph{Proof:}
Put $L=O_p(G)$.  Since $L \le P$, $[L, H] \subseteq H$ and so $[L, H, H]=1$ so
$x \in H \rightarrow [L,x,x]=1$ and thus $H/C_G(L) \subseteq O_p(G/C_G(L)$.  Since
$L$ is $p$-constrained, $C_G(L) \subseteq L$ and thus $H \subseteq L$.
\end{quote}
\section{Glauberman's $Z(J)$ Theorem:}
{\bf Glauberman Replacement Lemma:}  If $p \ne 2$, $P$, a $p$-group, $B \lhd P$,
$A \subseteq P$, $A'=1$, $B \nsubseteq N_P(A)$ and $A \cap B \supseteq B'$ then
$\exists A^* \subseteq P, (A^*)'=1$ with
(1) $A^* \cap B > A \cap B$;
(2) $|A^*|= |A|$;
(3) $A^* \subseteq N_P(A) \rightarrow [A^*, A, A]=1$.
\begin{quote}
{\bf Outline of proof:} 
(1) Reduce to $P=AB$, $N_P(A) \lhd P$; 
(2) $x \in B-N \ne \emptyset, A \cap B \subseteq A \cap A^x$;
(3) $u=A A*$, $V=A \cap A^*, W=U \cap B$;
(4) $[x, A]$ is abelian;
(5) $VW$ is abelian;
(6) $VW$ works.
\\
\\
\emph{Proof:}
Proof is by induction $|P|$.  If $|P|>|AB|$, we are done by induction,
so $P=AB$.  Suppose $N=N_P(A)$, $\exists M$ such that $N \le M \lhd P$ since
the maximal subgroups of $P$ are normal.  Let $B_1= B \cap M$ then by
Dedekind, $A B_1= M$ and $B_i \nsubseteq N_M(A)$.  Applying induction,
we find an $A^*$ such that $A^*$ satisfies (1), (2) and (3) in $A B_1$.  This
$A^*$ works in $P=AB$.  So we may assume, $N \lhd P$.\\
\emph{Claim:} $x \in B-N \rightarrow A \cap B \subseteq A \cap A^x$ and
$A^x \le N$.  Proof of claim:
$A \cap B \supseteq B'$ $\rightarrow A \cap B \lhd B$ $ \rightarrow A \cap B= (A \cap B)^x$
$\rightarrow A \cap B \subseteq A^x \cap B \rightarrow A \cap B \subseteq A \cap A^x$.
Now, $A<N$ so $A^x < N^x=N$ by the above and this proves the claim.
\\
\\
Let $U=A A^x<N$, $V=A \cap A^x$, $W= U \cap B$.  $A^x \subseteq  N_P(A)$ so $U$ is a
group and $Z \lhd U$.\\
Claim:  $U' \subseteq V \subseteq {\mathbb Z}(U)$ and $W= [x,A](A \cap B)$.  Proof of claim:
$A^x \subseteq N, A \subseteq N$ so $A^xA \subseteq N$ and $A^xA, A^xA] \subseteq A \cap A^x$,
which proves the claim.\\
Continuing, $A \cap B \subseteq U \cap B$ and $[x,a]= (a^{-1})^xa \in U$ and 
$[x,a]= x^{-1} x^a \in B$ so $[x,A](a \cap B) \subseteq W$.    If $y=a_1 a_2^x \in W$,
$a_1 , a_2 \in A \rightarrow a_1 a_2 [a_2,x]= a_1 a_2^x \in B$ so
$a_1 a_2 \in B$ and $y \in [x,A](A \cap B)$.
\\
\\
\emph{Major claim:} $[x, A]$ is abelian.
\\
Subclaim: $U$ has nilpotence class $\le 2$ ($U' \subseteq {\mathbb Z}(U)$) implies
$[ac,b] [a,b][c,b]$.  This is a calculation.\\
Let $a, a_1 \in A$.  $[x,a,a_1]= [[x, [x,a]^{-1} [x,a], a_1^x]$ but since $[x,a] \in B$,
$[x, [x,a]^{-1}] \in B \subseteq A \cap B \subseteq {\mathbb Z}(U)$,
$[x,a,a_1]^x = [[x,a],a_1^x]$ by the subclaim.
Now, $[[x,a],a_1^x]= [a,a_1^x]= [x,a_1,a]$.
So $\forall a, a_1 \in A: [x,a,a_1]^x=[x,a_1,a] \rightarrow [x,a,a_1]^{x^2}=[x,a,a_1]$ 
(Equation **).  Since $x$ has odd order, this becomes $[[x,a],a_1^x]= [x,a,a_1][[x,a],[a_1,x]]$.
So $[[x,a],[a_1,x]]=1$; these two generators of $[x,A]$ commute so $[x,A]$ is abelian.
\\
\\
Claim: $VW$ is abelian: $[x,A] \subseteq U$ is abelian and $A \cap B \subseteq {\mathbb Z}(U)$,
so $W= [x,A] (A \cap B)$ is abelian.  Finally, $V \subseteq {\mathbb Z}(U)$, so $VW$ is
abelian.
\\
$A^*=VW$ satisfies the theorem:\\
(1) $[x,A] \nsubseteq A$ since $x \notin N_P(A)$.
$A \cap B < W <B$ so $A^* \cap B > A \cap B$.\\
(2) $|A^*|= 
{\frac {|V| |W|} {|V \cap W|}} =
{\frac {|A \cap A^x| |U \cap B|} {|U \cap B \cap A \cap A^x|}} =
{\frac {|A \cap A^x| |U \cap B|} {|A \cap B|}}$.
$|P|=|AB|=
{\frac {|A| |B|} {|A \cap B|}} =
{\frac {|U| |B|} {|U \cap B|}}$.  So
$
{\frac {|U|} {|A|}}=
{\frac {|U \cap B|} {|A \cap B|}}$ and $|A^*|= {\frac {|A \cap A^x| |U|} {|A|}}$.
$|A^*| = {\frac {|A \cap A^x||A A^x|} {|A|}}= {\frac {|A| |A^x|} {|A|}}= |A|$.\\
(3)
$A^* \subseteq U$ since $V, W \subseteq U$ and $U \subseteq N$ since $A, A^x \subseteq N$.
\end{quote}
{\bf Glauberman's $Z(J)$ Theorem:}  Assume $p \ne 2$ is prime and that $G$ is a
$p$-stable, $p$-constrained group with $O_{p'}(G)=1$ and $P \in S_p(G)$ then
${\mathbb Z}(J(P)) \lhd G$.
\begin{quote}
\emph{Proof:}
Let $Z= {\mathbb Z}(J(P))$ and $H= O_p(G)$.  $J(P) \; char ;\ P$ and $Z \; char \; P$ so
$Z \subseteq H$.  Let $H_0$ be a minimal normal $p$-subgroup of $G$ such that $Z \cap H_0$ is not
normal in $G$.  Put $Z_0= H_0 \cap Z$ and let $K/C_G(H_0)= O_p(G/C_G(H_0))$ so $K \lhd G$.
Also put $P_0=P \cap K \in S_p(K)$.
\\
\\
(1) $K= P_0 C_G(H_0)$. Reason: Since $K/C_G(H_0)$ is a $p$-group and $P_0C_G(H_0) \subseteq K$,
$K \ne P_0 C_G(H_0)$ implies that $|K/C_G(H_0)| \ge p |(P_0C_G(H_0))/C_G(H_0)= p {\frac {|P_0 |}
{|P_0 \cap C_G(H_0)|}}$. Thus $|P_0| \mid p \cdot |K|$ which contradicts the fact that
$P_0 \in S_p(K)$.
\\
(2)  By Frattini, $G= KN_G(P_0)$.
\\
(3) $J(P) \nsubseteq P_0$.  Reason: If not, $J(P) < J(P_0) \lhd N_G(P_0)$ so
$Z_0 \lhd G$, a contradicton.
\\
(4) $H_0' \subseteq Z_0$. Proof: Since $H_0$ is a $p$-group $H_0' < H_0$ so by
maximality, $Z \cap H_0' \lhd G$ and $H_0 = \langle Z_0^g \rangle $.  
$[Z_0, H_0] \subseteq Z_0 \cap H_0' \lhd G$,
so $[Z_0, H_0]^g = [H_0, H_0] \subseteq Z_0 H_0'$ and $H_0' \subseteq Z_0 \cap H_0'$.
\\
(5) $H_0 \nsubseteq N_G(A)$ otherwise, $[H_0,A] \subseteq A$ and
$[H,A,A]=1$ which implies $A \subseteq K$ by $p$-stability.
\\
(6)  $H_0 \subseteq {\mathbb Z}(J(P_0))$.
\\
\\
Conclusion:
By (1) and (2), $G= N_G(P_0) C_G(H_0)$ so $Z_0^g=Z_0^n, n \in N_G(P_0)$ and
$H_0= \langle Z_0^g \rangle \subseteq {\mathbb Z}(J(P_0)), \forall g \in G$.  Therefore
$H_0 \subseteq A^*$ and $A^* \subseteq N_G(A)$ so $H_0 \subseteq N_G(A)$ which is
a contradiction.
\end{quote}
\section{Alperin-Lyons Proof of Baer}
{\bf Theorem 24:}  Suppose $x$ is a $p$-element of $G$.  
$x \in O_p(G)$ iff $ \langle x, x^g \rangle $ is a $\forall g \in G$.
\begin{quote}
\emph{Proof:}
Let $K= \langle x^G \rangle $ and suppose $z, y \in K \rightarrow \langle z, y \rangle $ 
is a $p$-group.  Suppose $K \nsubseteq O_p(G)$.
\\
\\
(1) $\exists P, Q \in S_p(G): P \cap K \ne Q \cap K$.  
Reason: If $K \subset R, \forall R \in S_p(G)$ then $K \subseteq  \bigcap R^g \lhd G$ so
$K \subseteq O_p(G)$.
\\
\\
(2) $|P \cap K| = |Q \cap K|$.  Reason: $Q= P^g$ so $Q \cap K= P^g \cap K^g= (P \cap K)^g$
so $|Q \cap K|= |P \cap K|$.
\\
\\
(3) $K \cap P \nsubseteq Q$ and $K \cap Q \nsubseteq P$ for $P, Q$ chosen such that
$P \cap K \ne Q \cap K$.  If $K \cap P \subseteq Q$, $K \cap P \subseteq K \cap Q$ if
since the cardinalities are the same, $K \cap P = K \cap Q$.
\\
\\
(4) We can choose $P$ and $Q$ such that $|K \cap P \cap Q|$ is maximal with respect to
$K \cap P \ne K \cap Q$.  Put $W=P_0< P_1< \ldots < P_n=p$ with $|P_i:P_{i-1}|=p$.
Now $P \cap K \nsubseteq W$ (otherwise $ \langle P \cap K \rangle \subset P \cap Q$ so 
$P \cap K \subset Q$
which is a contradiction).
Let $j$ be the smallest $j$ such that $P_j \cap K \nsubseteq W \cap K$ ($ j \ge 1$).  Pick
$x \in K\cap P_j-W$.
\\
\\
(5) Claim: $x \in N(W)$.  Proof of claim: $x$ normalizes $P_{j-1}$ so $P_{j-1} \cap K$
is normalized by $x$.  By the choice of $j$, $P_{j-1} \cap K \subseteq W \cap K$ so
$P_{j-1} \cap K = W \cap K$.  
Thus $W= \langle P_{j-1} \cap K \rangle = P_{j-1} \cap K \rangle^x= W^x$ which
proves the claim.\\
For the same reason, $\exists y \notin W: y \in K, y \in N(W)$ and 
$ \langle x, y \rangle $ is a $p$-group.
Thus if $R \in S_p(G)$ with $ \langle x, y \rangle W  \subset R$,
$K \cap P \cap R \supseteq K \cap W \cap \langle x \rangle $ and 
$|K \cap P \cap R| > |K \cap P \cap Q|$
and symmetrically,
$|K \cap Q \cap R| > |K \cap P \cap Q|$ thus $P \cap K = R \cap K= K \cap Q$ concluding the proof.
\end{quote}
\section{Goldschmidt's Proof of Burnside's Theorem for $p \ne 2$}
{\bf Burnside's Theorem:}  If $|G|= p^a q^b$ for $p \ne 2 \ne q$ then $G$ is solvable.
\begin{quote}
\emph{Proof:}
Let $G$ be a minimal counterexample, $r \in \{ p , q \}$.
\\
\\
\emph {Lemma 1:}  If $R \in S_r(G)$ then if $1 \ne S \in r'(G)$, $R \nsubseteq N_G(S)$.\\
Proof: Let $Q \in S_{r'}(G)$ with $S \subseteq Q$.  Since $G=RQ$, $\forall g, Q^g=Q^r$ for some
$r \in R$.  Now suppose $R \subseteq N_G(S)$ then $S \subseteq Q^r$ and thus
$1 \ne \bigcap_{r \in R} Q^r = \bigcap_{g \in G} Q^g \lhd G$ which is impossible since $G$
is simple.
\\
\\
\emph {Lemma 2:}  If $M$ is a maximal subgroup of $G$ then $F(M)$ is an $r$-group.\\
Proof:
From now on, let $F=F(M)= F_p \times F_q$, and $Z={\mathbb Z}(F) = Z_p \times Z_q$.
Observe that if $M$ is maximal, $M$ is solvable and so
$O_{p'}(N_M(P)) \subseteq O_{p'}(M)$.
\\
\\
\emph{Claim 1:}  $F$ is not cyclic.\\
Assume, by way of contradiction, that $F$ is cyclic.
Suppose $q > p$ and $Q \in S_q(M)$.  Since $Q$ acts on $F_p$ and $q \nmid Aut(F_p)$, $[Q, F_p]=1$.
$Q$ also acts on $F_q$ and
$Q/C_Q(F) \rightarrow Aut(F_q)$.
$C_M(F) \subseteq F$, since $M$ is solvable so $C_M(F) \subseteq F$ and $C_Q(F) \subseteq F_q$.  Clearly,
$F_q \subseteq C_Q(F)$ since $F_q$ is cyclic. So $Q/F_q \subseteq Aut(F_q)$.  Further, $F_q \thinspace char \thinspace Q$ and
$N_G(Q) \subseteq N_G(F_q)$.  $N_G(Q) \subseteq N_G(F_q) = M $.  Examining $N_M(F_q)/C_M(F_q) \subseteq Aut(F_q)$, we see $|N_M(F_q)|_q = |Q|$ so
$|N_M(F_q)|_q = |N_G(F_q)|_q$.
Thus, $Q \in S_q(G)$, contradicting Lemma 1.
\\
\\
\emph{Claim 2:}  $M$ is the unique maximal subgroup containing $Z$.
\\
Suppose $Z \subseteq M_1 \ne M$. $M_1$ a maximal subgroup of $G$.
$M= N_G(Z_p)= N_G(Z_q)$, $Z_p \subseteq O_{q'}(N_{M_1}(Z_q))$ and
$Z_q \subseteq O_{p'}(N_{M_1}(Z_p))$.
Because $M$, is solvable, $O_{p'}(N_M(F)) \subseteq O_{p'}(M)$,
$Z_p \subseteq F(M_1)_p \subseteq C(Z_q) \subseteq M$ and
$Z_q \subseteq F(M_1)_q \subseteq C(Z_p) \subseteq M$ so
$F(M_1) \subseteq F(M)$.  Since
the argument also applies to $M_1$, $F(M) \subseteq F(M_1)$. So
$F(M)= F(M_1)$.  This $M= N_G(F(M)) = M_1$, proving the claim.
\\
\\
Now, since $F$ is not cyclic, there is an Abelian subgroup
$V \subseteq F$ of type $(r,r)$.  $\forall x \in V^{\#}$, $Z \subseteq C(x)$
and by the uniqueness of $M$, $C(x) \subseteq M$.  Let $V \subseteq R \in S_r(M)$.
If $Q_0 \in r'(G)$  with $R \subseteq N(Q_0)$.  $Q_0 = \prod_{x \in V^{\#}} C_{Q_0}(x) \subseteq M$.
It follows that $F_{r'}$ is the unique maximal $r'$-subgroup of $G$ normalized by $R$, so
$N(R) \subseteq N(F_{r'}) = M$ so $R \in S_r(G)$, contradicting Lemma 1.  This proves Lemma 2.
\\
\\
\emph {Lemma 3:}  If $R \in S_r(G)$ then $R$ is contained in a unique maximal subgroup of
$G$.  Every maximal subgroup, $M \subseteq G$, contains a Sylow subgroup of $G$.
\\
\emph{Proof:} $R \subseteq M$ so $O_{p'}(M)=1$ by lemma 1.  Since $M$ is
$r$-constrained, $O_{r'}(M)=1$ and $M$ is solvable, $M$ is $r$-constrained.  Once
we know $M$ is $r$-stable, by Glauberman's $Z(J)$ theorem, we have: $M=N_G({\mathbb Z}(J(R)))$.
$p^aq^b$ is odd and $|SL_2(r)|= (r^2-1)r$ which is even, so $G$ is $p$-stable and $M=N_G({\mathbb Z}(J(R)))$.
So $M$ is unique.
Let $M_1$ be any maximal subgroup of $G$, we can choose $r$, such that $O_{r'}(M_1)=1$.
If $R_1 \in S_r(M)$, we have
$M_1 =N_G({\mathbb Z}(J(R_1)))$.
Since ${\mathbb Z}(J(R_1)) \; char \; R_1$,
$N_G(R_1) \subseteq M_1$, so $R_1 \in S_r(G)$ ,proving the second statement.
\\
\\
\emph {Lemma 4:}  If $R \in S_r(G)$ then ${\mathbb Z}(R)$ is contained in a 
unique maximal subgroup of $G$.
\\
\emph{Proof:}  Suppose ${\mathbb Z}(R) \subseteq M \cap M_1, M \ne M_1$ with $M, M_1$ maximal in $G$.
We may assume $M_1$ is chosen such that $|M \cap M_1|_r$ is maximal.  Now, let
$R_1 \in S_r(M \cap M_1)$ such that ${\mathbb Z}(R) \subseteq R_1$.  By the maximality of $|M \cap M_1|_r$,
$N_G(R_1) \subseteq M$ and $R_1 \in S_r(M_1)$
Conjugating $R$ by something in $M$, we may assume, by lemma 1,
$R_1 \subset R$ and $M_1$ contains a $r'$ sylow subgroup of $G$.
Thus $G=RM_1$ and
$1 \ne {\mathbb Z}(R) \subseteq \bigcap_{r \in R} M_1^r= \bigcap_{g \in G} M_1^g \lhd G$,
a contradiction.
\\
\\
\emph {Lemma 5:}  $\exists R_1, R_2 \in S_r(G)$ such that $R_1 \cap R_2 = 1$.
\\
\emph{Proof:}  Let $R_1 \in S_r(G)$ and let $M$ be a unique maximal subgroup containing
$R_1$.  Pick $R_2 \nsubseteq M$.  We can do this since $G$ is simple.\\
\\
Claim: $R_2 \cap M=1$.
\\
If not, choose $R_2$ such that $|R_2 \cap M|$ is as large as possible.  Put
$R_0 = (R_2 \cap M)$.
We can assume, possibly after conjugation in $M$, $R_0 \subseteq R_1$.  ${\mathbb Z}(R_1) \subseteq N(R_0)$.
By lemma 4, $N(R_0) \subseteq M$.  $R_0 < R_2$ so $N_{R_2}(R_0) > R_0$ and $|R_2 \cap M| > |R_0|$,
which is a contradiction.  So $R_2 \cap M=1$.
\\
\\
Conclusion: WLOG, assume $q^b > p^a$ and let $Q_1, Q_2 \in S_q(G)$ with $Q_1 \cap Q_2=1$.
Then $|G| \ge |Q_1| \cdot |Q_2| > |G|$ which is just plain wrong.
\end{quote}
\section{Thompson Complements}
{\bf Thompson Normal $p-$Complement:}
Let $p \ne 2$ and $P \in S_p(G)$.  Assume $N_G(J(P))$
and $C_G(\Omega_1({\mathbb Z}(P)))$ have a normal $p-$complement then so does $G$.
\begin{quote}
\emph{Proof:}  
Let $G$ be a counterexample of minimum order.
\\
\\
\emph{Step 1:}  Let ${\cal H}= \{ H \in p(G): N(H)$ does not have a normal $p$-complement $\}$.
If $H, K \in {\cal H}$.  We say $H \le K$ if one of the following holds:
(1) $|N(H)|_p < |N(K)|_p$;
(2) $|N(H)|_p = |N(K)|_p$ and $|H| < |K|$; or,
(3) $|H| = |K|$.  Choose $H$ minimal with respect to the ordering and set
$N=N(H)$.  Let $Q \in S_p(N)$ and $H \subseteq Q \subseteq P$.
\\
\\
\emph{Step 2:} $H \ne P$
\\
If $H=P$, $N \subseteq N(J(P))$ which is a contradiction.
\\
\\
\emph{Step 3:} ${\overline N}= N/H$ has a normal $p$-complement.
\\
Let ${\overline Q}= P/H$ and ${\overline Q} \in S_p({\overline N})$.  Suppose
${\overline N}$ does not have a
normal $p$-complement.
Since $|{\overline N}| < |{\overline G}|$, either 
$C_{\overline N}({\mathbb Z}({\overline P}))$ or
$N_{\overline N}(J({\overline Q}))$ does not have a 
normal $p$-complement.  Let $K$ be the inverse image of the one that fails to have a
normal $p$-complement.  $K \in p(G)$ and $N(K)$ has no
normal $p$-complement.  Since $Q \subseteq N(K)$, either
$|N(K)|_p > |N(H)|_p$ or
$|N(K)|_p = |N(H)|_p$ and $|H| < |K|$.  Hence, $K \ge \ge K$.  But $K \ne H$ and this
contradicts maximality of $H$ in ${\cal H}$.
\\
\\
\emph{Step 4:} $N=G$.
\\
$N$ satisfies the hypothesis of the theorem $H \subseteq Q \subseteq P$ so
${\mathbb Z}(P) \subseteq  N(H) =N$.
$Q {\mathbb Z}(P) \subseteq N \cap P$ so $Q {\mathbb Z}(Q) = Q$ and
${\mathbb Z}(P) \subseteq Q$ thus $
{\mathbb Z}(P) \subseteq
{\mathbb Z}(Q)$ and $C_N({\mathbb Z}(Q)) \subseteq 
C_G({\mathbb Z}(Q)) \subseteq C_G({\mathbb Z}(P))$.
If $P= Q$, $N(J(Q)) \subseteq N(J(P))$ has a
normal $p$-complement.
Suppose $Q < P$  so $N_P(Q) > Q$ and $|N_G(J(Q))|_p > |N|_p = |N_G(H)|_p$.
By the maximality of $H$ in ${\cal H}$, $J(Q) \notin {\cal H}$ and so
$N(J(Q))$ has a
normal $p$-complement and
so does $N_N(J(Q))$.  If $ N \ne G$, then by the minimality of $|G|$, $N$ has a
normal $p$-complement and so $N=G$.
\\
\\
\emph{Step 5:}  $O_{p'}(G)=1$.
\\
Let $J= O_{p'}(G)$ and ${\overline X} = X/L$.  If $K \subset P$,
$ {\overline {N(K)}} = N({\overline K})$,
${\overline {J(P)}} = J({\overline P})$,
${\overline {C(P)}} = C({\overline P})$, and
${\overline {{\mathbb Z}(P)}} = {\mathbb Z}({\overline P})$.  So 
$N({\mathbb Z}(P))$ has a 
normal $p$-complement.  If $|{\overline G}| < |G|$, ${\overline G}$ and the inverse image is
a normal $p$-complement of $G$.
\\
\\
\emph{Step 6:}  $H= O_p(G)$ and $G$ is $p$-solvable of $p$-length at most $2$.
\\
Set $K= O_p(G), K \subseteq H$.  Since $G$ has no
a normal $p$-complement,  $N(K)=G$ and $K \in {\cal H}$, by steps 3, 4,
$G/H$ has a
normal $p$-complement so $G$ has $p$-length $\le 2$.
\\
\\
\emph{Step 7:}  If ${\overline G}= G/H= {\overline P}{\overline M}$ where ${\overline M}$
is a normal $p$-complement
and ${\overline M}$ contains no $P$-invariant subgroup.
\\
Suppose ${\overline M} > {\overline M}_0 >1$ and
${\overline M}_0$ is $P$-invariant.  Let $M_0$ be the inverse image and set
$G_0= P M_0$.  $G_0 < G$ so $G_0$ has a
normal $p$-complement 
and $K_0 \lhd G_0, K_0 \cap P = 1$.  $[H, K_0] \subseteq K_0 \cap H =1$ so
$K_0 \subseteq C(H)$.  But then $K_0 \subseteq {\mathbb Z}(O_p(G))$ by Hall-Higman which is
a contradiction.
\\
\\
\emph{Step 8:} ${\overline M} = {\overline R}$ is an elementary abelian $r$-group for some
$r \ne p$ and $P$ acts irreducibly on ${\overline R}$. Hence, $P$ is maximal in $G$.
\\
Let $r \mid |{\overline M}|$.  $P$ permutes Sylow $r$-subgroups and by Sylow, the conjugacy
class has size $1$.   So ${\overline R} \in S_r({\overline M})$ is $P$-invariant and since
${\overline R}$ has no non-trivial characteristic subgroups, it is elementary abelian.
Since ${\overline M}$ has no proper ${\overline P}$ invariant subgroups, ${\overline P}$
acts irreducibly.
If $G > L >P$, $1 \le {\overline L} \cap {\overline R} < {\overline R}$
and ${\overline L} < {\overline R}$ is a proper ${\overline P}$-invariant subgroup of
${\overline R}$ which is a contradiction.
\\
\\
\emph{Step 9:} $\exists 1 \ne A \in P$ with $A$ abelian such that $m_p(A)= d(P)$ and
$A \nsubseteq H$.
\\
Let $A$ be a fixed one of minimal order.  If $A_0= A \cap H$,
$A/A_0$ is elementary abelian.  If $J(P) \subseteq H$ then $J(P) \; char \; H$ and
$J(P) \lhd G$ which is a contradiction.  So $J(P) \nsubseteq H$ and such an $A$ exists.
Choose $A$ as mentioned and set $A_0= A \cap H$,  $A_1= \Omega_1(A/A_0)$, $A_1 \nsubseteq H$
and $m(A)= m(A_1)$ so $A= A_1$ and $A/A_0$ is elementary abelian.
\\
\\
\emph{Step 10:}  Let ${\overline A}= (AH)/H$ then $
{\overline G} = {\overline A} {\overline Q} $ and
$|{\overline A} |=p$.
\\
${\overline G}= {\overline P} {\overline R}$ and
${\overline P}$ normalizes
${\overline R}$.  The action is faithful since $H=O_p(G)$ and the kernel would be a normal
$p$-subgroup of $G$.  Note ${\overline A}= (AH)/H \cong A/A_0 \ne 1$.
${\overline A}$ acts nontrivially on
${\overline R}$ by the previous result an we can find
$ {\overline R}_1 \subseteq {\overline R}$ on which ${\overline A}$ acts non-trivially and
irreducibly.  Let $G_1$ be the inverse image of
$ {\overline A} {\overline R} $ and $P_1 \in S_p(G_1)$.   $A \subseteq P_1 \subseteq P$.
Since $H \subseteq P_1$ and
$C_G(H) \subseteq H$, 
${\mathbb Z}(P) \subseteq C_G(H) \subseteq H \subseteq P$, and
${\mathbb Z}(P) \subseteq {\mathbb Z}(P_1)$ and the latter has a 
normal $p$-complement.  Since
$A \subseteq P_1$ and $m(A)= d(P)$, $d(P_1)=  d(P)$ and
$A \subseteq J(P_1 )$.  Let $R_1 \in S_r(N_G(J(P_1 )))$, then
$[A, R_1 ]= [J(P_1 ), Q_2] \subseteq J(P_1 )$ so
$[A, R_1]$ is a $p$ group.  Since
$ {\overline R}_2 \subseteq {\overline R}_1 $,
$ [A, {\overline R}_2] \subseteq {\overline R}_1 $ is an $r$-group
and so
$ [A, {\overline R}_2] =1$.   Thus, $R_2$ is an ${\overline A}$-invariant
subgroup of ${\overline R}_1$ centralized by
${\overline A}$ and ${\overline R}_2 =1$ and $R_2=1$.  So $N_{G_1}(J(P_1 ))$ is a 
$p$-group and has a
normal $p$-complement. If $G_1 < G$ then $G_1$ has a 
normal $p$-complement and would centralize $H$.   Contradiction.  Hence,
$G = G_1$ and
$ {\overline G} = {\overline A} {\overline R} $
and
$ {\overline A} $ acts faithfully and irreducibly on
${\overline R}$.
$ {\overline A} $ is elementary abelian and hence cyclic so $|A|= p$.
\\
\\
\emph{Step 11:}  Set $W= {\mathbb Z}(H), Z= \Omega_1(W)$.  If $R \in S_r(G)$ then
$[R, Z] \subseteq Z$ but $[R, Z] \ne 1$.
\\
${\mathbb Z}(P) \subseteq C_G(H) \subseteq H$ so ${\mathbb Z}(P) \subseteq W$.
If $[R, W] = 1$, $[R, {\mathbb Z}(P)] = 1$, hence
${\mathbb Z}(P)$ would be central in $G$ which is impossible.
\\
\\
\emph{Step 12:} Contradiction
\\
$Z \lhd G$ and $G$ acts by conjugation so the kernel of the action is $C(Z) \supset H$
and so ${\overline G}$ acts on $Z$.
Since ${\overline R}$ is the unique minimal normal subgroup of ${\overline G}$ and
${\overline G}$ acts nontrivially,
${\overline G}$ acts irreducibly on $Z$ by step 11.
By a previous result, $Z= C_Z({\overline R}) + V$ where $V$ is
${\overline R}$-invariant.
Moreover,
$ {\overline R} \lhd {\overline G} $ and so $V$ is
$ {\overline G} $ invariant and 
$ {\overline G} $ acts faithfully on $V$.
Since $|A|=p$, $m(A_0) \ge d(P) -1$.  Set $V_0= V \cap A_0$ and let $t= m(V_0 )$ and
$r= m(A/A_0)$.  $V \subseteq {\mathbb Z}(H)$ so $ \langle V, A_0 \rangle $ is abelian.
Since $V$ is elementary, $d(P) \ge m( \langle V, A_0 \rangle) = m(V)+m(A_0)- m(V \cap A_0)= t+r +
m(A_0)-t= d(P)-1+r$.  Hence, $r=0$ or $r=1$ and $V_0=V$ or $V_0$ is maximal.
Choose $a \in A \setminus A_0$, ${\overline b} \in {\overline R}^H$
such that
$[a, {\overline b}] \ne 1$,
$[{\overline a}, V_0] = 1$ and
$[{\overline a}^{\overline b}, V_0^{\overline b}] = 1$ then
$[\langle {\overline a}, {\overline a}^{\overline b} \rangle , V_0 \cap V_0^{\overline b}] = 1$ 
and $V_0 \cap V_0^{\overline b} \le p^2$.  Since $A= \langle a \rangle $ is maximal in $Z$ and
${\overline a}^{\overline b} \notin {\overline A}$, 
${\overline G}_1, V_0 \cap V_0^{\overline b} = 1$ hence $|V| \le p^2$.
So ${\overline G} \subseteq GL_2(p)$ and since
${\overline G}$ is generated by $2$ elements of order $p$, ${\overline G} \subseteq SL_2(p)$.
So we have an abelian $p'$ group, ${\overline R}$, which is normalized but not centralized by
${\overline A} \subseteq SL_2(p)$ of order $p$ and this is impossible. 
\end{quote}
{\bf Theorem (Frobenius):}  If $G$ is solvable and $|G| > 1$, 
$\exists p, P \in p(G): 1 \ne P \lhd G$.
\begin{quote}
\emph{Proof:}  
The proof is by induction.  It is clearly true for all $p$-groups.   Since $G$ is solvable,
$exists N \lhd G$ and we can assume $|G:N|= p \mid |G|$.  By induction, $exists q:  Q \in q(N), Q \lhd N$.
So $1 \ne O_q(N) \thinspace char \thinspace N \lhd G$.
\end{quote}
{\bf Theorem 25:}
If $P \in S_p(G)$, $G$ contains a normal $p$-complement iff
whenever two elements in $P$ are $G$-conjugate, they are $P$ conjugate.
\begin{quote}
\emph{Proof:}  
See the section on transfer.
\end{quote}
{\bf Theorem 26:}
If $G$ has a maximal subgroup, $M$, which is nilpotent of odd order, then $G$ is solvable.
\begin{quote}
\emph{Proof:}  
See Passman.
\end{quote}
{\bf Theorem 27:}
Let $p \ne 2$, $P \in S_p(G)$ and suppose for any $H<G: H \; char ;\ P$, $N_G(H)/C_G(H)$ is
a $p$-group, then $G$ has a normal $p$ complement.
\begin{quote}
\emph{Proof:}  
By induction on $|G|$.  $N_G(H)$ has a normal $p$-complement by induction.
If $a \in p'(H)$ then $[a,H]=1$.  So $H$ has  normal $p$-complement.  
The result follows from Thompson's
normal $p$-complement theorem.
\end{quote}
{\bf Theorem 28:}
Let $p \ne 2$, $G= SL_2(p)$.  The only abelian $p'$-subgroups of $G$ which are normalized
by an $S_p$ subgroup of $G$ lie in ${\mathbb Z}(G)$.
\begin{quote}
\emph{Proof:}  
See earlier.
\end{quote}
