\chapter{Representations}
\section {${\mathbb C}G$-modules}
{\bf Definition:} A \emph{representation} of $G$ is a group
homomorphism $\rho: G \rightarrow GL_n({\mathbb C})$, 
$n$ is called the \emph{degree} of $\rho$.
The representation $\rho: G \rightarrow GL_n({\mathbb C})$ is
called \emph{faithful} if $ker(\rho) =\{1\}$.
Thus $\rho$ is faithful if the only element $g\in G$ with $\rho(g)=I_n$ is the
identity of $G$. 
\\
\\
{\bf Theorem 1:} A representation $\rho$ of $G$ is faithful if
and only if $im(\rho) \cong G$.
\begin{quote}
\emph{Proof:}
We have $G/ker(\rho) \cong im(\rho)$ so if $ker(\rho)=\{1\}$, the
isomorphism theorem gives 
$G \cong im(\rho)$.
Conversely if $G \cong im(\rho)$ then these two finite groups
have the same order, and so $|ker(\rho)|=1$.
\end{quote}
{\bf Definition:} A finite-dimensional vector space $V$ over ${\mathbb C}$ is a
\emph{${\mathbb C}G$-module} if a multiplication $gv$ (for all $g\in G$ and $v\in V$) is
defined, satisfying the following conditions for all
$u,v\in V$, $\lambda \in {\mathbb C}$ and $g \in G$:
\begin{itemize}
\item[(i)] $gv\in V$;
\item[(ii)] $(gh)v=g(hv)$;
\item[(iii)] $1v=v$;
\item[(iv)] $g(\lambda v)=\lambda(gv)$;
\item[(v)] $g(u+v)=gu+gv$.
\end{itemize}
{\bf Definition:}
If $V$ is a ${\mathbb C}G$-module, $g$ induces a linear map
$\rho(g): V \rightarrow V$;
if ${\cal B}$ is a basis of $V$, the matrix for the matrix representing the linear
map $\rho(g)$ is denoted by $[g]_{\cal B}$.
\\
\\
{\bf Theorem 2:} Let $G$ be a finite group.
\begin{itemize}
\item[(i)] Given a representation
$\rho: G \rightarrow GL_n({\mathbb C})$, the vector space $V= {\mathbb C}^n$ is a
${\mathbb C}G$-module
if we define the multiplication by $gv=\rho(g)v$ for all $g\in G$ and
$v \in V$. Moreover, there is a basis ${\cal B}$ of $V$ such that
$\rho(g)=[g]_{\cal B}$ for all $g\in G$.
\item[(ii)] Given a ${\mathbb C}G$-module $V$ of dimension $n>0$ with basis ${\cal B}$, the
function $\rho: G \rightarrow GL_n({\mathbb C})$ defined by $\rho(g)=[g]_{\cal B}$ is a
representation of $G$.
\end{itemize}
\begin{quote}
\emph{Proof:}
\\
\\
(i) For all $u,v\in {\mathbb C}^n$, $\lambda\in{\mathbb C}$ and
$g,h\in G$ we have
\begin{eqnarray*}
\rho(g)v\!\!\!\!&\in&\!\!\!\!{\mathbb C}^n,\\
\rho(gh)v\!\!\!\!&=&\!\!\!\!\rho(g)\rho(h)v,\\
\rho(1)v\!\!\!\!&=&\!\!\!\!v,\\
\rho(g)(\lambda v)\!\!\!\!&=&\!\!\!\!\lambda\rho(g)v,\\
\rho(g)(u+v)\!\!\!\!&=&\!\!\!\!\rho(g)u+\rho(g)v.
\end{eqnarray*}
Defining $gv=\rho(g)v$ for all $g\in G$ and $v\in {\mathbb C}^n$ makes
${\mathbb C}^n$ into a ${\mathbb C}G$-module. Moreover, if we let ${\cal B}$ be the basis
$$\begin{matrix}1\\0\\0\\\vdots\\0\end{matrix},\qquad
\begin{matrix}0\\1\\0\\\vdots\\0\end{matrix},\qquad\cdots,\qquad
\begin{matrix}0\\0\\0\\\vdots\\1\end{matrix}$$
of ${\mathbb C}^n$, then we have $\rho(g)=[g]_{\cal B}$ for all $g\in G$.
\\
\\
(ii) Let $V$ be a ${\mathbb C}G$-module with basis ${\cal B}$. Since
$(gh)v=g(hv)$ for all $g,h\in G$ and all $v$ in the basis ${\cal B}$, it follows that
$[gh]_{\cal B}=[g]_{\cal B}[h]_{\cal B}$. In particular, 
$[1]_{\cal B}=[g]_{\cal B}[g^{-1}]_{\cal B}$ for all
$g\in G$. Now we have $1v=v$ for all $v\in V$, so $[1]_{\cal B}$ is the identity
matrix $I_n$; thus each matrix $[g]_{\mathbb B}$ is invertible (with inverse
$[g^{-1}]_{\cal B}$). Hence the function $g\mapsto [g]_{\cal B}$ is a homomorphism from
$G$ to $GL_n({\mathbb C})$, i.e., a representation of $G$.
\end{quote}
{\bf Theorem 3:} Let $V$ be a ${\mathbb C}G$-module with basis ${\cal B}$, and
let $\rho$ be the representation of $G$ defined by $\rho(g)=[g]_{\cal B}$ for all
$g\in G$. Then
\begin{itemize}
\item[(i)] if ${\cal B}'$ is a basis of $V$, the representation $\phi$ of
$G$ defined by $\phi(g)=[g]_{{\cal B}'}$ for all $g\in G$ is equivalent to $\rho$;
\item[(ii)] if $\sigma$ is a representation of $G$ which is equivalent to
$\rho$, there is a basis ${\cal B}''$ of $V$ such that $\sigma(g)=[g]_{{\cal B}''}$ for all
$g\in G$.
\end{itemize}
\begin{quote}
\emph{Proof:}
\\
\\
(i) Let $T$ be the change of basis matrix from ${\cal B}$ to ${\cal B}'$; then
$$\phi(g)=[g]_{{\cal B}'}=T^{-1}[g]_{\cal B} T=T^{-1}\rho(g)T\qquad\hbox{for all }g\in G,$$
and so $\phi$ is equivalent to $\rho$.
\\
\\
(ii) If $\sigma$ is equivalent to $\rho$, then there is an
invertible matrix $T$ such that we have $\sigma(g)=T^{-1}\rho(g)T$ for all
$g\in G$. If we let ${\cal B}''$ be the basis of $V$ such that the change of basis
matrix from ${\cal B}$ to ${\cal B}''$ is $T$, then
$$\sigma(g)=T^{-1}\rho(g)T=T^{-1}[g]_{\cal B} T=[g]_{{\cal B}''}
\qquad\hbox{for all }g\in G$$
as required.
\end{quote}
{\bf Lemma:} If $V$ and $W$ are ${\mathbb C}G$-modules, then
$V\cong W$ if and only if there exist bases ${\cal B}_1$ of $V$ and ${\cal B}_2$ of
$W$ such that $[g]_{{\cal B}_1}=[g]_{{\cal B}_2}$ for all $g\in G$.
\begin{quote}
\emph{Proof:}
Suppose that $\theta: V\rightarrow W$ is a ${\mathbb C}G$-isomorphism, and let
$v_1,\dots,v_n$ be a basis ${\cal B}_1$ of $V$; then
$\theta(v_1),\dots,\theta(v_n)$ is a basis ${\cal B}_2$ of $W$. Given
$g\in G$, write $gv_i=\sum a_{ji}v_j$; then
$$g\theta(v_i)=\theta(gv_i)=\theta\biggl(\sum a_{ji}v_j\biggr)
=\sum a_{ji}\theta(v_j),$$
and so $[g]_{{\cal B}_2}=(a_{ij})=[g]_{{\cal B}_1}$. Conversely, if $v_1,\dots,v_n$ and
$w_1,\dots,w_n$ are bases ${\cal B}_1$ of $V$ and ${\cal B}_2$ of $W$ such that
$[g]_{{\cal B}_1}=[g]_{{\cal B}_2}$ for all $g\in G$, let $\theta:V\rightarrow W$ be the
invertible linear map given by $\theta(v_i)=w_i$ for all $i$. Given
$g\in G$, let $[g]_{{\cal B}_1}=(a_{ij})$, so that $gv_i=\sum a_{ji}v_j$; then
because $[g]_{{\cal B}_2}=(a_{ij})$ as well, for all $i$  we have
$$g\theta(v_i)=gw_i=\sum a_{ji}w_j=\sum a_{ji}\theta(v_j)
=\theta\biggl(\sum a_{ji}v_j\biggr)=\theta(gv_i),$$
and so $\theta$ is a ${\mathbb C}G$-isomorphism.
\end{quote}
{\bf Theorem 4:} Let $V$ and $W$ be ${\mathbb C}G$-modules with bases
${\cal B}$ and ${\cal B}'$; let $\rho$ and $\sigma$ be the representations of $G$ given by
$\rho(g)=[g]_{\cal B}$ and $\sigma(g)=[g]_{{\cal B}'}$ for all $g\in G$. Then $V\cong W$ if
and only if $\rho$ and $\sigma$ are equivalent.
\begin{quote}
\emph{Proof:}
Assume that $V$ and $W$ are isomorphic ${\mathbb C}G$-modules. By the Lemma,
there exist bases ${\cal B}_1$ of $V$ and ${\cal B}_2$ of $W$ such that
$[g]_{{\cal B}_1}=[g]_{{\cal B}_2}$ for all $g\in G$. Define a representation $\tau$ of
$G$ by $\tau(g)=[g]_{{\cal B}_1}$; then by the Theorem (i), $\tau$ is equivalent to
both $\rho$ and $\sigma$, and so $\rho$ and $\sigma$ are equivalent. Conversely,
if $\rho$ and $\sigma$ are equivalent, by the Theorem (ii) there is a basis
${\cal B}''$ of $V$ such that $\sigma(g)=[g]_{{\cal B}''}$ for all $g\in G$, i.e.,
$[g]_{{\cal B}'}=[g]_{{\cal B}''}$ for all $g\in G$; thus $V\cong W$ by the Lemma.
\end{quote}
\section {${\mathbb C}G$-submodules}
{\bf Definition:} Let $V$ be a ${\mathbb C}G$-module. A subset $U$ of
$V$ is a \emph{${\mathbb C}G$-submodule} of $V$ if $U$ is a subspace of the vector
space $V$ which satisfies $gu\in U$ for all $g\in G$ and $u\in U$.
Thus a ${\mathbb C}G$-submodule of the ${\mathbb C}G$-module $V$ is a subspace $U$ of $V$ such
that $U$ is itself a ${\mathbb C}G$-module.
\\
\\
{\bf Definition:}
A non-zero ${\mathbb C}G$-module $V$ is called
\emph{irreducible} if it has no ${\mathbb C}G$-submodules apart from
$\{0\}$ and $V$, and
\emph{reducible} otherwise. A representation $\rho:G\rightarrow GL_n({\mathbb C})$ is
called \emph{irreducible} if the corresponding ${\mathbb C}G$-module ${\mathbb C}^n$, given by
$gv=\rho(g)v$ for $g\in G$ and $v \in {\mathbb C}^n$, is \emph{irreducible}, and is
\emph{reducible} otherwise.
\\
\\
{\bf Theorem 5:} If $U$ is a subspace of the complex vector
space $V$ with complex inner product $(\ ,\ )$, then $U^\perp$ is a subspace of
$V$, and $U\oplus U^\perp=V$.
\begin{quote}
\emph{Proof:}
If $v,w \in U^\perp$ and $\lambda \in {\mathbb C}$, then for all $u \in U$ we have
$(v,u)=(w,u)=0$, so
$$(v+w,u)=(v,u)+(w,u)=0+0=0,\qquad (\lambda v,u)=\lambda(v,u)=\lambda.0=0;$$
thus $v+w,\lambda v\in U^\perp$, and so $U^\perp$ is a subspace of $V$.
\\
\\
If $u \in U \cap U^\perp$, then $(u,u)=0$, so 
$u=0$ -- thus the sum $U+U^\perp$ is direct. To show that $U+U^\perp=V$,
take a basis $v_1,\dots,v_r$ of $U$, and extend to a basis
$v_1,\dots,v_r,v_{r+1},\dots,v_n$ of $V$. Apply the Gram-Schmidt process to
obtain an orthonormal basis $e_1,\dots,e_r,e_{r+1},\dots,e_n$ of $V$; then
$e_1,\dots,e_r$ form an orthonormal basis of $U$, and so
$e_{r+1},\dots,e_n\in U^\perp$ -- given $v\in V$ we may write
$$v=\sum_{i=1}^n\lambda_ie_i
=\sum_{i=1}^r\lambda_ie_i+\sum_{i=r+1}^n\lambda_ie_i\in U+U^\perp.$$
\end{quote}
{\bf Maschke's Theorem:}  If $V$ is a ${\mathbb C}G$-module and
$U$ is a ${\mathbb C}G$-submodule, there is a 
${\mathbb C}G$-submodule, $W$, of $V$ such that $V= U \oplus W$.
\begin{quote}
\emph{Proof:}  
Let $\langle\ ,\ \rangle$ be the inner product on $V$ 
and take the orthogonal complement $U^\perp$ with respect to it then
$V=U\oplus U^\perp$.
Given $g\in G$ and $v\in U^\perp$, for all $u\in U$ we have $g^{-1}u\in U$
as $U$ is a ${\mathbb C}G$-submodule, so that
$$\langle gv,u\rangle=\langle gv,g(g^{-1}u)\rangle=\langle v,g^{-1}u\rangle=0;$$
thus $gv\in U^\perp$. This shows that $U^\perp$ is a ${\mathbb C}G$-submodule of $V$; so
taking $W=U^\perp$ gives the ${\mathbb C}G$-module direct sum $V=U\oplus W$ as
required.
\\
\\
\emph{Alternate Proof:} $V= U \oplus W_0$ where $W_0$ is some
not necessarily ${\mathbb C}G$-invariant subspace.
Let $\pi_U$ be the projection operator onto $U$ so $\pi_U(u + w)= u$.
Put $\varphi(x)= {\frac 1 {|G|}} \sum_{g \in G} g^{-1} \pi_U g (x)$.
$U = \varphi(U)$ and $\varphi$ is a ${\mathbb C}G$-invariant projection.
$V= im( \varphi ) \oplus ker( \varphi )$.  But $im(\varphi)= U$ so
$V= U \oplus W$ where $W= ker( \varphi )$.
\end{quote}
{\bf Definition:} A ${\mathbb C}G$-module $V$ is called
\emph{completely reducible} if $V=U_1\oplus\cdots\oplus U_r$ where
each $U_i$ is an irreducible ${\mathbb C}G$-submodule of $V$.
\\
\\
{\bf Theorem 6:} If $V$ is a non-zero ${\mathbb C}G$-module, then $V$ is
completely reducible.
\begin{quote}
\emph{Proof:}
We use induction on $n= dim(V)$: the result is clear if $n=1$, as then
$V$ is itself irreducible, so assume $n>1$. The result holds if $V$ is
irreducible, so assume $V$ is reducible; then $V$ has a ${\mathbb C}G$-submodule
$U \neq \{0\},V$.  Theorem a previous result, $V$ has a
${\mathbb C}G$-submodule $W$ with $V=U\oplus W$; as $dim(U),dim(W)<n$, by
induction we have
$$U= U_1 \oplus \cdots \oplus U_r,\qquad W= W_1 \oplus \cdots \oplus W_s,$$
where each $U_i$ and $W_j$ is an irreducible ${\mathbb C}G$-module. It follows that
$$V= U_1 \oplus \cdots \oplus U_r \oplus W_1 \oplus \cdots \oplus W_s$$
as required.
\end{quote}
{\bf Schur's Theorem:}  
If $V, W$ are irreducible ${\mathbb C}G$-modules and $\theta: V \rightarrow W$ is a 
${\mathbb C}G$ homomorphism then either $\theta=0$ or $\theta$ is an isomorphism.
\begin{quote}
\emph{Proof:}  
(i) Suppose there exists $v\in V$ with $\theta(v)\neq0$. Then
$im(\theta) \neq \{0\}$; because $im(\theta)$ is a ${\mathbb C}G$-submodule of $W$, which
is irreducible, we must have $im(\theta)=W$. Also $ker(\theta) \neq V$; as
$ker(\theta)$ is a ${\mathbb C}G$-submodule of $V$, which is irreducible, we must have
$ker(\theta)=\{0\}$. Thus $\theta$ is invertible, and so it is a
${\mathbb C}G$-isomorphism.
\\
\\
(ii) Because $V$ is a vector space over ${\mathbb C}$, the endomorphism
$\theta$ has an eigenvalue $\lambda\in{\mathbb C}$, and so
$ker(\theta-\lambda1_V) \neq \{0\}$. Since $\theta-\lambda1_V$ is clearly a
${\mathbb C}G$-homomorphism, $ker(\theta-\lambda1_V)$ is a ${\mathbb C}G$-submodule of
$V$; because $V$ is irreducible, we must have $ker(\theta-\lambda1_V)=V$. Thus
$(\theta-\lambda1_V)(v)=0$ for all $v\in V$, and so $\theta-\lambda1_V=0$,
i.e., $\theta=\lambda1_V$ as required.
\end{quote}
{\bf Theorem 7:} If $V$ is a non-zero ${\mathbb C}G$-module such that
every ${\mathbb C}G$-homomorphism from $V$ to $V$ is a scalar multiple of $1_V$, then
$V$ is irreducible.
\begin{quote}
\emph{Proof:}
Suppose that $V$ is reducible; then it has a ${\mathbb C}G$-submodule $U$ not equal
to $\{0\}$ or $V$. There is a ${\mathbb C}G$-submodule $W$ of $V$ with
$V=U\oplus W$. The map $\pi:V\rightarrow V$ defined by
$\pi(u+w)=u$ for all $u\in U$ and $w\in W$ is a ${\mathbb C}G$-homomorphism; since it is
not a scalar multiple of $1_V$, this is a contradiction. Thus $V$ must be
irreducible.
\end{quote}
{\bf Corollary:} Let $\rho: G\rightarrow GL_n({\mathbb C})$ be a
representation of $G$; then $\rho$ is irreducible if and only if every
$n\times n$ matrix $A$ with complex entries which satisfies
$$A\rho(g)=\rho(g)A\qquad\hbox{for all }g\in G$$
has the form $A=\lambda I_n$ with $\lambda\in{\mathbb C}$.
\begin{quote}
\emph{Proof:}
Regard ${\mathbb C}^n$ as a ${\mathbb C}G$-module by defining
$gv=\rho(g)v$ for all $g\in G$ and $v\in{\mathbb C}^n$. Let $A$ be an $n\times n$ matrix
over ${\mathbb C}$. The endomorphism $v\mapsto Av$ of 
${\mathbb C}^n$ is a ${\mathbb C}G$-homomorphism
if and only if
$$A(gv)=g(Av)\qquad\hbox{for all }g\in G,\ v\in{\mathbb C}^n;$$
i.e., if and only if
$$A\rho(g)=\rho(g)A\qquad\hbox{for all }g\in G.$$
The result now follows from the Lemma and an earlier theorem.
\end{quote}
{\bf Theorem 8:} If $V, W$ are ${\mathbb C}G$-module and
$\varphi: V \rightarrow W$ is a ${\mathbb C}G$-homomorphism
then $ker(\varphi)$ and $im(\varphi)$ are 
${\mathbb C}G$-modules. 
\begin{quote}
\emph{Proof:}  
Straightforward.
\end{quote}
{\bf Theorem 9:} If $V$ is a non-zero ${\mathbb C}G$-module 
and $\theta: V \rightarrow W$ is a ${\mathbb C}G$ module
homomorphism, $\exists U$, 
a ${\mathbb C}G$-submodule of $V$ such that $V= ker(\theta) \oplus U$.
$Hom_{{\mathbb C}G}(V,W)$ is a vector space over ${\mathbb C}$.  
\begin{quote}
\emph{Proof:}  
$ker(\theta)$ is a ${\mathbb C}G$-module, so by 
Maschke, there is a ${\mathbb C}G$-module, $U$ such that $V= ker(\theta) \oplus U$.
The map ${\overline  \theta}(u) = \theta(u)$ is a ${\mathbb C}G$-isomorphism from $U$ into $im(\theta)$.
If $u \in ker({\overline \theta})$, $u \in ker(\theta) \cap U$.  If $v \in V$ and $\theta(v)= w$, $v = k + u$,
$k \in ker(\theta), u \in U$.  $w= \theta(v)= \theta(k) + {\overline \theta}(u) = \theta(u)$ and $Im(\theta)= Im({\overline \theta})$.
$U \cong Im(\theta)$.
\end{quote}
\section {${\mathbb C}G$-homomorphisms}
{\bf Definition:}
Let $V$ and $W$ be ${\mathbb C}G$ modules.  $\theta$ is a ${\mathbb C}G$ homomorphism if $\theta$ is a ${\mathbb C}$ linear map $\theta: V \rightarrow W$
with $\theta(gv) = g \theta(v), \forall v \in V$.  The set (group) of ${\mathbb C}G$ homomorphisms 
from $V$ into $W$ is denoted $Hom_{{\mathbb C}G}$.
\\
\\
{\bf Theorem 10:} If $V$ and $W$ are ${\mathbb C}G$-modules,
$\theta,\phi: V \rightarrow W$ are ${\mathbb C}G$-homomorphisms and $\lambda \in {\mathbb C}$, then
$\theta+\phi,\lambda\theta: V \rightarrow W$ are also ${\mathbb C}G$-homomorphisms.
\begin{quote}
\emph{Proof:}
We know that $\theta+\phi$ and $\lambda\theta$ are linear maps; and for
all $g\in G$ and $v\in V$ we have
\begin{eqnarray*}
(\theta+\phi)(gv)\!\!\!\!&=&\!\!\!\!\theta(gv)+\phi(gv)=g\theta(v)+g\phi(v)
=g(\theta(v)+\phi(v))=g((\theta+\phi)(v)),\\
(\lambda\theta)(gv)\!\!\!\!&=&\!\!\!\!\lambda.\theta(gv)
=\lambda.g\theta(v)=g(\lambda.\theta(v))=g(\lambda\theta)(v)
\end{eqnarray*}
as required.
\end{quote}
{\bf Definition:} If $V$ and $W$ are ${\mathbb C}G$-modules, the vector space
of all ${\mathbb C}G$-homomorphisms $\theta:V\rightarrow W$ is written
$Hom_{{\mathbb C}G}({\mathbb C}G(V,W))$.
\\
\\
{\bf Theorem 11:} If $V$ and $W$ are irreducible
${\mathbb C}G$-modules, then
$dim(Hom_{{\mathbb C}G}(V,W)) = 1$, if $V \cong W$, and
$dim(Hom_{{\mathbb C}G}(V,W)) = 0$, if $V \ncong W$.
\begin{quote}
\emph{Proof:}
If $V \not \cong W$ this is immediate from Schur's Lemma. If $V \cong W$,
let $\theta: V \rightarrow W$ be a ${\mathbb C}G$-isomorphism; if $\phi\in Hom_{{\mathbb C}G}(V,W)$,
then $\theta^{-1}\phi\in Hom_{{\mathbb C}G}(V,V)$. There exists
$\lambda \in {\mathbb C}$ with $\theta^{-1}\phi=\lambda1_V$; thus $\phi=\lambda\theta$,
and so we have $Hom_{{\mathbb C}G}(V,W)=\{\lambda\theta: \lambda \in {\mathbb C}\}$,
which is a $1$-dimensional vector space.
\end{quote}
{\bf Theorem 12:} Let $V$, $V_1$, $V_2$, $W$, $W_1$ and
$W_2$ be ${\mathbb C}G$-modules: then
\begin{itemize}
\item[(i)] $dim(Hom_{{\mathbb C}G}(V,W_1\oplus W_2))
=dim(Hom_{{\mathbb C}G}(V,W_1))+dim(Hom_{{\mathbb C}G}(V,W_2))$;
\item[(ii)] $dim(Hom_{{\mathbb C}G}(V_1\oplus V_2,W))
=dim(Hom_{{\mathbb C}G}(V_1,W))+dim(Hom_{{\mathbb C}G}(V_2,W))$.
\end{itemize}
\begin{quote}
\emph{Proof:}
\\
\\
(i) Define the maps $\pi_1:W_1\oplus W_2\rightarrow W_1$ and
$\pi_2:W_1\oplus W_2\rightarrow W_2$ by
$$\pi_1(w_1+w_2)=w_1,\qquad\pi_2(w_1+w_2)=w_2\qquad\hbox{for all }w_1\in W_1,\
w_2\in W_2;$$
then we see that $\pi_1$ and $\pi_2$ are
${\mathbb C}G$-homomorphisms. Given
$\theta\in Hom_{{\mathbb C}G}(V,W_1\oplus W_2)$, it is easy to see that
$\pi_i\theta\in Hom_{{\mathbb C}G}(V,W_i)$ for $i=1,2$. We may
thus define a map $f$ from the vector space
$Hom_{{\mathbb C}G}(V,W_1\oplus W_2)$ to the vector space direct sum
$Hom_{{\mathbb C}G}(V,W_1)\oplus Hom_{{\mathbb C}G}(V,W_2)$ by
$$f(\theta)=(\pi_1\theta,\pi_2\theta)\qquad\hbox{for all }\theta\in
 Hom_{{\mathbb C}G}(V,W_1\oplus W_2).$$
It is clear that $f$ is a linear map; we shall show that it is bijective.
\\
\\
Given $\phi_i\in Hom_{{\mathbb C}G}(V,W_i)$ for $i=1,2$, the map
$\phi: V\rightarrow W_1\oplus W_2$ defined by
$\phi(v)=\phi_1(v)+\phi_2(v)$ for all $v\in V$ lies in
$Hom_{{\mathbb C}G}(V,W_1\oplus W_2)$, and satisfies
$f(\phi)=(\phi_1,\phi_2)$; thus $f$ is surjective. If $f(\theta)=0$, then
$\pi_1\theta=0$ and $\pi_2\theta=0$, so for all $v\in V$ we have
$\theta(v)=\pi_1\theta(v)+\pi_2\theta(v)=0$; thus $\theta=0$, and so
$f$ is injective. Thus $f$ is an invertible linear map from
$Hom_{{\mathbb C}G}(V,W_1\oplus W_2)$ to the vector space direct sum
$Hom_{{\mathbb C}G}(V,W_1)\oplus Hom_{{\mathbb C}G}(V,W_2)$, and so these two spaces have the same
dimension as required.
\\
\\
(ii) Given $\theta\in Hom_{{\mathbb C}G}(V_1\oplus V_2,W)$, define
$\theta_i: V_i \rightarrow W$ for $i=1,2$ to be the restriction of
$\theta$ to $V_i$, i.e., the map defined by $\theta_i(v)= \theta(v)$ for all
$v \in V_i$; then $\theta_i \in Hom_{{\mathbb C}G}(V_i,W)$ for $i=1,2$. Now define a
map $h$ from the vector space $Hom_{{\mathbb C}G}(V_1\oplus V_2,W)$ to the vector space
$Hom_{{\mathbb C}G}(V_1,W) \oplus Hom_{{\mathbb C}G}(V_2,W)$ by
$$h(\theta)=(\theta_1,\theta_2)\qquad\hbox{for all }\theta\in
Hom_{{\mathbb C}G}(V_1\oplus V_2,W).$$
Clearly $h$ is a linear map, and is injective. Given
$\phi_i\in Hom_{{\mathbb C}G}(V_i,W)$ for $i=1,2$, the map 
$\phi: V_1 \oplus V_2 \rightarrow W$ defined by
$\phi(v_1+v_2)= \phi_1(v_1)+\phi_2(v_2)$ for all
$v_1\in V_1$, $v_2\in V_2$ lies in $Hom_{{\mathbb C}G}(V_1\oplus V_2,W)$, and
$h(\phi)=(\phi_1,\phi_2)$; thus $h$ is surjective. Thus $h$ is a
bijective linear map from $Hom_{{\mathbb C}G}(V_1\oplus V_2,W)$ to the vector space
direct sum $Hom_{{\mathbb C}G}(V_1,W)\oplus Hom_{{\mathbb C}G}(V_2,W)$, and so these two spaces
have the same dimension as required.
\end{quote}
{\bf Corollary:} If $V_1,\dots,V_r,W_1,\dots W_s$ are
${\mathbb C}G$-modules, then
$$dim(Hom_{{\mathbb C}G}(V_1\oplus\cdots\oplus V_r,W_1\oplus\cdots\oplus W_s))
=\sum_{i=1}^r\sum_{j=1}^s dim(Hom_{{\mathbb C}G}(V_i,W_j)).$$
\begin{quote}
\emph{Proof:}
For convenience write $W=W_1\oplus\cdots\oplus W_s$; then by induction,
we have
\begin{eqnarray*}
dim(Hom_{{\mathbb C}G}(V_1\oplus\cdots\oplus V_r,W))
\!\!\!\!&=&\!\!\!\!\sum_{i=1}^r dim(Hom_{{\mathbb C}G}(V_i,W))\\
\!\!\!\!&=&\!\!\!\!\sum_{i=1}^r dim(Hom_{{\mathbb C}G}(V_i,W_1\oplus\cdots\oplus W_s))\\
\!\!\!\!&=&\!\!\!\!\sum_{i=1}^r\sum_{j=1}^s dim(Hom_{{\mathbb C}G}(V_i,W_j))
\end{eqnarray*}
as required.
\end{quote}
{\bf Corollary:} Let $V$ be a ${\mathbb C}G$-module with
$V=U_1\oplus\cdots\oplus U_r$ where each $U_i$ is irreducible, and $W$ be any
irreducible ${\mathbb C}G$-module; then both $ dim(Hom_{{\mathbb C}G}(V,W))$ and
$ dim(Hom_{{\mathbb C}G}(W,V))$ are equal to the number of terms $U_i$ with $U_i\cong W$.
\begin{quote}
\emph{Proof:}
By the Corollary, we have
\begin{eqnarray*}
 dim(Hom_{{\mathbb C}G}(V,W))\!\!\!\!&=&\!\!\!\!\sum_{i=1}^r dim(Hom_{{\mathbb C}G}(U_i,W)),\\
 dim(Hom_{{\mathbb C}G}(W,V))\!\!\!\!&=&\!\!\!\!\sum_{i=1}^r dim(Hom_{{\mathbb C}G}(W,U_i));
\end{eqnarray*}
we have
$$ dim(Hom_{{\mathbb C}G}(U_i,W))= dim(Hom_{{\mathbb C}G}(W,U_i))=
1,  \textnormal { if } U_i \cong W;
0  \textnormal{ if } U_i \ncong W.
$$
The result follows.
\end{quote}
{\bf Theorem 13:} Let $V$ be a ${\mathbb C}G$-module, and write
$V=U_1\oplus\cdots\oplus U_r$ where each $U_i$ is an irreducible
${\mathbb C}G$-submodule of $V$. If $U$ is any irreducible ${\mathbb C}G$-submodule of $V$, then
$U \cong U_i$ for some $i$.
\begin{quote}
\emph{Proof:}
Given $u\in U$ we may write $u=u_1+\cdots+u_r$ for unique vectors
$u_i\in U_i$. Define $\pi_i:U\rightarrow U_i$ by $\pi_i(u)=u_i$ for
$1\leq i\leq r$; then each $\pi_i$ is a
${\mathbb C}G$-homomorphism. If we choose $i$ such that $u_i\neq0$ for some
$u\in U$, we have $\pi_i\neq0$. $\pi_i$ is a
${\mathbb C}G$-isomorphism, and so $U\cong U_i$.
\end{quote}
\section {Decomposition of ${\mathbb C}G$}
{\bf Definition:} If $V$ is a ${\mathbb C}G$-module and $U$ is an irreducible
${\mathbb C}G$-module, we say that $U$ is a \emph{composition factor} of $V$ if $V$ has
a ${\mathbb C}G$-submodule which is isomorphic to $U$.
\\
\\
{\bf Definition:} Two ${\mathbb C}G$-modules $V$ and $W$ are said to have a
\emph{common composition factor} if there is an irreducible ${\mathbb C}G$-module which
is a composition factor of both $V$ and $W$.
\\
\\
{\bf Theorem 14:} The ${\mathbb C}G$-modules $V$ and $W$ have a common
composition factor if and only if $ Hom_{{\mathbb C}G}(V,W)\neq\{0\}$.
\begin{quote}
\emph{Proof:}
Write $V=V_1\oplus\cdots\oplus V_r$, $W=W_1\oplus\cdots\oplus W_s$ with
each $V_i$ and $W_j$ irreducible. By the corollary,
$\dim( Hom_{{\mathbb C}G}(V,W))=\sum\sum\dim( Hom_{{\mathbb C}G}(V_i,W_j))$, and, 
$\dim( Hom_{{\mathbb C}G}(V_i,W_j))$ is $1$ if $V_i\cong W_j$, and $0$ if
$V_i \not\cong W_j$. Thus $ Hom_{{\mathbb C}G}(V,W) \neq \{0\}$ if and only if some $V_i$ is
isomorphic to some $W_j$.
\end{quote}
{\bf Definition:} The vector space ${\mathbb C}G$, with multiplication defined
by
$$\biggl(\sum_{g\in G}\lambda_gg\biggr)\biggl(\sum_{h\in G}\mu_hh\biggr)
=\sum_{g,h\in G}\lambda_g\mu_h(gh)\qquad\hbox{for all }\lambda_g,\mu_h\in{\mathbb C},$$
is called the  \emph{group algebra} of $G$.
\\
\\
{\bf Theorem 15:} For all $r,s,t\in{\mathbb C}G$ and $\lambda\in{\mathbb C}$, we
have the following:
\begin{itemize}
\item[(i)] $rs\in{\mathbb C}G$;
\item[(ii)] $r(st)=(rs)t$;
\item[(iii)] $r1=1r=r$;
\item[(iv)] $(\lambda r)s=\lambda(rs)=r(\lambda s)$;
\item[(v)] $r(s+t)=rs+rt$;
\item[(vi)] $(r+s)t=rt+st$;
\item[(vii)] $r0=0r=0$.
\end{itemize}
\begin{quote}
\emph{Proof:}
We prove (ii).  Let
$$r=\sum_{g\in G}\lambda_gg,\qquad s=\sum_{h\in G}\mu_hh,\qquad
t=\sum_{k\in G}\nu_kk,$$
where $\lambda_g,\mu_h,\nu_k\in{\mathbb C}$ for all $g,h,k\in G$; then
$$(rs)t=\sum_{g,h,k\in G}\lambda_g\mu_h\nu_k(gh)k
=\sum_{g,h,k\in G}\lambda_g\mu_h\nu_kg(hk)=r(st).$$
\end{quote}
{\bf Definition:}
We now define a ${\mathbb C}G$-module using the group algebra. Let $V={\mathbb C}G$, so that
$V$ is a vector space of dimension $n=|G|$ over ${\mathbb C}$. Given $g\in G$, we may
regard $g$ as an element of ${\mathbb C}G$, and so may form the product $gv$ for any
$v \in V$; by properties (i), (ii), (iii), (iv) and (v) of an earlier result,
for all $g,h \in G$, $\lambda \in {\mathbb C}$ and $u,v \in V$ we have
$$gv \in V,\quad (gh)v=g(hv),\quad 1v=v,\quad g(\lambda v)=\lambda gv,\quad
g(u+v)=gu+gv.$$
Thus $V$ is a ${\mathbb C}G$-module.
The ${\mathbb C}G$-module ${\mathbb C}G$ is called the \emph{regular}
${\mathbb C}G$-module. The corresponding representation $g\mapsto[g]_{\cal B}$, where ${\cal B}$ is
the natural basis of ${\mathbb C}G$, is called the {\emph{regular representation} of
$G$.
Note that the regular ${\mathbb C}G$-module has dimension $|G|$.
\\
\\
{\bf Theorem 16:} The regular ${\mathbb C}G$-module is faithful.
\begin{quote}
\emph{Proof:}
If $g\in G$ with $gv=v$ for all $v\in{\mathbb C}G$, then $g1=1$, and so $g=1$; 
thus ${\mathbb C}G$ is faithful.
\end{quote}
{\bf Theorem 17:} If $V$ is a ${\mathbb C}G$-module, the following
properties hold for all $r,s\in{\mathbb C}G$, $\lambda\in{\mathbb C}$ and $u,v\in V$:
\begin{itemize}
\item[(i)] $rv\in V$;
\item[(ii)] $(rs)v=r(sv)$;
\item[(iii)] $1v=v$;
\item[(iv)] $r(\lambda v)=\lambda(rv)=(\lambda r)v$;
\item[(v)] $r(u+v)=ru+rv$;
\item[(vi)] $(r+s)v=rv+sv$;
\item[(vii)] $r0=0v=0$.
\end{itemize}
\begin{quote}
\emph{Proof:}
We prove only (ii), leaving the rest as easy
exercises (some of whose results we shall assume). Let $r,s\in{\mathbb C}G$ and
$v\in V$, and set
$$r=\sum_{g\in G}\lambda_gg,\qquad s=\sum_{h\in G}\mu_hh\qquad\hbox{with }
\lambda_g,\mu_h\in{\mathbb C}\hbox{ for all }g,h\in G.$$
We then have

\begin{eqnarray*}
(rs)v
\!\!\!\!&=&\!\!\!\!\left(\sum_{g,h}\lambda_g\mu_h(gh)\right)v
\qquad\hbox{by definition of the multiplication in }{\mathbb C}G\\
\!\!\!\!&=&\!\!\!\!\sum_{g,h}\lambda_g\mu_h((gh)v)
\qquad\hbox{by (iv) and (vi)}\\
\!\!\!\!&=&\!\!\!\!\sum_{g,h}\lambda_g\mu_h(g(hv))
\qquad\hbox{by definition of a ${\mathbb C}G$-module}\\
\!\!\!\!&=&\!\!\!\!\left(\sum_g\lambda_gg\right)\left(\sum_h\mu_h(hv)\right)
\qquad\hbox{by (iv), (v) and (vi)}\\
\!\!\!\!&=&\!\!\!\!r(sv)\qquad\hbox{by (iv) and (vi)}
\end{eqnarray*}
as required.
\end{quote}
{\bf Theorem 18:} Write the regular ${\mathbb C}G$-module as
$${\mathbb C}G=U_1 \oplus \cdots \oplus U_r,$$
a direct sum of irreducible ${\mathbb C}G$-submodules; then any irreducible
${\mathbb C}G$-module is isomorphic to one of the $U_i$.
\begin{quote}
\emph{Proof:}
Let $W$ be an irreducible ${\mathbb C}G$-module, and choose a non-zero vector
$w\in W$. Define $\theta:{\mathbb C}G\rightarrow W$ by
$$\theta(r)=rw\qquad\hbox{for all }r\in{\mathbb C}G;$$
then clearly $\theta$ is a linear map. For all $g\in G$ and $r\in{\mathbb C}G$ we have
$$\theta(gr)=(gr)w=g(rw)=g\theta(r);$$
thus $\theta$ is a ${\mathbb C}G$-homomorphism. We know that
$im(\theta)$ is a ${\mathbb C}G$-submodule of $W$; since
$0\neq w=1w=\theta(1)\in im(\theta)$ and $W$ is irreducible, we must have
$im(\theta)=W$. There is a ${\mathbb C}G$-submodule $U$ of
${\mathbb C}G$ with
$${\mathbb C}G=ker(\theta) \oplus U\qquad\hbox{and}\qquad U \cong im(\theta)=W.$$
We have $U\cong U_i$ for some $i$, and so it follows that
$W\cong U_i$ as required.\
\end{quote}
{\bf Corollary:} Up to isomorphism, there are only finitely many
irreducible ${\mathbb C}G$-modules.
\begin{quote}
\emph{Proof:}
Follows immediately from previous result. 
\end{quote}
{\bf Theorem 19:} If $U$ is a ${\mathbb C}G$-module, then
$ dim(Hom_{{\mathbb C}G}({\mathbb C}G,U))= dim(U)$.
\begin{quote}
\emph{Proof:}
Let $dim(U)=d$, and choose a basis $u_1,\dots,u_d$ of $U$. For
$1\leq i\leq d$ define $\phi_i:{\mathbb C}G\rightarrow U$ by $\phi_i(r)=ru_i$ for all
$r\in{\mathbb C}G$. Clearly each $\phi_i$ is a linear map, and for all $g\in G$ and
$r\in{\mathbb C}G$ we have
$$\phi_i(gr)=(gr)u_i=g(ru_i)=g\phi_i(r),$$
so that $\phi_i\in Hom_{{\mathbb C}G}({\mathbb C}G,U)$. We shall prove that $\phi_1,\dots,\phi_d$ is
a basis of $Hom_{{\mathbb C}G}({\mathbb C}G,U)$. Given $\phi\in Hom_{{\mathbb C}G}({\mathbb C}G,U)$, write
$\phi(1)=\lambda_1u_1+\cdots+\lambda_du_d$ for some $\lambda_i\in{\mathbb C}$. For all
$r\in{\mathbb C}G$ we then have

\begin{eqnarray*}
\phi(r)\!\!\!\!&=&\!\!\!\!\phi(r1)=r\phi(1)=r\lambda_1u_1+\cdots+r\lambda_du_d
=\lambda_1\phi_1(r)+\cdots+\lambda_d\phi_d(r)\\
\!\!\!\!&=&\!\!\!\!(\lambda_1\phi_1+\cdots+\lambda_d\phi_d)(r);
\end{eqnarray*}

thus $\phi=\lambda_1\phi_1+\cdots+\lambda_d\phi_d$. Hence
$\phi_1,\dots,\phi_d$ span $Hom_{{\mathbb C}G}({\mathbb C}G,U)$. Now if
$\lambda_1\phi_1+\cdots+\lambda_d\phi_d=0$ for some $\lambda_i\in{\mathbb C}$, we have
$$0=(\lambda_1\phi_1+\cdots+\lambda_d\phi_d)(1)
=\lambda_1u_1+\cdots+\lambda_du_d,$$
so $\lambda_i=0$ for all $i$. Thus $\phi_1,\dots,\phi_d$ is
a basis of $Hom_{{\mathbb C}G}({\mathbb C}G,U)$; it follows that $dim(Hom_{{\mathbb C}G}({\mathbb C}G,U))=d$.
\end{quote}
{\bf Theorem 20:} Suppose that ${\mathbb C}G=U_1 \oplus \cdots \oplus U_r$ is a
direct sum of irreducible ${\mathbb C}G$-submodules. If $U$ is any irreducible
${\mathbb C}G$-module, then the number of terms $U_i$ isomorphic to $U$ is equal to
$dim(U)$.
\begin{quote}
\emph{Proof:}
$ dim(U)= dim(Hom_{{\mathbb C}G}({\mathbb C}G,U))$, and by the last theorem,
this is equal to the number of terms $U_i$ with $U_i\cong U$.
\end{quote}
{\bf Definition:} If $V_1,\dots,V_k$ are irreducible ${\mathbb C}G$-modules
such that no two are isomorphic and any irreducible ${\mathbb C}G$-module is isomorphic
to some $V_i$, we say that the $V_i$ form a \emph{complete set of
non-isomorphic irreducible ${\mathbb C}G$-modules\/}.
\\
\\
{\bf Theorem 21:} If $V_1,\dots,V_k$ form a complete set of
non-isomorphic irreducible ${\mathbb C}G$-modules, then
$$\sum_{i=1}^k( dim V_i)^2=|G|.$$
\begin{quote}
\emph{Proof:}
Let ${\mathbb C}G=U_1\oplus\cdots\oplus U_r$, a direct sum of irreducible
${\mathbb C}G$-modules; set $dim(V_i) =d_i$ for $1 \leq i \leq k$. For
each $i$ the number of terms $U_j$ isomorphic to $V_i$ is equal to $d_i$. Thus
$$dim({\mathbb C}G)= dim(U_1)+\cdots+ dim(U_r) =\sum_{i=1}^k d_i( dim(V_i))
= \sum_{i=1}^k {d_i}^2.$$
As $ dim({\mathbb C}G)=|G|$, the result follows.
\end{quote}
{\bf Theorem 22:}
$dim(Hom_{{\mathbb C}G}(U_1 \oplus \ldots \oplus U_r, W_1 \oplus \ldots \oplus W_s))=
\sum_{i=1, j=1}^{r,s} dim(Hom_{{\mathbb C}G}(V_i , W_j))$. Suppose $U$ is an irreducible
${\mathbb C}G$-module then 
$dim(Hom_{{\mathbb C}G}({\mathbb C}G, U))= dim(U)$.
\begin{quote}
\emph{Proof:}  
Let $d=dim(U)$ and $u_1, u_2, \ldots, u_d$ be a basis for $U$.  Define
$r \phi_i=u_i r$.  The $\phi_i$ are a basis for $Hom_{{\mathbb C}G}({\mathbb C}G, U)$.
\end{quote}
{\bf Theorem 23:}
Let $V$ be an ${\mathbb C}G$ module, 
$V= U_1 \oplus U_2 \oplus \ldots \oplus U_r$ with $U_i$ irreducible; (a) if
$W$ is an irreducible ${\mathbb C} G$ module then 
$dim_{{\mathbb C}}(Hom_{{\mathbb C}G}(V, W))= dim_{{\mathbb C}}(Hom_{{\mathbb C}G}(W, V))$ is
the number of $U_i \cong W$;
(b) each $U_i$ is a composition factor in the Jordan Holder series.
\begin{quote}
\emph{Proof:}  
$dim(Hom_{{\mathbb C}G}(U_1 \oplus \ldots \oplus U_r, W)
\sum_{i=1}^{r} dim(Hom_{{\mathbb C}G}(V_i , W))$. 
$dim(Hom_{{\mathbb C}G}(V_i , W))= 1$ when $U_i \cong W$ and is $0$ otherwise.  So the
sum is just the number of $U_i \cong W$ so is the sum where the $U_i$ and $W$ are reversed.
\end{quote}
\section{Representations of Abelian Groups}
{\bf Theorem 24:} If $G$ is abelian, then every irreducible
${\mathbb C}G$-module has dimension $1$.
\begin{quote}
\emph{Proof:}
Let $V$ be an irreducible ${\mathbb C}G$-module, and take any $x\in G$.
Because $G$ is abelian, we have
$$x(gv)=g(xv)\qquad\hbox{for all }g\in G,\ v\in V,$$
and hence the endomorphism $v\mapsto xv$ of $V$ is a ${\mathbb C}G$-homomorphism. 
This endomorphism must be a scalar multiple of the identity map
$1_V$, say $\lambda_x1_V$; thus $xv=\lambda_xv$ for all $v\in V$. Since this is
true for all $x\in G$, we see that every subspace of $V$ is a ${\mathbb C}G$-submodule;
so as $V$ is irreducible we must have $ dim(V)=1$.
\end{quote}
{\bf Theorem 25:} Let $G$ be the abelian group
$C_{n_1}\times\cdots\times C_{n_r}$. There are $|G|$ irreducible
representations of $G$, and any such is of the form
$\rho_{\lambda_1,\dots,\lambda_r}$; no two of these representations are
equivalent.
\begin{quote}
\emph{Proof:}
The above has shown that any irreducible representation of $G$ must be of
the form $\rho_{\lambda_1,\dots,\lambda_r}$; conversely if $\lambda_i$ is an
$n_i$th root of unity for $1\leq i\leq r$, the map
$\rho:G\rightarrow GL_1({\mathbb C})$ given by
$$\rho({g_1}^{i_1}\dots{g_r}^{i_r})=
({\lambda_1}^{i_1}\dots{\lambda_r}^{i_r})$$
is clearly an irreducible
representation. Since there are $n_i$ choices for each $\lambda_i$, the number
of representations of the form $\rho_{\lambda_1,\dots,\lambda_r}$ is
$n_1\dots n_r=|G|$; no two are equivalent, as
$T^{-1}AT=A$ for any $1\times 1$ invertible matrices $A$ and $T$.
\end{quote}
{\bf Theorem 26:} If every irreducible ${\mathbb C}G$-module has
dimension $1$, then $G$ is abelian.
\begin{quote}
\emph{Proof:}
Let $V$ be a faithful ${\mathbb C}G$-module. We can write
$V=V_1\oplus\cdots\oplus V_r$ with each $V_i$ irreducible. By assumption,
$\dim(V_i) =1$ for all $i$; let $V_i=\langle v_i\rangle$, so that
$v_1,\dots,v_r$ is a
basis ${\cal B}$ of $V$. Since each $V_i$ is a ${\mathbb C}G$-submodule, we have
$gv_i\in V_i$ for all $g\in G$; thus each matrix $[g]_{\cal B}$ is diagonal. As
diagonal matrices commute, for all $g,h\in G$ we have
$$[gh]_{\cal B}=[g]_{\cal B}[h]_{\cal B}=[h]_{\cal B}[g]_{\cal B}=[hg]_{\cal B},$$
and so as the representation is faithful we have $gh=hg$ -- so $G$ is
abelian.
\end{quote}
{\bf Note:} The above shows that no non-cyclic abelian group has a faithful representation.
\section{Characters}
{\bf Definition:} Let $\rho:G\rightarrow GL_n({\mathbb C})$ be a representation
of $G$; then the \emph{character} of $\rho$ is the function
$\chi:G\rightarrow{\mathbb C}$ given by $\chi(g)=Tr(\rho(g))$ for all $g\in G$.
\\
\\
{\bf Remark:}
Clearly the character of a representation $\rho$ of degree $n$ is ``simpler''
than $\rho$ itself, in that it involves only $|G|$ values rather than
$n^2|G|$ matrix entries. Our next result shows that this ``loss of detail''
means that we fail to distinguish between equivalent representations.
\\
\\
{\bf Theorem 27:} Equivalent representations of $G$ have the
same character.
\begin{quote}
\emph{Proof:}
Let $\rho,\sigma: G \rightarrow GL_n({\mathbb C})$ be equivalent representations of
$G$; then there is an invertible matrix $T$ such that for all $g\in G$ we have
$\sigma(g)=T^{-1}\rho(g)T$. Thus by the corollary, we have
$Tr(\sigma(g))=Tr((T^{-1}\rho(g)T)=Tr(\rho(g))$.
\end{quote}
{\bf Definition:} Let $V$ be a ${\mathbb C}G$-module, with basis ${\cal B}$; then the
\emph{character} of $V$ is the function $\chi:G\rightarrow{\mathbb C}$ given by
$\chi(g)=Tr([g]_{\cal B})$ for all $g\in G$.
\\
\\
{\bf Theorem 28:} Isomorphic ${\mathbb C}G$-modules have the same
character.
\begin{quote}
\emph{Proof:}
Combine the two previous results.
\end{quote}
{\bf Definition:} A function $\chi: G \rightarrow {\mathbb C}$ is called a
\emph{character} if it is the character of some ${\mathbb C}G$-module. A character is
called \emph{irreducible} if it is the character of an
irreducible ${\mathbb C}G$-module, and \emph{reducible} otherwise.
\\
\\
{\bf Theorem 29:} If $\chi$ is the character of a
${\mathbb C}G$-module $V$, then $\chi(1)= dim(V)$.
\begin{quote}
\emph{Proof:}
Let $n= dim (V)$; then the matrix $[1]_{\cal B}$ is the identity matrix $I_n$, and
so we have $\chi(1)=Tr([1]_{\cal B})=Tr( I_n)=n$ as required.
\end{quote}
{\bf Definition:} If $\chi$ is the character of the ${\mathbb C}G$-module $V$,
the dimension of $V$ is called the \emph{degree} of $\chi$.
\\
\\
{\bf Example:} (a) $G = D_8 = \langle a, b| a^4=b^2=1, a^b = a^{-1}\rangle$.
$\rho(a) =
\left(
\begin{array}{cc}
0&1\\
-1&0\\
\end{array}
\right)$,
$\rho(b) =
\left(
\begin{array}{cc}
1&0\\
0&-1\\
\end{array}
\right)$.\\
(b) $G = D_6 = S_3 = \langle a, b| a^3=b^2=1, a^b = a^{-1}\rangle$.  Basis is
$v_0 = 1 + a + a^2$, $w_0= b v_0$,
$v_1 = 1 + \omega^2a + \omega a^2$, $w_1= b v_1$,
$v_2 = 1 + \omega a + \omega^2 a^2$, $w_2= b v_2$.
$sp(v_0, w_0)$ is reducible as $sp(v_0 + w_0) \oplus sp(v_0 - w_0)$.
$sp(v_1, w_2) \cong sp(v_2, w_1)$ and they are irreducible.
The characters of $D_8$ have degree $1$, $1$, $1$, $1$, and $2$.
The characters of $S_3$
have degrees $1$, $1$ and $2$.
\\
\\
{\bf Theorem 30:} If $\chi$ is a character of $G$, and
$g,h\in G$ are conjugate, then $\chi(g)=\chi(h)$.
\begin{quote}
\emph{Proof:}
If $g$ and $h$ are conjugate, we have $h=x^{-1}gx$ for some $x\in G$;
thus if $\chi$ is the character of
the ${\mathbb C}G$-module $V$ and ${\cal B}$ is a basis for $V$, we have
$$[h]_{\cal B}=[x^{-1}gx]_{\cal B}=[x^{-1}]_{\cal B}[g]_{\cal B}[x]_{\cal B}=([x]_{\cal B})^{-1}[g]_{\cal B}[x]_{\cal B}.$$
By the corollary,
we then have $\chi(h)=Tr([h]_{\cal B})=Tr([g]_{\cal B})=\chi(g)$ as
required.
\end{quote}
{\bf Theorem 31:} If $V$ is a ${\mathbb C}G$-module, then for each
$g\in G$ there is a basis ${\cal B}$ of $V$ such that the matrix $[g]_{\cal B}$ is
diagonal; if $g$ has order $m$, the diagonal entries of $[g]_{\cal B}$ are $m$th
roots of unity.
\begin{quote}
\emph{Proof:}
Given $g\in G$, let $H=\langle g\rangle$; then $H$ is a cyclic subgroup of
$G$. By restricting the multiplication on $V$ to the elements of $H$, we may
consider $V$ as a ${\mathbb C} H$-module. We may write
$$V=U_1\oplus\cdots\oplus U_n,$$
where each $U_j$ is an irreducible ${\mathbb C} H$-submodule of $V$.
Each $U_j$ has dimension $1$; let $u_j$ be a vector spanning $U_j$.
If we set $\omega=e^{2\pi i/m}$, then for all $j$ there is an integer
$r_j$ with $gu_j=\omega^{r_j}u_j$. Thus if we let ${\cal B}$ be the basis
$u_1,\dots,u_n$ of $V$, we have
$[g]_{\cal B}=\left(\begin{array} {ccc}
\omega^{r_1} & \ldots & 0\\
\ldots & \ldots & \ldots\\
0 & \ldots & \omega^{r_n}\\
\end{array}\right)$
as required.
\end{quote}
{\bf Example:} Let $G=S_3$, and $g=(1\;2\;3)\in G$, so that $g$ has
order $3$; take $V$ to be the permutation module. We have seen that the
matrix of $g$ with respect to the natural basis $v_1,v_2,v_3$ is not diagonal.
However, if we write $\omega=e^{2\pi i/3}$ and set
$$w_1=v_1+v_2+v_3,\qquad w_2=v_1+\omega^2v_2+\omega v_3,\qquad
w_3=v_1+\omega v_2+\omega^2v_3,$$
then $w_1,w_2,w_3$ is a basis ${\cal B}$ of $V$, and we have
$$[g]_{\cal B}=\left(\begin{array}{ccc}
1&0&0\\0&\omega&0\\0&0&\omega^2\end{array}\right).$$
{\bf Theorem 32:} If $\chi$ is a character of $G$ of degree
$n$, and $g\in G$ has order $m$, then:
\begin{itemize}
\item[(i)] $\chi(g)$ is a sum of $n$ $m$th roots of unity;
\item[(ii)] $|\chi(g)|\leq n$;
\item[(iii)] $\chi(g^{-1})=\overline{\chi(g)}$;
\item[(iv)] if $g$ is conjugate to $g^{-1}$ then $\chi(g)  \in {\mathbb R}$.
\end{itemize}
\begin{quote}
\emph{Proof:}
Let $V$ be a ${\mathbb C}G$-module having $\chi$ as character.
There is a basis ${\cal B}$ of $V$ such that
$$[g]_{\cal B}=\left(\begin{array}{ccc}\omega_1&\cdots&0\\
\vdots&\ddots&\vdots\\0&\cdots&\omega_n\end{array}\right)$$
where each $\omega_j$ is an $m$th root of unity; this proves (i). The
triangle inequality gives
$$|\chi(g)|=|\omega_1+\cdots+\omega_n|\leq|\omega_1|+\cdots+|\omega_n|
=1+\cdots+1=n,$$
which proves (ii). Also we have
$$[g^{-1}]_{\cal B}=\left(\begin{array}{ccc}\omega_1^{-1}&\cdots&0\\
\vdots&\ddots&\vdots\\0&\cdots&{\omega_n}^{-1}\end{array}\right),$$
and so $\chi(g^{-1})={\omega_1}^{-1}+\cdots+{\omega_n}^{-1}$. As
$\overline{\omega}=\omega^{-1}$ for each root of unity $\omega$, we have
$\chi(g^{-1})=\overline{\omega_1}+\cdots+\overline{\omega_n}
=\overline{\chi(g)}$, giving (iii). Finally if $g$ is conjugate to
$g^{-1}$ then by Theorem 30 we have
$\chi(g)=\chi(g^{-1})=\overline{\chi(g)}$; thus $\chi(g)\in{\mathbb R}$, giving (iv).
\end{quote}
{\bf Corollary:} If $g\in G$ has order $2$, and $\chi$ is a
character of $G$, then $\chi(g) \in {\mathbb Z}$, and $\chi(g)\equiv\chi(1)\jmod{2}$.
\begin{quote}
\emph{Proof:}
$$\chi(g)=\omega_1+\cdots+\omega_n,$$
where $n=\chi(1)$ and each $\omega_j$ is a square root of unity. Suppose
$r$ terms $\omega_j$ are equal to $-1$; then the remaining $n-r$ are equal to
$1$, and so
$$\chi(g)=(n-r)-r=n-2r.$$
Hence $\chi(g)\in{\mathbb Z}$, and $\chi(g)\equiv\chi(1)\jmod{2}$ as required.
\end{quote}
{\bf Theorem 33:} Let $\rho:G\rightarrow GL_n({\mathbb C})$ be a
representation with character $\chi$. Then:
\begin{itemize}
\item[(i)] for $g\in G$ we have $|\chi(g)|=\chi(1)$ if and only if
$\rho(g)=\lambda I_n$ for some $\lambda\in{\mathbb C}$;
\item[(ii)] $ker(\rho)=\{g\in G:\chi(g)=\chi(1)\}$.
\end{itemize}
\begin{quote}
\emph{Proof:}
\\
\\
(i) Let $g\in G$ have order $m$. If $\rho(g)=\lambda I_n$ with
$\lambda\in{\mathbb C}$, then $\lambda$ is an $m$th root of unity, and
$\chi(g)=n\lambda$; thus $|\chi(g)|=n=\chi(1)$. Conversely, suppose that
$|\chi(g)|=\chi(1)$. We know by Theorem 31 that there is a basis ${\cal B}$ of
${\mathbb C}^n$ such that
$$[g]_{\cal B}=\left(\begin{array}{ccc}\omega_1&\cdots&0\\
\vdots&\ddots&\vdots\\0&\cdots&\omega_n\end{array}\right),$$
where each $\omega_i$ is an $m$th root of unity; thus
$\chi(g)=\omega_1+\cdots+\omega_n$. Since by assumption we have
$$|\omega_1+\cdots+\omega_n|=|\chi(g)|=\chi(1)=n
=|\omega_1|+\cdots+|\omega_n|,$$
each term must have the same argument; so $\omega_i=\omega_j$ for
all $i,j$, and thus
$$[g]_{\cal B}=\left(\begin{array}{ccc}\omega_1&\cdots&0\\
\vdots&\ddots&\vdots\\0&\cdots&\omega_1\end{array}\right)= \omega_1I_n.$$
Hence if ${\cal B}'$ is any basis of ${\mathbb C}^n$, there is a change of basis matrix
$T$ such that $[g]_{{\cal B}'}=T^{-1}[g]_{\cal B} T=T^{-1}\omega_1 I_n T=\omega_1 I_n$; so
$\rho(g)=\omega_1I_n$ as required.
\\
\\
(ii) Clearly if $g\in ker(\rho)$ then $\rho(g)=I_n$ so that
$\chi(g)=n=\chi(1)$. Conversely if $g\in G$ satisfies $\chi(g)=\chi(1)$, then
by (i) we have $\rho(g)=\lambda I_n$ for some $\lambda\in{\mathbb C}$; hence
$\chi(g)=\lambda\chi(1)$, and so $\lambda=1$, giving $\rho(g)=I_n$ and so
$g \in ker(\rho)$ as required.
\end{quote}
{\bf Example:} Let $G=D_8$, and let $\chi$ be the character given
above, with values as follows.
$$
\begin{array}{|c|cccccccc|}
\hline
g&1&a&a^2&a^3&b&ba&ba^2&ba^3\\
\hline
\chi(g)&2&0&-2&0&0&0&0&0\\
\hline
\end{array}
$$
{\bf Definition:} The \emph{kernel} of the character $\chi$ of
$G$ is the set $ker(\chi)=\{g\in G:\chi(g)=\chi(1)\}$.
\\
\\
{\bf Theorem 34:} If $\chi$ is a character of $G$ then so is
$\bar\chi$; if $\chi$ is irreducible then so is $\bar\chi$.
\begin{quote}
\emph{Proof:}
Let $\chi$ be the character of a representation
$\rho:G\rightarrow GL_n({\mathbb C})$; thus $\chi(g)=Tr((\rho(g))$ for all $g\in G$.
Now given an $n\times n$ matrix $A=(a_{ij})$ over ${\mathbb C}$, we set
$\bar A=(\overline{a_{ij}})$; then if $A=(a_{ij})$ and $B=(b_{ij})$ are
$n\times n$ matrices over ${\mathbb C}$ we have
$\overline{AB}=\overline{A}.\overline{B}$, because
$$(\overline{A}.\overline{B})_{ij}
=\sum_{k=1}^n\bar a_{ik}\bar b_{kj}=\sum_{k=1}^n\overline{a_{ik}b_{kj}}
=\overline{\sum_{k=1}^na_{ik}b_{kj}}=(\overline{AB})_{ij}.$$
Thus the function $\bar\rho:G\rightarrow GL_n({\mathbb C})$ defined by
$\bar\rho(g)=\overline{\rho(g)}$ for all $g\in G$ is a representation of $G$;
as $$Tr((\bar\rho(g))=Tr((\overline{\rho(g)})=\overline{Tr((\rho(g))}
=\overline{\chi(g)}=\bar\chi(g)\qquad\hbox{for all }g\in G,$$
the character of the representation $\bar\rho$ is $\bar\chi$. Clearly if
$\rho$ is reducible then so is $\bar\rho$; thus $\chi$ is irreducible if and
only if $\bar\chi$ is.
\end{quote}
{\bf Theorem 35:} If $V$ is a ${\mathbb C}G$-module and we have
$V=U_1\oplus\cdots\oplus U_r$ with the $U_i$ ${\mathbb C}G$-submodules of $V$, then the
character of $V$ is the sum of the characters of the $U_i$.
\begin{quote}
\emph{Proof:}
Let ${\cal B}_i$ be a basis of $U_i$ for $1\leq i\leq r$, and amalgamate the
bases ${\cal B}_i$ to form a basis ${\cal B}$ of $V$; then for all $g\in G$ we have
$$[g]_{\cal B}=\left(\begin{array}{ccc}[g]_{{\cal B}_1}&\cdots&0\\
\vdots&\ddots&\vdots\\0&\cdots&[g]_{{\cal B}_r}\end{array}\right).$$
Thus $Tr([g]_{\cal B}) =Tr([g]_{{\cal B}_1}) +\cdots+Tr([g]_{{\cal B}_r}) $, i.e., the character of
$V$ is the sum of those of the $U_i$ as required.
\end{quote}
{\bf Example:} Let $G=S_3$, and $V$ be the permutation module, with
character $\chi$. We saw in section 1.4 that $V=U_1\oplus U_2$, where
$U_1=\langle v_1+v_2+v_3\rangle$ and $U_2=\langle v_1-v_2,v_2-v_3\rangle$.
If we let ${\cal B}_1$ and ${\cal B}_2$ be the bases $v_1+v_2+v_3$ of $U_1$ and
$v_1-v_2,v_2-v_3$ of $U_2$, then the matrices $[g]_{{\cal B}_i}$ are as follows.

$$
\begin{array}{|c|c|c|c|c|c|c|}
\hline
g&1&(1\;2)&(1\;3)&(2\;3)&(1\;2\;3)&(1\;3\;2)\\
\hline
[g]_{{\cal B}_1}&(1)&(1)&(1)&(1)&(1)&(1)\\
\hline
[g]_{{\cal B}_2}&
\left(\begin{array}{cc}1&0\\0&1\end{array}\right)&
\left(\begin{array}{cc}-1&1\\0&1\end{array}\right)&
\left(\begin{array}{cc}0&-1\\-1&0\end{array}\right)&
\left(\begin{array}{cc}1&0\\1&-1\end{array}\right)&
\left(\begin{array}{cc}0&-1\\1&-1\end{array}\right)&
\left(\begin{array}{cc}-1&1\\-1&0\end{array}\right)\\
\hline
\end{array}
$$
Thus if we write $\chi_i$ for the
character of $U_i$ for $i=1,2$, the character values are as follows.
$$
\begin{array}{|c|cccccc|}
\hline
g&1&(1\;2)&(1\;3)&(2\;3)&(1\;2\;3)&(1\;3\;2)\\
\hline
\chi_1(g)&1&1&1&1&1&1\\
\chi_2(g)&2&0&0&0&-1&-1\\
\hline
\chi(g)&3&1&1&1&0&0\\
\hline
\end{array}
$$
By applying this result when each $U_i$ is irreducible, we see that any
character is a sum of irreducible ones. As with representations and
${\mathbb C}G$-modules, this concentrates attention on the irreducible characters.
\\
\\
{\bf Definition:} A character of degree $1$ is called a \emph{linear
character}.
\\
\\
{\bf Examples:}
\begin{itemize}
\item[(i)] The three irreducible characters of $C_3$ are all linear.
\item[(ii)] Of the three irreducible characters of $D_6$, the first two are
linear but the third is not, since it has degree $2$.
\end{itemize}
{\bf Remark:}
If $V$ is a $1$-dimensional ${\mathbb C}G$-module, then for all $g\in G$ there exists
$\lambda_g\in{\mathbb C}$ such that $gv=\lambda_gv$ for all $v\in V$; the linear
character $\chi$ of $V$ is given by $\chi(g)=\lambda_g$ for all $g\in G$.
Any irreducible character of an abelian group is linear.
Note that a linear character of any group is certainly irreducible; also a
linear character is in fact a homomorphism from $G$ to the multiplicative
group of non-zero complex numbers. (It is easy to see that the only characters
which are homomorphisms in this way are the linear ones: if $\chi$ is a
character of degree $d$ which is a homomorphism, we must have
$\chi(1)\chi(1)=\chi(1)$, i.e., $d^2=d$, and so $d=1$.)
\\
\\
{\bf Definition:} The character of the trivial representation of
$G$ is called the \emph{trivial character} of $G$, and is written $1_G$.
\\
\\
{\bf Definition:} The character $\chi$ is called \emph{faithful} if
$ker(\chi)=\{1\}$.
\\
\\
{\bf Examples:}
\begin{itemize}
\item[(i)] The irreducible characters of degree $2$ of $D_6$ and
$D_8$ above are both faithful.
\item[(ii)] The two linear characters of $D_6$ are not faithful, since their
kernels are $D_6$ and $\langle a\rangle\cong C_3$ respectively.
\end{itemize}
{\bf Definition:} The character of the regular ${\mathbb C}G$-module ${\mathbb C}G$ is
called the \emph{regular character} of $G$, and is written $\chi_{reg}$.
\\
\\
{\bf Theorem 36:} Let $V_1,\dots,V_k$ be a complete set of
non-isomorphic irreducible ${\mathbb C}G$-modules, and for $1\leq i\leq k$ let
$\chi_i$ be the character of $V_i$ and $d_i$ the dimension of $V_i$; then
$\chi_{reg}=d_1\chi_1+\cdots+d_k\chi_k$.
\begin{quote}
\emph{Proof:}
$${\mathbb C}G=\underbrace{(V_1\oplus\cdots\oplus V_1)}_{d_1\ \mathrm{terms}}\oplus
\cdots\oplus\underbrace{(V_k\oplus\cdots\oplus V_k)}_{d_k\ \mathrm{terms}};$$
the result now follows.
\end{quote}
{\bf Theorem 37:} $\chi_{reg}(1)=|G|$, while $\chi_{reg}(g)=0$ if
$1\neq g \in G$.
\begin{quote}
\emph{Proof:}
Let $G=\{g_1,\dots,g_n\}$, and let ${\cal B}$ be the natural basis of ${\mathbb C}G$.
We have $\chi_{reg}(1)=n=|G|$. Given $1\neq g\in G$, for all
$1\leq i\leq n$ we have $gg_i=g_j$ for some $j\neq i$; thus the $i$th column of
$[g]_{\cal B}$ has zero everywhere except in the $j$th row, and in particular the
$(i,i)$-entry of $[g]_{\cal B}$ is zero for all $i$. Thus
$\chi_{reg}(g)=Tr([g]_{\cal B}=0$ as required.
\end{quote}
{\bf Example:} 
Consider $G=D_6$. The irreducible characters are
$\chi_1$, $\chi_2$ and $\chi_3$, of degrees $1$, $1$ and $2$.
Calculating the values of $\chi_1+\chi_2+2\chi_3$, we get:
$$
\begin{array}{|c|cccccc|}
\hline
g&1&a&a^2&b&ba&ba^2\\
\hline
\chi_1(g)&1&1&1&1&1&1\\
\chi_2(g)&1&1&1&-1&-1&-1\\
\chi_3(g)&2&-1&-1&0&0&0\\
\hline
(\chi_1+\chi_2+2\chi_3)(g)&6&0&0&0&0&0\\
\hline
\end{array}
$$
$\chi_{reg}=\chi_1+\chi_2+2\chi_3$ and
$\chi_{reg}$ takes the value $|G|$ at the element $1$ and $0$ elsewhere.
\\
\\
{\bf Definition:} If $G$ is a subgroup of $S_n$, the character of the
permutation module for $G$ is called the \emph{permutation character} of $G$.
\\
\\
{\bf Theorem 38:} If $G$ is a subgroup of $S_n$, the function
$\nu:G\rightarrow{\mathbb C}$ defined by $\nu(g)=|Fix(g)|-1$ is a character of $G$.
\begin{quote}
\emph{Proof:}
Let $V$ be the permutation module for $G$, and let $v_1,\dots,v_n$ be the
natural basis of $V$; set $u=v_1+\cdots+v_n$, and let $U$ be the
$1$-dimensional subspace of $V$ spanned by $u$. Since $gu=u$ for all $g\in G$,
we see that $U$ is a trivial ${\mathbb C}G$-submodule of $V$, with character $1_G$.
There is a ${\mathbb C}G$-submodule $W$ of $V$ such that $V=U\oplus W$;
let $\nu$ be the character of $W$. We have
$$\pi=1_G+\nu,$$
and so $|Fix(g)|=1+\nu(g)$ for all $g\in G$; thus $\nu(g)=|Fix(g)|-1$ for all
$g\in G$.
\end{quote}
{\bf Example:} Let $G=A_4$, a subgroup of $S_4$; then
$G$ has four conjugacy classes, represented by $1$, $(1\;2)(3\;4)$,
$(1\;2\;3)$ and $(1\;3\;2)$. The values of the character $\nu$ are as follows.
$$
\begin{array}{|c|cccc|}
\hline
g&1&(1\;2)(3\;4)&(1\;2\;3)&(1\;3\;2)\\
\hline
\nu(g)&3&-1&0&0\\
\hline
\end{array}
$$
\section{The space of functions $G \rightarrow {\mathbb C}$}
{\bf Definition:} Given $\theta:G\rightarrow{\mathbb C}$ and
$\phi: G \rightarrow {\mathbb C}$, we define
$$\langle\theta,\phi\rangle
=\frac{1}{|G|}\sum_{g \in G} \theta(g) \overline{\phi(g)}.$$
\\
\\
{\bf Theorem 39:} If $G$ has precisely $\ell$ conjugacy
classes $C_1,\dots,C_\ell$, with representatives $g_1,\dots,g_\ell$, and
$\chi$ and $\psi$ are characters of $G$, then
$$\!\langle\chi,\psi\rangle=\langle\psi,\chi\rangle
=\frac{1}{|G|}\sum_{g\in G}\chi(g)\psi(g^{-1})
=\frac{1}{|G|}\sum_{i=1}^\ell|C_i|\chi(g_i)\overline{\psi(g_i)}
=\sum_{i=1}^\ell\frac{\chi(g_i)\overline{\psi(g_i)}}{|C_G(g_i)|}\in{\mathbb R}.\!$$
\begin{quote}
\emph{Proof:}
We have five things to prove. First note that because
$\overline{\psi(g)}=\psi(g^{-1})$ for all $g\in G$ by Theorem 32 (iii),
we have
$$\langle\chi,\psi\rangle=\frac{1}{|G|}\sum_{g\in G}\chi(g)\psi(g^{-1}).$$
Because $g^{-1}$ runs through $G$ as $g$ does, we also have
$$\langle\chi,\psi\rangle=\frac{1}{|G|}\sum_{g\in G}\chi(g^{-1})\psi(g)
=\langle\psi,\chi\rangle.$$
Since $\langle\chi,\psi\rangle=\langle\psi,\chi\rangle
=\overline{\langle\chi,\psi\rangle}$, we must have
$\langle\chi,\psi\rangle\in{\mathbb R}$. Next, because characters are constant on
conjugacy classes, we have
$$\sum_{g\in C_i}\chi(g)\overline{\psi(g)}
=|C_i|\chi(g_i)\overline{\psi(g_i)};$$
thus as $G$ is the disjoint union of the conjugacy classes $C_i$, we have
$$\langle\chi,\psi\rangle
=\frac{1}{|G|}\sum_{g\in G}\chi(g)\overline{\psi(g)}
=\frac{1}{|G|}\sum_{i=1}^\ell\sum_{g\in C_i}\chi(g)\overline{\psi(g)}
=\frac{1}{|G|}\sum_{i=1}^\ell|C_i|\chi(g_i)\overline{\psi(g_i)}.$$
Finally, as $|C_i|=|G|/|C_G(g_i)|$, we have
$$\langle\chi,\psi\rangle
=\sum_{i=1}^\ell\frac{|C_i|}{|G|}\chi(g_i)\overline{\psi(g_i)}
=\sum_{i=1}^\ell\frac{1}{|C_G(g_i)|}\chi(g_i)\overline{\psi(g_i)}$$
as required.
\end{quote}
{\bf Example:} Let $G=A_4$ then 
$G$ has four conjugacy classes, with
representatives 
$$g_1=1,\qquad g_2=(1\;2)(3\;4),\qquad g_3=(1\;2\;3),\qquad g_4=(1\;3\;2).$$
The conjugacy class sizes $|C_i|$ are as follows:
$$
\begin{array}{rlrl}
C_G(g_1)\!\!\!\!&=G&\quad |C_1|\!\!\!\!&=|G|/|C_G(g_1)|=1,\\
C_G(g_2)\!\!\!\!&=\{1,(1\;2)(3\;4),(1\;3)(2\;4),(1\;4)(2\;3)\},&
|C_2|\!\!\!\!&=|G|/|C_G(g_2)|=3,\\
C_G(g_3)\!\!\!\!&=\{1,(1\;2\;3),(1\;3\;2)\},&|C_3|\!\!\!\!&=|G|/|C_G(g_3)|=4,\\
C_G(g_4)\!\!\!\!&=\{1,(1\;3\;2),(1\;2\;3)\},&|C_4|\!\!\!\!&=|G|/|C_G(g_4)|=4.
\end{array}
$$
Let $\omega=e^{2\pi i/3}$, then $G$ has characters $\chi$ and
$\psi$ and:
$$
\begin{array}{|c|cccc|}
\hline
g&g_1&g_2&g_3&g_4\\
\hline
|C_G(g)|&12&4&3&3\\
\hline
|C_g|&1&3&4&4\\
\hline
\chi&1&1&\omega&\omega^2\\
\psi&4&0&\omega^2&\omega\\
\hline
\end{array}
$$
Since $g_2$ has a total of $3$ conjugates, and $g_3$ and $g_4$ have
$4$ each, we may calculate
$$\langle\chi,\psi\rangle
=\frac{1}{12}(1.4+1.0+1.0+1.0+\omega.\overline{\omega^2}
+\omega.\overline{\omega^2}+\omega.\overline{\omega^2}
+\omega.\overline{\omega^2}+\omega^2.\overline{\omega}
+\omega^2.\overline{\omega}+\omega^2.\overline{\omega}
+\omega^2.\overline{\omega})=0;$$
however, it is simpler to compute
$$\langle\chi,\psi\rangle
=\sum_{i=1}^\ell\frac{\chi(g_i)\overline{\psi(g_i)}}{|C_G(g_i)|}
=\frac{1.4}{12}+\frac{1.0}{4}+\frac{\omega.\overline{\omega^2}}{3}
+\frac{\omega^2.\overline{\omega}}{3}=0.$$
Note that we also have
$$\langle\psi,\chi\rangle
=\sum_{i=1}^\ell\frac{\psi(g_i)\overline{\chi(g_i)}}{|C_G(g_i)|}
=\frac{4.1}{12}+\frac{0.1}{4}
+\frac{\omega^2.\overline{\omega}}{3}+\frac{\omega.\overline{\omega^2}}{3}=0,$$
so that $\langle\chi,\psi\rangle=\langle\psi,\chi\rangle\in{\mathbb R}$. Similarly we
find that
\begin{eqnarray*}
&\langle\chi,\chi\rangle
=\frac{1.1}{12}+\frac{1.1}{4}+\frac{\omega.\overline{\omega}}{3}
+\frac{\omega^2.\overline{\omega^2}}{3}=1,&\\
&\langle\psi,\psi\rangle
=\frac{4.4}{12}+\frac{0.0}{4}+\frac{\omega^2.\overline{\omega^2}}{3}
+\frac{\omega.\overline{\omega}}{3}=2.&
\end{eqnarray*}
\\
\\
{\bf Remark:} 
Sometimes it is possible to obtain information about
class sizes from knowledge of characters and their inner products. Here it
tends to be more convenient to use the inner product formula in the form
$$\langle\chi,\psi\rangle
=\frac{1}{|G|}\sum_{i=1}^\ell c_i\chi(g_i)\overline{\psi(g_i)},$$
where $g_1,\dots,g_\ell$ are representatives of the conjugacy classes, and
$c_i=|C_i|$ for $1\leq i\leq\ell$.
\\
\\
{\bf Theorem 40:}
If $G \lhd N$ and $\chi$ is a character of $G/N$,
define $\tilde{\chi}(g)= \chi(gN)$.  $\tilde{\chi}$ is a character of $G$ and is irreducile iff
$\chi$ is irreducible.
\begin{quote}
\emph{Proof:}  
This follows from the natural homomorphism $G \rightarrow G/N$.
\end{quote}
\subsection{Orthonormality}
{\bf Theorem 41:} For all $w_1\in W_1$ and
$w_2\in W_2$ we have
$$e_1w_1=w_1,\qquad e_1w_2=0,\qquad\qquad e_2w_1=0,\qquad
e_2w_2=w_2.$$
\begin{quote}
\emph{Proof:}
Given $w_2\in W_2$, the map $\theta:W_1\rightarrow W_2$ defined by
$\theta(w_1)=w_1w_2$ for all $w_1\in W_1$ is clearly a ${\mathbb C}G$-homomorphism. 
Since $W_1$ and $W_2$ have no common composition factor we
have $ Hom_{{\mathbb C}G}(W_1,W_2)=\{0\}$, and so $\theta=0$. Thus $w_1w_2=0$ for all
$w_1\in W_1$ and $w_2\in W_2$; in particular, $e_1w_2=0$ for all $w_2\in W_2$.
Similarly $e_2w_1=0$ for all $w_1\in W_1$. Since we then have
\begin{eqnarray*}
&w_1=1w_1=(e_1+e_2)w_1=e_1w_1\qquad\hbox{for all }w_1\in W_1,&\\
&w_2=1w_2=(e_1+e_2)w_2=e_2w_2\qquad\hbox{for all }w_2\in W_2,&
\end{eqnarray*}
the result follows.
\end{quote}
{\bf Corollary:} ${e_1}^2=e_1$, ${e_2}^2=e_2$ and
$e_1e_2=e_2e_1=0$.
\begin{quote}
\emph{Proof:}
Take $w_1=e_1$ and $w_2=e_2$ in the previous result.
\end{quote}
{\bf Definition:}
An element $e$ with $e^2=e$ is called an \emph{idempotent},
from the Latin for ``same power''.
\\
\\
{\bf Theorem 42:} If $\chi$ is the character of $W_1$, then
$$e_1=\frac{1}{|G|}\sum_{g\in G}\chi(g^{-1})g.$$
\begin{quote}
\emph{Proof:}
Let $x \in G$; then the map $\theta: {\mathbb C}G \rightarrow {\mathbb C}G$ given by
$\theta(w)=x^{-1}e_1w$ is linear. We shall calculate the trace of
$\theta$ (i.e., the trace of the matrix of $\theta$ with respect to
any basis of ${\mathbb C}G$) in two ways; comparing the two answers will give the
result.
\\
\\
First, for any $w_1\in W_1$ and $w_2\in W_2$ we have
$$\theta(w_1)=x^{-1}e_1w_1=x^{-1}w_1,\qquad\theta(w_2)=x^{-1}e_1w_2=0$$.
Writing elements of $W_1\oplus W_2$ as
ordered pairs $(w_1,w_2)$, we have
$\theta(w_1,w_2)=(\theta_1(w_1),\theta_2(w_2))$ where
$\theta_1(w_1)=x^{-1}w_1$ and $\theta_2=0$. Hence
$$Tr(\theta) =Tr(\theta_1) +Tr(\theta_2) =\chi(x^{-1}).$$
Secondly, write $e_1=\sum_{g\in G}\lambda_gg$. 
The endomorphism of ${\mathbb C}G$ which sends any element $w$ to $x^{-1}gw$ has trace
$|G|$ if $x^{-1}g=1$, i.e., if $g=x$, and $0$ otherwise. Thus as
$\theta(w)=\sum_{g\in G}\lambda_gx^{-1}gw$ for all $w\in{\mathbb C}G$, we see that
$$Tr(\theta)=\lambda_x|G|.$$
Comparing the two expressions gives $\lambda_x=\frac{1}{|G|}\chi(x^{-1})$,
and so
$$e_1=\frac{1}{|G|}\sum_{g\in G}\chi(g^{-1})g.$$
\end{quote}
{\bf Example:} With $G=C_3$ and $W_1$, $e_1$ as above, the character
$\chi$ of $W_1$ is given by $\chi(1)=1$, $\chi(a)=\omega$ and
$\chi(a^2)=\omega^2$; thus $\frac{1}{|G|}\sum_{g\in G}\chi(g^{-1})g
=\frac{1}{3}(1+\omega^2a+\omega a^2)=e_1$.
\\
\\
{\bf Corollary:} If $\chi$ is the character of $W_1$, then
$\langle\chi,\chi\rangle=\chi(1)$.
\begin{quote}
\emph{Proof:}
We calculate the coefficient of the basis element $1$ in ${e_1}^2$.
We have
$${e_1}^2=\frac{1}{|G|^2}\sum_{g,h\in G}\chi(g^{-1})\chi(h^{-1})gh$$
and so the coefficient of $1$ is
$$\frac{1}{|G|^2}\sum_{g\in G}\chi(g^{-1})\chi(g)
=\frac{1}{|G|}\langle\chi,\chi\rangle.$$
On the other hand, we have ${e_1}^2=e_1$ 
and again the coefficient of $1$ in $e_1$ is $\frac{1}{|G|}\chi(1)$. Thus
$\langle\chi,\chi\rangle=\chi(1)$ as required.
\end{quote}
{\bf Theorem 43:} Let $U$ and $V$ be non-isomorphic irreducible
${\mathbb C}G$-modules, with characters $\chi$ and $\psi$; then
$\langle\chi,\chi\rangle=1$ and $\langle\chi,\psi\rangle=0$.
\begin{quote}
\emph{Proof:}
We know that ${\mathbb C}G=U_1\oplus\cdots\oplus U_r$ with the $U_i$ irreducible
${\mathbb C}G$-submodules of ${\mathbb C}G$; let $dim (U)=m$ and $dim(V)=n$, so 
there are precisely $m$ terms $U_i$ which are isomorphic to $U$ and
$n$ which are isomorphic to $V$. We shall apply the corollary to two different
decompositions ${\mathbb C}G=W_1\oplus W_2$ in which $W_1$ and $W_2$ have no
common composition factor.
\\
\\
First, let $W_1$ be the sum of the $U_i$ which are isomorphic to
$U$, and $W_2$ be the sum of the rest. The character of $W_1$ is
$m\chi$, since $W_1$ is the direct sum of $m$ ${\mathbb C}G$-submodules each having
character $\chi$. Thus by the corollary,  we have
$$m\chi(1)=\langle m\chi,m\chi\rangle=m^2\langle\chi,\chi\rangle;$$
since $\chi(1)=dim(U) =m$ we have $\langle\chi,\chi\rangle=1$.
\\
\\
Next, let $W_1$ be the sum of the $U_i$ which are isomorphic
to either $U$ or $V$, and $W_2$ be the sum of the rest. The
character of $W_1$ is $m\chi+n\psi$. The corollary gives
$$m\chi(1)+n\psi(1)=\langle m\chi+n\psi,m\chi+n\psi\rangle
=m^2\langle\chi,\chi\rangle+n^2\langle\psi,\psi\rangle
+mn(\langle\chi,\psi\rangle+\langle\psi,\chi\rangle).$$
Since $\langle\chi,\chi\rangle=\langle\psi,\psi\rangle=1$ by the above, and
$\chi(1)=m$ and $\psi(1)=n$, we have
$$\langle\chi,\psi\rangle+\langle\psi,\chi\rangle=0;$$
as $\langle\chi,\psi\rangle=\langle\psi,\chi\rangle$,
we must have $\langle\chi,\psi\rangle=0$.
\end{quote}
{\bf Example:} Let $G=D_6$, so that the irreducible characters 
$\chi_1,\chi_2,\chi_3$ are as follows.
$$
\begin{array}{|c|ccc|}
\hline
g&1&a&b\\
\hline
|C_G(g)|&6&3&2\\
\hline
\chi_1&1&1&1\\
\chi_2&1&1&-1\\
\chi_3&2&-1&0\\
\hline
\end{array}
$$
Thus we have
$$
\begin{array}{ll}
\langle\chi_1,\chi_1\rangle=\frac{1.1}{6}+\frac{1.1}{3}+\frac{1.1}{2}=1,
&\langle\chi_1,\chi_2\rangle=\frac{1.1}{6}+\frac{1.1}{3}+\frac{1.(-1)}{2}=0,\\
\langle\chi_2,\chi_2\rangle=\frac{1.1}{6}+\frac{1.1}{3}+\frac{(-1).(-1)}{2}=1,
&\langle\chi_1,\chi_3\rangle=\frac{1.2}{6}+\frac{1.(-1)}{3}+\frac{1.0}{2}=0,\\
\langle\chi_3,\chi_3\rangle=\frac{2.2}{6}+\frac{(-1).(-1)}{3}+\frac{0.0}{2}=1,
&\langle\chi_2,\chi_3\rangle=\frac{1.2}{6}+\frac{1.(-1)}{3}+\frac{(-1).0}{2}=0.
\end{array}
$$
{\bf Theorem 44:} If $\chi$ is any character of $G$, then
$\chi=d_1\chi_1+\cdots+d_k\chi_k$ for some non-negative integers
$d_1,\dots,d_k$; moreover $d_i=\langle\chi,\chi_i\rangle$ for $1\leq i\leq k$,
and $\langle\chi,\chi\rangle=\sum_{i=1}^k{d_i}^2$.
\begin{quote}
\emph{Proof:}
Let $V$ be a ${\mathbb C}G$-module with character $\chi$.
$V$ is a
direct sum of irreducible ${\mathbb C}G$-submodules, each of which is isomorphic to some
$V_i$; thus there exist non-negative integers $d_1,\dots,d_k$ such that
$$V\cong\underbrace{(V_1\oplus\cdots\oplus V_1)}_{d_1\ \mathrm{terms}}\oplus
\cdots\oplus\underbrace{(V_k\oplus\cdots\oplus V_k)}_{d_k\ \mathrm{terms}}.$$
Thus the character $\chi$ is given by $\chi=d_1\chi_1+\cdots+d_k\chi_k$. Taking
inner products now gives
\begin{eqnarray*}
\langle\chi,\chi_i\rangle
\!\!\!\!&=&\!\!\!\!\langle d_1\chi_1+\cdots+d_k\chi_k,\chi_i\rangle
=\sum_{j=1}^kd_j\langle\chi_j,\chi_i\rangle=d_i,\\
\langle\chi,\chi\rangle
\!\!\!\!&=&\!\!\!\!\langle d_1\chi_1+\cdots+d_k\chi_k,
d_1\chi_1+\cdots+d_k\chi_k\rangle
=\sum_{i=1}^k\sum_{j=1}^kd_id_j\langle\chi_i,\chi_j\rangle
=\sum_{i=1}^k{d_i}^2
\end{eqnarray*}
as required.
\end{quote}
{\bf Remark:}
Thus if we are given any character $\chi$, we can write it as a linear
combination of the irreducible characters, and the coefficients can be found
simply by calculating inner products. This result motivates the following
definition.
\\
\\
{\bf Burnside's Algorithm:} Let the conjugacy classes of a finite group $G$ be
$C_1, C_2, \ldots , C_r$. $C_i C_j = \sum_{s=1}^r c_{ijs} C_s$.  Thus
$({\frac {|C_i| \chi_k(g_i)} {\chi_k(1)}}) ({\frac {|C_j| \chi_k(g_j)} {\chi_k(1)}}) =
\sum_{s=1}^r c_{ijs} ({\frac {|C_s| \chi_k(g_s)} {\chi_k(1)}})$.  Multiply this
equation by $a_{ki}$ and sum over $i$ to get:
$(\sum_{i=1}^r a_{ki} {\frac {|C_i| \chi_k(g_i)} {\chi_k(1)}})
({\frac {|C_j| \chi_k(g_j)} {\chi_k(1)}}) =
\sum_{s=1}^r (\sum_{i=1}^r a_{ki} c_{ijs}) {\frac {|C_s| \chi_k(g_s)} {\chi_k(1)}}$
Put $Y_{ki} = {\frac {|C_i| \chi_k(g_i)} {\chi_k(1)}}$,
$A_k = \sum_{i=1}^r a_{ki} Y_{ki}$ and
$B_{js}^{(k)} = \sum_{i=1}^r c_{ijs} a_{ki}$, for $k = 1,2, \ldots, r$.
Then, 
\\
$
\left(\begin{array} {ccc}
B_{11}^{(k)} & \ldots  &  B_{1r}^{(k)}\\
\ldots & \ldots & \ldots \\
B_{r1}^{(k)} & \ldots  &  B_{rr}^{(k)}\\
\end{array}\right) 
\left(\begin{array} {c}
Y_{k1} \\
Y_{k2} \\
\ldots \\
Y_{kr} \\
\end{array}\right) =
A_k
\left(\begin{array} {c}
Y_{k1} \\
Y_{k2} \\
\ldots \\
Y_{kr} \\
\end{array}\right)$\\
So, we can solve for the $Y_{kj}$ by computing the eigenvalues of 
$det(B_{is}^{(k)} - \lambda I)$ and finding the corresponding eigenvectors.
\\
\\
\emph{Example, $S_3$}:\\
\\
Let $C_1 = (1)$, $C_2 = (123) + (132)$ and $C_3 = (12) + (13) + (23)$.
$C_1 C_1 = C_1$,
$C_1 C_2 = C_2 C_1 = C_2$,
$C_1 C_3 = C_3 C_1 = C_3$,
$C_2 C_3 = C_3 C_2 = 2 C_3$,
$C_2 C_2 = C_2 C_2 = 2 C_1 + C_2$, and
$C_3 C_3 = C_3 C_3 = 3 C_1 + 3C_3$.
\\
The matricies $m_1 = (C_{1ij})$, $m_2 = (C_{2ij})$, and
$m_3 = (C_{3ij})$ are:\\
$m_1 = \left(\begin{array} {ccc}
1 & 0 & 0 \\
0 & 1 & 0 \\
0 & 0 & 1 \\
\end{array}\right)$\\\\
$m_2 = \left(\begin{array} {ccc}
0 & 1 & 0 \\
2 & 1 & 0 \\
0 & 0 & 2 \\
\end{array}\right)$\\
\\
$m_3 = \left(\begin{array} {ccc}
0 & 0 & 1 \\
0 & 0 & 2 \\
3 & 3 & 0 \\
\end{array}\right)$\\
\\
$S_3$ character table is
$\left(\begin{array} {ccc}
1 & 1 & 1 \\
1 & 1 & -1 \\
2 & -1 & 0 \\
\end{array}\right)$\\
\\
This is computed as follows:  The eigenvalues of the respective matrix are $1$; $2, 1, -1$; and
$3, -3, 0$.  The respective eigenvectors are
$\left(\begin{array} {c}
1 \\
0 \\
0 \\
\end{array}\right)$,
$\left(\begin{array} {c}
0 \\
1 \\
0 \\
\end{array}\right)$, and
$\left(\begin{array} {c}
0\\
0\\
1\\
\end{array}\right)$;
$\left(\begin{array} {c}
0\\
0\\
1\\
\end{array}\right)$,
$\left(\begin{array} {c}
1\\
-1\\
0\\
\end{array}\right)$, and,
$\left(\begin{array} {c}
1\\
2\\
0\\
\end{array}\right)$; and, finally,
$\left(\begin{array} {c}
1\\
2\\
3\\
\end{array}\right)$,
$\left(\begin{array} {c}
1\\
2\\
-3\\
\end{array}\right)$.\\
Consider three of the eigenvalues
$
\left(\begin{array} {c}
{\frac {|C_1| \chi_1(1)} {\chi_1(1)}}\\
{\frac {|C_2| \chi_1((123))} {\chi_1(1)}}\\
{\frac {|C_3| \chi_1((12))} {\chi_1(1)}}\\
\end{array}\right) =
\left(\begin{array} {c}
1\\
2\\
3\\
\end{array}\right)
$,
$
\left(\begin{array} {c}
{\frac {|C_1| \chi_2(1)} {\chi_2(1)}}\\
{\frac {|C_2| \chi_2((123))} {\chi_2(1)}}\\
{\frac {|C_3| \chi_2((12))} {\chi_2(1)}}\\
\end{array}\right) =
\left(\begin{array} {c}
1\\
2\\
-3\\
\end{array}\right)
$, and,
$
\left(\begin{array} {c}
{\frac {|C_1| \chi_3(1)} {\chi_3(1)}}\\
{\frac {|C_2| \chi_3((123))} {\chi_3(1)}}\\
{\frac {|C_3| \chi_3((12))} {\chi_3(1)}}\\
\end{array}\right) =
\left(\begin{array} {c}
1\\
-1\\
0\\
\end{array}\right)
$.  Remembering 
$|C_1| = 1$,
$|C_2| = 2$, and
$|C_3| = 3$, we can
solve for the $\chi_k(g_i)$ giving the character table entries.
\\
\\
{\bf Definition:} If $\chi$ is a character of $G$ and we write
$\chi=d_1\chi_1+\cdots+d_k\chi_k$, then we call the irreducible character
$\chi_i$ a \emph{constituent} of $\chi$ if the coefficient $d_i$ is
non-zero.
Thus $\chi_i$ is a constituent of $\chi$ if and only if
$\langle\chi,\chi_i\rangle>0$.
\\
\\
{\bf Theorem 45:} If $V$ is a ${\mathbb C}G$-module with character $\chi$,
then $V$ is irreducible if and only if $\langle\chi,\chi\rangle=1$.
\begin{quote}
\emph{Proof:}
If $V$ is irreducible, then $\langle\chi,\chi\rangle=1$.
Conversely, assume that $\langle\chi,\chi\rangle=1$; write
$$V\cong\underbrace{(V_1\oplus\cdots\oplus V_1)}_{d_1\ \mathrm{terms}}\oplus
\cdots\oplus\underbrace{(V_k\oplus\cdots\oplus V_k)}_{d_k\ \mathrm{terms}},$$
so that $\chi=d_1\chi_1+\cdots+d_k\chi_k$, and then we have
$$1=\langle\chi,\chi\rangle={d_1}^2+\cdots+{d_k}^2.$$
As the $d_i$ are non-negative integers, one (say $d_j$) must be $1$ and the
remainder $0$; then $V\cong V_j$ and so $V$ is irreducible.
\end{quote}
{\bf Theorem 46:} If $V$ and $W$ are ${\mathbb C}G$-modules, with
characters $\chi$ and $\psi$ respectively, then $V\cong W$ if and only if
$\chi=\psi$.
\begin{quote}
\emph{Proof:}
We know that if $V\cong W$ then
$\chi=\psi$; it is the converse which we must show. Thus we assume that
$\chi=\psi$, and seek to show that $V\cong W$. There are non-negative integers
$c_1,\dots,c_k$ such that
$$V\cong\underbrace{(V_1\oplus\cdots\oplus V_1)}_{c_1\ \mathrm{terms}}\oplus
\cdots\oplus\underbrace{(V_k\oplus\cdots\oplus V_k)}_{c_k\ \mathrm{terms}},$$
and similarly $d_1,\dots,d_k$ such that
$$W\cong\underbrace{(V_1\oplus\cdots\oplus V_1)}_{d_1\ \mathrm{terms}}\oplus
\cdots\oplus\underbrace{(V_k\oplus\cdots\oplus V_k)}_{d_k\ \mathrm{terms}}.$$
Since $\chi=\psi$, for $1\leq i\leq k$ we have
$$c_i=\langle\chi,\chi_i\rangle=\langle\psi,\chi_i\rangle=d_i,$$
and so $V\cong W$ as required.
\end{quote}
{\bf Theorem 47:} The irreducible characters
$\chi_1,\dots,\chi_k$ are linearly independent vectors in the vector
space of functions $G\rightarrow{\mathbb C}$.
\begin{quote}
\emph{Proof:}
Assume that $\lambda_1,\dots,\lambda_k\in{\mathbb C}$ with
$$\lambda_1\chi_1+\dots+\lambda_k\chi_k=0.$$
Taking inner products with $\chi_i$ gives
$$0=\langle\lambda_1\chi_1+\dots+\lambda_k\chi_k,\chi_i\rangle=\lambda_i;$$
since this is true for all $1\leq i\leq k$, we see that the $\chi_i$ are
linearly independent.
\end{quote}
{\bf Theorem 48:} If $V$ and $W$ are ${\mathbb C}G$-modules with characters
$\chi$ and $\psi$, then we have
$dim( Hom_{{\mathbb C}G}(V,W))=\langle\chi,\psi\rangle$.
\begin{quote}
\emph{Proof:}
As before there exist non-negative integers $c_1,\dots,c_k$ and
$d_1,\dots,d_k$ such that

\begin{eqnarray*}
&V\cong\underbrace{(V_1\oplus\cdots\oplus V_1)}_{c_1\ \mathrm{terms}}
\oplus\cdots\oplus
\underbrace{(V_k\oplus\cdots\oplus V_k)}_{c_k\ \mathrm{terms}},&\\
&W\cong\underbrace{(V_1\oplus\cdots\oplus V_1)}_{d_1\ \mathrm{terms}}
\oplus\cdots\oplus
\underbrace{(V_k\oplus\cdots\oplus V_k)}_{d_k\ \mathrm{terms}}.&
\end{eqnarray*}
So
\begin{eqnarray*}
dim( Hom_{{\mathbb C}G}(V,W))
\!\!\!\!&=&\!\!\!\!\sum_{i=1}^k\sum_{j=1}^kc_id_j dim( Hom_{{\mathbb C}G}(V_i,V_j))\\
\!\!\!\!&=&\!\!\!\!\sum_{i=1}^k\sum_{j=1}^kc_id_j\delta_{ij}
=\sum_{i=1}^kc_id_i.
\end{eqnarray*}

On the other hand, we have $\chi=\sum_{i=1}^kc_i\chi_i$ and
$\psi=\sum_{j=1}^kd_j\chi_j$, and so
$$\langle\chi,\psi\rangle
=\sum_{i=1}^k\sum_{j=1}^kc_id_j\langle\chi_i,\chi_j\rangle
=\sum_{i=1}^k\sum_{j=1}^kc_id_j\delta_{ij}=\sum_{i=1}^kc_id_i.$$
The result follows.
\end{quote}

\subsection{Center of the group algebra}
{\bf Definition:} The \emph{center} of the group algebra ${\mathbb C}G$ is the
subspace
$${\mathbb Z}({\mathbb C}G)=\{z\in{\mathbb C}G:zr=rz\hbox{ for all }r\in{\mathbb C}G\}.$$
\\
\\
{\bf Definition:} Let $C_1,\dots,C_\ell$ be the distinct conjugacy
classes of $G$, and for $1\leq i\leq\ell$ define
$$\bar C_i=\sum_{g\in C_i}g\in{\mathbb C}G;$$
the elements $\bar C_1,\dots,\bar C_\ell$ of ${\mathbb C}G$ are called the \emph{class
sums}.
\\
\\
{\bf Theorem 49:} The class sums
$\bar C_1,\dots,\bar C_\ell$ form a basis of ${\mathbb Z}({\mathbb C}G)$.
\begin{quote}
\emph{Proof:}
We first show that each $\bar C_i$ lies in ${\mathbb Z}({\mathbb C}G)$. Let $g\in C_i$, and
set $|C_i|=r$; write
$$C_i=\{{y_1}^{-1}gy_1,{y_2}^{-1}gy_2,\dots,{y_r}^{-1}gy_r\}$$
for some $y_1,\dots,y_r\in G$, so that
$\bar C_i=\sum_{j=1}^r{y_j}^{-1}gy_j$. For all $h\in G$, we have
$$h^{-1}\bar C_ih=\sum_{j=1}^rh^{-1}{y_j}^{-1}gy_jh
=\sum_{j=1}^r(y_jh)^{-1}g(y_jh).$$
This is a sum of $r$ conjugates of $g$; and they are distinct, since
$$h^{-1}{y_j}^{-1}gy_jh=h^{-1}{y_k}^{-1}gy_kh
\iff{y_j}^{-1}gy_j={y_k}^{-1}gy_k.$$
Thus $h^{-1}\bar C_ih=\bar C_i$, and so $\bar C_ih=h\bar C_i$; since this is
true for all $h\in G$, we see that $\bar C_i$ commutes with all $r\in{\mathbb C}G$,
i.e., $\bar C_i\in {\mathbb Z}({\mathbb C}G)$.
\\
\\
Now for $1\leq i\leq\ell$ let $g_i$ be a representative of $C_i$.
It is clear that the $\bar C_i$ are linearly independent,
since if $\sum_{i=1}^\ell\lambda_i\bar C_i=0$ then considering the
coefficient of $g_i$ shows that $\lambda_i=0$ for all $i$. Thus we must
show that the $\bar C_i$ span ${\mathbb Z}({\mathbb C}G)$. Let
$z=\sum_{g\in G}\lambda_gg\in {\mathbb Z}({\mathbb C}G)$. For
all $h\in G$ we have $zh=hz$, and so $h^{-1}zh=z$, i.e.,
$$\sum_{g\in G}\lambda_gh^{-1}gh=\sum_{g\in G}\lambda_gg.$$
Since the coefficient of $g$ in the sum on the left is $\lambda_{hgh^{-1}}$,
we must have
$$\lambda_{hgh^{-1}}=\lambda_g\qquad\hbox{for all }g,h\in G;$$
i.e., the coefficients in $z$ of two conjugate elements $g$ and $hgh^{-1}$ are
equal. Thus we have $z=\sum_{i=1}^\ell\lambda_{g_i}\bar C_i$; so the
$\bar C_i$ do indeed span ${\mathbb Z}({\mathbb C}G)$ as required.
\end{quote}
{\bf Examples:}
\begin{itemize}
\item[(i)] Let $G=S_3$, then a basis of ${\mathbb Z}({\mathbb C}G)$ is
$$1,\qquad(1\;2)+(1\;3)+(2\;3),\qquad(1\;2\;3)+(1\;3\;2).$$
\item[(ii)] Let $G=D_8$, then a basis of ${\mathbb Z}({\mathbb C}G)$ is
$$1,\qquad a^2,\qquad a+a^3,\qquad b+ba^2,\qquad ba+ba^3.$$
\end{itemize}
{\bf Theorem 50:} If $V$ is an irreducible ${\mathbb C}G$-module
and $z\in {\mathbb Z}({\mathbb C}G)$, then there exists $\lambda\in{\mathbb C}$ such that
$zv=\lambda v$ for all $v\in V$.
\begin{quote}
\emph{Proof:}
For all $r\in{\mathbb C}G$ and $v\in V$, we have
$$z(rv)=r(zv);$$
thus the map $\theta:V\rightarrow V$ defined by $\theta(v)=zv$ is a
${\mathbb C}G$-homomorphism. By a previous Lemma (ii), $\theta=\lambda1_V$ for some
$\lambda\in{\mathbb C}$, i.e., $zv=\lambda v$ for all $v\in V$.
\end{quote}
{\bf Observation:}
We may now give our second basis of ${\mathbb Z}({\mathbb C}G)$.
Recall that we have the complete set of non-isomorphic
irreducible ${\mathbb C}G$-modules $V_1,\dots,V_k$. We write
$${\mathbb C}G=W_1\oplus\cdots\oplus W_k,$$
where each $W_i$ is isomorphic to a direct sum of copies of $V_i$; the summands
$W_i$ are called the \emph{homogeneous components} of ${\mathbb C}G$. We set
$$1=e_1+\cdots+e_k,$$
where $e_i\in W_i$ for $1\leq i\leq k$.
\\
\\
{\bf Theorem 51:} The elements $e_1,\dots,e_k$ form a basis
of ${\mathbb Z}({\mathbb C}G)$.
\begin{quote}
\emph{Proof:}
We begin by showing that each $e_i\in {\mathbb Z}({\mathbb C}G)$; it clearly suffices to
consider the case $i=1$. If we set $X=W_2\oplus\cdots\oplus W_k$, we have
$${\mathbb C}G=W_1\oplus X,$$
and $W_1$ and $X$ have no common composition factor; thus,
we see that $w_1x=0=xw_1$ for all $w_1\in W_1$ and $x\in X$.
Thus if we set $e=e_2+\cdots+e_k\in X$, for all $w_1\in W_1$ we have
$$w_1=w_11=w_1(e_1+e)=w_1e_1+w_1e=w_1e_1.$$
It follows that we have $e_1w_1=w_1=w_1e_1$ for all $w_1\in W_1$, and
$e_1x=0=xe_1$ for all $x\in X$;
since ${\mathbb C}G=W_1\oplus X$ we see that $e_1\in {\mathbb Z}({\mathbb C}G)$ as required.
\\
\\
Now the $e_i$ are certainly linearly independent, as the sum
of the $W_i$ is direct. To show that they span ${\mathbb Z}({\mathbb C}G)$, take
$z\in {\mathbb Z}({\mathbb C}G)$; then for $1\leq i\leq k$ there
exists $\lambda_i\in{\mathbb C}$ such that
$$zv=\lambda_iv\qquad\hbox{for all }v\in V_i.$$
Hence $zw=\lambda_iw$ for all $w\in W_i$, and in particular
$ze_i=\lambda_ie_i$, for $1\leq i\leq k$; thus
$$z=z1=z(e_1+\cdots+e_k)=ze_1+\cdots+ze_k=\lambda_1e_1+\cdots+\lambda_ke_k.$$
Therefore $z$ is a linear combination of the $e_i$, so $e_1,\dots,e_k$ do
indeed span ${\mathbb {\mathbb {\mathbb {\mathbb Z}}}}({\mathbb C}G)$; 
thus they form a basis of ${\mathbb Z}({\mathbb C}G)$ as required.
\end{quote}
{\bf Corollary:} The number of irreducible characters of $G$ is
equal to the number of conjugacy classes of $G$.
\begin{quote}
\emph{Proof:}
We have $k= dim({\mathbb Z}({\mathbb C}G))= \ell$.
\end{quote}
{\bf Example:} Let $G=D_6$; we have
${\mathbb C}G=U_1\oplus U_2\oplus U_3\oplus U_4$ with $U_1$ the trivial ${\mathbb C}G$-module,
$U_2$ the other $1$-dimensional ${\mathbb C}G$-module, and $U_3\cong U_4$ with
$dim (U_3)= dim(U_4) = 2$. We therefore have $W_1=U_1$, $W_2=U_2$ and
$W_3=U_3\oplus U_4$; the elements $e_i$ are
$$e_1={\ts\frac{1}{6}}(1+a+a^2+b+ba+ba^2),\quad
e_2={\ts\frac{1}{6}}(1+a+a^2-b-ba-ba^2),\quad
e_3={\ts\frac{1}{3}}(2.1-a-a^2).$$
If we consider $e_3$, we have
$$e_3a={\ts\frac{1}{3}}(2a-a^2-1)=ae_3,\qquad
e_3b={\ts\frac{1}{3}}(2b-ab-a^2b)={\ts\frac{1}{3}}(2b-ba^2-ba)=be_3;$$
as $a$ and $b$ generate $G$ we see that $e_3\in {\mathbb Z}({\mathbb C}G)$.
\subsection{The space of class functions}
{\bf Definition:} A \emph{class function} on $G$ is a function
$\psi:G\rightarrow{\mathbb C}$ with the property that $\psi(g)=\psi(h)$ if
$g,h\in G$ are conjugate.  The set of class functions on $G$ is written $ cl$. 
It is clear that $ cl$ is
a subspace of the vector space of all functions $G\rightarrow{\mathbb C}$. Moreover, it
is easy to provide a basis of $ cl$.
\\
\\
{\bf Definition:} If $C$ is a conjugacy class of $G$, the function
$\psi_C:G\rightarrow{\mathbb C}$ defined by
$\psi_C(g)=1, g\in C$,
$\psi_C(g)=0, g \notin C$
is called the \emph{characteristic function} of the class $C$.
\\
\\
{\bf Example:} Let $G=D_6$, and write $C_1=\{1\}$,
$C_2=\{a,a^2\}$ and $C_3=\{b,ba,ba^2\}$ as before; then the
characteristic functions $\psi_{C_i}$ are as follows.
$$
\begin{array}{|c|cccccc|}
\hline
&1&a&a^2&b&ba&ba^2\\
\hline
\psi_{C_1}&1&0&0&0&0&0\\
\psi_{C_2}&0&1&1&0&0&0\\
\psi_{C_3}&0&0&0&1&1&1\\
\hline
\end{array}
$$
\\
\\
{\bf Theorem 52:} The characteristic functions
$\psi_{C_1},\dots,\psi_{C_\ell}$ form a basis of $ cl$.
\begin{quote}
\emph{Proof:}
It is clear that $\psi_{C_i}\in cl$ for $1\leq i\leq\ell$. If
$\lambda_1\psi_{C_1}+\cdots+\lambda_\ell\psi_{C_\ell}=0$, evaluating at
$g_j$ gives $\lambda_j=0$; as this is true for all $1\leq j\leq\ell$, the
$\psi_{C_i}$ are linearly independent. Given $\psi\in cl$, set
$\lambda_i=\psi(g_i)$ for all $i$; then
$\sum_{i=1}^\ell\lambda_i\psi_{C_i}$ is a class function agreeing
with $\psi$ on all $g_i$, so it must be equal to it. Thus the
$\psi_{C_i}$ span $ cl$; so they form a basis as required.
\end{quote}
{\bf Example:} If $G=D_6$ as above, the class function
$\psi$ given by $\psi(1)=3$, $\psi(a)=\psi(a^2)=1$ and
$\psi(b)=\psi(ba)=\psi(ba^2)=0$ is equal to $3\psi_{C_1}+\psi_{C_2}$.
\\
\\
{\bf Theorem 53:} The irreducible characters
$\chi_1,\dots,\chi_k$ form a basis of $ cl$; indeed if
$\psi$ is a class function then $\psi=\sum_{i=1}^k\lambda_i\chi_i$, where
$\lambda_i=\langle\psi,\chi_i\rangle$ for $1\leq i\leq k$.
\begin{quote}
\emph{Proof:}
The $\chi_i$ are linearly independent, so they
span a subspace of $ cl$ of dimension $k$; since $dim(cl) = \ell =k$,
they form a basis of $ cl$. Given $\psi\in cl$ we may therefore write
$\psi=\sum_{i=1}^k\lambda_i\chi_i$ for some $\lambda_i\in{\mathbb C}$; since
$\langle\chi_i,\chi_j\rangle=\delta_{ij}$, taking inner products with
$\chi_i$ gives $\langle\psi,\chi_i\rangle=\lambda_i$ as required.
\end{quote}
{\bf Corollary:} If $g,h\in G$, then $g$ is conjugate
to $h$ if and only if $\chi(g)=\chi(h)$ for all characters $\chi$ of $G$.
\begin{quote}
\emph{Proof:}
If $g$ is conjugate to $h$ then $\chi(g)=\chi(h)$ for
all characters $\chi$ of $G$. Conversely, if $\chi(g)=\chi(h)$ for all
characters $\chi$, then by the Theorem, we have $\psi(g)=\psi(h)$ for all
$\psi\in cl$. In particular, this is true for the characteristic function
$\psi_C$ of the class $C$ containing $g$; thus $\psi_C(h)=\psi_C(g)=1$, and so
$h\in C$, i.e., $h$ lies in the same conjugacy class as $g$.
\end{quote}
{\bf Corollary:} If $g\in G$, then $g$ is conjugate to
$g^{-1}$ if and only if $\chi(g)\in {\mathbb R}$ for all characters $\chi$ of $G$.
\begin{quote}
\emph{Proof:}
Since $\chi(g)\in {\mathbb R}$ if and only if
$\chi(g)=\overline{\chi(g)}=\chi(g^{-1})$, the
result follows immediately from the corollary.
\end{quote}
\subsection{Character tables and orthogonality relations}
{\bf Definition:} The $k\times k$ matrix with $(i,j)$-entry
$\chi_i(g_j)$ is called the \emph{character table} of $G$.
\\
\\
{\bf Definition:} The relations
$$\sum_{i=1}^k\frac{\chi_r(g_i)\overline{\chi_s(g_i)}}{|C_G(g_i)|}=\delta_{rs}$$
are called the \emph{row orthogonality relations} for $G$.
\\
\\
{\bf Definition:} The relations
$$\sum_{i=1}^k\chi_i(g_r)\overline{\chi_i(g_s)}=\delta_{rs}|C_G(g_r)|$$
are called the \emph{column orthogonality relations} for $G$.
\\
\\
{\bf Theorem 54:} The character table for $G$ satisfies the row
and column orthogonality relations.
\begin{quote}
\emph{Proof:}
We already know that the row orthogonality relations hold. For
$1\leq s\leq k$, we may write the characteristic function $\psi_{C_s}$ as a
linear combination of $\chi_1,\dots,\chi_k$; say
$\psi_{C_s}=\lambda_1\chi_1+\cdots+\lambda_k\chi_k$.
Taking inner products with $\chi_i$ gives
$$\lambda_i=\langle\psi_{C_s},\chi_i\rangle
=\frac{1}{|G|}\sum_{g\in G}\psi_{C_s}(g)\overline{\chi_i(g)}.$$
Now the only elements $g$ for which $\psi_{C_s}(g)\neq0$ are those lying in the
conjugacy class $C_s$; since there are $|G|/|C_G(g_s)|$ of
them, each having $\psi_{C_s}(g)=1$, we have
$$\lambda_i=\frac{1}{|G|}\sum_{g\in C_s}\psi_{C_s}(g)\overline{\chi_i(g)}
=\frac{\overline{\chi_i(g_s)}}{|C_G(g_s)|}.$$
Thus
$$\delta_{rs}=\psi_{C_s}(g_r)=\sum_{i=1}^k\lambda_i\chi_i(g_r)
=\sum_{i=1}^k\frac{\chi_i(g_r)\overline{\chi_i(g_s)}}{|C_G(g_r)|}.$$
\end{quote}
{\bf Examples:}
\begin{itemize}
\item[(i)] Let $G=D_6$.
$$\sum_{i=1}^3\chi_i(g_1)\overline{\chi_i(g_1)}=1.1+1.1+2.2=6,\quad
\sum_{i=1}^3\chi_i(g_1)\overline{\chi_i(g_2)}=1.1+1.1+2.(-1)=0.$$
\item[(ii)] Suppose $G$ is a group of order $12$ with four conjugacy classes,
and we are given the following part of the character table, in which
$\omega=e^{2\pi i/3}$.
$$
\begin{array}{|c|cccc|}
\hline
g&g_1&g_2&g_3&g_4\\
\hline
|C_G(g)|&12&4&3&3\\
\hline
\chi_1&1&1&1&1\\
\chi_2&1&1&\omega&\omega^2\\
\chi_3&1&1&\omega^2&\omega\\
\chi_4&&&&\\
\hline
\end{array}
$$
We use the column orthogonality relations to determine the
final row of the table. The entries in the first column are the degrees of
the $\chi_i$, so they are all positive integers.  The relation
with $r=s=1$ implies that the sum of the squares of these entries is $12$, and
so $\chi_4(g_1)=3$. Next, the relation with $r=1$ and $s=2$ yields
$$0=1.1+1.1+1.1+3\overline{\chi_4(g_2)};$$
so $\chi_4(g_2)=-1$. Similarly, those with $r=1$ and $s=3$ or $4$ give
$\chi_4(g_3)=0=\chi_4(g_4)$. Thus the full character table is:
$$
\begin{array}{|c|cccc|}
\hline
g&g_1&g_2&g_3&g_4\\
\hline
|C_G(g)|&12&4&3&3\\
\hline
\chi_1&1&1&1&1\\
\chi_2&1&1&\omega&\omega^2\\
\chi_3&1&1&\omega^2&\omega\\
\chi_4&3&-1&0&0\\
\hline
\end{array}
$$
\end{itemize}
{\bf Theorem 55:} If $N\norm G$ and $\tilde\chi$ is a character
of $G/N$, then the function $\chi:G\rightarrow{\mathbb C}$ defined by
$\chi(g)=\tilde\chi(gN)$ for all $g\in G$ is a character of $G$; the characters
$\chi$ and $\tilde\chi$ have the same degree, and $\chi$ is irreducible if and
only if $\tilde\chi$ is.
\begin{quote}
\emph{Proof:}
Let $\tilde\rho:G/N\rightarrow GL_n({\mathbb C})$ be a representation of $G/N$ with
character $\tilde\chi$. Define a function $\rho:G\rightarrow GL_n({\mathbb C})$ by
$$\rho(g)=\tilde\rho(gN)\qquad\hbox{for all }g\in G;$$
then $\rho(g)\rho(h)=\tilde\rho(gN)\tilde\rho(hN)=\tilde\rho(gN.hN)
=\tilde\rho(ghN)=\rho(gh)$, so $\rho$ is a homomorphism, i.e., a representation
of $G$. The character $\chi$ of $\rho$ is given by
$$\chi(g)=Tr(\rho(g)=Tr((\tilde\rho(gN))=\tilde\chi(gN)\qquad\hbox{for all }
g\in G;$$
moreover $\chi(1)=\tilde\chi(N)$ so that $\chi$ and $\tilde\chi$ have the
same degree. Finally, let $U$ be a subspace of the vector space ${\mathbb C}^n$; then by
definition of $\rho$ we have $\rho(g)u=\tilde\rho(gN)u$ for all $g\in G$ and
$u\in U$. Thus
\begin{eqnarray*}
U\hbox{ is a ${\mathbb C}G$-submodule of }{\mathbb C}^n
\!\!\!\!&\iff&\!\!\!\!\rho(g)u\in U\hbox{ for all }g\in G,\ u\in U\\
\!\!\!\!&\iff&\!\!\!\!\tilde\rho(gN)u\in U\hbox{ for all }g\in G,\ u\in U\\
\!\!\!\!&\iff&\!\!\!\! U\hbox{ is a ${\mathbb C}(G/N)$-submodule of }{\mathbb C}^n;
\end{eqnarray*}
so the representation $\rho$ is reducible if and only if
$\tilde\rho$ is, and thus $\chi$ is irreducible if and only if
$\tilde\chi$ is.
\end{quote}
{\bf Definition:} If $N\norm G$ and $\tilde\chi$ is a character of
$G/N$, then the character $\chi$ of $G$ defined by
$$\chi(g)=\tilde\chi(gN)\qquad\hbox{for all }g\in G$$
is called the \emph{lift} of $\tilde\chi$ to $G$.
\\
\\
{\bf Example:} Let $G=S_4$; then if we write $v_{ij}$ for
$v_{\{i,j\}}$, the second permutation module for $G$ has basis
$v_{12},v_{13},v_{14},v_{23},v_{24},v_{34}$. If we take conjugacy class
representatives $1$, $(1\;2)$, $(1\;2)(3\;4)$, $(1\;2\;3)$ and $(1\;2\;3\;4)$,
then $1$ fixes all six basis elements, $(1\;2)$ and $(1\;2)(3\;4)$ each fix
just $v_{12}$ and $v_{34}$, while $(1\;2\;3)$ and $(1\;2\;3\;4)$ both fail to
fix any basis elements. Thus the values of the second permutation character
are as follows.

$$
\begin{array}{|c|ccccc|}
\hline
&1&(1\;2)&(1\;2)(3\;4)&(1\;2\;3)&(1\;2\;3\;4)\\
\hline
\pi_2&6&2&2&0&0\\
\hline
\end{array}
$$

 Together with the first permutation character, lifts from the
quotient $S_4/A_4\cong C_2$ and the orthogonality relations, this enables us
to obtain the full character table of $G$.

$$
\begin{array}{|c|ccccc|}
\hline
&1&(1\;2)&(1\;2)(3\;4)&(1\;2\;3)&(1\;2\;3\;4)\\
\hline
\chi_1&1&1&1&1&1\\
\chi_2&1&-1&1&1&-1\\
\chi_3&2&0&2&-1&0\\
\chi_4&3&1&-1&0&-1\\
\chi_5&3&-1&-1&0&1\\
\hline
\end{array}
$$
We may also define the \emph{third permutation character} of $G$ by
considering unordered triples, and so on.
\\
\\
The third method applies when we have a non-trivial linear character of
$G$.
\\
\\
{\bf Theorem 56:} If $\chi$ and $\lambda$ are characters of
$G$ and $\lambda$ is linear, then the function
$\lambda\chi:G\rightarrow{\mathbb C}$ defined by
$\lambda\chi(g)=\lambda(g)\chi(g)$ for all $g\in G$ is a character of $G$;
if $\chi$ is irreducible, so is $\lambda\chi$.
\begin{quote}
\emph{Proof:}
Let $\rho:G\rightarrow GL_n({\mathbb C})$ be a representation of $G$ with character
$\chi$, and define a map $\lambda\rho: G \rightarrow GL_n({\mathbb C})$ by
$$(\lambda\rho)(g)=\lambda(g)\rho(g)\qquad\hbox{for all }g \in G;$$
thus $(\lambda\rho)(g)$ is the matrix obtained by multiplying $\rho(g)$ by the
complex number $\lambda(g)$. For all $g,h\in G$ we have

\begin{eqnarray*}
(\lambda\rho)(gh)\!\!\!\!&=&\!\!\!\!\lambda(gh)\rho(gh)\\
\!\!\!\!&=&\!\!\!\!\lambda(g)\lambda(h)\rho(g)\rho(h)\\
\!\!\!\!&=&\!\!\!\!\lambda(g)\rho(g)\lambda(h)\rho(h)\\
\!\!\!\!&=&\!\!\!\!(\lambda\rho)(g)(\lambda\rho)(h),
\end{eqnarray*}

 and so $\lambda\rho$ is a homomorphism, i.e., a
representation of $G$. The trace
of the matrix $(\lambda\rho)(g)$ is $\lambda(g)Tr(\rho(g)=\lambda(g)\chi(g)$;
thus $\lambda\chi$ is the character of the representation $\lambda\rho$. For
irreducibility, we note that for all $g\in G$ the complex number
$\lambda(g)$ is a root of unity, and so
$\lambda(g)\overline{\lambda(g)}=1$; thus
$$\langle\lambda\chi,\lambda\chi\rangle
=\frac{1}{|G|}\sum_{g\in G}\lambda(g)\chi(g)\overline{\lambda(g)\chi(g)}
=\frac{1}{|G|}\sum_{g\in G}\chi(g)\overline{\chi(g)}
=\langle\chi,\chi\rangle,$$
so $\chi$ is irreducible if and only if $\lambda\chi$ is.
\end{quote}
{\bf Examples:}
\begin{itemize}
\item[(i)] Let $G=S_4$. We
obtain a non-trivial linear character of $G$ by lifting from the quotient
$S_4/A_4\cong C_2$; this gives the character $\chi_2$ (which corresponds to
the sign homomorphism). If we write $\lambda=\chi_2$, we see that
$$\lambda\chi_1=\chi_2,\quad\lambda\chi_2=\chi_1,\quad
\lambda\chi_3=\chi_3,\quad\lambda\chi_4=\chi_5,\quad
\lambda\chi_5=\chi_4.$$
\item[(ii)] Let $G=A_4$, we have two non-trivial linear characters,
$\chi_2$ and $\chi_3$. If we set $\lambda=\chi_2$, we have
$$\lambda\chi_1=\chi_2,\quad\lambda\chi_2=\chi_3,\quad
\lambda\chi_3=\chi_1,\quad\lambda\chi_4=\chi_4.$$
\end{itemize}
{\bf Example:} Let $G=S_5$; then $G$ has seven conjugacy classes,
with representatives
$$1,\quad(1\;2),\quad(1\;2)(3\;4),\quad(1\;2\;3),\quad(1\;2\;3\;4),
\quad(1\;2\;3\;4\;5),\quad(1\;2\;3)(4\;5),$$
and centralizer sizes $120$, $12$, $8$, $6$, $4$, $5$ and
$6$ respectively. We have the trivial character $1_G$ and
we have the normal subgroup $A_5$ with quotient $S_5/A_5 \cong C_2$, so we may
lift the non-trivial irreducible character of $C_2$ to $G$ to obtain a second
linear character $\lambda$ (which again corresponds to the sign homomorphism).
We also have the first and second permutation characters $\pi_1$ and $\pi_2$;
the values of the characters found so far are as follows.
$$
\begin{array}{|c|ccccccc|}
\hline
&1&(1\;2)&(1\;2)(3\;4)&(1\;2\;3)&(1\;2\;3\;4)&(1\;2\;3\;4\;5)&(1\;2\;3)(4\;5)\\
\hline
1_G&1&1&1&1&1&1&1\\
\lambda&1&-1&1&1&-1&1&-1\\
\pi_1&5&3&1&2&1&0&0\\
\pi_2&10&4&2&1&0&0&1\\
\hline
\end{array}
$$
Of these, $1_G$ and $\lambda$ are irreducible, and shall be called
$\chi_1$ and $\chi_2$ respectively. We find that
$\langle\pi_1,\chi_1\rangle=1$, and so $\nu_1=\pi_1-\chi_1$ is a
character; then $\langle\nu_1,\nu_1\rangle=1$, so
that $\nu_1$ is a third irreducible character $\chi_3$. Since
$\lambda\nu_1 \neq \nu_1$, we have a fourth irreducible character
$\chi_4=\lambda\nu_1$. We then find that 
$$\langle\pi_2,\chi_1\rangle=\langle\pi_2,\chi_3\rangle=1,$$
and so $\nu_2=\pi_2-\chi_1-\chi_3$ is a character; as
$\langle\nu_2,\nu_2\rangle=1$, we see that $\nu_2$ is a fifth irreducible
character $\chi_5$. Again, $\lambda\nu_2\neq\nu_2$, so we have a sixth
irreducible character $\chi_6=\lambda\nu_2$. Finally, the seventh irreducible
character may be determined using the orthogonality relations; the full
character table is as follows.
$$
\begin{array}{|c|ccccccc|}
\hline
&1&(1\;2)&(1\;2)(3\;4)&(1\;2\;3)&(1\;2\;3\;4)&(1\;2\;3\;4\;5)&(1\;2\;3)(4\;5)\\
\hline
\chi_1&1&1&1&1&1&1&1\\
\chi_2&1&-1&1&1&-1&1&-1\\
\chi_3&4&2&0&1&0&-1&-1\\
\chi_4&4&-2&0&1&0&-1&1\\
\chi_5&5&1&1&-1&-1&0&1\\
\chi_6&5&-1&1&-1&1&0&-1\\
\chi_7&6&0&-2&0&0&1&0\\
\hline
\end{array}
$$
\section {Characters and group structure}
{\bf Observation:}
The character table determines the normal subgroups and the nilpotent groups.
General procedure for calculating characters: (1) Derive a faithful representation,
(2) generate group elements, (3) determine conjugacy classes, (4) determine structure
constants ($|C_i||C_j|= \sum_k \alpha_{ijk} |C_k|$), (5) get characters from structure
constants.
\\
\\
{\bf Theorem 57:}
If $\chi$ is an irreducible character, $\chi(1) \mid |G|$ 
\begin{quote}
\emph{Proof:}
If $g_i$ is in the $i$-th conjugacy class,
${\frac {|G|} {|C_G(g_i)|}} {\frac {\chi(g_i)} {\chi(1)}}$ and ${\overline {\chi(g)}}$
are algebraic integers so
$\sum_{i=1}^k {\frac {|G|} {|C_G(g_i)|}} {\frac {\chi(g_i)} {\chi(1)}}{\overline {\chi(g)}}
={\frac {|G|} {\chi(1)}}$ is and algebraic integer.  Since it is rational, it must be in
${\mathbb Z}$.
\end{quote}
{\bf Theorem 58:}
If $G \subseteq S_n$, $\alpha: G \rightarrow {\mathbb C}$ by $\alpha(g)= |Fix(g)|-1$,
then $\alpha$ is a character of $G$. Define $ker(\rho)= \{g: \chi_{\rho}(g)= \chi_{\rho}(1) \}$.
$\rho$ is faithful iff $ker(\rho)=1$.  $N= \{n: |\chi(n)|= \chi(1) \} \lhd G$. 
\begin{quote}
\emph{Proof:}  
This is easy.
\end{quote}
{\bf Theorem 59:}
$N \lhd G, \exists \chi_i: \bigcap_{i=1}^r ker(\chi_i) =N$.  
$g \sim h$ iff $\chi(g) = \chi(h), \forall  \chi$.  
\begin{quote}
\emph{Proof:}  
If $g, h$ are conjugate, it's clear $\chi(g) = \chi(h), \forall  \chi$.
If $\chi(g) = \chi(h), \forall  \chi$, it's also true for any class function.  Pick the class
function which is $1$ in $ccl_G(g)$ and $0$ elsewhere and the result follows.
\end{quote}
{\bf Theorem 60:}
$ccl_{A_n}(x)=ccl_{S_n}(x)$ otherwise
$ccl_{S_n}(x)$ splits into two conjugacy classes in $A_n$.
\begin{quote}
\emph{Proof:}  
$S_n = A_n \cup A_n (12)$ and the result follows.
\end{quote}
{\bf Theorem 61:}
Let $C_i= \sum_{x \in ccl(y)} x$ then the $C_i$ form a basis for ${\mathbb Z}(FG)$.
There are $|G/G'|$ inequivalent linear representations (characters) of $G$.
\begin{quote}
\emph{Proof:}  
If $A$ is abelian, there are $|A|$ inequivalent linear representations (characters) of $G$
by the decomposition results.  Each gives rise to an inequivalent linear representation of $G/G'$.
So $G/G'$ has at least $|G/G'|$ inequivalent linear representations.
\end{quote}
{\bf Theorem 62:}
$\chi(g)$ is real iff
$\chi(g)= \chi(g^{-1}), \forall \chi$.  $N \lhd G$ iff $\exists \chi_i, i= 1, \ldots, k$
such that $\bigcap_{i=1}^k ker(\chi_i)=N$.
\begin{quote}
\emph{Proof:}  
In ${\mathbb C}$, $\chi(g^{-1}) = {\overline {\chi(g)}}$.
\end{quote}
{\bf Theorem 63:}
$G$ is not simple iff $\exists \chi, g \ne 1:
\chi(g)= \chi(1)$.
\begin{quote}
\emph{Proof:}  
If such a $g$ exists, it is in the kernel of $\chi$.
\end{quote}
{\bf Theorem 64:}
$G$ has $|G/G'|$ linear characters.
If all irreducible representations of $G$ have dimension $1$, $G$ is abelian.
\begin{quote}
\emph{Proof:}  
Every linear character, $\chi$, has $G' \subseteq ker(\chi)$.  Further, if $N \lhd G$,
$G' \subseteq N$ means $G/N$ is abelian.  Further, every linear character of $G$ is a
lift of linear character of $G/G'$.
\end{quote}
{\bf Theorem 65:}
Let $H$ be the kernel of $\theta$ then (i) $|\theta(g)| \leq \theta(1)$,
(ii) $\theta(g) = \theta(1)$, iff $g \in H$,
(iii) $|\theta(g)| = \theta(1)$, iff $gH$ is in the center of $G/H$.
\begin{quote}
\emph{Proof:}  
$\theta(g)$ is the sum of $\theta(1)$ roots of unity showing (i).
(ii) is the definition.
\end{quote}
{\bf Definition:}
Define $(\theta , \eta)= {\frac 1 {|G|}} \sum_g \theta(g) {\overline {\eta(g)}}$.
\\
\\
{\bf Theorem 66:}
If $U = U_{1} \otimes ... \otimes U_{s}$, the number of these similar to $U_{1}$
is ${\frac {(\theta, \eta)} {(\eta, \eta)}}$.
$(\theta, \rho_{G})= \theta(1)$,
$(\chi_{i}, \chi_{j})= \delta_{ij}$,
$\sum_{g} \chi(g) = |G| \delta_{i1}$,
$\sum_{i} \chi_{i}^{2}(1) = |G| $.
$\omega_{i}(R_{j})= |r_{j}|\chi_{i}(g)/\chi_{i}(1)$,
$\omega_{t}(R_{i}) \omega_{t}(R_{j})= \sum_{s} a_{ijs} \omega_{t}(R_{s})$.
$\sum_{t} \chi_{t}(g_{i}) {\overline \chi_{t}(g_{j})} =
{\frac {|G|}{|R_{j}|}} \delta_{ij}$.
\begin{quote}
\emph{Proof:}  
Calculations based on the decomposition theorem.
\end{quote}
{\bf Theorem 67:}
The number of conjugacy classes = number of irreducible representations.
$\omega_{i}(R_{j})$ is an algebraic integer.
\begin{quote}
\emph{Proof:}  
Each element of the representation affording each $\omega$ is an algebraic integer and a
character is just field operations of these elements.
\end{quote}
\section {Some applications of character theory}
{\bf Theorem 70:}
Suppose $\chi$ is a character of a ${\mathbb C}G$-module, $V$, and $g \in G$ has order
$m$ then (1) $\chi(1)=dim(V)$, (2) $\chi(g)$ is a sum of $m$-th roots of unity,
(3) $\chi(g^{-1})= {\overline {\chi(g)}}$ and (4) $\chi(g)$ is real iff $g \sim g^{-1}$.
\begin{quote}
\emph{Proof:}  
(a) is trivial.
For (b), put the matrix affording $\chi$ into Jordan form (maybe by extending the field).  Raising
the matrix to the $n$-th power (where $g^n = 1$), we get the identity.  The character of $g$ is thus
the sum of roots of $1$.
\end{quote}
{\bf Burnside's Lemma:} 
$|{\frac {\chi(g)} {\chi(1)}}| \le 1$.  If
$|{\frac {\chi(g)} {\chi(1)}}| \ne 1$ it is not an algebraic integer.  
\begin{quote}
\emph{Proof:}  
$\chi(g)$ is the sum of $\chi(1)$ roots of unity.
\end{quote}
{\bf Theorem 71:}
Let $\chi$ be an irreducible character and $R$ a conjugacy class.
If $(\chi(1), |R|)=1$ then either $|\chi(g)|= \chi(1)$ (equivalently, 
$R \subseteq {\mathbb Z}(\chi)$) or $\chi(g)=0$.
\begin{quote}
\emph{Proof:}  
$\exists s, t \in {\mathbb Z}: s|R|+ t \chi(1) =1$, so,
$s|R| \chi(g) + t \chi(1) \chi(g)  = \chi(g)$.  
$a_1= {\frac {\chi(g)} {\chi(1)}}$ is an algebraic integer.  Let its
conjugates be $a_2 , \ldots a_m$.  For each $a_1$, $|a_i| \le 1$ and
$\prod_{i=1}^m |a_i| \le 1$ is a rational integer so
$\prod_{i=1}^m |a_i| = 0$ or
$\prod_{i=1}^m |a_i| = 1$.  In the former case, $\chi(g)= 0$ and in the latter
case, $|\chi(g)|= \chi(1)$.
\end{quote}
{\bf Theorem 72:}
Let $p$ be a prime and $G$ a finite
group with conjugacy class of size $p^r, r \ge 1$, then $G$ is not a non-abelian simple group.
\begin{quote}
\emph{Proof:}  
Let $g \in R$.  Every non-principal irreducible character $\chi_i, i>1$ of $G$ is faithful.
$\sum_{i=2}^k \chi_i(1) \chi_i(g) + 1 = 0$ so $\exists i: \chi_i(g) \chi_i(g) \ne 0 \jmod{p}$.
Thus $(|R|, \chi_i(1)) =1$ and since $\chi_i(g) \ne 0$, applying the previous result,
we have $g \in {\mathbb Z}(G)$.
\end{quote}
{\bf Burnside's Theorem:} Every group of order $p^a q^b$ is solvable.
\begin{quote}
\emph{Proof:}  
Let $G$ be a minimal counterexample.  If $G$ is abelian, the theorem is true.  
If not, some element of
$G$ has a conjugacy class of prime power order.  This contradicts the previous result.
\end{quote}
{\bf Theorem 73:} The number of real irreducible characters of $G$ is the number of real conjugacy
classes in $G$,
\begin{quote}
\emph{Proof:}  
Let $X$ be the character table of $G$.  Since ${\overline X}$ is also a matrix of irreducible
characters so $P X = {\overline X}$, where $P$ is a permutation matrix.  Similarly,
$X Q = {\overline X}$ and $Q= X^{-1}PX$.  The number of irreducible real characters of $G$
is $tr(X)$ and the number of irreducible conjugacy classes is $tr(Q)$, and the result holds.
\end{quote}
{\bf Corollary:}
$G$ has a non-trivial real irreducible character iff $G$ has even order.
\begin{quote}
\emph{Proof:}  
Straightforward.
\end{quote}
\section {Feit's moduleless treatment}
{\bf Maschke:} If $char(F)$ does not divide $|G|$, then $F$-representations of $G$
are completely reducible.  For $\phi$ irreducible,
if $\exists S: \forall g, S \phi(g) = \phi(g) S$ then S is non-singular.
\\
\\
{\bf Theorem 74:}
If $A(g), B(g)$ are $k$-irreducible then (i) if $A$ is not similar to $B$,
and,
$\sum_{g} a_{is}(g) b_{tj}(g^{-1}) = 0$; or,
(ii) $A$, $B$ are absolutely irreducible and
$\sum_{g} a_{is}(g^{-1}) a_{tj}(g) =
{\frac {|G|} {n}} \delta_{ij} \delta_{st}$, where $n \times n$ is the
dimension of $(a_{is}(g))$.
\\
\\
{\bf Theorem 75:}
If $A^{s}$ is absolutely irreducible then $a^{s}_{ij}(g)$ are linearly
independent and $\sum_{s=1}^k n_{s}^{2} \leq |G|$.
\section{Induced Representations and Characters}
{\bf Theorem 76:}
Let $U$ be a ${\mathbb C}H$ submodule then the map $\theta(x) = xr, r \in {\mathbb C}G$ is
a ${\mathbb C}H$-homomorphism from $U$ into ${\mathbb C}G$.  Any ${\mathbb C}H$
homomorphism into ${\mathbb C}G$ can be represented this way.
\begin{quote}
\emph{Proof:}
The first statement is obvious.  If $\theta$ is a ${\mathbb C}H$ homomorphism into
${\mathbb C}G$, ${\mathbb C}H= U \oplus W$ for some ${\mathbb C}H$ invariant, $W$.
Define 
$\varphi(u+w)= \theta(u)$ then
$\varphi(u)= \theta(u), u \in U$.  If $r= \varphi(1)$, 
$\varphi(u)= \varphi(u \cdot 1)=
u \varphi(1)= ur $.
\end{quote}
{\bf Definition:}
If $X \subseteq {\mathbb C}G$, define $X({\mathbb C}G)= span \{ xg, g \in G, x \in X \}$.
If $H \le G$ and $U$ is a ${\mathbb C}H$ module,
further define the induced representation $U^G= U({\mathbb C}G)$.
\\
\\
{\bf Theorem 77:}
If $U, V$ are isomorphic ${\mathbb C}H$ modules, $U^G, V^G$ are isomorphic  ${\mathbb C}G$
modules.
\begin{quote}
\emph{Proof:}  
Straightforward.
\end{quote}
{\bf Theorem 78:}
If $U, V$ are ${\mathbb C}H$ modules and $U \cap V= \{ 0 \}$, then $U^G \cap V^G = \{ 0 \}$.
\begin{quote}
\emph{Proof:}  
Let $\theta: U \rightarrow W$ be a ${\mathbb C}H$-homomorphism.  
$\exists r \in {\mathbb C}G: \theta(u)= ru$ and
$\exists s \in {\mathbb C}G: \theta^{-1}(v)= sv$.  If $a \in U^G$, $a$ is a linear combination of
the elements $ug$.  So $ra$ is a linear combination of elements $rug$ and $ra \in V^G$.  Moreover,
$\phi \in Hom_{{\mathbb C} G}$, $a\phi(g)= \phi(ag)$.  $sra=a$ and $rsb=b$ so $b \mapsto sb$ is $\phi^{-1}$.
So $\phi$ is a ${\mathbb C} G$ isomorphism.
\end{quote}
{\bf Theorem 79:}
If $U$ is a ${\mathbb C}H$ module of ${\mathbb C}H$ and
If $V$ is a ${\mathbb C}G$ module of ${\mathbb C}G$ then
$ dim(Hom_{{\mathbb C}G}(U^G, V))= dim(Hom_{{\mathbb C}H}(U, V_{|H}))$
\begin{quote}
\emph{Proof:}  
If $\theta \in
Hom_{{\mathbb C}G}(U^G, V)) $, 
$\exists r \in {\mathbb C}G$: $\theta(s)= sr, s \in U^G$.  Define
${\overline \theta}= \theta_{|H}$.   The map $\theta \mapsto {\overline \theta}$
is a linear transformation from
$ Hom_{{\mathbb C}G}(U^G, V)$ to $Hom_{{\mathbb C}H}(U, V_{|H})$.  First we show it
is invertible: 
If $\varphi \in Hom_{{\mathbb C}H}(U, V_{|H})$,
$\exists r \in {\mathbb C}G: \varphi(u)= ur$.  Define $\theta: U^G \rightarrow {\mathbb C}G$
by $\theta(s)= sr, s \in U^G$ then
$\theta \in Hom_{{\mathbb C}G}(U^G, V) $ and ${\overline \theta}= \varphi$.  Now
the transformation is injective because if
$r_1 , r_2 \in {\mathbb C}G$ and  $u r_1 = u r_2, \forall u \in U$, then
$r_1 s = r_2 s, \forall s \in U^G$.
\end{quote}
{\bf Definition:}
If $H \le G$ and $\varphi$ a class function on $H$, 
$\varphi^G(g)= {\frac 1 {|H|}} \sum_{x \in G} \varphi^* (x^{-1}gx)$ is called an
\emph{induced character}, where $\varphi^*(x)= 0, x \notin H$ and
$\varphi^*(x)= \varphi(x), x \in H$.  It actually is a character.
\\
\\
{\bf Frobenius Reciprocity Theorem:} 
$(\varphi^G , \theta)= (\varphi, \theta_{|H})$.
\begin{quote}
\emph{Proof:}  
Suppose first that 
$(\varphi$ and $\theta$ are irreducible with underlying representation modules
$U$ and $V$.  Then
$(\varphi^G , \theta)_G = dim(Hom_{{\mathbb C}G}(U^G, V))$ and
$(\varphi , \theta_{|H})_H = dim(Hom_{{\mathbb C}H}(U, V_{|H})$ and the result follows from
the earlier theorem.  Since any character is the linear combination of irreducible characters,
the result follows from the bilinarity of the form.
\end{quote}
{\bf Remark:}  Note that $deg(\phi^G)= \phi^G(1)= {\frac {|G|} {|H|}} \phi(1)$.
We define  $f^G_x(y) = 1$, if $y \in x^G$ and
$f^G_x(y) = 0$, otherwise.
\\
\\
{\bf Theorem 80:}
$(\chi, f^G_x) = {\frac {\chi(x)} {|C_G(x)|}}$.
If no element of $g^G$ lies in $H$ then $\phi^G(g)= 0$.
If some element of $g^G$ lies in $H$ then 
$\phi^G(g)= 
|C_G(g)|(
{\frac {\phi(x_1 )} {|C_H(x_1)|}} + \ldots +
{\frac {\phi(x_m )} {|C_H(x_m)|}} ) $ where $(f^G_g)_{|H}= 
f^H_{x_1} + \ldots +
f^H_{x_m}$.
\begin{quote}
\emph{Proof:}  
$(\chi , {f_x}^G) = {\frac 1 {|G|}} \sum_{g \in G} \chi(g) {f_x}^G(g)= $
${\frac 1 {|G|}} \sum_{g \in G} \chi(g) {f_x}^G(g)= {\frac {|x^G|} {|G|}} \chi(x) = {\frac {\chi(g)} {|C_G(x)|}}$.
\end{quote}
{\bf Brauer's Characterization of Characters:} $p$-elementary groups are the products of
a cyclic $p'$ group and $p$ group.  Every irreducible character is an induced character
of a linear character of a $p$ elementary subgroup for some $p$.
\begin{quote}
\emph{Proof:}
\\
\\
\emph{Step 1:}  
Let $\chi_1 , \ldots , \chi_h$ be the irreducible characters of $G$
over ${\mathbb C}$ and let $X_R(G)= \{ \sum a_i \chi_i \}$,
$V_R(G) = \{ \sum r_i \phi_1 , \phi_i $ and irreducible character of an
elementary abelian subgroup of $G \}$, $U_R(G)= \{ \chi: G \rightarrow {\mathbb C} \}$ where
$\chi$ is a class function of an elementary abelian subgroup $E$ of $G$, $\chi_{|E} \in X_R(E)$.
\\
\\
\emph{Step 2:}  
$V_R(G) \subseteq X_R(G) \subseteq U_R(G)$ and $V_R(G)$ is an ideal in $U_R(G)$.
\\
\\
\emph{Step 3:}  
Let $E= A \times B$ be elementary with $(|A|, |B|)= 1$.  $\exists \psi = \psi_a \in V_S(G)$
such that (1) $\psi(g) \in {\mathbb Z}, \forall g \in G$,
(2) if $g$ is not conjugate to an element of $gB$ then $\psi(g)= 0$ and
(3) $\psi(a)= |C(a):B|$.
\\
\\
\emph{Step 4:} Let $ {\cal C_1}, {\cal C_1}, \ldots, {\cal C_k} $
be the conjugacy classes of $G$ which consist of $p'$-elements then $\forall i, 1 \le i \le k$,
$\exists \tau_i \in V_S(G):$
(1) $\tau_i(g) \in Z, g \in G$,
(2) $\tau_i(g) = 0$ if the $p'$ part of $g$ is not in ${\cal C}_i$,
(3) $\tau_i(g) = 1 \jmod{p} $ if the $p'$ part of $g \in {\cal C}_i$.
\\
\\
\emph{Step 5:} 
If $p$ is prime, $\exists \phi \in V_S(G)$ such that $\forall g \in G, \phi(g) \in Z$ and
$\phi(g)= 1 \jmod{p}$.
\\
\\
\emph{Step 6:} 
If $\alpha: G \rightarrow {\mathbb C}$ is a class function and $\alpha(g) \in |G|S,
\forall g \in G$ then $\alpha \in V_S(G)$.
\\
\\
\emph{Step 7:} 
$|G|= p^n g_0$, $p \nmid g_0$ then $\alpha(x)= g_0 \in V_S(G)$.
\\
\\
\emph{Step 8:} 
$1 \in V_S(G)$.
\\
\\
\emph{Step 9:} 
$1 \in V_Z(G)$.
\\
\\
\emph{Step 10:}  $1 \in V_Z(G) \le V_R(G)$ so by step $2$,
$V_R(G) = X_R(G)= U_R(G)$.
\end{quote}
{\bf RSK correspondence} for representations
of the symmetric group: $\exists$ bijection between $S_n$ and the set of ordered
tableau of the same shape $g \leftrightarrow (S,T)$, further
$g^{-1} \leftrightarrow (T,S)$.
\\
\\
{\bf Young's diagram:} $D(\lambda )$, $n= n_1 + n_2 + \ldots + n_k$,
$n_1 \geq n_2 \geq \ldots \geq n_k$.
Number of tableaus with shape $\lambda$:
$f_{ \lambda } = \frac { n!} {\prod_{i,j \in D(\lambda )} {h(i, j)}}$, where
$h(i, j)$ = number of cells in hook $H_{i,j}$.
\section {Miscellaneous}
{\bf Theorem:} Let $\chi$ be an irrecucible character of $G$ then $\chi(1) \mid |G:{\mathbb Z}(G)|$.
If $|\theta(g)| = \theta(1)$ then $g ker(\theta) \in {\mathbb Z}(G/ker(\theta))$.
\begin{quote}
\emph{Proof:}
\end{quote}
{\bf Character Table:} Let $\chi_1, \chi_2, \ldots, \chi_s$ be the irreducible characters of $G$ and
$R_1 , R_2 , \ldots , R_s$ be the conjugacy classes.  $X= (\chi_i(g_j))$ is the character table.
\\
\\
{\bf Theorem:} If $N \lhd G$, $N \subseteq ker (\rho_{G/N})$.  If $H < G$ and $\theta$ is a character
that vanishes on $H^\#$ then $|H| \mid \theta(1)$. If $\chi$ is a non-linear irreducible character
of $G$ then $\exists g \in G: \chi(g)= 0$.
\begin{quote}
\emph{Proof:} ${\frac {\theta(1)} {|H|}} = {\frac 1 {|H|}} \sum_h \theta(h)$ which is an integer.
\end{quote}
{\bf Theorem:} Let $A(g)= (a_{ij}(g))$ and $B(g)= (b_{ij}(g))$ be $F-$irreducible representations of $G$.
(a) If $A \nsim B$ then $\sum_g a_{is}(g) b_{tj}(g^{-1}) = 0$.
(b) If $A$ is absolutely irreducible, $\sum_g a_{is}(g) a_{tj}(g^{-1}) = {\frac {|G|} n} \delta_{ij} \delta_{st}$,
where $n = deg(A)$.  If $A$ affords $\chi$ and $B$ affords $\eta$, $(\chi, \eta) = 0$ and
$(\chi, \chi)= 1$.
\begin{quote}
\emph{Proof:} Put $f(S) = \sum_g A(g^{-1})SB(g)$.  $A(h)f(S)=F(S)B(h), \forall h$, so by Schur, $f(S)$ is $0$.
Put $E_{ij} = (\delta_{ij})$.  $f(E_{ij}) = 0$.  This gives (a).  Since $A = B$ is absolutely irreducible,
$f(E_{st}) = \lambda I_n$.  $e_{st} \delta_{ij} = \sum_g a_{is}(g^{-1}) a_{tj}(g)$ or
$\sum_g a_{is}(g^{-1}) a_{tj}(g) = {\frac {|G|} {n \lambda}} \delta_{ij} \delta_{tj}$.  This gives b.
Summing over $i, j, s, t$ gives $(\chi, \eta) = 0$.
Summing over $i$, we get
$\sum_g a_{ss}(g^{-1}) a_{tj}(g) = {\frac {|G|} {n \lambda}} \delta_{sj} \delta_{tj}$.  Summing over $j$, we get
$\sum_g a_{ss}(g^{-1}) a_{tt}(g) = {\frac {|G|} {n \lambda}} \delta_{ts}$.  Finally, summing over $t,s$, we get
$\sum_g a_{ss}(g^{-1}) a_{tt}(g) = {\frac {|G|} {n \lambda}} \delta_{ts}$.  Finally, summing over $t,s$, we get
$\sum_g \chi(g^{-1}) \chi(g) = {\frac {|G|} {n \lambda}} \delta_{ts}$.  Finally, summing over $t,s$, with
$\lambda = 1$, we get
$\sum_g \chi(g^{-1}) \chi(g) = |G|$.
\end{quote}

\section {Some character tables}

$$
\begin{array}{|c|ccc|}
\hline
C_3&&&\\
\hline
|ccl(g)|&1&1&1\\
|C_G(g)|&3&3&3\\
g&1&a&a^2\\
\hline
\chi_1(g)&1&1&1\\
\chi_2(g)&1&\omega&\omega^2\\
\chi_3(g)1&&\omega^2&\omega\\
\hline
\end{array}
$$

$$
\begin{array}{|c|ccc|}
\hline
S_3&&&\\
\hline
|ccl(g)|&1&2&3\\
|C_G(g)|&6&2&3\\
g&1&(123)&(12)\\
\hline
\chi_1(g)&1&1&1\\
\chi_2(g)&1&1&-1\\
\chi_3(g)&2&-1&0\\
\hline
\end{array}
$$

$$
\begin{array}{|c|cccc|}
\hline
A_4&&&&\\
\hline
|ccl(g)|&1&3&4&4\\
|C_G(g)|&12&4&3&3\\
g&1&(12)(34)&(123)&(132)\\
\hline
\chi_1(g)&1&1&1&1\\
\chi_2(g)&3&-1&0&0\\
\chi_3(g)&1&1&\lambda&\lambda^2 \\
\chi_4(g)&1&1&\lambda^2&\lambda \\
\hline
\end{array}
$$

$$
\begin{array}{|c|ccccc|}
\hline
A_5&&&&&\\
\hline
|ccl(g)|&1&20&15&12&12\\
|C_G(g)|&60&3&4&5&5\\
g&1&(123)&(12)(34)&(12345)&(21345)\\
\hline
\chi_1(g)&1&1&1&1&1\\
\chi_2(g)&4&1&0&-1&-1\\
\chi_3(g)&5&-1&1&0&0\\
\chi_4(g)&3&0&-1&{\frac {1 + \sqrt{-5}} {2}}&{\frac {1 - \sqrt{-5}} {2}}\\
\chi_5(g)&3&0&-1&{\frac {1 - \sqrt{-5}} {2}}&{\frac {1 + \sqrt{-5}} {2}}\\
\hline
\end{array}
$$

$$
\begin{array}{|c|ccccccc|}
\hline
S_5&&&&&&&\\
\hline
|ccl(g)|&1&10&20&15&30&20&24\\
|C_G(g)|&120&12&6&8&4&6&5\\
g&1&(12)&(123)&(12)(34)&(1234)&(123)(45)&(12345)\\
\hline
\chi_1(g)&1&1&1&1&1&1&1\\
\chi_2(g)&1&-1&1&1&-1&-1&1\\
\chi_3(g)&4&2&1&0&0&-1&-1\\
\chi_4(g)&4&-2&1&-2&0&1&-1\\
\chi_5(g)&6&0&0&1&0&0&1\\
\chi_6(g)&5&1&-1&1&-1&1&0\\
\chi_7(g)&5&-1&-1&1&1&-1&0\\
\hline
\end{array}
$$

