\chapter{Signalizers and Thompson Transitivity}
\section{Thompson Transitivity}
{\bf Motivation for transitivity:}
In ${\cal N}^*(P, q)$, very often maximal elements are conjugate.
$Q \in {\cal N}^*(P, q) \rightarrow P \subseteq N(Q)$.  If
$Q_1, Q_2 \in {\cal N}^*(P, q) \rightarrow \exists c \in C(P): Q_1^c = Q_2$.
In common application, $g \in N(P)$ and $Q, Q^g \in
{\cal N}^*(P, q)$ so $\exists c \in C(P): Q^{gc}=Q$ and thus $N(P)= C(P) (N(P) \cap N(Q))$.
\\
\\
{\bf Theorem 1:}  Suppose  $A$ is an elementary abelian $p$-group such that $r(A) \ge 3$.
If $P, Q$ are $A$-invariant $p'$-groups, $\exists a \in A^{\#}: C_P(a) \ne 1 \ne C_Q(a)$.
\begin{quote}
\emph{Proof:}  
Let $V$ be a subgroup of $A$ of type $(p,p)$.  Since $P= \langle C_P(v): v \in v^{\#} \rangle $,
$\exists v \in V^{\#}: C_P(v) \ne 1$.  Let $W \subseteq A$ be a group of type $(p,p)$ and
$W \cap \langle v \rangle = 1$.  $\exists w \in W: C_P(w) \cap C_P(v) \ne 1$ since
$C_P(v)$ is $W$-invariant.  $ \langle v,w \rangle $ is of type $(p,p)$ and acts on $Q$.
$\exists a \in \langle v,w \rangle^{\#}: C_Q(a) \ne 1$.  Since 
$C_P(a) \supseteq C_P(w) \cap C_P(v) \ne 1$, we are done.
\end{quote}
{\bf Definition:} 
${\cal N}_G(E, \pi)$ is the set of all $E$-invariant $\pi$-subgroups of $G$.
${\cal N}_G^*(E, \pi)$ is the set of all maximal elements of
${\cal N}_G(E, \pi)$.
${\cal SCN}(P)$ is the set of self-centralizing normal subgroups of $P$.
${\cal SCN}_G(p)= {\cal SCN}(P)$ where $P \in S_p(G)$.
$r(P)$ is the rank of the largest elementary abelian subgroup of $P$.
\\
\\
{\bf Theorem 2:}  Let $G$ be $p$-constrained.  If $p=2$ assume further
that the sylow subgroup has class $\le 2$.  Let $E$ be an abelian $p$ subgroup of $G$ with
$r(E) \ge 3$ and $E$ contains every $p$-element of $C= C_G(E)$.  Then
every $E$-invariant $p'$ subgroup $H$ of $G$ satisfies $H \subseteq O_{p'}(G)$.  In other
words, $ \langle {\cal N}(E, p') \rangle \subseteq O_{p'}(G)$.
\begin{quote}
\emph{Proof:}  
Let $G$ be a minimal counterexample.
\\
\\
\emph{Step 1:} $O_{p'}(G)= 1$.
\\
If not, set ${\overline G}= G/O_{p'}(G)$.  Since 
${\overline {C_G(E))}}= C_{\overline G}({\overline E}))$, ${\overline H} \subseteq O_{p'}({\overline G})=1$ and
$H \subseteq O_{p'}(G)$.
\\
\\
\emph{Step 2:} Set $R= O_p(G)$ and let $Q \ne 1$ be a minimal $E$-invariant $p'$ subgroup 
of $G$ then $G= RQE$.
\\
If $RQE \subset G$ then $Q \subseteq O_{p'}(RQE)$ by induction hence
$[R, Q] \subseteq O_{p'}(RQE) \cap R = 1$.  By $p$-constraint, $C_G(R) \subseteq R$ proving the
result.
\\
\\
Let $S$ be a $QE$-invariant $p$ subgroup of $G$ minimal with respect to $[Q, S] \ne 1$, then
$S$ is a special $p$-group.
\\
\\
If $S$ is abelian, $S= C_S(Q) \oplus [S, Q]$.  But $[S, Q]$ is an $E$-invariant $p$ group so
$C_P(E) \cap [S,Q] \ne 1, P \in S_p(G)$.  Thus $E \cap [S,Q] \ne 1$.  But
$[E \cap [S,Q], Q] \subseteq Q \cap S = 1$ which is a contradiction.  So $S$ must be non-abelian.
Further, since $[S, Q] \ne 1$ and $S$ is minimal, $S= \Omega_1(S)$ has exponent $p$.
\\
\\
Suppose $p$ is odd.
Define a new group $T$ whose elements are te elements of $S$ and define the operation as follows:
Every element of $S$ has a square root since $p$ is odd.  
Define $x \cdot y = {\sqrt x} y {\sqrt x}$.  Then $T$ is elementary abelian and $QE$ acts
on $T$.  $Fix(S)=Fix(T)$ and this contradicts what we just showed. So $p=2$.
\\
\\
Now $p=2$ and $S$ is non-abelian.  $[E,S] \subseteq  {\mathbb Z}(S)$ since $cl(SE) \le 2$.  So
$[S,E,Q] = 1$, and 
$[Q, S,E] \subseteq {\mathbb Z}(S)$.
If $[E, Q] \ne 1$, then
$[E, Q] \subseteq Q$ stabilizes $S \supseteq {\mathbb Z}(S) \supseteq 1$ hence
$[E, Q] \subseteq {\mathbb Z}(S)$. But 
$[E,Q]$ is an $E$ invariant subgroup of $Q$ and by minimality, $Q= [E,Q] \subseteq {\mathbb Z}(S)$
which is a contradiction.  So $[E, Q] =1$.
Thus we may assume $[S, E] \subset S$ is $Q$-invariant and the minimality of $S$ gives
$[S,E,Q]= 1$.  We also have $[E,Q,S]=1$ so $[Q,S,E]=[S,E]=1$, so $S \subseteq E$ and
$[S,Q] \subseteq Q \cap S =1$ and this contradiction concludes the proof.
\end{quote}
{\bf Observation:}  If $E \in {\cal SCN}(P), P \in S_p(G)$ and $D \in S_p(C_G(E))$.  If
$Q \in S_p(N_G(E))$ and $D \subseteq Q$, $\exists n \in N_G(E)$: $Q^n=P$ and so
$D^n \subseteq P \cap C_G(E) = E$ so $D=E$.  Hence $D \in S_p(C_G(E)$ and by Burnside,
$C_G(E)= E \times O_{p'}(C_G(E))$.
\\
\\
{\bf Thompson Transitivity Theorem (weakened result):}  Let $G$ be a group in which the
normalizer of every non-trivial $p$-subgroup is $p$-constrained.  If $p=2$ assume further
that the sylow subgroup has class $\le 2$.  Let $E$ be an abelian $p$ subgroup of $G$ with
$r(E) \ge 3$ and $E$ contains every $p$-element of $C= C_G(E)$.  Then
$O_{p'}(C)$ acts transitively on the elements of ${\cal N}^*(E, q)$ where $q \ne p$.
\begin{quote}
\emph{Proof:}  
Let $ {\cal S}_1, \ldots , {\cal S}_t$ be $O_{p'}(G)$ orbits of 
${\cal N}^*(E,Q)$ and suppose $t > 1$; obviously, ${\cal N}^*(E,Q) \ne \{ 1 \}$.
Let $R \ne 1$ be choosen of minimal order subject to  
$R= S_i \cap S_j, S_i \in {\cal S}_i, S_j \in {\cal S}_j, i \ne j$.  WLOG $i=1, j=2$.
$N= N_G(R) \supseteq E$.  Put ${\overline N}= N/R$, $T_i = N \cap S_i \supseteq R$.
${\overline {T_i}}$ is $E$-invariant so $\exists e \in E^{\#} : C_{\overline {T_i}}(e) \ne 1$.
Since
$ C_{\overline {T_i}}(e) = {\overline {C_{T_i}(e)}} $,  $C_G(e) \cap T_i \supseteq R$.
Since $H= C_G(e)$ is $p$-constrained, $E \subseteq C_G(e)$.
\\
\\
Let $P_i= T_i \cap H$.  $R \subset P_i R$ and $P_i \subseteq N(R)$.
$P_i$ is $E$-invariant so by the previous result, $P_i \subseteq O_{p'}(H)$.
Let $L= R(N \cap H)$.  $P_i \subseteq O_{p'}(H) \cap N \subseteq O_{p'}(N \cap H)$ so
$P_i \subseteq O_{p'}(L)$ since $R \lhd L$ and $R$ is a  $p'$ group and thus
$R P_i \subseteq O_{p'}(L)$.
Now let $Q_i \supseteq Q_i R$ be ab $E$-invariant Sylow $q$ subgroup of $O_{p'}(L)$.
$\exists x \in C_G(E) \cap O_{p'}(L): Q_1^x = Q_2$.  Since $ \langle x \rangle $ is an $E$-invariant
$p'$ subgroup of $N_G(E)$ which is $p$-constrained, another application of the previous
result gives $x \in O_{p'}(N_G(E))$ and so $x \in O_{p'}(C_G(E))$.
Let $U \in {\cal N}_G^*(E,q)$ so $U \supseteq Q_1$.  Then 
$U \cap S_1 \supseteq Q_1 \cap S_1 \supseteq P_1R \supseteq R$.  But
$U^x \cap S_2 \supseteq Q_1^x \cap S_2 \supseteq P_2R \supseteq R$.
Since $x \in C(E)$, $U^x$ is maximal in ${\cal N}_G(E,q)$ and by the choice of $R$,
$U^x \in {\cal S}_2$ and so
$U \in {\cal S}_2$ and ${\cal S}_1 = {\cal S}_2$.
\end{quote}
\section{Signalizers}
Let $A$ be a non-cyclic elementary abelian $p$ group acting on $G$.  $C_G(a)$ is $A$-invariant for
$a \in A^{\#}$.  By an earlier theorem, if $U$ is an $A$-invariant $p'$ group then
$U= \langle C_G(a) \cap U, a \in A^{\#} \rangle$.
\\
\\
Let $\theta$ be a map from $A^{\#}$ into
$p'$, $A$-invariant subgroups of $G$ contained in $C_G(a)$. Define $\theta(C_G(a)) = \theta(a)$.
\\
\\
Define ${\cal N}_{\theta}(A)$ be the solvable $A$-invariant $p'$ subgroups $U \leq G$ such that
$U \cap \theta(C_G(a)) \subseteq \theta(C_G(a)), \forall a \in A^{\#}$.
\\
\\
If $U \in {\cal N}_{\theta}(A)$, $\langle U \cap \theta(C_G(a)), a \in A^{\#} \rangle = U$.
\\
\\
$\theta$ is solvable if $\theta(C_G(a)) \cap C_G(b) \subseteq \theta(C_G(b)), \forall a, b \in A^{\#}$.
Equivalently, $\theta(C_G(a)) \in {\cal N}_{\theta}(A)$.
\\
\\
$\theta$ is complete if ${\cal N}_{\theta}(A)$ contains a unique maximal element $E$.
Clearly, $E = \langle \theta(C_G(a)), a \in A^{\#} \rangle$ if it exists.  So $\theta$
is complete iff $E \in {\cal N}_{\theta}(A)$.
\\
\\
Note: If $\theta(G)$ exists and $\theta(G) \ne 1$ and $G$ is simple, $M=N_G(\theta(G))$
contains the normalizers of many $p$-subgroups of $G$; this is a uniqueness subgroup.
If $\theta(G) = 1$, $G$ is often of characteristic $p$-type and we can use Thompson
factoring.
\\
\\
{\bf Definition:}
Suppose $r$, prime, $G$ finite and $A$ an abelian $r-$subgroup of $G$. An
$A-$\emph{signalizer} is a map $\theta: A^{\#} \rightarrow 
{\cal N}_{\theta}(A)$ 
where ${\cal N}_{\theta}(A)$ is a
set of $r'$ $A-$invariant subgroups such that $a,b \in A^{\#}$ and $\theta(a) \le C_G (a)$
and $\theta(a) \cap C(b) \le \theta(b)$.  Example: $X \in r'(G)$, $X= \theta(G)$,
$\theta(a)= C_X(a), a \in A^{\#}$.
\\
\\
{\bf Example signalizer:} $X \in r'(G)$, $A \in r(G)$, $A'=1$, $X^A=X$, $\theta(a)=C_X(a), a \in A^{\#}$.
\\
\\
{\bf Note:} If $P \in S_p(G)$, $A \subseteq P$ then $N_G(A)$ permutes
${\cal N}_{G}(A, q)$ and $C_G(A)$ acts transitively so $N(P)$ normalizes some
$Q \in {\cal N}^{*}_{G}(A, q)$.  Pushing up gives $Q \in{\cal N}^{*}_{G}(P, q)$.
\\
\\
{\bf Definition:} A signalizer function is solvable if
$\theta(C_G(a) \cap C_G(b)) \le \theta(C_G(b)), \forall a, b \in A^{\#}$.  It is
\emph{complete} if ${\cal N}_{\theta}(A)$ contains a unique maximal element.
\\
\\
{\bf Motivation for signalizers:}  Suppose $G$ has no solvable normal subgroups, what is
$H= C_G(z), z \in Inv(G)$?  $\theta(G,A)$ is a proper $p'$-group and either (i)
$\theta(G, A) \lhd G$ or (ii) $N_G(\theta(G,A))$ is strongly $p$-embedded.
Below, we show $m(A) \ge 3$ then every $A-$signalizer functor is complete.  
If $H$ is strongly $p$-embedded then the representation of $G$ on the right
cosets of $H$ in $G$ has the property that all the $p$-elements of $G$ fixes
exactly $1$ point.  Simple groups with strongly $p$-embedded subgroups were classified
by Bender.
\\
\\
{\bf Theorem 3:}
Let $p$ be a prime and $\theta: a \mapsto O_{p'}(C_G(a))$. 
(1) If $C_G(a)$ is
solvable $\forall a \in A^{\#}$ then $\theta$ is a solvable $A$-signalizer
functor on $G$,
(2) Suppose $G$ is solvable then $\theta$ is complete and $\theta(G)= O_{p'}(G)$.
\begin{quote}
\emph{Proof:}  
The theorem that if $O_{p'}(G)=1$ and $C_G(O_p(G)) \le O_p(G)$ means that for
$p \in p(G)$, $O_{p'}(N_G(P)) =O_{p'}(G) \cap N_G(P)$ insures the inclusion property
for $\theta$ holds  It also insures that 
$O_{p'}(G)= \langle \theta(C_G(a)) : a \in A^{\#} \rangle$.
\end{quote}
{\bf Theorem 4:}
Let $X, Y \in {\cal N}_{\theta}(A)$ and $XY = YX$ and suppose that either
(1) $Y \le N_G(X)$ or (1') $XY$ is solvable, then $XY \in {\cal N}_{\theta}(A)$.
\begin{quote}
\emph{Proof:}  
(1) insures $XY$ is an $A$-invariant solvable $p'$-group and so
$C_{XY}(a)= C_X(a) C_Y(a) \le C_a , \forall a \in A^{\#}$.  This proves the result.
\end{quote}
{\bf Theorem 5:}
Let $N$ be an $A$-invariant normal $p'$-subgroup of $G$ and ${\overline G}= G/N$ then
${\overline {\theta}}: a \mapsto {\overline {\theta}}(C_{\overline G}(a))= {\overline C}_a$
is a solvable signalizer functor  of ${\overline G}$ and 
${\overline {{\cal N}_{\theta}(A)}} \subseteq {\cal N}_{\overline {\theta}} (A)$.
\begin{quote}
\emph{Proof:}  
Let $a, b \in A^{\#}$ and $M= N C_a$ so
$ {\overline M} = 
{\overline {\theta}}(C_{\overline G}(a)) $
and $ C_{\overline M}(b) = {\overline {C_M(b)}}$.  It follows that
$C_M(b)= C_N(b) C_{C_a}(b)= C_N(b) (C_a \cap C_{G}(b)) \le 
{\overline {\theta}}(C_{\overline G}(a)) $.
Similarly for $U \in {\cal N}_{\theta}(A)$ 
$C_{\overline U}(a)= {\overline {C_U(a)}}= {\overline {C_a \cap U}}
\le {\overline {\theta}}(C_{\overline G}(a)) $.
So ${\overline U} \in {\cal N}_{\overline {\theta}}(A)$.
\end{quote}
{\bf Theorem 6:}
Assume in the previous result that $N \in
{\cal N}_{\theta}(A)$  then $
{\overline {{\cal N}_{\theta}(A)}} = 
{\cal N}_{\overline {\theta}}(A)$.  In particular, $\theta$ is complete iff ${\overline {\theta}}$
is complete.
\begin{quote}
\emph{Proof:}  
Let $N \le U < G$ such that 
${\overline U} \in {\cal N}_{\overline {\theta}}(A)$.  STS
$U \in {\cal N}_{\theta}(A)$.  For $a \in A^{\#}$,
$ {\overline {C_U(a)}} = C_{\overline U}(a)
\le {\overline {\theta}}(C_{\overline G}(a))= {\overline C}_a= C_a N/N$ and thus
$C_U(a) \le N C_a \cap C_G(a) = C_N(a) C_a \le C_a$  since $N \in
{\cal N}_{\theta}(A) $ so
$U \in {\cal N}_{\theta}(A)$.  Set $\pi(\theta) = \bigcup_{a \in A^{\#}} \pi(C_a)$.
\end{quote}
{\bf Theorem 7:} Let $U \in 
{\cal N}_{\theta}(A)$ then $\pi(U) \subseteq \pi(C_a)$.
\begin{quote}
\emph{Proof:}  
This is clear.
\end{quote}
{\bf Definition:} 
$C_A = \theta(C_G(A))$.
If $p \notin \pi$, $C_a= \theta(C_G(a))$ then $C_a$ contains a unique 
maximal $AC_A$-invariant $\pi$-subgroup denoted by $\theta_{\pi}(C_G(a))$.
\\
\\
{\bf Theorem 8:}
The mapping
$\theta_{\pi}: a \mapsto
\theta_{\pi}(C_G(a))$ is a solvable $A$-signalizer functor on $G$ satisfying
$\pi(\theta_{\pi}) \subseteq \pi$ and
$\{ U \in
{\cal N}_{\theta}(A, \pi): U^{C_A} = U \} \subseteq
{\cal N}_{\theta_{\pi}}(A)$.
\begin{quote}
\emph{Proof:}  
Let $a, b \in A^{\#}$ then 
$\theta_{\pi}(C_G(a)) \cap C_G(b)$ is an $A C_A$-invariant
$\pi$ subgroup of $C_b$  Thus
$\theta_{\pi}(C_G(a)) \cap C_G(b)$ is contained in the unique maximal 
$A C_A$-invariant $\pi$-subgroup 
$\theta_{\pi}(C_G(b))$,  This shows that $\theta_{\pi}$ is a solvable
$A$-signalizer functor on $G$.
Clearly, $\pi(\theta_{\pi}) \subseteq  \pi$ and the other property 
follows from the uniqueness of $\theta_{\pi}(C_G(a))$.
\end{quote}
{\bf Theorem 9:} Let $U \in 
{\cal N}_{\theta}(A)$ then $\pi(U) \subseteq \pi(\theta)$.
\begin{quote}
\emph{Proof:}  
For every $q \in \pi(U)$, there is an $A$-invariant Sylow $q$-subgroup
$Q$ of $U$.  Moreover, since $A$ is non-cyclic, $\exists a \in A$ such that
$C_Q(a) \ne 1$.  Thus $C_Q(a) \le C_U(a) \le C_a$, we get $q \in \pi(\theta)$.
\end{quote}
{\bf Theorem 10:}
Let $A$ be an elementary abelian $p$-group with $r(A) \ge 3$ that acts on the
group $X$ and let $p \ne q \in \pi(X)$.  Suppose $Q_1$
and $Q_2$ are two $A$-invariant $q$-subgroups of $X$ such that for $D= Q_1 \cap Q_2$
$Q_1 \ne D \ne Q_2$ then $\exists a \in A^{\#}$ such that
$N_G(D) \cap C_{Q_i}(a) \nleq D, i= 1, 2$.
\begin{quote}
\emph{Proof:}  
Since $Q_1 \ne D \ne Q_2$, we get $D < N_{Q_i}(D)= N_i , i= 1, 2$ and
$N_i= \langle C_{N_i}(B) \rangle$, $B \le A$ and $r(A/B) \le 1$.  For $i= 1, 2$
there is a maximal subgroup $B_i$ of $A$ such that $C_{N_i}(B_i) \nleq D$.
Because $r(A) \ge 3$, we get $B_1 \cap B_2 \ne 1$.  Now choose $1 \ne a \in B_1 \cap B_2$.
\end{quote}
{\bf Transitivity Theorem:}   Let $\theta$ be a solvable $A$-signalizer functor on $G$, 
$q \in \pi(\theta)$ and suppose $r(A) \ge 3$ then the elements in ${\cal N}^{*}_{\theta}(A, q)$
are conjugate in $C_A$.
\begin{quote}
\emph{Proof:}  
Assume the assertion is false.  Among all pairs of elements
${\cal N}^*_{\theta}(A, q)$ that are not conjugate under $C_A$,
we choose $Q_1$ and $Q_2$ such that $D = Q_1 \cap Q_2$ is maximal.
Set $N= N_G(D)$ and $N_a = N \cap C_a, a \in A^{\#}$.  There is an
$a \in A^{\#}$, such that 
$N_a \cap Q_1 \nleq D$ and
$N_a \cap Q_2 \nleq D$.
$N_a$ is an $A$-invariant $p'$-group, thus $\exists c \in C_{N_a}(A) \le C_A$
such that $E= \langle (N_a \cap Q_1)^c, N_a \cap Q_2 \rangle$ is an
$A$-invariant $q$-subgroup.  Since $D$ and $E$ are in
${\cal N}_{\theta}(A, q)$.  So $\exists Q_3 \in {\cal N}^*_{\theta}(A, q)$
containing $DE$, so
$D < D(N_a \cap Q_1)^c \le Q_1^c \cap Q_3$ and.
$D < D(N_a \cap Q_2) \le Q_2 \cap Q_3$.  The maximal choice of $D$ implies that
$Q_1^c$ and $Q_3$ are conjugate under $C_A$ as well as $Q_2$ and $Q_3$.  But
then also $Q_1$ and $Q_2$ are conjugate under $C_A$, a contradiction.
\end{quote}
{\bf Theorem 11:}
Let $q \in \pi(\theta)$ and
$Q \in {\cal N}^*_{\theta}(A, q)$ and suppose $r(A) \ge 3$ then
(1) $\forall H \in 
{\cal N}_{\theta}(A)$  $\exists c \in C_A$ such that $Q^c \cap H$ is an $A$-invariant
Sylow $q$-subgroup of $H$;  and (2)
$C_Q(B)$ is an $A$-invariant Sylow $q$-subgroup of $C_B$ for every $1 \ne B \le A$.
\begin{quote}
\emph{Proof:}  
Every $A$-invariant Sylow $q$-subgroup, $Q_1$ of $H$ is in
${\cal N}_{\theta_{\pi}}(A)$ and is thus contained in an element $Q_2 \in
{\cal N}^*_{\theta_{\pi}}(A)$.  So $\exists c \in C_A: Q_2= Q^c$ and 
$Q^c \cap H = Q_1$.  Part 2 follows from 1 with $H= C_B$ since $C_A \le C_B$.
\end{quote}
{\bf Theorem 12:}
If $|\pi(\theta)| \le 1$ and $r(A) \ge 3$ then $\theta$ is complete.
\begin{quote}
\emph{Proof:}  
The case $\pi(\theta) = \emptyset$ gives $\theta(G) = 1$.
Assume $\pi(\theta)= \{ q \}$.  Then
$ {\cal N}_{\theta}(A) {\cal N}_{\theta}(A, q) $, and 
$\exists Q \in {\cal N}^*_{\theta}(A) $
such that $C_A \le Q$ and by the previous result, $Q$ is the only element in
${\cal N}^*_{\theta}(A, q) $.
\end{quote}
{\bf Theorem 13:}
Let $G$ be a solvable $p'$ group and $q \in \pi(G)$.  Suppose $U$ is a $q'$ group of
$G$ such that $[U,C_G(B)]$ is a $q'$ group for every maximal subgroup $B$ of $A$.  Then
$U \leq O_{p'}(G)$.
\begin{quote}
\emph{Proof:} See Stellmacher, p 325.
\end{quote}
{\bf Glauberman's Theorem:} Let $\theta$ be a solvable signalizer functor on $G$
and $r(A) \ge 3$ then $\theta$ is complete.
\begin{quote}
\emph{Proof:} See Stellmacher, p 327.
\end{quote}
\section {Factorizations}
{\bf Theorem 14a:}
Suppose the action of $A$ on $G$ is coprime and $G=XY$ with $X$ and $Y$ $A$-invariant then
$C_G(A)= C_X(A) C_Y(A)$.
\begin{quote}
\emph{Proof:} 
Let $g=xy \in G$.  $xy = (xy)^a=x^a y^a$.  So $x^{-1}x^a = yy^{-a} \in X \cap Y$.
$(xU)^a=xU$ and $(Uy)^a=Uy$, so 
$\exists c \in C_X(A), x=cu$ and
$\exists d \in C_Y(A), y=wu$.  $cuwd \in C_G(A) \cap X \cap Y$, so $uw \in C_X(A) C_Y(A)$.
\end{quote}
{\bf Theorem 14:}
If $p \in \pi(G)$ and ${\overline G}= G/ O_{p'}(G)$ and suppose ${\overline G}$ is $p$-constrained
then $\forall P \in p(G)$, $O_{p'}(N_G(P)) = O_{p'}(G) \cap N_G(P)$.
\begin{quote}
\emph{Proof:} 
$C_G(P) \lhd N_G(P) \rightarrow O_{p'}(N_G(P))= O_{p'}(C_G(P))$ so it STS
$O_{p'}(G) \cap C_G(P) = O_{p'}(C_G(P))$.  Assume $O_{p'}(G)=1$ and put $Q=O_{p'}(C_G(P))$.
$C_G(O_p(G)) \subseteq O_p(G)$ and $[P,Q]=1$ so we can apply Thompson's $p \times q$ lemma
to show $Q$ acts trivially on $C_G(P)$ and $Q \leq C_G(O_p(G)) \leq O_p(G)$ and $Q=1$.
\end{quote}
{\bf Theorem 15:}
Let $P \in p(G)$ and $U \leq O_{p'}(N_G(P))$.  Suppose $U$ and $P$ are contained in some solvable group
$L$.  Then $U \leq O_{p'}(L)$.
\begin{quote}
\emph{Proof:} Todo.
\end{quote}
{\bf Definition:} For $q \in \pi(\theta)$, put $\theta_{q'} = \theta_{\pi(\theta) \setminus Q}$.
$\theta$ is locally complete if $\theta_{N_G(U)}$ is complete for $U \in {\cal N}_{\theta}(A)$ and
$\theta_{q'}$ is complete for all $q \in \pi(\theta)$.
\\
\\
{\bf Theorem 16:}
Let $G$ be a $p'$-group and $X$ and $Y$ be two $A$-invariant subgroups if $G$.  Suppose further,
(1) $C_G(a)= C_X(a)C_Y(a)$ for all $a \in A^{\#}$ and (2) $X$ is $C_G(A)$-invariant then $G=XY$.
\begin{quote}
\emph{Proof:}
By induction.  Let $q$ be a prime divisor of $|G|$ and $G$ has a non-trivial $q$ subgroup.
$\mathbb{Z}(G) \ne 1$ and since $A$ is non-cyclic, $\exists a \in A: N=C_{\mathbb{Z}(G)}(a) \ne 1$.
$N$ is $A$-invariant.  Put $\overline(G) = G/N$.  By induction,
$G=XYN=XNY=X C_G(a)Y=XY$.
See Stellmacher, p 312 for the rest.
\end{quote}
\section{More from Thompson}
{\bf Thompson Transitivity Theorem:}  If $G$ is a group in which
the normalizer of every non-identity $p$-subgroup is $p$-constrained
and if $A \in {\cal SCN}_3(p)$ then $C_G(A)$ permutes all
maximal $A$-invariant $q$ groups of $G$, $q \ne p$. 
\begin{quote}
\emph{Proof:}  
Generalization of earlier result.
\end{quote}
{\bf Consequence:}
Under the TTT conditions, if $P \in S_p(G), A \in {\cal SCN}_3(P)$ and $\forall q \ne p$,
$P$ normalizes some $A-$invariant $q-$subgroup of $G$; so if $P$ normalizes no
$p'$ subgroup of $G$, neither does $A$.  
\\
\\
{\bf Maximal Subgroup Theorem:}
If $P \in S_p(G), {\cal SCN}_3(P) \ne \emptyset, p \ne 2$ and every element of 
${\cal N}^*(P)$
is $p-$constrained and $p-$stable and $\exists H, 1 \ne H \lhd P$: $[Q,P]=1$ if $H \in p'(G)$
and $H^P=H$ then ${\cal N}^*(P)$ has a unique maximal element.
\begin{quote}
\emph{Proof:}  
Generalization of earlier result.
\end{quote}
{\bf Thompson (from N-group paper):} $G$ is not solvable iff 
$\exists x, y, z \in G \setminus \{1\}$
with $(|x|, |y|)=(|y|,|z|)=(|x|,|z|)=1$ such that $xy=z$.  If $G$ is a non-abelian
simple group all of whose $p-$locals are solvable then $G$ is isomorphic to one
of the following: (1) $PSL_2 (q), q > 3$, (2) $Sz(q), q= 2^{2m+1}, m \ge 1$ or (3)
$A_7$, $PSL(2(3)$, $U_3(3)$, or $M_{11}$.
\begin{quote}
\emph{Proof:}  
The proof is too long for this paper.
\end{quote}
{\bf Theorem 13:} If $M$ is a maximal subgroup of $G$, $p \in \pi(M)$, $O_{p'}(M) \ne 1$ 
then all maximal primitive subgroups are conjugate to $M$.
\begin{quote}
\end{quote}
{\bf Thompson Transfer Theorem:} Let $S \in S_2(G)$ and suppose $\exists U \le G$ maximal
and $t \in Inv(S) \rightarrow t^G \cap U = \emptyset$ then $t \notin O^2(G)$.
\begin{quote}
\end{quote}
\section{Bender Uniqueness}
{\bf Observation:} The following uniqueness result is used in the simplification of the
odd order result.
\\
\\
{\bf Bender's Theorem:}
Let $G$ be a minimal subgroup of odd order and $U$ be a elementary abelian subgroup of
order $p^3$ then there is one maximal subgroup of $G$ containing $U$.
