\chapter{Permutation representations}
\section {Basic Results}
{\bf Lemma 1:} 
Let $H \le G$ and $Hg_1 , ..., Hg_n$ be the cosets; the map
$\pi(g): \langle Hg_1, ..., Hg_n \rangle \mapsto \langle Hg_1 g, ..., Hg_n g \rangle $ 
is a map from $G$ to $\Sigma_n$
whose kernel is the largest normal subgroup of $G$ in $H$; in fact,
$ker(\pi) = \bigcap_{i=1}^n H^{x_i}$.
\\
\\
{\bf Corollary:}  If $G$ is simple and $G > H$ with $|G:H| = k$ then
$|G| \mid k!$.
Equivalently, if $|G| \nmid k!$, $|ker(\pi)| > 1$ as above, so $G$ has
a normal subgroup.
\section {Imprimitivity}
{\bf Definition 1:}
A \emph{system of imprimitivity for permutation 
group $G$:} is a set, ${\cal B} = \{ \Delta_i\}$ $|\Delta_i| > 1$, with
the property that for $\Delta \in {\cal B}, g \in G$ either 
$\Delta \cap g \Delta = \emptyset$ or
$\Delta = g \Delta$.
\\
\\
{\bf Theorem: 1}
If $H<G$ and $G$ is simple then $|G| \mid |G:H|!$. 
\begin{quote}
\emph{Proof:}  
Let  $\pi: G \rightarrow Perm(G/H)$ be the map from $G$ to the permutations on the right
cosets of $H$ in $G$.  $G/ker(\pi)$ is an injection into $S_{|G:H|}$ and $ker(\pi)=1$ so
$|G| \mid |G:H|!$.
\end{quote}
{\bf Definition 2:}
A permutation is \emph{primitive} if there is no non-trivial set of imprimitivity.
\\
\\
{\bf Definition 3:}
$\Gamma$ is $G$ invariant if $\Gamma^G= \Gamma$ so $\Gamma$ is a union of $G$ orbits.
$G/G_{\Gamma} \equiv G^{\Gamma}$.
\\
\\
{\bf Normal Subgroups of Primitive Groups Theorem:}  
Let $G$ be primitive on $\Omega$ and $1 \ne N \lhd G$.  Then either $N \subseteq G_a$ or
$N$ acts transitively on $\Omega$.  If the action is regular, $N$ is a minimal normal subgroup.
\begin{quote}
\emph{Proof:}  
Let $\Delta(a)= Na$.  If $\Delta(a)= \{ a \}$, $N \subseteq G_a$ and we're done.
Suppose $a \ne b \in Na$.  Since $G$ is transitive, $\exists g \in G: ga=b$ so
$g \Delta(a)= gNa= (gN g^{-1} ) ga= Nb = \Delta(b)$.  But $b \in Na$ so $Nb = Na$.
Thus $\Delta(a)$ is an imprimitive block.  Since $G^{\Omega}$ is primitive, $\Delta(a)= \Omega$
and $N$ is transitive.
\end{quote}
{\bf Iwasawa's Theorem:}  Let $G$ be a primitive permutation group and suppose (1) $G'=G$ and
(2) $\exists A$ with $A$ solvable  and $A \lhd G_a: G=
\langle A^G \rangle $ then $G$ is simple.
\begin{quote}
\emph{Proof:}  
Let $1 < N \lhd G$.  Since $N$ is transitive by the previous result, $NG_a=G$.  Further,
$G_a \subseteq N_G(NA)$ and $N \subseteq N_G(NA)$ and so $G= NG_a= N_G(NA)$. So $NA \lhd G$ and
$(NA)^G =NA \lhd G$ but $ \langle A^G \rangle = G$ so $NA=G$. 
Since $A$ is solvable, so is $G/N$ and
$(G/N)'=(G/N)$ and so $N = G$.
\end{quote}
{\bf Theorem 2:}
If $\Delta \subseteq \Omega$ and $\alpha \in \Omega$ then 
$\psi( \alpha ) = \bigcap_{\alpha \in g \Delta} g \Delta$ is a block of a transitive
group $G \subseteq Sym(\Omega)$.
\begin{quote}
\emph{Proof:}  
Let $\beta= h \alpha$ and
note that 
$h \psi( \alpha ) = \bigcap_{h \alpha \in hg \Delta} hg \Delta= \psi( \beta )$ so
$ |\psi( \alpha )|= |\psi( \beta )| $.   Now suppose $\beta \in \psi( \alpha )$ then
$\alpha \in g \Delta \rightarrow \beta \in g \Delta$ so 
$\bigcap_{\beta \in g \Delta} g \Delta \supseteq \bigcap_{\alpha \in g \Delta} g \Delta$ and
so, $\psi( \beta ) = \psi ( \alpha )$ since they both have the same number of elements.  We have
shown $\beta \in \psi( \alpha ) \rightarrow  \psi( \beta ) = \psi( \alpha )$ and thus
$\psi( \alpha ) \cap \psi( \beta ) = \emptyset$  or 
$\psi( \alpha ) = \psi( \beta )$.
\end{quote}
{\bf Theorem 3:}
$G$ is primitive iff $G_{\alpha}$ is maximal.
A transitive group is imprimitive iff $\exists Z$: $G_{\alpha} < Z < G$.
\begin{quote}
\emph{Proof:}  
Suppose $b \ne a$.  Suppose $a, b \in \Delta_i$ with $\{ \Delta_i \}$ a set of
imprimitivity and suppose $g a= b$.  Then $g \Delta_i = \Delta_i$ and 
$ \langle g, G_a \rangle > G_a$
stabilises the block containing $a$.  So if $G_a$ is not primitive then $G_a$ is not maximal.
If $G > M > G_a$ and $M$ is maximal, $Ma$ is a set of imprimitivity so $G$ is not primitive.
\end{quote}
{\bf Definition 4:}
$G$ acts \emph{regularly} on $\Omega$ if $\forall \alpha, \beta \in \Omega,
\exists ! g: \alpha^g= \beta$.  
\\
\\
{\bf Theorem 4:}  Let $G$ be $n-$fold transitive on $\Omega$, $n \ge 2$ and $N$ a regular
normal subgroup of $G$ then $n \le 4$ and 
(1) If $n=2$, $N$ is an elementary abelian $p-$group; 
(2) If $n=3$, $N$ is an elementary abelian $2-$group or $N=C_3$ and $G=S_3$;
(3) If $n=4$, $N= C_2 \times C_2$ and $G=S_4$.
\begin{quote}
\emph{Proof:}  
Let $n \geq 2$ and $\alpha \in \Omega$.  $G_{\alpha}$ is $(n-1)$-fold transitive on $\Omega - \{ \alpha \}$ and
hence on $N^{\#}$ so $\exists g: x^g=y, x,y \in N^{\#}$.  Thus every element of $N^{\#}$ has the same order, $p$.
$N= {\mathbb Z}(N)$ and $N$ is abelian (and hence elementary abelian).
If $n \geq 3$, $3 \leq |N|=|\Omega|$.  If $n=3$, $G=S_3$.  If $p \geq 3$, $x_1 \neq x_1^{-1}=x_2$.  Let $x_3$ be another 
element.  Then $x_1^g= x_2, x_2^g=x_3$ for some $g$ and this is contradictory so $p=2$.
If $n \geq 4$, $N$ is an elementary abelian $2$-group.  Let $U= \langle x_1 \rangle \times \langle x_2 \rangle$.
If $N \neq U$, put $x_3=x_1 x_2$ and choose $x_4 \notin U$.  Then $\exists g \in G:$
$x_1^g= x_1$,
$x_2^g= x_2$, and
$x_3^g= x_4$ which is again contradictory.
\end{quote}
{\bf Definition 5:}
Define ${\cal G}(G, \Omega )$ as the graph of $G$ acting on $\Omega$ as follows:
$G$ acts on $\Omega \times \Omega$.  Diagonal orbital is
$\Delta_1 = \{ (\alpha , \alpha )\}$. If $\Delta = \{ (\alpha , \beta ) \}$,
$\Delta^*= \{ (\beta, \alpha) \}$.  Self paired if $\Delta^* = \Delta$.
$\Delta(\alpha)= \{ \beta : (\alpha , \beta ) \in \Delta \}$ --- corresponds to 
orbits of $G_{\alpha}$. The \emph{rank of the permutation group} is number of orbitals.
$\Delta^p= \{ (y,x): (x, y) \in \Delta \}$ is called the \emph{paired orbit}.
\\
\\
{\bf Theorem 5.1:} 
(a) A non-diagonal orbital is self-paired iff $(x,y)$ and $(y,x)$ are in the same orbit.\\
(b) $G$ has a non-diagonal self paired orbital iff $G$ has even order. \\
(c) If $G$ is of even order and rank-3, both non-diagonal orbitals are self paired.
\begin{quote}
\end{quote}
{\bf Theorem 5.2:}
On a self-paired orbit $\Delta$, the graph ${\cal G}=(G, X,\Delta)$ is 
symmetric and $G$ is transitive on edges.
\begin{quote}
\emph{Proof:}  
Since $\Delta$ is self-paired, the graph is symmetric.  
\end{quote}
{\bf Theorem 6:}
$G$ is primitive iff ${\cal G}$ is connected.
\begin{quote}
\emph{Proof:}  
$G$ is maximal so $G= \langle G_x , G_y \rangle , x \ne y$.
\end{quote}
{\bf Definition 6:}
A transitive permutation group is \emph{regular} if $|X|= |G^X|$ or, equivalently 
$|G_x|=1, \forall x \in X$ and $G^X$, transitive.
\\
\\
{\bf Theorem 7:}
Let $X$ be a faithful primitive $G-set$ with $G_x$ simple.  Then either $G$ is simple or
every non-trivial normal subgroup $H$ of $G$ is a regular normal subgroup.
\begin{quote}
\emph{Proof:}  
Suppose $1 \ne H \lhd G$.  Since $G$ is primitive, $H$ is transitive.  $H_x= H \cap G_x \lhd G_x$
so, since $G_x$ is simple,  $H_x = 1$ of $H_x= G_x$.  If $H_x=1$, $H$ is a regular normal subgroup
of $G$.  So $H_x = G_x$ and since $H$ is transitive, $|H:H_x|= |G:G_x|$, so $G$ is simple.
\end{quote}
{\bf Definition 7:} $A$ acts \emph{semi-regularly} on $G$ if $C_G(a)=1, \forall a \in A^{\#}$.  
\\
\\
{\bf Theorem 8:}
Suppose $A$ acts semi-regularly
on $G$.  Then (1) $|G| = 1 \jmod{|A|}$, (2) $A$ is semi-regular on each $A-$invariant
subgroups factor group of $G$, (3) $\forall p \in \pi(G)$, $\exists! A-$invariant
Sylow $p-$subgroup of $G$, (4) $\forall a \in A, g \mapsto [g,a]$ is a permutation of
$G$, (5) if $2 | |A|, \exists t: |t|=2, t \in A: g^t = g^{-1}, g \in G$ and
$G^{(1)}=1$.
\begin{quote}
\emph{Proof:}  
(1) each orbit of $A$ on $G^{\#}$ has length $|A|$.  (2) is clear (second part comes from coprime action).
(3) By Schur, the set of $A$-invariant Sylow $p$-groups is non-empty.  Since $C_G(A)= 1$ is transitive on this
set, the subgroup is unique. (4) $[g,a]= [h,a]$ iff $h g^{-1} \in C_G(a)$.  The commutator map is already an
injection and since $G$ is finite, it is a bijection. (5)  Let $t \in Inv(A)$ and $g \in G$.  By previous result,
$g= [h,t]$ for some $h \in G$ and $g^t = (h^{-1} h^t)^t= g^{-1}$.  Therefore $x^t =x^{-1}$ and $G$ is abelian.
Finally, if $s \in Inv(A)$, $s$ inverts $G$ so $st \in C_A(G)=1$ and $t$ is unique.
\end{quote}
{\bf Theorem 9:}
Let $\Delta$ be an orbit of $G$ and let $\delta \in \Delta$.  For each $\gamma \in \Delta$
let $v(\gamma ) \in G$ be such that $\delta \mapsto \gamma$.  Finally, suppose $S$
generates $G$.  
Then $G_{\delta}= \langle v(\gamma)sv(\gamma^{s})^{-1} | \gamma \in \Delta, s \in S \rangle$.
\begin{quote}
\emph{Proof:}  
See, Stellmacher.
\end{quote}
\section {Fixed point free automorphisms} 
{\bf Fixed Point Free:}  An automorphism, $\phi$ acting on $G$ is fixed point free if $C_G(\phi) =1$.
\\
\\
{\bf Lemma 1:} Let $\phi$ be a fixed point free automorphism acting on $G$ with $|\phi|=n$.  Then
(1) $y \in G$ then $y= x^{-1}(x \phi)$ for some $x$, and 
(2) $\forall x \in G, x(x \phi) (x \phi^2) \ldots (x \phi^{n-1}) = 1$.
\begin{quote}
\emph{Proof:}  
$x^{-1}(x\phi) = y^{-1}(y\phi)$ $\rightarrow$ $y x^{-1} = (y x^{-1})\phi$, so $x=y$.  Thus
$|{x^{-1}(x\phi): x \in G}| = |G|-1$ and (1) holds. Since $\exists y: x= y^{-1}(y\phi)$, so
$x(x \phi) (x \phi^2) \ldots (x \phi^{n-1}) =
y^{-1}(y \phi) (y \phi^2) \ldots (y \phi^{n-1})^{n-1}) = y^{-1}y = 1$.
\end{quote}
{\bf Lemma 2:} Let $\phi$ be a fixed point free automorphism acting on $G$ then $\phi$
leaves a unique $S_p$subgroup of $G$ invariant.
\begin{quote}
\emph{Proof:}  
Let $Q \in S_p(G), (Q)\phi = y^{-1}Qy$.  $y^{-1}= (z\phi)z^{-1}$ and $y = z(z^{-1}]phi)$.
$Q\phi = (z\phi) (z^{-1}Q z) (z^{-1}\phi)$ and $(z^{-1}Qz)\phi= z^{-1}Qz$ and thus
$\phi$ leaves $P=z^{-1}Qz$ fixed.  If both $P, Q$ are $\phi$-invariant and 
$Q=x^{-1}Px$ then
$Q=(x^{-1}\phi)P(x(phi)$ and $(x\phi)x^{-1} \in N(P) = N$ but $N$ is $\phi$-invariant and
$\phi$ is fixed point free on $N$ so $\exists z: y=(z\phi)z^{-1}$.
$(z\phi)z^{-1}=
(x\phi)x^{-1}$ $\rightarrow$ $x=z$.
\end{quote}
{\bf Lemma 3:} Let $\phi$ be a fixed point free automorphism acting on $G$, $H \lhd G$ and $H= H\phi$ then
$\phi$ is fixed point free on $G/H$.
\begin{quote}
\emph{Proof:}  
Let ${\overline G} = G/H$ and ${\overline x} = {\overline x}\phi$.  ${\overline x}^{-1}({\overline x}\phi)$
so $y= x^{-1}(x\phi) \in H$.  Since $\phi$ induces a fixed point free automorphism on $H$, $y= z^{-1}(z\phi)$.
So $x = z$, $x \in H$ and ${\overline x} = 1$.  Thus $\phi$ induces a fixed point free automorphism on
${\overline G}$.
\end{quote}
{\bf Lemma 4:} Let $\phi$ be a fixed point free automorphism acting on $G$ of order $2$, then $G$ is abelian.
\begin{quote}
\emph{Proof:}  
Since $x(x\phi) = 1$, $x^{-1} = (x\phi)$.  If 
$x, y \in G$, $y^{-1}x^{-1} = (xy)^{-1}= ((x\phi)(y\phi))^{-1} = x^{-1} y^{-1}$, so $G$ is abelian.
\end{quote}
{\bf Lemma 5:} Let $\phi$ be a fixed point free automorphism acting on $G$ of order $3$, then $G$ is
nilpotent and $[x, x\phi]=1, \forall x \in G$.
\begin{quote}
\emph{Proof:}  
$x(x\phi)(x\phi^2)=1$, so $x(x\phi)= (x\phi^2)^{-1}$ and $x$ commutes
with $x\phi$ for any $x \in G$.
Let $P$ be the unique $\phi$-invariant $S_p$ subgroup of $G, \forall p \in \pi(G)$.  We show
$P \lhd G$.
Suppose not and
$Q \ne P \in S_p(G)$ and pick $x \in Q \setminus P$, put $H=\langle x, x\phi \rangle$.
$H$ is a $p$-group since $[x, x\phi]=1$ and $H'=1$ and $H \phi = H$.  So $H \subseteq P$ but then
$x \in P$.
\end{quote}
{\bf Thompson:}
Let $G$ be a transitive permutation group on $X$ and $1 \ne g \in G$ fixes
no more than one element then $N= \{g: X_g= \emptyset \}$ is a normal subgroup of $G$.
Thompson showed any finite group having a fixed point free automorphism of prime order is nilpotent.
\begin{quote}
\emph{Proof:}  
Let $G$ be a counterexample of minimal order and $\phi$ be a fixed point free automorphism of prime order $r$.
$G$ has a proper normal subgroup, $H \ne 1$ with $H\phi=H$.  By induction, $H$ is nilpotent and $\phi$
is a fixed point free automorphism on $G/H$ so $G/H$ is nilpotent and $G$ is solvable.  If $G$ has no non-trivial
normal subgroup, which is $H$-invariant.  Let $P \in S_p(G), p \ne 2$ with $P\phi=P$.  Put
$N= N_G({\mathbb Z}(J(P)))$, $ {\mathbb Z}(J(P))) \; char \; P$.  If $N < G$, $N$ is nilpotent:
$N$ has a normal $p$-complement, $K$.  By the Thompson $p$-complement theorem, $G=KP$, $K\phi = K$ so
$K = 1$ and $P=G$ so $G$ is nilpotent.  We may assume $G$ is solvable.
Suppose $H_1, H_2 \lhd G$ $H_i\phi = H_i$, $H_1 \cap H_2 = 1$.  ${\overline G_i} = G/H_i$ is nilpotent
so ${\overline L} = {\overline G_1} \times {\overline G_2}$ is too.  $x\phi = (H_1x, H_2x)$.
$\psi: G \rightarrow {\overline G_1} \times {\overline G_2}$.  $G\psi$ is nilpotent.
Let $N$ a minimal normal subgroup of $G$, $N$ is elementary abelian and ${\overline G} = G/N$ is nilpotent.
${\overline G}$ is not a $p$-group.  Let ${\overline Q} \in S_q({\overline G})$ with 
${\overline Q} = {\overline Q} \phi$, $q \ne p$.  Let ${\overline M}$ be a minimal $\phi$-invariant
subgroup of $\Omega_1({\mathbb Z}({\overline Q}))$.  ${\overline M} \ne 1$ and ${\overline M} \lhd {\overline G}$
since ${\overline G}$ is nilpotent.  Let $H$ be the inverse image of ${\overline M}$ then
$H=NM$ and $M$ is a non-trivial elementary abelian $q$-group.  $H \lhd G$, $H\phi = H$ and $M\phi=M$.
$\phi$ acts irreducibly on $M$.  $H \subset G$ then $H$ is nilpotent.
$M \; char \; H \lhd G$ and $M$ and $N$ are two minimal normal $\phi$-invariant subgroup and
$M \cap N \ne 1$.  $C_M(N)\phi = C_M(N)$ since $\phi$ acts irreducibly on $M$.  Either
$C_M(N) = 1$ or $C_M(N) = N$.
If $C_M(N) = N$, $G$ is nilpotent so $C_M(N) = 1$.  Let $G^*$ be the semidirect product of
$M$ by $\langle \phi \rangle$.  $G^*$ acts irreducibly on $N$ as a vector space since
$C_M(N) = 1$ and $\phi$ is fixed point free.  But $G^*$ is a $p'$-group $C_{M^*}(M) = M$ and
$G^*/M$ has order $q$ so $C_N(\phi) \ne 1$.
\end{quote}
{\bf O-Nan-Scott:}  Let $G$ be a finite primitive permutation group of degree $n$ and
$H=soc(G)$.  Then either (1) $H$ is a regular elementary abelian $p$ group for
some $p$ and $G$ is isomorphic to a subgroup of
$AGL_m(P)$ ; or, (2) $H$ is isomorphic to $T^m$ where $T$ is a non-abelian simple group
with a bunch of conditions.
\section {Mathieu groups are simple}
We will use one result from a future section here.
\\
\\
{\bf Theorem 1:} 
Let $N$ be a finite group $G \subseteq Aut(N)$.  Then $G$ acts as a permutation group on $N^{\#}$.  Further, \\
(i) If $G$ is transitive, then $N$ is an elementary abelian $p$-group.
\\
(ii) If $G$ is $2$-transitive, then $N$ is either an elementary abelian $2$-group or $|N|=3$.
\\
(iii) If $G$ is $3$-transitive, then $|N|=4$.
\\
(iv) $G$ cannot be $4$-transitive.
\\
\begin{quote}
\emph{Proof:}  These are all clear.
\end{quote}
{\bf Theorem 2:} 
Let $G$ be an transitive permutation group and $N \lhd G$.  Then $N$ is ${\frac 1 2}$ transitive.  If
$N \ne 1$ and $G$ is primitive then $N$ is transitive.
\begin{quote}
\emph{Proof:}
Let $B$ be an orbit of $N$ of minimal length. $B^{Ng}= B^{gN} = B^g$, so $B^g$ is a union of orbits of $N$.
By minimality of $|B|$, $B^g$ is a single $N$ orbit.  These form a system of imprimitivity.
\end{quote}
{\bf Theorem 3:} 
Let $G$ be primitive with no regular normal subgroups.  If $G_a$ is simple then $G$ is simple.
\begin{quote}
\emph{Proof:}
Let $N \lhd G$.  $N_a \leq (N \cap G_a) \lhd G_a$ so either $N_a =1$ or $N=G_a$.  If $N_a=1$, $N$
is a regular normal subgroup.
\end{quote}
{\bf Theorem 4:} 
Let $G$ be $m$-transitive on $A$, $|A|=n$ with a regular normal subgroup, $N$.\\
(i) if $m=2$, then $n = |N|= p^k$
\\
(ii) if $m=3$, then $n = 3$ or $n = 2^k$ or $n = 3$.
\\
(iii) if $m=4$, then $n = 4$.
\\
(iv) We cannot have $m>4$.
\begin{quote}
\emph{Proof:}
\end{quote}
{\bf Theorem 5:}  Let $G$ be transitive and $H < G$ also be transitive.  The $G= H G_a$
\begin{quote}
\emph{Proof:}
Clear.
\end{quote}
{\bf Theorem:}  The Mathieu groups are simple.
\begin{quote}
\emph{Proof:}
\\
Step 1: $M_{11}$ is simple.\\
\\
Put $G = M_{11}$ so $|G| = 11 \cdot 10 \cdot 9 \cdot 8$.  Let $P=\langle x\rangle$ be a subgroup of order
$11$. $P$ acts transitively on $\Omega$.  If $P \subseteq A$, $A$, abelian, then $A$ is transitive and regualr so
$A=P$.  Thus $C_G(P)=P$.  Since $|Aut(P)|=10$, $|N_G(P)/P| \mid 10$.
If $2 \mid N_G(P)$ let $y \in N_G(P), |y|=2$. $y$ has a fixed point since $11$ is odd and $x^y = x^{-1}$.  Thus
$y$ is a product of $5$ transpositions thus $y \notin A_{11}$ which is a contradiction.  So $|N_G(P)/P|$ is $1$ or $5$.
Let $1 \ne H \lhd G, H \ne G$.  Since $G$ is primitive, $H$ is transitive and $11 \mid |H|, P \subseteq H$.
$G= H G_a$ and $G_a \subseteq N_G(P)$ so $G= H N_G(P)$ and $N_G(P) \nsubseteq H$ and $N_H(P)=P$.  So, in $H$,
$P$ is in the center of its normalizer and $H$ has a normal $11$ complement, $K$.  $K \lhd G$ and $11 \nmid |K|$ so
$K = 1$.  Thus $H=K$ and $|G|=55$, which is a contradiction.
\\
Step 2:  $PSL_3(4)$ is simple. This was already shown.
\\
\\
Step 3:
By the above result, $M_{12}, M_{22}, M_{23}, M_{24}$ cannot ahve a regular normal subgroup.
These groups are all $3$-transitive.  $M_{11} = (M_{12})_a$ so $M_{12}$ is simple.
$PSL_3(4)  = (M_{22})_a$, $M_{22} = (M_{23})_a$, and $M_{23} = (M_{24})_a$ proving their simplicity.
\end{quote}
