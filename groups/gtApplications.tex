\chapter{Applications of Signalizers and Amalgams for groups with solvable proper subgroups}
\section {Outline and definitions}
We refer to the major classification theorems in the Classification of Finite Simple Groups section.  The
goal here is to classify the $N$-groups, as defined below by reducing to these cases.
\section {N-groups}
{\bf $N$-group:}  Non solvable groups all of whose $2$-local subgroups are solvable.
\\
\\
{\bf Condition Z:}  $N_G(\Omega({\mathbb Z}(S))) \le N_G(S)$ for $S \in S_2(G)$.
\\
\\
{\bf Definition ZN-group:} An $N$-group which satisfies condition $Z$.
\\
\\
{\bf Condition {\cal C}:}  $G$ is of even order and $C(t)$ is solvable $\forall t \in Inv(G)$.
\\
\\
{\bf Recall:}  $M$ is \emph{strongly-embedded} if
$|M|_2 > 1$ and $|M \cap M^g|_2=1, \forall g \in G \setminus M$.  $C_L(O_2(L)) \le O_2(L)$ for all
$2$-locals, $L$.
\\
\\
{\bf Lemma:} Let $G$ be $p$-separable, $p \in \pi(G)$ and $P \in S_p(O_{p', p}(G))$ then
$O_{p'}(G)=1 \rightarrow C_G(O_p(G)) \subseteq O_p(G)$.
\begin{quote}
\emph{Proof:}
Given the definition of $p$-separable, this was proved earlier.
\end{quote}
{\bf Observation:} For an $N$-group, $O_{2'}(L)=1$ for all $2$-locals, $L$.
\\
\\
{\bf Modified $ZN$ Theorem (Theorem 1):}  Let $G$ be a $ZN$-group
with $O_{2'}(G)=1=O_2(G)$ and $S \in S_2(G)$.  Put
$Z= O^2(G)$, $R= S \cap H$.  One of the following holds:\\
(1) $H$ contains a strongly embedded subgroup.\\
(2) $R$ is dihedral or semi-dihedral.\\
(3) $Z \cap R \cong C_2$ and $Z \cap R$ is weakly closed in $R$ with respect to 
$H$.\\
(4) $\Omega(R)=Z= C_2 \times C_2$.
\\
The first outcome is handled by Bender's classification.  The second by the Gorenstein-Walter
classification and the Alperin-Brauer-Gorenstein classification.  The third case is handled by the $Z^*$
theorem and the fourth case is handled by Goldschmidt's classification of groups with a strongly-closed
abelian $2$-subgroup.
\\
\\
{\bf Definition:}  $G$ has local characteristic $2$-type if $|G|$ has even order and
$C_L(O_2(L)) \leq O_2(L)$.
\\
\\
{\bf Theorem 2:}  Let $G$ be a $ZC$-group with $O_{2'}(G)=1=O_2(G)$ then case (1), (3) or (4)
of the Modified $ZN$ Theorem holds or $O^2(G) \Omega({\mathbb Z}(S))$ has local characteristic $2$.
\\
\\
{\bf Theorem 3:} Let $G$ be a $ZN$-group with $O_2(G)= 1$ then either (a)
$G$ possesses a strongly embedded subgroup or (b) there are two maximal $2$-local subgroups,
$M_1, M_2$ of $O^2(G)$ such that $M_1 \cong M_2 \cong S_4$ and $M_1 \cap M_2 \in S_2(M_i), i = 1,2$.
\\
\\
Note: Conclusion (b) of theorem 3 implies case (2) of theorem 1.
\section{Application of the Completeness Results}
{\bf Theorem:}  Suppose $G$ satisfies ${\cal C}$ and $O_{2'}(C(t)) \ne 1, \forall t \in Inv(G)$,
then $G$ has local characteristic $2$-type.
\begin{quote}
\end{quote}
{\bf Theorem:} Let ${\cal B}$ be the set of maximal Abelian $2$-subgroups of $G$ that contain
an elementary abelian subgroup of rank $3$ then 
$\theta_B: a \mapsto O_{2'}(C_G(a)), a \in \Omega(B)^{\#}$ for $B \in {\cal B}$ is a solvable
signalizer function.  Denote its maximal element as $\theta_B(G)$.
\begin{quote}
\end{quote}
Now we begin the proof of theorem 2 above.
\\
\\
{\bf Condition {\cal S}:}  $G$ is a $ZC$ group with $O(G)= O_2(G)=1$.
$H=O^2(G)$ and
$S \in S_2(G), Z= \Omega({\mathbb Z}(S)), T= S \cap H \in S_2(H)$. ${\cal B}(G)$ is the set
of maximal abelian $2$-groups of $G$ that contain a subgroup of order $8$.   For $B \in {\cal B}(G)$,
$\theta_B(G): a \mapsto O_{2'}(C_G(a)), a \in \Omega(B)^\#$.  $\theta_B$ is a solvable $\Omega(B)$
signalizer functor.  For $R = \theta_B(G)$, (1) $C_R(a) = O_{2'}(C_G(a)), a \in B^\#$,
(2) $C_R(B_0) = O_{2'}(C_G(B_0)), 1 \neq B_0 \leq B$, (3) $R= \langle O_{2'}(C_G(a)), a \in B^\# \rangle$,
(4) $R^g = \theta_{B^g}(G)$ for $g \in G$ so $N_G(B) \leq N_G(R)$.
\\
\\
\section{$J(T)$-Components}
{\bf Theorem 3:}  Let $G$ be a $ZN$-group ofl local characteristic $2$ with $O_2(G)=1$ then
$G$ possesses a strongly embedded subgroup or there are two maximal $2$-locals $M_1$ and $M_2$
of $O^2(G)$ such that $M_1 \cong S_2 \cong M_2$ and $M_1 \cap M_2 \in S_2(M_i)$.
\begin{quote}
\end{quote}

\section{$N$-groups of characteristic $2$-type}
$G$ is a $ZN$-group of local characteristic $2$-type with $O_2(G)=1$, $S \in S_2(G), Z = \Omega(Z(S))$
and $M$ is a $2$-local containing $N_G(J(S))$.
\\
\\
{\bf Definition:}  ${\cal T}(M)$ is a set of $2$ subgroups $T \leq M$ such that
there is a $2$-local $L$, $T \leq L \leq G$ and $L \not\leq M$.
\\
\\
{\bf Theorem:} $M$ is strongly embedded in $G$ iff ${\cal T}(M)= \emptyset$.
\begin{quote}
\emph{Proof:} If ${\cal T}(M)= \emptyset$ then $M$ is strongly embedded.  
Suppose ${\cal T}(M) \neq \emptyset$ and $M$ is strongly embedded.  $N_G(T) \leq M, T \in {\cal T}(M)$.
Choose $T \in {\cal T}(M)$ with $|M|$ maximal and let $T \leq L \leq G$ and $L \not\leq M$.
$N_{N_0}(T) \leq M \cap L$ for $T \leq T_0 \in S_2(L)$ so $T=N_{T_0}(T)=T_0$ by maximality.
$O_2(L) \leq T_0 \leq M \cap L$ so $O_2(L) \in {\cal T}(M)$ thus $L \leq N_G(O_2(L)) \leq M$, a contradiction.
Apply Thompson Transfer lemma.
\end{quote}
The Amalgam results give.\\
{\bf Theorem:} Let $T \in {\cal T}(M)$.  There exist two different maximal $2$-locals, $P_1 , P_2$ with
$P_i \in S_2(T), i = 1, 2$ and either $P_1 \cong P_2 \cong S_4$ or $P_1 \cong P_2 \cong S_4 \times {\mathbb Z}_2$.
\begin{quote}
\end{quote}
{\bf Proof of theorem 3}
\begin{quote}
Assume $M$ is not strongly embedded. Suffices to show $M_1 \cong M_2 \cong S_4$
\end{quote}
{\bf Proof of theorem 1}
\begin{quote}
We may assume $HZ$ has local characteristic $2$.  If $HZ$ has a strongly embedded subgroup, we're done.  Otherwise,
$H$ contains a maximal $2$-local, $P \cong S_4$.  $O_2(P) \cong {\mathbb Z}_2 \times {\mathbb Z}_2$ and $P= N_H(O_2(P))$.
$D \in S_2(P)$ is dihedral of order $8$.  After conjugation, $D \leq S \cap H = R$. Put $Z^*= \Omega(Z(R))$. 
$Z^* \leq O_2(P)$ and $\exists t \in Inv(O_2(P))$ with $O_2(P) = Z^* \times \langle t \rangle$.  Thus
$C_R(t) = C_R(O_2(P)) = O_2(P)$ and $R$ is dihedral or semi dihedral.
\end{quote}
