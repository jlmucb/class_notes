\chapter{Miscellaneous Results}
\section{Thompson like characterization}
{\bf Netto:}  Let $x,y \in S_n$ be selected randomly.  
$Pr[ \langle x,y \rangle =S_n]= {\frac 3 4}$.  
\\
\\
{\bf Thompson Order Formula:} Suppose $G$ has $s\geq 2$ classes of involutions, $C_1, \ldots , C_s$.
Put $d_{ijk} = |\{ (u,v): u \in C_i, v \in C_j, w=(uv)^m, w \in C_k \}|$.  Put $n_i = {\frac {|G|} {|C_i|}}$ then
$|G| = n_i n_j \sum_k d_{ijk} $. (Proved earlier)
\\
\\
{\bf Sample characterization of $S_5$:}
If $G$ is a finite group with two conjugacy classes of involutions ($ccl_1, ccl_2)$ having
centralizers, $C_1= C(u_1) = \langle u_1 \rangle \times S_3$ and
$C_2= C(u_2) = D_8$ then $G = S_5$.
\\
In $S_5$, $u_1= (12)$, $u_2= (12)(34)$.
\\
\\
1. $C_2 \in S_2(G)$.
\\
\\
2.  After conjugation, we can assume, $u_1 \in C_2, u_2 \in C_1$.
Let $x_1 \in ccl_1, x_2 \in ccl_2$ be involutions.  Let $S_i= \{ (u, v):
u \in x_1^G, v \in x_2^G, x_i \in \langle u, v \rangle \}$. Put $s_i = |S_i|$.
$|G|= s_1|C_G(x_2)| + s_2 |C_G(x_1)|$.
(In the real $G$, $|ccl_G(u_1)| = 10$ and $|ccl_G(u_2)|= 15$.)
Note that if $x \in x_1^G$ and $y \in x_2^G$, $|xy|$ is even (otherwise they would be
conjugate), $x,y \in C((xy)^{\frac {|xy|} 2})$.
\\
\\
3. 
$C_2 \cong D_8$ has three conjugacy classes of involutions.  
$D_8= \langle u, v \rangle $, $u^2=v^2=1$, $t= uv$ and $|t| = 4$.
$C_2= \langle (12),(14)(23) \rangle $, $t= (1324)$.
$u_2= (12)(34)$.
$ccl_{C_2}((14)(23))= \{ (14)(23), (13)(24) \}$,
$ccl_{C_2}((12))= \{ (12), (34) \}$,
$ccl_{C_2}((12)(34))= \{ (12)(34) \}$.  If $x \ne u_2$, $ x \sim x u_2$.
$G$ contains a non-cyclic group, $V$, of order $4$.
$V= ccl_{C_2}((14)(23)) \cup \{ 1 \} \cup ccl_{C_2}((12)(34))$.
Since $G$ has two classes of involutions, $s_2 = 0$ or $s_2 = 4$ and all the
involutions in $V$ are $G$-conjugate.
\\
\\
4.
$C_1$ has three conjugacy classes of involutions.  $x \ne u_1 \rightarrow x \nsim x u_1$.
$ \langle u_1 \rangle \times S_3$. 
$ccl_{C_1}((12))= \{ (12) \}$,
$ccl_{C_1}((12)(34))= \{ (12)(34), (12)(35), (12)(45), \}$,
$ccl_{C_1}((34))= \{ (34) , (35), (45) \}$.  If $x \ne u_1$, $ x \nsim x u_1$.
Since $G$ has two classes of involutions and for any $u_1 \ne x \in Inv(C_1 )$,
exactly one of $x$  or $xu_1$ is conjugate to $u_1$, $s_1 = 9$.
\\
\\
5.
$|G|= 9 \cdot 8 + 4 \cdot 12 = 120$, if $s_2=4$ or $|G|= 9 \cdot 8 = 72$, if $s_2 = 0$.
\\
\\
6. $|G| \ne 72$: If $|G|=72$, let
$P \in S_3(C_1)$, $P \subseteq Q \in S_3(G)$.  
$|Q|=9$ and $ \langle C_1 , Q \rangle \subseteq N(P)$,
$36 \mid |N(P)|$.
$\exists H: C(P) \subset H: |H|=36$.  $u_1 \in H$ and $u_2$ is a square in $H$.
$H \lhd G$ and $H$ contains all involutions in $G$ hence $C_2 \cap H$ contains all involutions of
$C_2$ but $|C_2 \cap H|=4$ and there are $5$ involutions in $C_2$.
\\
\\
7. By 3, $C_2$ has a normal four-group, $V$, with $u_2 \in V$ and all the involutions
in $V$ are $G$-conjugate.
Let $x \in G$ then $x^{-1} u_2 x \ne u_2$, $x^{-1} u_2 x \in V$ then
$x^{-1} C_2 x \ne C_2$ and
$u_2 \in C(x^{-1} u_2 x) = x^{-1} C_2 x$.
\\
\\
8. By 7, $N(V)$ contains at least two Sylow subgroups of $G$.
\\
\\
9.  $C(V)=V$.
$N(V)/V \cong S_3$ (permutes all $3$ involutions of $V$), by 7 above, and $|N(V)|= 24$.
\\
\\
10. $G:N(V)= 5$. Let $\phi: G \rightarrow \{N(V)x_i\}$ be the permutation representation
on the cosets of $N(V)$ in $G$.  $ker(\phi) = 1$ so $G \cong S_5$.
If $g \in G$ is an element
of order $5$, $g$ acts as a $5$-cycle on the cosets.  If $t$ is an element of order $2$,
it is of the form $(12)$ or $(12)(34)$.
\section{Isaac's treatment of central extensions}
{\bf Definition:} A central extension of $G$ is a pair $(\Gamma, \pi)$ with $\pi: \Gamma \rightarrow G$ and $ker(\pi) = {\mathbb Z}(\Gamma)$.
\\
\\
{\bf Theorem 1:}
Let $(\Gamma, \pi)$ be a central extension of $G$, $A=ker(\pi)$ and let $X= \{x_g| g \in G \}$ be coset representatives for $\Gamma/A$.
Define $\alpha$ by $x_g x_h = \alpha(g,h) x_{gh}$, then $\alpha \in Z(G,A)$ is well defined and the resulting multiplication is associative.
\begin{quote}
\emph{Proof:} Compute $x_g x_h x_k$ both ways and compare.  Put $y_g = \mu(g)x_g$ then
$y_g y_h = \mu(g) \mu(h) \mu(gh)^{-1} \alpha(g,h)$.
\end{quote}
{\bf Definition:} $A$ divisible if $\forall a \in A, n \in {\mathbb Z}^+, \exists b \in A: b^n=a$.  $M(G)= H(G, {\mathbb C}^{\times})$.
$\hat{A}$ is the set of linear characters of $A$.
\\
\\
{\bf Theorem 2:}
Let $A$ be abelian, $Q \subseteq A$, $Q$, divisible and $|A:Q| < \infty$ then $Q$ is complemented in $A$.
\begin{quote}
\emph{Proof:} By induction on $|A:Q|$.  True if $|A:Q|=1$.  Choose $a \in G \setminus A$. $n=|aQ|$ in $A/Q$.
$u = a^n$, $v \in Q: v^n=u$.  Put $b= av^{-1}$ and $\langle b \rangle \cap Q =1$.  ${\overline A}= A/\langle b \rangle$.
${\overline Q} = Q\langle b \rangle / \langle b \rangle$ and ${\overline Q} = Q$.  
$|{\overline A}:{\overline Q}| = |A:Q\langle b \rangle| < |A:Q|$, so ${\overline Q}$ is complemented by induction and 
$\exists B \subseteq A: B \cap Q\langle b \rangle = \langle b \rangle$ and $QB =A$.  $Q \cap B = Q \cap Q\langle b \rangle \cap B= $
$Q \cap \langle b \rangle = 1$.
\end{quote}
{\bf Theorem 3:}   If $F$ is algebraically closed and $G$ is finite, $H(G,A)| \mid |G|$.
\begin{quote}
\emph{Proof:} 
Claim: $B(G, F^{\times})$ is divisible:
$B(G, F^{\times})$ is complemented in $Z(G, F^{\times})$.  For $\beta \in B(G, F^{\times})$, $\beta = \delta(\mu)$ for some
$\mu:G \rightarrow F^{\times}$.  Choose $\nu(g) \in F^{\times}: \nu(g)^n = \mu(g)$, we can do this since $F$ is algebraically closed.
$\delta(\nu)^n = \delta(\mu)= \beta$.  
We can apply previous lemma and once we show $|H(G, F^{\times})| < \infty$.\\
\\
For $\alpha \in Z(G, F^{\times})$, put
$\mu(g) = \prod_{x \in G} \alpha(g,x)$.
$\alpha(g, hx) \alpha(h, x) = \alpha(gh, x)\alpha(g, h)$ so
$\prod_{x \in G} \alpha(g, hx) \alpha(h, x) = \prod_{x \in G} \alpha(gh, x)\alpha(g, h)$ so
$\mu(g) \mu(h) = \mu(gh) \alpha(g, h)^{|G|}$.  $\alpha(g, h)^{|G|} \in B(G, F^{\times})$ and
$exp(H(G, F^{\times})) \mid |G|$.
Putting $U = \{ \alpha \in Z(G, F^{\times}): \alpha^{|G|} = 1 \}$ and $A = \langle B(G, F^{\times}), \alpha \rangle$.
$|A : B(G, F^{\times})| \mid |G|$.  By the previous result, $B(G, F^{\times})$ is complemented in $Z(G, F^{\times})$ so
$B(G, F^{\times})U = Z(G, F^{\times})$.   $\forall u \in U, u: G \times G \rightarrow \{y \in F: y^{|G|} = 1 \}$.
$|H(G, F^{\times})| = B(G, F^{\times})U:B(G, F^{\times})| \leq |U| < \infty$, and we can apply the previous result, proving the lemma.
\end{quote}
{\bf Definition:}  Choose a set of coset representatives in $\Gamma/A$, as before and $\pi(x_g) = g$.  Let
$\alpha \in Z(G, A)$, defined by $x_g x_h = \alpha(g,h) x_{gh}$.  For $\lambda \in \hat{A}$, define $\eta(\lambda) = {\overline {\lambda(\alpha)}}$.
$\lambda(\alpha)(g,h) = \lambda(\alpha(g,h))$.  Bar is the canonical map from $Z(G, {\mathbb C}^{\times})$ to $H(G, {\mathbb C}^{\times}) = M(G)$.
$\eta: \hat{A} \rightarrow M(G)$ is defined by that map, called the \emph{standard map}.
\\
\\
{\bf Theorem 4 (Schur):} Given $G$ and $A$, abelian, there is a central extension with $ker(\pi)=A= M(G)$
\begin{quote}
\emph{Proof:}
Let $M$ be a complement of $B(G, F^{\times})$ in $Z(G, F^{\times})$. Let $M = \hat{A}$.
Define $\alpha(g,h)$ by $\alpha(g, h)(\gamma) = \gamma(g,h), \gamma \in M$.  $\alpha(g,h) \in A$.
$\alpha(gh,k) \alpha(g,h) (\gamma) =  \gamma(gh, k) \gamma(g,h)$.  
Since $\gamma$ runs over factor sets, $\alpha \in Z(G, F^{\times})$.  Let $\Gamma, G, X$ be as in the earlier results.
$\lambda(\alpha(g,h))) = \alpha(g,h)(\gamma) = \gamma(g,h)$ and $\lambda(\alpha) = \gamma$.  
$|A| = |\hat{A}| \geq |\eta(\hat(A))| = |M(G)| = |A|$.
\end{quote}
