\chapter{Nilpotent Groups and Frattini Subgroup}
\section{Series}
{\bf Definition of lower central series:} $G^{(0)}=G$, $G^{(n+1)} = [G^{(n)},G]$.
\\
\\
{\bf Definition 1:} $G$ is {\bf nilpotent} if
$G^{(n)}=1$ for some $n$.  Note $G^{(n)}/G^{(n+1)} \subseteq {\mathbb Z}(G/G^{(n+1)})$.
\\
\\
{\bf Definition of upper central series:}
$G^{[0]}=1$. Define $G^{[n+1]}= $ preimage of ${\mathbb Z}(G/G^{[n]})$.
\\
\\
{\bf Definition 2:}
A normal series, $N_k \subseteq N_{k-1} \subseteq  \ldots \subseteq N_1 = G$,
is \emph{central} if $[N_i , G] \subseteq N_{i+1}$.
\\
\\
{\bf Theorem 1:}  Let $G$ be a finite group, the following are equivalent:
(1) $G$ has a central series which reaches $1$;
(2) $G^{(n)} = 1$ for some $n$;
(3) $G^{[n]} = G$ for some $n$;
(4) $N_G(H) > H$ if $H \ne G$;
(5) Every maximal subgroup of $G$ is normal in $G$ and the quotient group has index $p$;
(6) Every Sylow subgroup of $G$ is normal in $G$.  When these hold, we say $G$ is
\emph {nilpotent}.
\begin{quote}
\emph{Proof $1 \rightarrow 2$:}
Let $G=N_0 \supseteq N_1 \supseteq \ldots \supseteq N_n$ with $[G, N_i] \subseteq N_{i+1}$.
$G^{(0)} \subseteq N_0$ and $G^{(i+1)}= [G^{(i)}, G] \subseteq [N_i, G] \subseteq N_{i+1}$
so $G^{(n)}=1$.
\\
\\
\emph{Proof $2 \rightarrow 3$:}
We prove that $G^{[n-i]} \supseteq G^{(i)}$, $G^{(0)}=G=G^{[n]}$ and $G^{(n)}=1=G^{[0]}$.
\\
\\
\emph{Proof $3 \rightarrow 4$:}
$\exists i$ such that $G^{[i]} \subseteq H$ and $G^{[i+1]} \nsubseteq H$.  Then
$[G^{[i+1]}, G] \subseteq G^{[i]} \subseteq H$ and there is a $x \in G^{[i+1]}$,
$x \notin H$ with $xHx^{-1}=H$.
\\
\\
\emph{Proof $4 \rightarrow 5$:}
Let $M$ be a maximal subgroup.  $N_G(M) > M$ so $G= N_G(M)$.
\\
\\
\emph{Proof $5 \rightarrow 6$:}
Let $M$ be a maximal subgroup containing $N_G(P)$, $P \in S_p(G)$ then $M \lhd G$ and
by Frattini, $G= M N_G(P)= N_G(P)$.
\\
\\
\emph{Proof $6 \rightarrow 1$:}
Let $P_1, P_2, \ldots, P_n$ be the Sylow subgroups and put $H= P_1 P_2 \ldots P_n$,
$H \lhd G$.  Put $G^*= [P_1, G]= [P_1, P_1] \ne P_1$ and consider
$G \supseteq G^*$.  $G^*$ has a central series by induction.
\end{quote}
{\bf Corrollary:}  If $G$ is nilpotent, $G= \bigoplus_{p \mid |G|} O_p(G)$ and 
$O_p(G)$ is a Sylow subgroup of $G$.
\section{Frattini Subgroup}
{\bf Lemma:} Let $1 \ne H \lhd G$ and $J$, a $p$-group in $G$.  If $|H| \neq 1 \jmod{p}$ then
$H \cap C_G(J) \neq 1$.
\begin{quote}
\end{quote}
{\bf Theorem 2:} If $G$ is nilpotent and $1 \ne H \lhd G$ then (1) $[H,G] \nsubseteq H$ and
(2) $1 \ne H \cap {\mathbb Z}(G)$.
\begin{quote}
\emph{Proof:} (1) If $[H,G]=H$ then $[H, G, G, \ldots, G]=H$. (2) Count!
\end{quote}
{\bf Definition 3:} $\Phi(G)= \bigcap_{M \in {\cal M}(G)} M$ where ${\cal M}(G)$ is the set of
maximal subgroups of $G$.  
It is called the \emph {Frattini subgroup} of $G$.  
\\
\\
{\bf Theorem 3:} 
(1) If $\langle \Phi(G), X \rangle = G$, $\langle X \rangle = G$.
$\Phi(G) \; char \; G$; further,
(2) $\Phi(G)$ is nilpotent. (3)
$F(G)/\Phi(G)= F(G/\Phi(G))$.
\begin{quote}
\emph{Proof:}  
For (1), if not put $H=\langle X \rangle$.  $H$ is in some maximal subgroup of $G$, say $M$,
So $\Phi(G) \subseteq M$ and $\langle \Phi(G), X \rangle \subseteq \langle \Phi(G), H \rangle \subseteq M$,
contradiction!.
Automorphisms, $\phi$, and their inverses take maximal subgroups in $G$ into
maximal subgroups in $G$, so 
$ \bigcap_{M \in {\cal M}} M = \bigcap_{M \in {\cal M}} \phi(M)= \Phi(G)$.
Let $P \in S_p(\Phi(G))$.  Since $\Phi(G) \lhd G$, $G= \langle \Phi(G), N_G(P) \rangle = N_G(P)$ and so
$P \lhd G$ so $\Phi(G)$ is nilpotent.  (3) follows from (2).
\end{quote}
{\bf Theorem 4:} If $G$ is nilpotent, $G' \subseteq \Phi(G)$.
\begin{quote}
\emph{Proof:}  Suppose $G$ is nilpotent and $M$ is a maximal subgroup of $G$.
$N_G(M)>M$ so $M \lhd G$.  If $1 \ne a \in G/M$, $ \langle a, M \rangle = G$ so $G/M$ is cyclic and
$G' \subseteq M$.  This is true for every $M$ so $G' \subseteq \Phi(G)$.
\end{quote}
{\bf Theorem 4:} $G$ is nilpotent iff it is solvable and $p$-closed $\forall p$.
\begin{quote}
\emph{Proof:} If $G$ is nilpotent, $O_p(G) = P \in S_p(G)$ and $G= \bigotimes_{p \mid |G|} O_{p_i}(G)$ and the result follows.
The reverse inclusion goes by induction $G$ is solvable, $O_p(G) \neq 1$, some $p$, applying induction to ${\overline G}=
G/O_p(G)$ gives the result.
\end{quote}
{\bf Theorem 5:} If $\Phi(G)$ and ${\overline G}= G/\Phi(G)$ are nilpotent then so is $G$.
\begin{quote}
\emph{Proof:} Let ${\overline P} \in S_p({\overline G})$ then ${\overline P} \lhd {\overline G}$.
Put $N= P \Phi(G) \lhd G$.  
By Frattini, $P \in S_p(G) \rightarrow N_G(P)N= N_G(P) \Phi(G)= N_G(P)$.
\end{quote}
{\bf Theorem 6:} $\Phi(G) \subseteq F(G)$.
\begin{quote}
\emph{Proof:} $\Phi(G)$ is nilpotent and so is $F(G)$ and so is their product.  Since
$F(G)$ is maximal, this product is in $F(G)$.
\end{quote}
{\bf Theorem 7:} If $P$ is a $p$-group then ${\overline P}= P/\Phi(P)$ is elementary abelian.
\begin{quote}
\emph{Proof:}  Suppose $M$ is maximal in $P$.  As above, $M \lhd P$ and $P' \subseteq M$.
Suppose ${\overline x} \in {\overline P}$.  $|x|= p^m$ and there is a the characteristic
subgroup $N$ of $P$ with index $p$ must be in $M$ (otherwise $P=NM$).  
Thus $xM$ has order $p$.  This is
true for all $M$ so ${\overline x}$ has order $p$ and as noted ${\overline P}$ is abelian so
it is elemenetary abelian.
\end{quote}
{\bf Theorem 8:} $\Phi(G/N)= \Phi(G)N/N$.
\begin{quote}
\emph{Proof:} Let ${\overline A}= A/N$, $N \lhd G$.  Maximal subgroups of ${\overline G}$ are
the maximal subgroups of $G$ containing $N$.
\end{quote}
{\bf Theorem 9:} If $S \subseteq G$ and $ \langle S, \Phi(G) \rangle =G$ then $ \langle S \rangle
=G$.  If $H/\Phi(H)$ is cyclic then $H$ is cyclic.
\begin{quote}
\emph{Proof:}  Suppose $ \langle S \rangle
=H < G$. Let  $M$ be a maximal subgroup containing $H$.  
$\Phi(G) \subseteq M$, so $ \langle \Phi(G), S \rangle \subseteq H \ne G$.  Contradiction.
\end{quote}
