\chapter{Basic Definitions}
\section{Notation, commutator calculus, characteristic subgroups}
{\bf Definition 1:}
$Core_G(H)= \bigcap_{g \in G} H^g$ (Can use this to show $|G:Core_G(H)|\le |G:H|!$).
$O^{\cal A}(G)= \bigcap_{A \lhd G, G/A \in {\cal A}} A$.
$O_{\cal A}(G)= \prod_{A \lhd G, A \in {\cal A}} A$.\\
\\
{\bf Definition 2:} Let ${\cal M}_G= \{ M: M$ is a non-trivial minimal normal subgroup of $G\}$;
the \emph{socle of $G$}, denoted
$soc(G)$, is defined by
$soc(G)= \langle M \rangle_{M \in {\cal M}}$.
$O_{\pi}(G)=$ maximal normal $\pi-$subgroup of $G$.
$O^{\pi}(G)=$ smallest normal subgroup of $G$ such that $G/O^{\pi}(G)$ is a $\pi$-group.
$G$ is $p-$closed if $O_p(G) \in S_p(G)$.
\\
\\
{\bf Definition 3:}
A homomorphism $\varphi: G \rightarrow X$ is \emph {faithful} if it is injective.
\\
\\
{\bf Definition 4:}
${\cal SCN}(P)=$ set of self centralizing normal subgroups of $P$.
${\cal SCN}(p)= {\cal SCN}(P)$ where $P \in {\cal SCN}(P)$.
\\
\\
{\bf Definition 4.5:} If $\pi = \{ p_1, p_2, \ldots , p_k\}$ is a set of primes, we say
$G$ is a $\pi$ group if the only primes dividing $|G|$ are in $\pi$.  If $G$ is a finite
group and $H \leq G$, we say $H$ is an $S_{\pi}$ group if $H$ is a $\pi$ group and 
$p \nmid |G:H|$  if $p \in \pi$.  $S_{\pi}(G)$  is the set of all $S_{\pi}$ subgroups of
$G$. For $\pi = \{ p \}$, we call an $S_{\pi}$ group and $S_p$ group and put $S_{\pi}(G) = S_p(G)$.
$S_p(G)$ is just the sylow subgroups of $G$.
\\
\\
{\bf Definition 5:}
${\cal N}_G(A, \pi)=$ set of all $A-$invariant $\pi$ subgroups of $G$.
${\cal N}_G^*(A, \pi)=$ maximal subgroups in ${\cal N}_G(A, \pi)$.  For a $p-$group, $P$,
$\Omega_n(P)= 
\langle x \in P: x^{p^n}=1 \rangle $ and
$\mho_n(P)= \langle x^{p^n}: x \in P \rangle $.
\\
\\
{\bf Definition 6:} Let $H \subseteq G$ and let $S$ be an $H$-invariant subset of $G$;
$H$ is said to 
control fusion in $S$ if for $s \in S$, $s^G \cap S = s^H$.
Let $X \le H \le G$.  $X$ is \emph{weakly closed} in $H$ with respect to $G$ if
$X^g \cap H = \{X\}$.\\
\\
{\bf Definition 7:}
$G$ is $p$\emph{-constrained} if 
$P \in S_p ( O_{p',p} (G))$ implies $C_G(P) \subseteq O_{p',p} (G)$.
Equivalently, if $O_{p'}(G)=1$ then $C_G(O_p(G)) \subseteq O_p(G)$ and $F(G)= O_p(G)$..
\\
\\
{\bf Definition 8:}
$Z$ is \emph {weakly closed} in $P$ with respect to $G$ if $Z^g \subseteq P \rightarrow Z^G=Z$.
The \emph {weak closure} of $Z$ in $P$ with respect to $G$ is $wcl_G(Z, P)= 
\langle Z^g, Z^g \subseteq P \rangle $.
\\
\\
{\bf Definition 9:}
$G$ is $p$\emph{-stable} if $p \ne 2$ and 
if $A \in p(N(P))$ with $[P,A,A]= 1$ implies $A C(P)/C(P) \subseteq O_p (N(P)/C(P))$.
\\
\\
{\bf Definition 10:}
$m_p(P)$ is the rank of the largest elementary abelian $p$-group in $P$.
\\
\\
{\bf Definition 11:}
$O_{\infty}(G)=$ largest solvable normal subgroup of $G$.
\\
\\
{\bf Definition 12:} $F(G)$, the \emph{Fitting subgroup of $G$}
is the unique maximal, normal, nilpotent subgroup of $G$ and
$F(G)= \prod_p O_p(G)$.
\\
\\
{\bf Definition 13:} $A^{\pi}(G)$ is the unique smallest normal subgroup of $G$ such that
$G/A^{\pi}(G)$ is an abelian $\pi$-group.  If $P \subseteq H \subseteq G$ then
$ |G:A^p(G)| \le |H:A^p(H)|$ if equality holds we say $H$ \emph {controls $p$-transfer in $G$}.
\\
\\
{\bf Definition 14:}
$G$  is
$\pi$\emph{-solvable} if there is a normal series whose factors
consist of either $\pi'$-groups or a solvable
$\pi$-groups.
\\
\\
{\bf Definition 15:} $G$ is $\pi$\emph{-closed} if $G/O_{\pi}(G)$ is a $\pi'$ group and thus
$O_{\pi}(G)=O^{\pi'}(G)$.
\\
\\
{\bf Definition 16:}
$E_{p^n}$ denotes the elementary abelian $p-$group of rank $n$.
$m_{2,p}(G) = max \{ m_p (H) \}$, where $H$ is 2-local.
$e(G) = max \{ m_{2,p} (G), p \ne 2 \}$ ($e(G)$ is a good approximation of
the Lie rank.).
\\
\\
{\bf Definition 17:} The group $A/B$ is called a section of $G$ if $B \lhd A <G$.
\\
\\
{\bf Definition 18:} $G$ is $\pi$-separable if there are characteristic subgroups
$A_0, A_1, \ldots , A_n$: $1=A_0<A_1<\ldots<A_n=G$ such that $A_i/A_{i-1}$ is
either a $\pi$ group or a $\pi'$ group.
\\
\\
{\bf Definition 19:} $G$ is $p$-closed if $O_p(G)= O^{p'}(G)$.
\\
\\
{\bf Definition 20:} 
$O^{\cal K}(G)= \bigcap_{A \lhd G, G/A \in {\cal K}} A$.
$O_{\cal K}(G)= \bigcap_{A \lhd G, A \in {\cal K}} A$.
\\
\\
{\bf Definition 21:}
$O_{p'}(G)$ is called the $p$\emph{-core} of $G$. $O_{2'}(G)$ is often called the
\emph{core} of $G$ and is sometimes denoted by $O(G)$.
\\
\\
{\bf Definition 22:}
$G$ is 
\emph{metacyclic} if $\exists H \lhd G: G/H ,H$ are cyclic. 
\\
\\
{\bf Notation:}  Let $G$ be a finite group, $x, y \in G$, $U, V \subseteq G$,
$A \subseteq Aut(G)$.
\begin{enumerate}
\item $ \langle U \rangle $ denotes the smallest subgroup of $G$ containing $U$.
\item $x^g= g^{-1}xg$.
\item $[x,g]= x^{-1} g^{-1} x g$.
\item $U^g= \{ u^g: u \in U \}$
\item ${\overline U} = U/N$.
\end{enumerate}
{\bf Theorem 1:}
$ [x, y^{-1}, z] [y, z^{-1}, x] [z, x^{-1}, y]=1$
(\emph{Jacobi Identity}).
$[ab,c]= [a,c]^b [b,c]$ and $[a,bc]=[a,c] [a,b]^c$.  
\begin{quote}
\emph{Proof:} Straightforward calculations.
\end{quote}
{\bf Three Subgroups:} $A, B, C \subseteq G$ and
$N \lhd G$ with
$[A,B,C] \subseteq N$ and
$[B,C,A] \subseteq N$ then
$[C,A,B] \subseteq N$.
\\
\\
{\bf Dedekind's Lemma:}  If $G=HK$ and $H \subseteq G_0$ then $G_0 = H(G_0 \cap K)$.
\begin{quote}
\emph {Proof:} 
$g_0=hk$ so $g_0 h^{-1} \in G_0 \cap K$ (since $h \in G_0$).
\end{quote}
{\bf Theorem 2:}  If $x,y \in G$ , $z=[x,y]$, $[z, x]=1=[z,y]$ then (1) $[x^n, y^m]=z^{mn}$ and
(2) $(yx)^n= z^{(n(n-1))/2}y^n x^n$.
\begin{quote}
\emph {Proof of 1:} Claim: Under the conditions of the theorem, $[x^n,y]= z^n$.  Proof by induction.
It is true for $n=1$.  $[x^{n+1},y]= x^{-(n+1)}y^{-1}x^{n+1}y= x^{-n}x^{-1} y^{-1} x x^n y$.
So $[x^{n+1},y]= x^{-1} x^{-n} y^{-1} x^n y y^{-1} x y= x^{-1} z^n y^{-1} x y= z^{n+1}$.
Proof of theorem is by induction on $m$.  It is true for $m=1$, by the claim.  Now,
$[x^n,y^{m+1}]= (x^{-n} y^{-1} x^n y) y^{-1} x^{-n} y^{-m} x^n y^m y= z^n y^{-1} (z^{mn}) y$;
so $[x^n,y^{m+1}]= z^n y^{-1} (z^{mn}) y = z^n z^{mn}= z^{n(m+1)}$ which proves the result.
\\
\\
\emph {Proof of 2:} Again, the proof is by induction on $n$; it is true for $n=1$.
$(yx)^{n+1}= (yx)^n (yx)= 
z^{\frac {n(n-1)} {2}} y^n x^n y x= y^{n+1} x^n (x^{-n} y^{-1} x^n y) x
= z^{\frac {n(n-1)} {2}} y^{n+1} x^n z^n x= z^{\frac {n(n+1)} 2} y^{n+1} x^{n+1}$.
\end{quote}
{\bf Theorem 3:}
If  $H \; char \; K$ and $K \;  char \; G$ then $H \; char \; G$.
\\
\\
{\bf Theorem 4:}
If $H, K < G$, $[H,K] \lhd \langle H, K \rangle$.
If $H, K, L \lhd G$ then $[HK,L]= [H,L] [K, L]$. 
$[xy,z]= [x,z] [x,z,y] [y,z]$.
$[x, yz]= [x,z] [x,y] [x,y,z]$.
\begin{quote}
\emph{Proof:}  
Straightforward.
\end{quote}
{\bf Theorem 5:}
If $A$ is a cyclic group of maximal order in and abelian group $G$, then $A$ is a direct
factor of $G$.
\begin{quote}
\emph{Proof:}  
By induction on $|G|$.
$G$ must possess a non-cyclic abelian $S_p$ group for some $p$.
$\exists H: pH=1, H \cap A = \{ 1 \}$, set ${\overline G}= G/H$.
If $|{\overline A}|= |A|$, the result follows by induction so
${\overline G} = {\overline A} {\overline B}$,
${\overline A} \cap {\overline B} = \{ 1 \}$.  Let $A$ and $B$ be the inverse images
under the homomorphism.  $A \cap B \subseteq H$ but by construction, $H \cap A= \{ 1 \}$
and the result follows.
\end{quote}
{\bf Theorem 6:}
If $H \lhd G$ and $(|G:H|, |H|)=1$ then $H \; char \; G$.
\begin{quote}
\emph{Proof:}  
Let $\phi$ be an automorphism that does not fix $H$ and put $H'= \phi(H)$.
$H H'$ is a subgroup of $G$ and
$|H H'| > |H|$.  ${\frac {(H H')} {H}} \cong {\frac {H'} {H \cap H'}}$.
Put $k= |H H'|, m= |G:H|, n= |H|, d= | H \cap H' |$,  then 
$ 1 < k'= {\frac {k} {n}} =  {\frac n d}$ and $k' \mid n$.  So $dk'= n$ and $(k', d)=1$,
thus $d=1, H = H'$ and the result follows.
\end{quote}
{\bf Theorem 7:}
Let $H$ be an elementary abelian $p$-group of $G$ and $x \in N_G(H)$ and let $\phi_x$
be the linear transformation induced by $x$ on $H$ then the minimal polynomial of
$\phi_x$ is $(x-1)^r$, where $[h,x, \ldots, x]= 1$ and $x$ appears $r$ times.
\begin{quote}
\emph{Proof:}  
Straightforward calculation.
\end{quote}
{\bf Theorem 8:}  If $G$ has no non-trivial characteristic subgroups then
$G= \bigoplus_{i=1}^n H_i$, where $H_1 \cong H_i, \forall i$, and $H_i$ is
simple.
\begin{quote}
\emph{Proof:} Let $H_1$ be a minimal normal sugroup of $G$.  If $H_1 = G$, we're done.
Now let $H \le G$ be maximal subject to (a) $H=H_1 H_2 \ldots H_n$, (b) $H_1 \cong H_i$,
(c) $H_i \lhd G$ and (d)
$H=H_1 H_2 \ldots H_{i-1} H_{i+1} \ldots H_n \cap H_i = 1$.  Claim:  $H=G$.  Observe
$1 \ne H \; char \; G$; if $\phi \in Aut(G)$, 
$H_i^{\phi} \lhd G$.
So $H_i^{\phi} \cap H \lhd G$.
If $H_i^{\phi} \cap H =1$, $H$ is not maximal and if
$H_i^{\phi} \cap H 1\ge $, $H_1$ is not minimal.
Thus $H^{\phi} = H$, $H \; char \; G$, so $G=H$.
\end{quote}
{\bf Corrollary:} If $G$ is solvable and $H \le G$ is characteristically simple, $H$
is an elementary abelian $p$-group.

