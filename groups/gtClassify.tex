\chapter{Classification of Finite Simple Groups}
\section{Early Results}
{\bf Feit-Thompson (FT):}  The only finite simple groups of odd order are
${\mathbb Z}_p, p\ne 2$. The proof follows the CN classification.\\
\\
Proof of FT is similar to the proof that all $CA$ groups are solvable. (In a $CA$-group,
the centralizer of any element is abelian.).  The plan is to try and prove the following:
Let $G$ be a minimal counterexample, for every maximal subgroup $M<G$, $\exists M_0 \lhd M$,
such that:\\
(a) $C_{M_0}(a) =1, a \in M - M_0$;\\
(b) $M \cap {M_0}^g = 1$ for $g \in G-M$;\\
(c) $M_0$ is nilpotent;\\
(d) $M/M0$ is cyclic; \\
(e) the union of all such $M_0$'s is a partition of $G$.\\
The notion of stability shortens the original proof.
\\
\\
{\bf Glauberman $ZJ$:}  If $C_G(O_p(G)) \le O_p(G)$ and the action of
$G$ on its chief factors of $G$ is $p-$stable then $G=N_G(Z(J(S)))$. \\
\\
{\bf Glauberman's $Z^*$ Theorem:}
Let $G$ be a finite group and $t$ and involution in $G$ which is weakly closed in
$C(t)$.  Then $t^* \in Z(G^*)$  where $G^*= G/O_{2'} (G)$.
\section{Major Classification Results}
{\bf Brauer, Suzuki:} Suppose the Sylow $2$-subgroups of $G$ are quaternion then
${\mathbb Z}^*(G)/O_{2'}(G) \cong C_2$.
\\
\\
{\bf Gorenstein, Walter:} Suppose $O_{2'}(G)=1$ and that the sylow $2$ subgroups of $G$ are
dihedral then $F^*(G)$ is isomorphic to $PSL_2(q), q=1 \jmod{2}$ or $A_7$.
\\
\\
{\bf Alperin, Brauer, Gorenstein:} Suppose $G$ is simple and the Sylow 
$2$-subgroups are semi-dihedral then $G$ is isomorphic to 
$PSL_3(q), q= -1 \jmod{4}$,
$PSU_3(q), q= 1 \jmod{4}$,
or $M_{11}$.
\\
\\
{\bf Bender:} Suppose $G$ possesses a strongly embedded subgroup then the Sylow
$2$-subgroups of $G$ are cyclic or quaternion or $G$ possesses a normal series
$1 \lhd M \lhd L \lhd G$ such that $M$, and $G/L$ have odd order and
$L/M$ is isomorphic to 
$PSL_2(2^n)$, $Sz(2^{2n-1})$, or $PSU_3(2^n), n \ge 2$.
\\
\\
{\bf Goldschmidt:} Suppose $S \in S_2(G)$ and $A$ and Abelian subgroup of $S$
such that $a \in A, a^g \in S \rightarrow a^g \in A$.  Suppose that $G= \langle A^G \rangle $ and
$O_{2'}(G)=1$ then $G=F^*(G)$, $A=O_2(G) \Omega(T)$ and for every component
$K$ of $G$, the factor group $K/{\mathbb Z}(K)$ is isomorphic to
$PSL_2(2^n)$, $Sz(2^{2n-1})$, $PSU_3(2^n), n \ge 2$, $PSL_2(q), q = 3, 5 \jmod {8}$,
$R(3^{2n+1})$, or $J_1$.
\\
\\
{\bf Thompson:} Suppose $G$ is a non-solvable group all of whose
$2$-locals are solvable $\forall p \in \pi(G)$, then $F^*(G)$ is isomorphic to
$PSL_2(q), q>3$, $Sz(2^{2n-1})$, $PSU_3(2^n), n \ge 2$, $A_7$, $M_{11}$, $PSL_3(3)$,
$PSU_3(3)$
or $^2F_4(2)'$.
\\
\\
{\bf Gorenstein, Lyons, Janko, Smith:} Suppose $G$ is a non-solvable group all of whose
$2$-locals are solvable, then $F^*(G)$ is isomorphic to
$PSL_2(q), q > 3$, $Sz(2^{2n-1})$, $PSU_3(2^n), n \ge 2$, $A_7$, $M_{11}$, $PSL_3(3)$,
$PSU_3(3)$ or $^2F_4(2)'$.
\\
\\
{\bf Walter:} Let $G$ be a group with 2 rank $\ge 5$ and $O_{2'}(G)=1$ with the property
that the centralizer of every involution is $2-$constrained then $O_{2'}(C(x))=1$ for
every involution $x$.
\\
\\
{\bf Example of groups of semi-simple type:} $GL_n(q)$, $q$, odd, $t \in Inv(G)$ has
form 
$t= \left(
\begin{array}{cc}
-I_m & 0\\
0 & I_r\\
\end{array}
\right), n= m+r$ and 
$C_G(t)= \left(
\begin{array}{cc}
A & 0\\
0 & B\\
\end{array}
\right)$, 
$A \cong GL_m(q)$ and
$B \cong GL_r(q)$. \\
\\
{\bf Definition:} 
$e(G)= max \{ m_p(H),
H \le G  \textnormal{ is a } 2-\textnormal{local and } p \ne 2 \}$.
A group $G$ is \emph{quasi-thin} if 
$m_2(G) \ge 3$ but every $e(G) \le 2$.
\\
\\
{\bf Classification decomposes as follows:} (1) Minimal counterexample has characteristic
$2$-type ($C_H(O_p(H)) \le O_p(H)$) where $H$ is a $2$-local (Lie type $q= 2^n$) small
groups are quasi-thin (2) $G$ is not of characteristic $2$-type, small groups have
$m_2(G) \le 2$.
\section{Connected Groups}
{\bf Definition:}  
Let $\Omega_G$ be a collection of subgroups of $G$ \emph{define}
${\cal D}(\Omega_G)$ as the graph with points $\Omega_G$; two points
$A,B \in \Omega_G$ are joined by an edge if $[A,B]=1$.  Note that $G$ acts as a
group of automorphisms on ${\cal D}(\Omega_G)$ via conjugation.
\\
\\
{\bf Theorem:} 
Let $\Delta$ be a $G$-invariant collection of subgroups of $G$ and $H < G$.
The following are equivalent:
(1) $H$ controls fusion in $H \cap \Delta$ and $N_G(X) \le H$ for $x \in H \cap \Delta$;
(2) $H \cap H^g \cap \Delta= \emptyset$ if $g \in G \setminus H$;
(3) the members of $H \cap \Delta$ fix a unique point in the permutation representation
of $G$ on $G/H$ by right multiplication;
(4) $H \cap \Delta$ is the uniion of a set, $\Gamma$, of connected components of
${\cal D}(\Delta)$ and $\Gamma \cap \Gamma^g= \emptyset$ if $g \in G \setminus H$.
\begin{quote}
\emph{Proof:}  
\\
\\
$1 \rightarrow 2$:
Let $g \in G$ with $H \cap H^g \cap \Delta \ne \emptyset$.  $\exists X \in H \cap \Delta$
with $X^g \le H$ so $H$ controls fusion in $H \cap \Delta$ and
$X^{gh}=X, h \in H$.  $gh \in N_G(X) \le H$ so $g \in H$.
\\
\\
$2 \rightarrow 3$:
Consider the representation $G \rightarrow G/H$.  $X \in H \cap \Delta$ fixes
$Hg$ iff $X \le H^g$ which happens iff $g \in H$.
\\
\\
$3 \rightarrow 4$:
For $A \in H \cap \Delta$, $\{ H \} = Fix(A)$ so $N_G(A) \le H$.  If $B \in \Delta$ is incident
to $A$ in ${\cal D}(\Delta)$ then $B \le C(A) \le H$ so $B \in H \cap \Delta$ and
$H \cap \Delta$ is the union of some set $\Gamma$ of connected components of ${\cal D}( \Delta )$.
Further, if $A \in \theta \in \Gamma$ and $\theta^g \in \Gamma$ then
$A^g \le H$ so $\{ H \} = Fix(A^g ) = \{ Hg \}$ and $g \in H$.
\\
\\
$4 \rightarrow 1$:
If $X \in H \cap \Delta$ and $g \in G$ with $X^g \le H$ then 
$X \in \theta \in \Gamma$ and
$X^g \in \theta' \in \Gamma$, so $\theta' = \theta^g \in \Gamma \cap \Gamma^g$ hence
$g \in H$.
\end{quote}
{\bf Definition:}  
Define ${\cal E}_k^p(G)$ to be the set of all elementary abelian
$p$-subgroups of $G$ of $p$-rank at least $k$. $G$ is said to be $k$\emph{-connected}
for the prime $p$ if ${\cal D}({\cal E}_k^p(G))$ is connected.  
Define $\Gamma_{P,k}(G) = \langle N_G(X), X \rangle \le P, m(X) \ge k \rangle$ and
$\Gamma_{P,k}^0(G) = \langle N_G(X), X \le P, m(X) \ge k, m(XC_P(X)) > k \rangle$.  
If $P \in S_p(G)$ then
$\Gamma_{P,k}(G)$ is called the \emph{$k$-generated $p$-core of $G$}.
\\
\\
{\bf Theorem:} 
Let $P$ be a $p$-group.  Then 
(1) $P$ is $1$-connected for the prime $p$;
(2) If $m(P) > 2$, $\exists E_{p^2} \cong U \lhd P$ and $X \in {\cal E}_2^p(P)$ is in
the same connected component of ${\cal D}({\cal E}_k^p(P))$ as $U$ when
$m(C_P(X)) > 2$;
(3) If $p=2$ and $\exists E_{8} \cong U \lhd P$ then $P$ is $2$-connected for the prime $2$;
(4) If $p=3$ and $m(P) > 3$ then $P$ is $2$-connected for the prime $3$.
\begin{quote}
\emph{Proof:}  
(1) follows from ${\mathbb Z}(G) \ne 1$.  Assume $m(G) > 2$ then
there is an $E_{p^2} \cong  U \lhd G$ and $G/C_G(U) \le SL_2(p)$ and
$SL_2(p)$ is of $p$-rank $1$.  If $A \in {\cal E}^2_3(G)$ then 
$m(C_A(U) )  \ge 2$
so $m(UC_A(U)) > 2$ and
$m(C_A(U) ) > 2$.  If 
$U \ne E \in {\cal E}^p_2(G)$ is in the same connected component
as $U$, then
$\exists D: E \ne D \in {\cal E}^p_2(G)$ and $E$ is adjacent to $D$ and
further,  $m(C_G(E)) \ge m(DE) \ge 3$.  All we need to show is that
$A \in {\cal E}^p_2(G)$ is in the same connected component as $U$. 
But $\langle A, C_A(U), U C_A(U), U \rangle$ is in a path of
${\cal D}({\cal E}^p_2(G))$ completing 2.
Assume $p = 2$ and $E_8 \cong V \lhd G$ and let
$E \in {\cal E}^p_2(G)$, we must show there is a path from $E$ to $V$.
For $e \in E: m(C_V(e)) \ge 2$ and so we can assume $e \in E \setminus V$ so
$\langle E, \langle e, C_V(E) \rangle, C_V(e), V \rangle$ is a path in
${\cal D}({\cal E}^p_2(G))$.
$\exists E_8 \cong V \lhd G$
and the argument above establishes 4.
\end{quote}
{\bf Theorem:} 
Let $H \le G, P \in S_p(H)$ and $k \in {\mathbb Z}^+$.  Then
(1) If $m(p) \ge k$ and $\Gamma_{P,k}(G) \le H$ then $P \in S_p(G)$; and
(2) If $m(p) > k$ and $\Gamma_{P,k}^0(G) \le H$ then $P \in S_p(G)$.
\begin{quote}
\emph{Proof:}  
Follows from previous result.
\end{quote}
{\bf Theorem:} 
Let $H \le G$, $P \in S_p(G)$ and $m(p) \ge k$.  Then the following are equivalent:
(1) $\Gamma_{P,k}(G)<H$;
(2) $H$ controls fusion in ${\cal E}_k^p(H)$ and $N_G(X) \le H$ for
$x \in {\cal E}_k^p(H)$;
(3) $m(H \cap H^g)<k$ for $g \in G \setminus H$;
(4) each member of
${\cal E}_k^p(H)$ fixes a unique point in the permutation representation of $G$ on $G/H$.
\begin{quote}
\emph{Proof:}  
Parts 2, 3, 4 are equivalent by the above result except
${\cal E}^p_k(G)$ should be
${\cal E}^p_k(H)$.  It remains to show 1 implies 2.
Assume
$\Gamma_{P, k}(G) \le H$ and $X \in {\cal E}^p_k(H)$.  It STS
that if $g \in G$ with $X^g \le H$ then $g \in H$.  By Sylow,
$\langle P, P^g \rangle \le P$ and by the above
$P \in S_p(G)$.  By Alperin, $\exists P_i \in S_p(G), 1 \le i \le n$ and
$g_i \in N_G(P \cap P_i )$ with $g= g_1 g_2 \ldots g_n$, $X \le P_1$
and $X^{g_1 g_2 \ldots g_i} \le P \cap P_i$ and $m(P \cap P_i) \ge k$
so $g_i \in N_G(P \cap P_i) \le
\Gamma_{P, k}(G) \le H $ and so
$g= g_1 g_2 \ldots g_n \in H$ and 2 holds.
\end{quote}
{\bf Definition:}  If $k=1$ in any of the above equivalent conditions we say $H$ is
\emph{strongly $p$-embedded in $G$}.
\\
\\
{\bf Theorem:} 
Let $P \in S_p(G)$ and suppose $P$ is $k$-connected then
(1) ${\cal E}_k^p(\Gamma_{P,k}(G))$ is a connected component of
${\cal D}({\cal E}_k^p(G))$ and
$\Gamma_{P,k}(G))$ is the stabiliser of that component;
(2) $G$ is $k$-disconnected for the prime $p$ iff $G$ has a proper $k$-generated $p$-core.
\begin{quote}
\emph{Proof:}  
1 implies 2.  Let $H= \Gamma_{P, k}(G)$.  By the earlier resutls,
${\cal E}^p_k(G)$
is a union of connected components of
${\cal D}({\cal E}^p_2(G))$
while
${\cal E}^p_k(P)$ is contained in some component since
$P$ is $k$-connected and $H \le N_G( \Delta )$.  Hence
${\cal E}^p_k(H) \subseteq \Delta$ by Sylow. Thus
$\Delta = {\cal E}^p_k(H)$ and $H= N_G( \Delta )$ and 2 holds.
\end{quote}
{\bf Theorem:} 
$G$ posseses a strongly $p$-embedded subgroup iff $G$ is $1$-disconnected for the prime $p$.
\begin{quote}
\emph{Proof:}  
Follows from previous result.
\end{quote}
{\bf Theorem:} 
Let $m_p(G) > 2, P \in S_p(G)$ and
${\cal E}_k^p(G)^0$ be the set of subgroups 
$X \in {\cal E}_k^p(G)$ with $m_p(XC_G(X)) > 2$.  Then
(1) ${\cal E}_k^p(G)^0$ is the set of points that are not isolated in
${\cal D}({\cal E}_k^p(G))$;
(2) ${\cal E}_2^p(\Gamma_{P,2}^0(G))^0$ is a connected component of
${\cal D}({\cal E}_2^p(G))$ and
$\Gamma_{P,2}^0(G))$ is the stabilizer of this connected component;
(3)
${\cal D}({\cal E}_2^p(G))^0$ is connected iff
$G= \Gamma_{P,2}^0(G))$.
\begin{quote}
\emph{Proof:}  
1 is trivial and 3 easily follows from 2.   By the previous result,
${\cal E}^p_2(G)^0$ is contained in a connected component $\Delta$ of
${\cal D}({\cal E}^p_2(G))$.  Thus
$H=
\Gamma_{P, 2}^0(G) \le N_G( \Delta )$.
Let $\Gamma= {\cal E}^p_2(G)^0$.
Since ${\cal E}^p_2(G)^0 \subseteq \Delta$ and $H$ acts on $\Delta$,
$\Gamma \subseteq \Delta$ by Sylow.  If $\Delta \ne \Gamma$, $\exists x \in \Gamma$,
$Y \in \Delta \setminus \Gamma$ with $X$ and $Y$ are adjacent
${\cal D}( \Delta )$.   WLOG,
$X \in {\cal E}^p_2(G)^0$, so $Y \le C_G(X) \le H$ and hence
$m_P(C_H(Y)) \ge m_P(XY) \ge 3$, $Y \in \Gamma$, a contradiction.
So $\Delta = \Gamma$, it remains to show that if 
$X, X^g \in \Gamma$ then $g \in G$.
Suppose
$X, X^g \in \Gamma$ then $N_G(X) \le H \ge N_G(X^g )$ so there is a
$E_{p^3} \cong A \le C_G(X) \le H$ and $A^g \le H$.  So by Sylow take $A$,
$A^g \le P$.  Now apply Alperin, using 
$ \Gamma_{P, 3}(G) \le \Gamma_{P, 2}(G)^0 \le H $ to conclude $g \in H$.
\end{quote}
{\bf Theorem:} 
Let $\Gamma$ be the elements $a \in G$ of order $p$ with $m_p(C_G(a)) > 2$ and
$\theta: \Gamma \rightarrow p'(G)$ such that $\forall a,b \in \Gamma, [a,b]=1$ and
all $g \in G: \theta(a^g)=\theta(a)^g$ and $\theta(a) \cap C_G(b) \le \theta(b)$.
Let $P \in S_p(G)$ and assume $G=\Gamma_{P,2}(G)$ and $O_{p'}(G)=1$ and finally assume
that either $\theta(a)$ is solvable for each $a \in \Gamma$ or the Signalizer Functor Theorem
holds on $G$ then $\theta(a)=1, \forall a \in \Gamma$.
\begin{quote}
\emph{Proof:}  
For $A \in {\cal E}^p_3(G)$, $\theta$ is an $A$-signalizer functor.  For
$B \in {\cal E}^p_2(G)$, define $W_B= \langle \theta(b): b \in B^{\#} \rangle$.  Then
$\exists A \in {\cal E}^p_3(G)$ with $B \le A$ and hence $W_B = W_A$ and
$\Gamma_{A,2}(G) \le N_G(W_A)$.  In particular, if $B, D$ are distinct members of
${\cal E}^p_2(G)$ adjacent in
${\cal D}({\cal E}^p_2(G))$ , $BD \in
{\cal E}^p_2(G)$ so $W_B = W_{BD} = W_D$.  Thus
$G = \Gamma^0_{P, k}(G) \le N_G(W)$ since
$N_G(B) \le \Gamma_{A,2}(G) \le N_G(W)$.  But by 1 and the solvable signalizer functor theorem,
$W$ is a $p'$ group so $W \le O_{p'}(G)$.  
Since $O_{p'}(G) = 1$ and $\theta(a) \le W, \forall a \in \Gamma$, the lemma is proved.
\end{quote}
{\bf Theorem:} 
Let $\Gamma$ be the elements $a \in G$ of order $p$ with $m_p(C_G(a)) > 2$ and
$P \in S_p(G)$ and assume $G=\Gamma_{P,2}(G)$ and $O_{p'}(G)=1$, $C_G(a)$ is balenced
for the prime $p$ and for each $a \in \Gamma$ either
$O_{p'}(C_G(a))$ is solvable or the Signalizer Functor Theorem holds on $G$ then
$O_{p'}(C_G(a))=1, \forall a \in \Gamma$.
\begin{quote}
\emph{Proof:}  
For $a \in \Gamma$, let $\theta(a)= O_{p'}(C_G(a))$ and we know
$\theta(a) \cap C_G(b) \le \theta(b), \forall a, b \in \Gamma$ with $[a,b]= 1$ and applying the
previous result, we're done.
\end{quote}
\section{Aschbacher's Plan}
{\bf Definition:} 
$m_{2,p}(G)= max_H \{ m_p(H) \}$, where $H$ is a $2$-local of $G$.
$e(G)= max \{m_{2,p}(G) \}, p$ odd.
\\
\\
{\bf Definition:}
$O_{p', E}(G)/O_{p'}(G)= E(G/O_{p'}(G))$.
\\
\\
{\bf Theorem:} $P \in p(G)$ then 
(1) $O_{p', E}(N_G(P)) \le C_G(O_p(G))$ and
(2) if $P \le O_p(G)$ then $O^p(F^*(N_G(P))= O^p(F^*(G))$.
\begin{quote}
\emph{Proof:}  
Todo.
\end{quote}
{\bf Definition:} 
$G$ is \emph{balenced} for $p$, prime if $O_{p'}(C_G(X)) \le O_{p'}(G), \forall X < G: |X|=p$.
$m_{2,p}(G)= max_H \{ m_p(H) \}$, where $H$ is a $2$-local of $G$.
$e(G)= max \{m_{2,p}(G) \}, p$ odd.
\\
\\
{\bf Theorem:} Let $O_{p'}(G)=1$ and $Aut_H(L)$ is balenced for the prime $p$ 
for each $L \in Comp(G)$ and for each $L \in Comp(G)$ and each $H \le G$ with $L \lhd H$
then $G$ is balenced for the prime $p$.
\begin{quote}
\end{quote}
{\bf Theorem:} If $G$ is a non-abelian simple group with $m_2(G) \le 2$ then either
(1) a Sylow $2$-group of $G$ is dihedrael, semi-dihedral, or
${\mathbb Z}_{2^n} \wr {\mathbb Z}_2$ and $G \cong L_2(q), L_3(q), U_3(q), q$, odd, or
$M_{11}$; or (2) $G \cong U_3(4)$.
\begin{quote}
\end{quote}
{\bf Theorem:} If $G$ is a non-abelian simple group with $m_2(G) > 2$  and
$G$ has a proper $2$-generated $2$-core then either $G$ is of Lie type of characteristic 2
and Lie rank 1 ($L_2(2^n), L_2(2^n), Sz(2^n)$ or $J_1$).
\begin{quote}
\end{quote}
{\bf Definition of $B_p$ property:}   Let $p$ be a prime and $O_{p'}(G)=1$ and $x$ and element of
order $p$ in $G$ then $O_{p',E}(C_G(x))= O_{p'}(C_G(x))E(C_G(x))$.
\\
\\
{\bf Theorem:}
Let $p \ne 2$ and suppose $G$ contains no normal abelian subgroups of rank $3$ then
$G$ has $p$-rank $\le 2$.
\begin{quote}
\end{quote}
{\bf Theorem:} If $G$ is a non-abelian simple group with $m_2(G) \le 2$ then either
(1) a Sylow $2$-group of $G$ is dihedrael, semi-dihedral, or
${\mathbb Z}_{2^n} \wr {\mathbb Z}$ and $G \cong L_2(q), L_3(q), U_3(q), q$, odd, or
$M_{11}$; or (2) $G \cong U_3(4)$.
\begin{quote}
\end{quote}
{\bf Theorem:} If $G$ is a non-abelian simple group with $m_2(G) > 2$  and
$G$ has a proper $2$-generated $2$-core then either $G$ is of Lie type of characteristic 2
and Lie rank 1 ($L_2(2^n), L_2(2^n), Sz(2^n)$ or $J_1$.
\begin{quote}
\end{quote}
{\bf Definition of $B_p$ property:}   Let $p$ be a prime and $O_{p'}(G)=1$ and $x$ and element of
order $p$ in $G$ then $O_{p',E}(C_G(x))= O_{p'}(C_G(x))E(C_G(x))$.
\\
\\
{\bf Unbalenced Group Theorem:} If $G$ is a group with $F^*(G)$, simple which is unbalenced for
the prime $2$, then $F^*(G)$ is a group of Lie type and odd characteristic $A_{2n+1}$, 
$L_3(4)$ or $He$.
\\
\\
{\bf Component Theorem:} Let $G$ be a group with $F^*(G)$ simple satisfying the $B_2$ property
and possessing an involutions, $t$, such that $O_{2',E}(C_G(t)) \ne O_{2'}(C_G(t))$ then
$G$ posseses a standard subgroup for the prime $2$.
\\
\\
{\bf Standard Form Problem for $(L,r)$:}   Determine all finite groups $G$ possessing a 
standard subgroup $H$ for the prime $r$: $E(H) \cong L$.
\\
\\
{\bf Theorem:} Let $G$ be a minimal counterexample to the classification theorem and assume $G$
is generic of even characteristic then one of the following holds:
(1) $G$ possesses a standard subgroup  for some $p \in \sigma(G)$,
(2) there is a symplectic involution, $t$, such that $F^*(C_G(t))$ is a $2$-group
of symplectic type or 
(3) $G$ is in the uniqueness case.
\\
\\
{\bf Aschbacher:} Suppose $G$ is a finite simple group, the $B$-Theorem holds, and $E(C_G(t)) \neq 1$
for some $t \in Inv(G)$.  $\exists x \in Inv(G)$ and a quasi-simple group,
$K$ of $C(z)$ such that $C_{C(z)}(K)$ either has $2$-rank 1 or is solvable with elementary abelian
or dihedral $2$-Sylow subgroups.
\\
\\
{\bf Standard form:} Suppose $x \in Inv(G)$ and  $H = C(x)$ has a normal quasi-simple subgroup $L$,
of Lie type with odd characteristic such that $C_H(L)$ has $2$-rank 1.   We say $H$ is in standard
form with standard component $L$.
\section{Outline of Thompson}
J.G. Thompson proved that Frobenius kernels are nilpotent.
It is enough to prove that a finite group which admits a fixed-point free automorphism $\alpha$
of prime order $p$ is nilpotent, which Thompson did in his thesis.
Before Thompson's, it was known that a finite solvable group which admits a fixed-point free automorphism 
of prime order is nilpotent. This was proved by G. Higman and J. Witt.
Here is a quick outline, making use of the more modern version of the normal $p$-complement theorems
(and standard properties of coprime automorphisms).\\
\\
\\
{\bf Theorem 1:}  Let $G$ be a finite group which admits a fixed-point-free automorphism 
$\alpha$ of prime order $p$ such that $G$ is not nilpotent, 
and let $G$  have minimal order subject to this. Then $|G|=1 \jmod{p}$.
\begin{quote}
\emph{Proof:}
We may assume that $G$ is not solvable. Hence we may choose an 
odd prime divisor $q$ of $|G|$.
Then $G$ has an $\alpha$-invariant Sylow $q$-subgroup $Q$, and $N(Z(J(Q)))$
is also $\alpha$-invariant. 
If $Z(J(Q)) \lhd G$ then $\alpha$ induces a fixed-point free automorphism on 
$G/Z(J(Q))$ so, by induction, that $G/Z(J(Q))$ is nilpotent,
and $G$  is solvable, contrary to assumption.  Hence $N(Z(J(Q)))$
is a proper subgroup of $G$, so is nilpotent by the minimality of $G$.
The Thompson's normal $q$-complement theorem shows that $G$ has a normal $q$-complement too.
Since $q$ was an arbitrary odd prime divisor of $|G|$, $G$ has a normal Sylow 2-subgroup $U$ and 
$G/U$ is nilpotent by minimality of $G$. So $G$ is again solvable, contrary to assumption.
\\
This now reduces to proving the result in the case that $G$ is solvable which .
reduces to representation theory. 
Further reducing to the case that $Q=F(G)$ is a minimal normal subgroup of $G$. This means $Q$ is 
an elementary Abelian $q$-group for some prime $q$. 
Furthermore, $G/M$ is an elementary Abelian $r$-group for some prime $r$.
Let $R$ be an $\alpha$-invariant Sylow $r$-subgroup of $G$.
Consider the semi-direct product $H=G \langle \alpha \rangle= Q R \langle \alpha \rangle$.
$R\langle \alpha \rangle$ is a Frobenius group of order $p|R|$, and 
acts faithfully as a group of automorphisms of $Q$.
Now by Clifford's theorem,
$\alpha$ can't act without non-trivial fixed-points on $Q$, contrary to the assumption that it does.
\end{quote}
\section{Dickson's Theorem}
{\bf Theorem 2:} (i) $SL_2(q)$ has cyclic groups of order $q-1$ and $q+1$, (ii) $P \in S_2(SL_2(q)$ is
elementary abelian if $q$ is even and generalized quaternion if $q$ is odd.
\begin{quote}
\emph{Proof:}
$ \left(
\begin{array}{cc}
\lambda  &  0 \\
0 &  {\lambda}^{-1} \\
\end{array}
\right)$ shows (i). Identify the two dimensional space over $GF(q)$ with the additive group of $GF(q^2)$;
now look at the kernel of $det$ into the cyclic group of order $q^2-1$ in $GF_2(q)$.  The kernel has order $q+1$.
\end{quote}
{\bf Theorem:} Let $\lambda$ be a generator for $GF(p^r )$, $p \ne 2$ and put
$L= \langle
\left(
\begin{array}{cc}
1 &  0 \\
\lambda &  1 \\
\end{array}
\right),
\left(
\begin{array}{cc}
1 &  1\\
0 &  1\\
\end{array}
\right)
\rangle$. Then, either (i) $L=SL_2(p^r)$ or (ii) $p^r=9$, $L/Z(L) \cong A_5$, and
$L$ contains a subgroup isomorphic to $SL_2(3)$.
\begin{quote}
\emph{Proof:}
Gorenstein 2.8.4
\end{quote}
\section{Simplifying $CN$-theorem}
{\bf Theorem 1:} Suppose $G$ is a minimal simple $CN$group of odd order and $P, Q \in S_p(G)$ with
$P \cap Q \ne 1$  Then $P=Q$.
\begin{quote}
\end{quote}
{\bf Theorem 2:} Suppose $G$ is a minimal simple $CN$group of odd order and $P \in S_p(G)$ then
$P \subseteq N(P)'$.
\begin{quote}
\end{quote}
From Theorem 1, it is clear the minimal counterexample cant be a $3$-step group which gives 14.2.2 
in Gorenstein.  With these two theorems, we can use $N_G(P)$ in place of $N_G(Z(J(P)))$ in 14.2.3.
\section{Some Bender Results}
{\bf Theorem 3:} If $P \in p(G)$ and $N_G(P)$ is $p$-constrained, so is $C_G(P)$.
\begin{quote}
\emph{Proof:}
\end{quote}
{\bf Theorem 4:} If $A < G$ is and elementary abelian $p$-groups with $r(A) \geq 3$ and $P, Q $ are
$A$-invariant $p'$ subgroups of $G$ then $\exists a \in A^{\#}: C_P(a) \ne 1 \ne C_Q(a)$.
\begin{quote}
\emph{Proof:}
\end{quote}
{\bf Theorem 5:} Suppose $G$ is a group in which every normalizer of a $p$ subgroup is $p$-constrained.
If $p=2$ assume $cl(P) \leq 2$, $P \in S_p(G)$.  Let $E$ be an abelian $p$ groups with $r(E) \geq 3$ such that
$E$ contains every $p$-element of $C_G(E)$.  Then $O_{p'}(C_G(E))$ acts transitively on ${\cal N}_G^*(E,q)$,
$p \ne q$.
\begin{quote}
\emph{Proof:}
\end{quote}
{\bf Definition:} $S \in S_2(G), S' = 1$ is an $A^*$ group if there is a normal series 
$1 \subseteq N \subseteq M \subseteq G$ where $N$ and $G/M$ have odd order and $M/N$
is a direct product of $2$-group and $L_2(q)$ or $JR$.
\\
\\
{\bf Bender UniquenessTheorem:} Let $G$ be a minimal simple group of odd order and $U$ an elementary
abelian subgroup with $|U|=p^3$.   Then there is one and only one maximal subgroup $U<M<G$.
\begin{quote}
\emph{Proof:}
\end{quote}
{\bf Bender A:} Let $G$ be an group with abelian $S_2$ subgroups then $G$ is an $A^*$ group.
\begin{quote}
\emph{Proof:}
\end{quote}
{\bf Theorem 6:} If $G$ is $p$-solvable and $R \in p(G)$ is $p$-stsable if $p \geq 5$ or $SL_2(3)$ is not involved.
\begin{quote}
\emph{Proof:}
\end{quote}
{\bf Theorem 7:} If $T \in p(G)$ and $M \in p'(G)$ with $M \lhd G$.  Write ${\overline X} = XM/M$.
$C_{\overline G}({\overline T}) = {\overline {C_G(T)}}$ and
$N_{\overline G}({\overline T}) = {\overline {N_G(T)}}$.
\begin{quote}
\emph{Proof:}
\end{quote}
{\bf Theorem 8:} If $G$ is solvable and $R \in p(G)$ then $O_{p'}(C_G(R)) \subseteq O_{p'}(G)$.
\begin{quote}
\emph{Proof:}
\end{quote}
{\bf Theorem 9:} If $G$ is solvable of odd order and $O_{p}(G)=1$, then $O_{p', p}(G)$ contains every
normal abelian subgroup in $S in S_p(G)$.
\begin{quote}
\emph{Proof:}
We can assume $O_{p'}(G)=1$. $R= O_p(G)$, $V= R/\Phi(R)$, ${\overline G}= G/R$.
${\overline G}$ acts faithfully on $V$. $O_p({\overline G}) =1$.  Let $A$ be an abelian
normal subgroup of $P \in S_q(G)$. $[P, A] \subseteq A$ so $[P, A, A] = 1$ and $[P, A] =1$
so $[R,A,A]=1$ and ${\overline A}$ has minimal polynomial $(x-1)^2$.  Since $G$ is $p$-stable,
the minimal polynomial is $x-1$.
\end{quote}
{\bf Bender B:} If $A \ne B$ are two maximal subgroups of a simple group $G$ with
$F^*(A) \subseteq B$ and $F^*(B) \subseteq A$ then
$F^*(A)$ and $F^*(B)$ are $p$-groups.
\begin{quote}
\emph{Proof:}
\end{quote}
{\bf Theorem:} If $G$ has a fixed-point-free automorphism of prime order, $p$ then $G$ is nilpotent.
\begin{quote}
\emph{Proof:}
\end{quote}
