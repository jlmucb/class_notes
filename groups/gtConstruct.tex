\chapter{Constructions}
\section {Semidirect Product}
{\bf Definition 1:}
$G$ is an \emph{extension} of $K$ by $Q$ if $G \triangleright K$ and $G/K \cong Q$.  Equivalently,
there is a surjective homomorphism $\pi:G \rightarrow Q$ (i.e.: $1 \rightarrow K \rightarrow G \rightarrow H \rightarrow 1$).
\\
\\
{\bf Definition 2:}
Let $\pi: H \rightarrow Aut(K)$, $K \lhd G$, $H<G$ and $H \cap K = 1$ define the
\emph {semidirect} product of $K$ by $H$ (denoted $H \ltimes K$) as the group with elements
$(a, g) \in H \times K$ with the product $(a, g) \cdot (b, h)= (ab, \pi(b)(g)h)$ [or $(ab, (g)^{\pi(b)}h)$].
\emph{Example:} The dihedrael group, $D_n : |D_n|=2n$ is $H \ltimes K$, where $K= \langle a \rangle, |a|=n$ and
$H$ is a group of order 2, the non-identity element is a reflection.  $(a, g)^{-1} = (a^{-1}, (g^{-1})^{\pi(a^{-1})})$.
\\
\\
{\bf Definition 3:} An extension $G$ of $K$ by $H$ \emph {splits} if G is a semidirect product of $H$ and $K$.
\\
\\
{\bf Theorem 1:}
If $1 \rightarrow N \rightarrow_{i} G \rightarrow_{\varphi} Q \rightarrow 1$, the following
are equivalent
(1) $\exists H \subseteq G$ such that $\varphi: H \rightarrow Q$ is an isomorphism;
(2) $\exists s:Q \rightarrow G$ such that $\varphi \cdot s = id$;
(3) $G$ is a semidirect product of $N$ by $H$ written $Q \ltimes N$; in this
case, we say $G$ is a split extension of $N$ by $H$.
\begin{quote}
\emph{Proof:}
\\
\\
$1 \rightarrow 2$:
Put $s= \varphi_{|H}^{-1}$, then (2) holds.
\\
\\
$2 \rightarrow 3$:
Let $H= s(Q)$, $N= ker(\varphi)$.  Suppose $x \in G$ and $q= \varphi(x)$.  Set
$h= s(q)$.  $\varphi(x h^{-1})= \varphi(x) \varphi(h)^{-1} = q q^{-1}= 1$ so
$x h^{-1} \in N$ and $x= nh$.
\\
\\
$3 \rightarrow 1$:
If $G= N \ltimes H$, the map $\varphi_{|H}$ is an isomorphism onto $Q$ and $H$ satisfies the
conditions of (1).
\end{quote}
\section {Presentations}
{\bf Theorem 2:}  Let $G=
\langle x_1, \ldots, x_n| R_1(x_1, \ldots, x_n), R_2(x_1, \ldots, x_n), \ldots R_m(x_1, \ldots, x_n)
\rangle $. There is a one to one correspondance between homomorphisms
$\rho: G \rightarrow H$ and solutions to
$R_1(y_1, \ldots, y_n)=1, R_2(y_1, \ldots, y_n)=1, \ldots R_m(y_1, \ldots, y_n)=1, y_i \in H$.
\\
\\
{\bf Theorem 3:}  Two groups are isomorphic iff they admit the same presentation up to renaming.
\\
\\
{\bf Theorem 4:}  Let $G \cong F/R$, $F$, free.  Then for arbitrary abelian $A$, the transgression map
$Hom(R/[F,R],A) \rightarrow H^2(G,A)$ is associated with the natural exact sequence
$1 \rightarrow R/[F,R] \rightarrow F/[F, R] \rightarrow G \rightarrow 1$.
\\
\\
{\bf Projective representations and Schur:} Let $\rho: G \rightarrow GL_n({\mathbb C})/{\mathbb Z}$
so that $\rho(g)$ is a coset $\pmod {\mathbb Z}$.  Choose $A(g) \in \rho(g)$ then
$\rho(ab)= r_{a,b} \rho(a) \rho(b)$ and $r_{a,b} r_{ab,c}= r_{a, bc} r_{b,c}$ (cocycle identity).
Conversely, given an $r$ satisfying the cocycle identity, there is a projective representation
of degree $|G|=m$ giving rise to it.  If $B(g) \in \rho(g)$ is another, $A(g)= d_g B(g)$, and
if $B(ab)= s_{a,b}B(a)B(b)$, $r_{a,b}= {\frac {d_a d_b} {d_{ab}}} s_{a,b}$ (Relation 1).
If $r, s$ satisfy relation 1, they are called \emph{equivalent}.
\\
\\
{\bf Theorem 5:} Let $\rho^*: H \rightarrow GL_n({\mathbb C})$ be an ordinary irreducible
representation of $H$ with $A \le {\mathbb Z}(H)$ then $\rho^*(a)= j_a I_n$ and we can
define $\rho: H/A \rightarrow GL_n({\mathbb C})/{\mathbb Z}$ by
$\rho(hA)= \rho^*(h) {\mathbb Z}$.  Schur showed the converse:
$\exists H: A \le {\mathbb Z}(H)$ with $H/A \cong G$.
\begin{quote}
\emph{Proof:}
\end{quote}
Note that if $G= \langle F|R \rangle $ is of rank $k$,
$(F/[F,R])/(R/[F,R])= G$ and $R/[F,R]$ is central.
\section {Central Product}
{\bf Definition of Central Product:}
$G= \langle G_i \rangle , 1 \le i \le n$, $[G_i , G_j ]=1$ for $i \ne j$.  Equivalently,
$\rho: (x_1, x_2, \ldots , x_n) \mapsto x_1 x_2 \ldots x_n$ is a surjective homomorphism
from $D= (G_1 \times G_2 \times \ldots \times G_n)$ to $G$ with $\rho(D_i) = G_i$ with
$\pi_i (G_1, \ldots , G_n)= D_i$ and $ker(\rho) \cap D_i = 1$, $ker(\rho) \subseteq {\mathbb Z}(G)$.
\begin{quote}
\emph{Proof of equivalence:}
Let $\alpha_i: {\mathbb Z}(G_1) \rightarrow {\mathbb Z}(G_i)$.
$E= \langle z(\alpha_i(z^{-1})) \rangle $.  $E$ is a complement to ${\mathbb Z}(D_i )$ in
$Z= \langle {\mathbb Z}(D_i) \rangle $ so $D/E$ is a central product.  Suppose $G$ is a
central product of the $G_i$ with ${\mathbb Z}(G_i)={\mathbb Z}(G_1)$ and $\rho$ the surjective
homomorphism.  Let $\beta_i:{\mathbb Z}(D_1) \rightarrow {\mathbb Z}(G_i)$ be the composition of
$\rho_{|{\mathbb Z}(D_1)}$ and
$\rho_{|{\mathbb Z}(D_i)}^{-1}: {\mathbb Z}(G_1) \rightarrow {\mathbb Z}(D_i)$ then
$ker(\rho)= \langle x \beta_i(z^{-1}), z \in {\mathbb Z}(D_1) \rangle = A$ is a complement
to ${\mathbb Z}(D_i)$ in $Z$ and $G \cong D/A$.  Define $\gamma \in Aut(D)$ with $\gamma(D_1)=D_i$
and $\gamma(E)=A$.  This induces and isomorphism of $D/E$ with $D/A$.  Let
$\delta_i= \beta_i (\alpha_i)^{-1}$ then $\gamma_{i|{\mathbb Z}(D_i)}= \delta_i$ and
define $\gamma: D \rightarrow D$ by $(x_1, x_2 , \ldots , x_n) \mapsto (x_1, \gamma_1(x_2), \ldots,
\gamma_n(x_n))$.
\end{quote}
{\bf Example:}
Both $D_8$ and
$Q_8$ are central products of $Z_2$ by $Z_2 \times Z_2$.  Note that $D_8$ is also a direct product but $Q_8$
is not.
\\
\\
{\bf Theorem 6:}
Let $G_i, 1 \le i \le n$ be a family of groups with $Z(G_1)=Z(G_i)$ and
$Aut_{G_i}(Z(G_i))=Aut(Z(G_i))$.  The up to isomorphism there is a unique
central product with $Z(G_1)=Z(G_i)$.
\begin{quote}
\emph{Proof:}
By hypothesis, there are isomorphisms
$\alpha_i : {\mathbb Z}( D_1 ) \rightarrow {\mathbb Z}( D_i )$, $1 \le i \le n$.
Let $E$ be the subgroup of $D$ generates by $\{ \alpha_i (z^{-1})z, z \in {\mathbb Z}(D_1 )$.
$E$ is a complement to ${\mathbb Z}(D_i)$ in $Z= \langle {\mathbb Z}(D_i), 1 \le i \le n \rangle = {\mathbb Z}(D)$ so
$D/E$ is a central product of the $G_i$ with ${\mathbb Z}(G_1 ) = {\mathbb Z}(G_i )$ by
the foregoing equivalence.  Now suppose $G$ is a central product of the $G_i$ with
${\mathbb Z}(G_1 ) = {\mathbb Z}(G_i )$ and let $\pi: D \rightarrow G$ be the surjective
homomorphism in the equivalence proof.  Let $\beta_i : {\mathbb Z}(D_1 ) \rightarrow {\mathbb Z}(D_i )$ be the composition of
$\pi_{|{\mathbb Z}(D_1 )}$ and $\pi_{|{\mathbb Z}(D_i )}^{-1}$.
$ker(\pi ) = \langle z \beta_i(Z^{-1} ): z \in {\mathbb Z}(D_1 ), 1 \le i \le n \rangle
= A$ is a complement
to ${\mathbb Z}(D_i )$ in $Z$ for each $i$ and $G \cong D/A$.  Now let $\delta_i = \beta_i \alpha_i^{-1}$ and $\delta_i \in Aut({\mathbb Z}(D_i ))$.  By
hypothesis, $\exists \gamma_i \in Aut( D_i )$ with $\gamma_{i | {\mathbb Z}(D_i )}= \delta_i$.
Define $\gamma : D \rightarrow D$ by $(x_1 , x_2 , \ldots , x_n ) \mapsto
(x_1 , \gamma_2(x_2), \ldots , \gamma_n(x_n))$.  $\gamma \in Aut(D)$ and
$\alpha_i(\gamma (z^{-1})) z$ so $\gamma(E)=A$ and we're done.
\end{quote}
\section {Wreath Product}
{\bf Wreath Product:} $G^*= G^{X}$ - maps from $X$ to $G$.  $fg(x)= f(x) g(x)$.
Let $H$ act on $X$: $f^h(x)= f(xh^{-1})$.  Let $\phi$ be the natural action of
$H$ induced on $G^{|H|}$, then $G \wr H = H \rtimes_{\phi} G^*$. If $G_x =
\{f: f(y)= 1 \; if \; x \ne y \}$.  $G^* = \prod_X G_x$.
Put $g_x (y)= g (y)$ if $x=y$, 1 otherwise.  Note that
${g_x}^h= g_{xh}$.
\\
\\
{\bf Theorem 7:} If $D$ and $Q$ are groups with $Q$ finite then the regular wreath product
$D \wr_r Q$ contains an isomorphic copy of every extension of $D$ by $Q$.
\begin{quote}
\emph{Proof:}  If $G$ is an extension of $D$ by $Q$ then there is a surjective homomorphism
$G \rightarrow Q$ with kernel $D$ denoted by $a \mapsto {\overline a}$.
Choose a transversal $l: Q \rightarrow G$.  For $a \in G$, define $\sigma_a(x) = l(x)^{-1} a l({\overline {a^{-1}}} x)$.  If $a, b \in G$ then
$
\sigma_a(x) \sigma_b^{\overline a}(x) =
\sigma_a(x) \sigma_b({\overline {a^{-1}}} x) =
l(x)^{-1} a l({\overline {a^{-1}}} x)
l(x)^{-1} a l({\overline {a^{-1}}} x)
l({\overline {a^{-1}}} x)^{-1} b l({\overline { b^{-1} a^{-1}}} x) =
l(x)^{-1} a b l({\overline { b^{-1} a^{-1}}} x) = \sigma_{ab}(x)$.  Define $\varphi : G \rightarrow D \wr_r Q$ by $\varphi(a) = (\sigma_a, {\overline a}), \forall a \in G$.  A simple calculation shows
$\varphi$ is a injective homomorphism.
\end{quote}
{\bf Theorem 8 (Gaschutz):} Let $V$ be an abelian normal subgroup of a $p$-group, $G$, $P \in S_p(G)$.
$G$ splits over $V$ iff $P$ splits over $V$.
\begin{quote}
\emph{Proof:}
$V \leq P$.  $P = P \cap G = P \cap HV$ so $P = V (P \cap H)$ by Dedekind.
\end{quote}
