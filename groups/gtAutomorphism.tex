\chapter{Automorphisms}
\section {Cyclic Groups}
{\bf Theorem 1:}
If $G= \langle x \rangle $ and $|p^t|=n$.  For $p \ne 2$, $|Aut(G)|= (p-1)p^{t-1}$.  For $p=2$, $|Aut(G)|= 2^{t-1}$;
moreover, it is a direct product of a group generated by an  involution and a cyclic group of order $2^{t-2}$.
\begin{quote}
\emph{Proof:} Suppose $\sigma \in Aut(G)$.  $\sigma(x)= x^r$, so $(p,r)=1$ and the result follows.
If $p \ne 2$, $Aut(G)$ has a unique element of order $p-1$.
\end{quote}
{\bf Theorem 2:} $G= \langle x \rangle $ and $|p^t|=n$.  If $p \ne 2, t>1$, $Aut(G)$ has a unique subgroup of order
$p$ generated by by $x \mapsto x^{1+p^{t-1}}$.  If $p=2, t > 2$, there are three subgroups of order $2$ generated by
$x \mapsto x^{-1}$,
$x \mapsto x^{-1 + 2^{t-1}}$, and
$x \mapsto x^{1 + p^{t-1}}$.
\begin{quote}
\emph{Proof:}
\end{quote}
{\bf Theorem 3:} If $G= \langle x \rangle $ and $|p^t|=n$. Let 
$G_1 = \langle x^{p} \rangle$ and
$G_2 = \langle x^{p^{t-1}} \rangle$ so $|G_1| = p^{t-1}$ and $|G_2|=p$. Suppose $\sigma \in Aut(G)$
of order $p$.
If $sigma$ centralizes $G/G_1$ or $sigma$ centralizes $G/G_2$, $\sigma = 1$
\begin{quote}
\emph{Proof:}
\end{quote}
{\bf Theorem 4:} Suppose $G$ is a $p$-group of order $p^{n+1}$ with a cyclic subgroup of order $p$ then either
$G$ is cyclic or $G= \langle x, y \rangle$ with $|x|=p^n$ and one of the following holds for $y$:
(i) $|y| = p$ and $[x,y]=1$,
(ii) $p > 2$ or $p=2, n>2$ and $y^{-1}x y = x^{1+p^{n-1}}$;
(iii) $p=2$ and $y^{-1}x y = x^{-1}$ [dihedral],
(iv) $p=2$ and $y^{-1}x y = x^{-1 + 2^{n-1}}$ [semi-dihedral], or
(v) $p=2$ and $y^2= x^{2^{n-1}}$, $y^{-1} x y = x^{-1}$ [quaternion].
\begin{quote}
\emph{Proof:}
\end{quote}
{\bf Theorem 5:} Suppose $G$ is a $p$-group and $G$ is not cyclic, dihedral, semi-dihedral or quaternion.  Then
$G$ has a subgroup of type $(p,p)$.
\begin{quote}
\emph{Proof:}
\end{quote}
{\bf Theorem 6:} Suppose $G$ is a $p$-group and ${\mathbb Z}({\mathbb Z}(G))$ is cyclic then $G$ is cyclic,
dihedral, semi-dihedral or quaternion.
\begin{quote}
\emph{Proof:}
\end{quote}
{\bf Theorem 7:} Suppose $G$ is a $p$-group and every normal abelian subgroup is cyclic.  Then $G$ is cyclic,
dihedral, semi-dihedral or quaternion.
\begin{quote}
\emph{Proof:}
\end{quote}
{\bf Theorem 8:} Suppose $G$ is a $p$-group with exactly one subgroup of order $p$. Then $G$ is cyclic or quaternion.
\begin{quote}
\emph{Proof:}
\end{quote}
{\bf Theorem 9:} If $G$ is quaternion of order $8$ then (i) every subgroup is cyclic, (ii) $Aut(G) \cong S_4$,
(iii) an element of order $3$ transitively permutes the $3$ non-central classes of $G$.
\begin{quote}
\emph{Proof:}
\end{quote}
{\bf Theorem 10:} If $G$ is quaternion of order $>8$ then (i) every subgroup is cyclic or quaternion,
(ii) if $N$ is a non-ableian normal subgroup, $|G:N| = 1, 2$,
(iii) $G$ has a characteristic cyclic subgroup, $A$ of index $2$,
$Aut(G)$ is a $2$-group, and,
(iv) $G$ has three conjugacy classes of elements of order $4$, one consists of the two elements of $A$ or order $4$
and the other two consist of half the elements of $G \setminus A$.
\begin{quote}
\emph{Proof:}
\end{quote}
\section {Non-cyclic groups}
{\bf Theorem 1:}
If $G= E(p^n)$ then $Aut(G)= L_n(p)$.
\\
\\
{\bf Theorem 2:}
If $G= S_n$ then $Aut(G)= S_n$ if $n \ne 6$ and a covering group with factor $2$ of $S_6$ for
$S_6$ (an outer automorphism).
\\
\\
{\bf Theorem 3:}
If $G$ is simple, $1 \rightarrow G \rightarrow Aut(G)$.
\\
\\
{\bf Definition:} $Q_8 =
\langle
\left(\begin{array} {cc}
i & 0\\
0 & i\\
\end{array}\right),
\left(\begin{array} {cc}
0 & -1\\
1 & 0\\
\end{array}\right),
\left(\begin{array} {cc}
0 & 1\\
-1 & 0\\
\end{array}\right)
\rangle$. $Q_8 \in S_2(SL_2(5))$.
\\
\\
{\bf Theorem 4:}  The automorphism group of $Q_8$ is $S_4$.
\\
\\
{\bf Notation:} Suppose $G$ can be represented as a subgroup of $GL_p(V)$.
Let $\phi_{x}$ be the automorphism induced by $x$ on $V$ in such a representation.
\\
\\
{\bf Theorem 5:} If $H \lhd G \subseteq GL_p(V)$ then $C_V(H)$ is $G$-invariant.
\begin{quote}
\emph{Proof:}
Let $w \in W = C_V(G), h \in H, g \in G$ and let $\phi_{x}$ be as above.
$g^{-1} h g \in H$ so $\phi_{g^{-1} h g}(w) = 1$.
so $\phi_{g^{-1}} \phi_{h} \phi_{g} (w) = w$ and so $\phi_{h} \phi_{g} (w) = \phi_{g} (w)$.
Thus $W$ is $G$-invariant.
\end{quote}
{\bf Theorem 6:} Let $P \in p(GL_p(V))$.  $\exists v \in V: \phi_{x}(v)= v, \forall x \in P$.
\begin{quote}
\emph{Proof:}
By induction on $|P|$.  Let $M \lhd P$ be maximal so $|P:M|=p$ and put $W=C_V(M)$.
By induction, $W \ne 1$ and is $P$-invariant.  Choose $y \in P \setminus M$ then
$y^p \in M$ so the minimal polynomial for $y$ is $x^p-1 = (x-1)^p$ so there is a $w \in W: [w,y]=1$.
thus $[w, \langle y, M \rangle ] = 1$.
\end{quote}
{\bf Theorem 7:} Let $K \lhd P$ then $K \cap {\mathbb Z}(P) \ne 1$.
\begin{quote}
\emph{Proof:}
$H= \Omega_1({\mathbb Z}(K) \ne 1$.  $H \textnormal{ char } K \lhd P$ and $H$ is elementary
abelian. Let ${\overline P} = P C_H(P) / C_H(P)$.  Then $C_H({\overline P}) \ne 1$ by Theorem 7.
So, $1 \ne H \cap {\mathbb Z}(P) \subseteq {\mathbb Z}(P) \cap K$.
\end{quote}
{\bf Note:} Theorem 7 holds if $P$ is nilpotent.
\\
\\
{\bf Theorem 8:}  Let $H$ be an elementary abelian subgroup of $P$ and $x \in N_P(H)$.  Define
$H^{(0)} = H$ and $H^{(i+1)}= [H^{(i)}, x]$ then the minimal polynomial for $x$ is $(x-1)^r$
where $H^{(r)}=1$.
\begin{quote}
\emph{Proof:}
Put $\psi_x= \phi_x -1$. $\psi_x(h)$ represents $[h, x]$ and $\psi_x^n(h)$ represents $[h,x,\ldots,x]$.
Since $\phi_x^{p^n} = 1$, $(\phi_x(h) - 1)^r = 0$ for some $r \mid p^n$, wiht $r$ is the least such integer.
\end{quote}
{\bf Theorem 9:} If $G$ has a faithful, irreducible representation over $GF(p)$ then $G$
has no normal $p$ subgroup.
\begin{quote}
\emph{Proof:}
Suppose $P \lhd G$.  $W= C_V(\phi_{P}) \ne 1$.  But $\phi$ is irreducible so $W=V$.  $\phi_{P}$
acts trivially on $V$ so $P=1$ since $\phi$ faithful.
\end{quote}

