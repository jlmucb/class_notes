\chapter{Automorphisms}
\section {Specific Groups}
{\bf Theorem 1:}
If $G= \langle a \rangle $ and $|a|=n$ then $Aut(G)= \langle \rho \rangle $, $|\rho|= \phi(n)$.
$Aut(G) \cong {{\mathbb Z}_n}^*$.
\\
\\
{\bf Theorem 2:}
If $G= E(p^n)$ then $Aut(G)= L_n(p)$.
\\
\\
{\bf Theorem 3:}
If $G= S_n$ then $Aut(G)= S_n$ if $n \ne 6$ and a covering group with factor $2$ of $S_6$ for
$S_6$ (an outer automorphism).
\\
\\
{\bf Theorem 4:}
If $G$ is simple, $1 \rightarrow G \rightarrow Aut(G)$.
\\
\\
{\bf Definition:} $Q_8 = 
\langle 
\left(\begin{array} {cc}
i & 0\\
0 & i\\
\end{array}\right),
\left(\begin{array} {cc}
0 & -1\\
1 & 0\\
\end{array}\right),
\left(\begin{array} {cc}
0 & 1\\
-1 & 0\\
\end{array}\right)
\rangle$. $Q_8 \in S_2(SL_2(5))$.
\\
\\
{\bf Theorem 5:}  The automorphism group of $Q_8$ is $S_4$.
\\
\\
{\bf Notation:} Suppose $G$ can be represented as a subgroup of $GL_p(V)$.
Let $\phi_{x}$ be the automorphism induced by $x$ on $V$ in such a representation.
\\
\\
{\bf Theorem 6:} If $H \lhd G \subseteq GL_p(V)$ the $C_V(H)$ is $G$-invariant.
\begin{quote}
\emph{Proof:}
Let $w \in W = C_V(G), h \in H, g \in G$ and let $\phi_{x}$ be as above.
$g^{-1} h g \in H$ so $\phi_{g^{-1} h g}(w) = 1$.
so $\phi_{g^{-1}} \phi_{h} \phi_{g} (w) = w$ and so $\phi_{h} \phi_{g} (w) = \phi_{g} (w)$.
Thus $W$ is $G$-invariant.
\end{quote}
{\bf Theorem 7:} Let $P \in p(GL_p(V))$.  $\exists v \in V: \phi_{x}(v)= v, \forall x \in P$.
\begin{quote}
\emph{Proof:}
By induction on $|P|$.  Let $M \lhd P$ be maximal so $|P:M|=p$ and put $W=C_V(M)$.
By induction, $W \ne 1$ and is $P$-invariant.  Choose $y \in P \setminus M$ then
$y^p \in M$ so the minimal polynomial for $y$ is $x^p-1 = (x-1)^p$ so there is a $w \in W: [w,y]=1$.
thus $[w, \langle y, M \rangle ] = 1$.
\end{quote}
{\bf Theorem 8:} Let $K \lhd P$ then $K {\mathbb Z}(P) \ne 1$.
\begin{quote}
\emph{Proof:}
$H= \Omega_1({\mathbb Z}(K) \ne 1$.  $H \textnormal{char} K \lhd P$ and $H$ is elementary
abelian. Let ${\overline P} = P C_H(P) / C_H(P)$.  Then $C_H({\overline P}) \ne 1$ by Theorem 7.
So, $1 \ne H \cap {\mathbb Z}(P) \subseteq {\mathbb Z}(P) \cap K$.
\end{quote}
{\bf Note:} Theorem 8 holds if $P$ is nilpotent.
\\
\\
{\bf Theorem 9:}  Let $H$ be an elementary abelian subgroup of $P$ and $x \in N_P(H)$.  Define
$H^{(0)} = H$ and $H^{(i+1)}= [H^{(i)}, x]$ then the minimal polynomial for $x$ is $(x-1)^r$
where $H^{(r)}=1$.
\begin{quote}
\emph{Proof:}
\end{quote}
{\bf Theorem 10:} If $G$ has a faithful, irreducible representation over $GF(p)$ then $G$
has no normal $p$ subgroup.
\begin{quote}
\emph{Proof:}
Suppose $P \lhd G$.  $W= C_V(\phi_{P}) \ne 1$.  But $\phi$ is irreducible so $W=V$.  $\phi_{P}$
acts trivially on $V$ so $P=1$ since $\phi$ faithful.
\end{quote}
{\bf Theorem 11:}
\begin{quote}
\emph{Proof:}
\end{quote}

