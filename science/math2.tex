\section{Algebra}
\subsection{General Algebra}
{\bf Solving low degree univariate polynomials:}
For \emph{quadratic},  $x^2+px+q=0$, $(x_1-x_2)= {\sqrt D}, D= p^2-4q, x_1+x_2=p$.
For \emph{cubic}, $az^3 + bz^2 + cz + d= f(z)$, substitute $z= x - {\frac {b} {3a}}$ and
divide by $a$ to get $x^3+ px+ q= 0$.  Put $x= (u+v)$, get $p=3uv$,
$q= u^3 + v^3$.  Following Galois, note $S_3 \supseteq A_3 \supseteq 1$,
$(x_1 -x_2)(x_1-x_3)(x_2-x_3)= {\sqrt D}, D= -4p^3-27q^2$.  After adjoining
${\sqrt D}$, we are left with an irreducible cubic.  Put $\rho=
-{\frac 1 2} + {\frac 1 2} {\sqrt {-3}}$ and 
$(1,x_1 )= x_1 +x_2 + x_3$, $(\rho, x_1 )= x_1 + \rho x_2 + \rho^2 x_3$,
$(\rho^2, x_1 )= x_1 + \rho^2 x_2 + \rho x_3$, then
$(\rho, x_1 )^3= \sum x_i^3  - {\frac 3 2} \sum x_i^2 x_j + 6 x_1 x_2 x_3$.
$(\rho, x_1 )^3= - {\frac {27} {2}}q + {\frac 3 2} {\sqrt {-3}} {\sqrt {D}}$ and
$(\rho^2, x_1 )^3= - {\frac {27} {2}}q - {\frac 3 2} {\sqrt {-3}} {\sqrt {D}}$.
Solution is:
$x= ({-{\frac {p} {2}} +- {\sqrt {{\frac{q^2} {4}} + {\frac {p^3} {27}}}}})^{\frac {1} {3}}$.  
For \emph{quartic}, $az^4 + bz^3 + cz^2 + dz+e= f(z)$, 
substitute $x= x - {\frac {b} {4a}}$ and
divide by $a$ to get $x^4+ px^2 + qx + r$.  Again following Galois,
$S_4 \supseteq A_4 \supseteq C_4 \supseteq Z_2 \supseteq 1$.  
$\Theta_1 = (x_1+x_2)(x_3+x_4)$ is fixed by $C_4$ but not $A_4$.
The $\Theta_i$ are solutions of $\Theta^3 -b_1 \Theta^2 + b_2 \Theta_3 -b_3$ with
$b_1= 2p, b_2= p^2-4r, b_3= -q^2$ and
$D=16 p^4r-4p^3q^2-128p^2r^2+144 p q^2r -27 q^4 +256 r^3$.
Look at
$(y^2 + p)^2 = py^2 -qy -r$ and pick $z$ to make right hand side
$(y^2 + p + z)^2 = (p+2z)y^2 -qy + (p^2 -r +2pz + z^2 )$ a perfect square.
\\
\\
{\bf Fundamental Theorem of Algebra:}  If $f(z) \in {\mathbb C}[x]$ then $f(z)=0$
has a solution (\emph{root}) in ${\mathbb C}$.
\begin{quote}
\emph{Proof:}
Let $f(z)= z^n+ a_{n-1} x^{n-1} + ... + a_0$ and 
$\mu= \textnormal{inf}(|f(z)|)$.   If 
$\mu = 0$, we're done since the minimum must occur in bounded ball.  So assume
$\mu \ne 0$.  Let the minimum occur at $z_{0}$ and put
$f(z_{0})= w_{0}$,
$w=f(z_{0}+ \zeta)$.  ${\frac {w} {w_{0}}}= 1+q \zeta^{\nu}(1+\zeta \xi)=
1-h \rho^{\nu}(1+ \zeta \xi)$ where $\zeta= \rho (cos(\theta) + i sin(\theta))$ and
$q= h(cos(\lambda) + i sin(\lambda))$.  So we can find a point with smaller modulus than
$w_0$.  This contradicts the assumed minimality at $z_0$.
\end{quote}
{\bf Facts about roots of unity:}  
Consider $f(x)= x^h -1$ over $F$ where 
$(char(F),h)=1$ or $char(F)=0$.
The roots of $f$ form an abelian group, $G$.  $x \in G \rightarrow |x| \mid |G|$.  Since
$(f, f')=1$ there are $h$ distinct roots, set $h= \prod_{i=1}^m q_i^{v_i}$.  
$\{ x: x^{h/q_i} = 1 \}$ is a group of order
$h/q_i$ so $\forall i, \exists x_i \in G: 
x^{h/q_i} \ne 1$.  Setting $b_i= x_i^{h/{q_i}^{v_i}}$, then $\chi = \prod b_i$ has order
exactly $h$ and is a primitive $h$th root of unity.  
Let the number of such roots be $\varphi(h)$; if $(r,s)=1, \varphi(rs)= \varphi(r) \varphi(s)$
so $\varphi(\prod_i {q_i}^{v_i})= 
\prod_i \varphi({q_i}^{v_i})= 
\prod_i ({q_i}^{v_i}- {q_i}^{v_i-1})= h \prod_i (1- {\frac 1 {q_i}})$.  Set $n=\varphi(h)$ and
$\Phi_n(x) = \prod_i (x- \psi_i)$ where $\psi_i$ are the primitive roots.
$x^h-1 = \prod_{d \mid h} \Phi_d(x)$ and by Moebius inversion,
$\Phi_h(x)= \prod_{d \mid h} (x^d -1)^{\mu({\frac h d})}$.
\\
\\
{\bf Theorem:} $\Phi_h(x)$ is irreducible of
degree $\varphi(h)$.
\begin{quote}
\emph{Proof:} Let $\zeta \in {\mathbb C}$ be a primitive root of
$\Phi_h(x)$ with minimal polynomial $f(x)$ and $(p,h)=1$.  Let $g(x)$ be the minimal
polynomial for $\zeta^p$ so $g(\zeta^p)=0$.  $x^h -1 = f(x) g(x) h(x)$ and
$g(x^p)=f(x)k(x)$.  $g(x^p)=g(x)^p \jmod{p}$.  If $\phi(x) \mid f(x)$ then
$\phi(x) \mid g(x)^p \jmod{p}$.  So $\phi(x)^2 \mid x^h-1$ but this contradicts the
fact that $x^h -1$ does not have roots of multiplicity 2.  It follows that if $(p_i , h)=1$,
$\zeta^{p_1 p_2 \ldots p_k}$ is a primitive root and the degree of $f(x)$ is $\varphi(h)$.
\end{quote}
Note this shows that $Aut({\mathbb Q}[\zeta]) \cong {\mathbb Z}_h^*$.
So if $h=q=p^n$ ,
the Galois group is cyclic and the subfields correspond to the
cyclic subgroups of ${\mathbb Z}_q^*$.  The $q$th roots of 1 are expressible as radicals
if $char(F)=0$ or $char(F)>q$.
If $N_p (d)=$ number of irreducible monic polynomials of degree $d$ in
$GF(p)[x]$ then $p^n = \sum_{d \mid n} d N_p (d)$ and $N_p (d)= {\frac 1 n}
\sum_{d \mid n} \mu ( {\frac {n} {d} }) p^d$.
$x^{p^n} - x = \prod_{f,\small{irred, monic}, deg(f) \mid n} f$.
\\
\\
{\bf Eisenstein's Criteria:} If $f(x)= \sum_{i=0}^n a_n x^n$, $a_n \ne 0 \jmod{p}$,
$a_i = 0 \jmod{p}, i<n$ and $a_0 \ne 0 \jmod{p^2}$ then $f$ is irreducible.
\\
\\
{\bf Factoring in finite number of steps:} Let $g(x) \in {\mathbb Z}[x]$ if
$f(x) \mid g(x)$ then $f(n) \mid g(n)$ for
all $n$. $deg(f) = s \le \lfloor {\frac {deg(g)} 2} \rfloor$.  Pick $s$ integers $i_j$ and
use the integer factors of $g(i_j)$ to get possible $f(i_j)$; there are a finite
number of ways to pick the factors.  For each possibility, we can solve for the $s$ 
coefficients of $f$.
\subsection{Free groups, rings and modules}
{\bf Theorem:}
Every group is the homomorphic image of a free group.
\begin{quote}
\emph{Proof:}
Let $F$ be the elements of $G$ and $R$ be the relations $abc^{-1}=1$.
$G$ is the free group on the symbols $F$ with relations $R$.
\end{quote}
{\bf Theorem:}
If $x_1, x_2, \ldots , x_n$ is a basis for a free abelian group for a free abelian
group and $y_i = \sum_{j=1}^n a_{ij} x_j$ with $a_{ij} \in {\mathbb Z}$ then
$\langle y_i \rangle$ is a basis iff $det(a_{ij}) = \pm 1$.
\begin{quote}
\emph{Proof:}  
Since $\langle y_i \rangle$ is a basis, $x_i= \sum_j b_{ij} y_j$ for some
$b_{ij} \in {\mathbb Z}$.  Let $A= (a_{ij})$ and $B= (b_{ij})$ then
$BA=I$ and $det(B)= det(A) \in {\mathbb Z}$, so $det(A)= \pm 1$.
$\langle y_i \rangle$ is a basis iff $det(a_{ij}) = \pm 1$.
\end{quote}
{\bf Theorem:}
Every subgroup $H \le G$ of a free abelian group $G$ of rank $n$ is free abelian group
of rank $s \le n$.  Moreover, $\exists  u_1, \ldots, u_s \in G$ and 
$\alpha_1, \ldots, \alpha_s \in {\mathbb Z}$ such that $\alpha_1 u_1 , \ldots, \alpha_s u_s$
is a basis of $H$.
\begin{quote}
\emph{Proof:}  
By induction on $n$.  True for $n=1$.  Pick a basis $w_1, \ldots, w_n$ of $G$ and
for $h \in H: h= h_1 w_1 + \ldots + h_n w_n$.  Let $\alpha_1 \ne 0$ be the
smallest (in absolute value) coefficient in any such sum and assume WLOG it occurs as the coefficient of $w_1$ for
some $h$.
Let $v_1= \alpha_1 w_1 + \beta_2 w_2 + \ldots + \beta_n w_n \in H$. $\exists q_i, r_i:
\beta_i = \alpha_1 q_i + r_i$.  Put $u_1 = w_1 + q_2 w_2 + \ldots + q_n w_n$ then
$u_1, w_2 , \ldots , w_n$ is another basis for $G$ and $v_1= \alpha_1 u_1 \in H$.
Now put $H'= \{ h= m_2 w_2 + \ldots + m_n w_n: h \in H \}$.  $H' \cap (v_1) = 0$
and $H= H' \oplus (v_1)$.
\end{quote}
{\bf Theorem:}
If $G$ is a free abelian group of rank $r$ and $H \le G$ then $G/H$ is finite iff
$H$ has rank $r$.  If $G$ has basis $x_1, \ldots , x_s$ and $H$ has basis
$y_1, \ldots, y_s$ with $y_i= \sum_{j=1}^s a_{ij} x_j$ then $|G/H|= det(a_{ij})$.
\begin{quote}
\emph{Proof:}  
By the structure theorem, viewing $G$ as a ${\mathbb Z}$-module,
$G \approx {\mathbb Z}^s/H$.  In the diagonal form, the order of the direct sums appears
on the diagonal and the product is $det(a_{ij})$.
\end{quote}
{\bf Theorem:}
Every finitely generated abelian group with $n$ generators is the direct product of a
free abelian group and a finite abelian group.
\begin{quote}
\emph{Proof:}  
Let $G= \langle x_i , \ldots ,x_n \rangle$ and $H= \{ x \in G: x^n=1, n \in {\mathbb Z} \}$.
$G/H$ is free and finitely generated.  Apply the theorem (just before the structure theorem)
to $g \rightarrow G/H$ to conclude the proof.
\end{quote}
{\bf Calculating with free groups:}
Let $F_m$ be a free abelian group generated by $a_1 , a_2 , \ldots , a_m$ and
define $E_i= r_{i1} a_1 + r_{i2} a_2 + \ldots + r_{im} a_m$ where
$r_{ij} \in {\mathbb Z}$ and $1 \le i \le n$; further, put
$b_i= E_i$ and let $K= \langle b_i \rangle$.  
Suppose $G$ is the free abelian group generated
by $a_i$ subject to $E_i = 0$.  Then $G \cong F_m / K$.  Let $R$ represent the matrix
$(r_{ij})$ then 
(1) if the matrix $S=(s_{ij})$ is obtained from $R$ by elementary row operations then
$c_i= s_{i1} a_1 + \ldots + s_{im} a_m \in K$; and,
(2) if the matrix $S=(s_{ij})$ is obtained from $R$ by elementary column operations then
$\exists a_i' \in F_m: b_i= s_{i1} a_1' + \ldots + s_{im} a_m'$ (so the $a_i'$ generate
$K$).  By applying elementary row and column operations, we can transform $R$ into
the diagonal matrix $D= diag(d_1 , d_2 , \ldots , d_r , 0, \ldots , 0)$ where
$d_i \mid d_{i+1}$
and $G \cong {\mathbb Z}/(d_1) \times {\mathbb Z}/(d_2) \times \ldots 
\times {\mathbb Z}/(d_r) \times {\mathbb Z} \times \ldots \times {\mathbb Z}$ where there
are $m-r$ copies of ${\mathbb Z}$ in the product.
\\
\\
{\bf Definition:} If $f(x)= \sum_{i=0}^n a_i x^i$, $a_i \in R$, a UFD, the \emph{content} of
$f$ is $cont(f)= gcd(a_0, a_1, \ldots, a_n)$.  
\\
\\
{\bf Gauss' Lemma}: 
If $D$ is a UFD and $f, g \in D[x]$ then
$cont(f(x)g(x))= cont(f(x)) cont(g(x))$.  If $f(x) \in R[x], deg(f)>0$ and $f(x)$
is irreducible in $R[x]$ then it is irreducible in $K[x]$.
\begin{quote}
\emph{Proof:}
Suffices to show that if $cont(f)=cont(g)=1$ then $cont(fg)=1$.  Let
$f(x)= a_0 + a_1 x + \ldots +a_n x^n$ and
$g(x)= b_0 + b_1 x + \ldots +b_m x^m$ with $a_i, b_i \in R$.  Suppose, by way of
contradiction that $p \mid cont(fg)$.  Let 
$i$ be maximal subject to $p \nmid a_i$ and
$j$ be maximal subject to $p \nmid b_j$.  The coefficient of $x^{i+j}$ in $fg$ is
$c_{i+j}= \sum_{k=0}^{i+j} a_k b_{i+j-k}$.  
$p \mid a_k, k<i$ and
$p \mid b_{i+j-k}, k>i$ but $p \nmid a_i b_j$ so $p \nmid c_{i+j}$.
\end{quote}
{\bf Theorem:} If $R$ is a UFD then
$R[x]$ is a UFD. 
\begin{quote}
\emph{Proof:}
Let the field of quotients of $R$ be $K$.  If $f(x) \in R[x]$, $f(x)= cont(f) f'(x)$
with $cont(f')=1$.  By Gauss' lemma, it suffices to prove the result
when $f$ is primitive.
$f(x)= f_1(x) \cdot f_2(x) \ldots \cdot f_n(x)$ over $K$ with each $f_i$ irreducible;
further, this factorization is unique up to units in $K$.  So $\exists c_i, d_i \in R$ 
such that 
$d_i f_i(x)= c_i p_i(x)$ where $p_i \in R[x]$ satisfies $cont(p_i)=1$.
Put $c= \prod_{i=1}^n c_i$ and
$d= \prod_{i=1}^n d_i$  then $d f(x)=
c p_1(x) \cdot p_2(x) \ldots \cdot p_n(x)$ so $c$ and $d$ are units of $R$ by Gauss.
This shows a factorization exists.
If $f(x)= d q_1(x) \cdot q_2(x) \ldots \cdot q_m(x)$
is another such factorization then $m=n$  
and $p_i= u_i q_j$, $u_i \in K$ by the factorization result in $K[x]$.  Thus
$t_i p_i= r_i q_j$, $r_i, t_i \in R$ with $(t_i, r_i)= 1$.  By Gauss, $t_i= u_i r_i$
where $u_i$ is a unit in $R$.
Finally, 
$d$ is a unit in $R$ again by Gauss's lemma and the proof is complete.
\end{quote}
{\bf Ring theoretic Chinese Remainder Theorem:} 
If $I_j, j= 1,2, \ldots, n$ are ideals of $R$ and
$I_j + I_k = R$ for $j \ne k$, then $\forall x_1 , x_2, \ldots , x_n \in R, \exists
x \in R$ such that $x= x_j \jmod{I_j}$.
\\
\\
{\bf Corollaries:} Under the same assumptions as the theorem,
$\psi: R \rightarrow R/I_1 \times R/I_2 \times \ldots \times R/I_n$ given by
$x \mapsto x \jmod{I_1} \times \ldots \times x \jmod{I_n}$ is surjective and
$R/(\bigcap_{j=1}^n I_j) \cong R/I_1 \times R/I_2 \times \ldots \times R/I_n$.
If $m= \prod_{i=1}^n p_i^{r_i}$,
${\mathbb Z}/(m{\mathbb Z}) \cong \prod_i {\mathbb Z}/({p_i}^{r_i}{\mathbb Z})$ and
$\psi(m)= \prod_i \psi({p_i}^{r_i})$.  If $R$ is cyclic of order $n$ then
$End(R) \cong {\mathbb Z}/(n{\mathbb Z})$ and
$({\mathbb Z}/(n{\mathbb Z}))^* \cong Aut(R)$.
\\
\\
{\bf Hilbert Basis Theorem: }
If $R$ is a ring with identity such that every ideal is finitely 
generated then $R[x]$ has the same property.
\begin{quote}
\emph{Proof:}
Let $I$ be an ideal of $R[x]$ and $I_j$ be the set of coefficients
of the $x^j$ terms in $I$.  $I_j$ is an ideal of $R$ and $I_j \subseteq I_{j+1}$.
Since $R$ is finitely generated, $\exists m: I_j \subseteq I_m, \forall i$.
Let the generators of $I_j$ be 
$\langle b^{(1)}_j, b^{(2)}_j,  \ldots , b^{(i_j)}_j, \rangle$.  So there are
polynomials $f_j^{(i)}(x) \in I_j$:
$f_j^{(i)}(x)= b_j^{i} x^j + g_j^{(i)}(x), deg (g_j^{(i)}(x)< j)$.
The polynomials
$$\langle
f_0^{(1)}(x),
f_0^{(2)}(x), \ldots,
f_0^{(i_0)}(x), 
f_1^{(1)}(x),
f_1^{(2)}(x), \ldots,
f_1^{(i_1)}(x), 
\ldots,
f_m^{(1)}(x),
f_m^{(2)}(x), \ldots,
f_m^{(i_m)}(x)
\rangle$$
generate $I$.
\end{quote}
{\bf Groups with operators} ($M$) and invariant subgroups:  Projection commutes with
all inner automorphisms; such an endomorphism is called normal.  An $M-$group
$G$ is decomposable 
iff there are projections.  Any $M-$group satisfying DCC is a direct
product of a finite number of indecomposable $M-$groups.  If $\eta \in  End(G)$
then ${\sqrt \eta}= \{z \in G: z \eta^s=1 \}$.
\\
\\
{\bf Fitting Lemma:}  Let $G$ be an $M-$group that satisfies ACC and DCC and $\eta$ is a
normal endomorphism of $G$ then $G= {\sqrt \eta} \times H$ and $H \eta = H$.
If $G$ is an indecomposable $M-$group satisfying ACC and DCC then any normal
$M-$endomorphism of $G$ is either {\bf nilpotent} or an automorphism.  Suppose
$\eta_1, \eta_2$ are normal nilpotent $M-$endomorphisms, if $\eta_1 + \eta_2$
is an endomorphism it is nilpotent.  {\bf Krull-Schmidt} follows
from this.  
\\
\\
{\bf Definition:} $M$ is \emph{unitary} if $RM=M$.  
\\
\\
{\bf Primary decomposition:}
If $A,B$ are ideals, we say $A \mid B$ if $B \subseteq A$. $Q$ is \emph{primary} iff
$ab= 0 \jmod{I} \rightarrow a = 0 \jmod{Q}$ or $b \in {\sqrt I}$.  If
$Q$ is primary then ${\sqrt Q}$ is prime.  
Every irreducible ideal in a Noetherian ring is primary.
Every ideal in a Noetherian ring is the finite intersection of
primary ideals.  If $Q_1, Q_2$ are primary and
${\sqrt {Q_1}}= {\sqrt {Q_2}}$ then $Q_1 \cap Q_2$ is primary.
If $Q_1 \cap Q_2 \cap \ldots \cap Q_r = Q_1' \cap Q_2' \cap \ldots \cap Q_s'$
are two irredundant representations into primary ideals whose associated primes
are distinct, then $r=s$ and the set of associated primes is identical.
If $R^2=R$ is a commutative ring then every maximal ideal is prime.  Let
$P$ be a prime ideal of $R$ ($1 \in R$) then (1)
There is a 1-1 correspondence between the set of prime ideals of $R$ contained
in $P$ and (2) the prime ideals of $R_P$ given by 
${\mathbb Q} \mapsto {\mathbb Q}_P$.  A local ring is a commutative ring
with identity containing a unique maximal ideal.  If $R$ is a commutative ring
with identity, the following are equivalent: (1) $R$ is a local ring; (2)
all non-units of $R$ are contained in an ideal 
$M \ne R$; (3)  the non-units form an ideal.
Substitution from a the polynomial ring to the ring of coefficients is a homomorphism.
\\
\\
{\bf Theorem:} 
If $R$ is Noetherian and $a \in M$ is $R-$\emph{integral} iff $\exists$ a finitely
generated submodule of $M$ that contains all powers of $a$.  The totality
$G$ of elements of $M$ that are $R-$integral is a subring of $M$ containing
$R$.  The ring $G$ or $R-$integral elements is integrally closed in $R$.
\begin{quote}
\emph{Proof:} $\rightarrow$.  Suppose $a^n + r_{n-1} a^{n-1} + \ldots + r_0 = 0$.
Put $M= \langle a^{n-1}, \ldots, 1 \rangle$.  $a^k \in M, \forall k$.
$\leftarrow$. Consider 
$\langle 1 \rangle \subseteq \langle 1, a \rangle \subseteq
\langle 1, a, a^2 \rangle \subseteq \ldots$.  Since $R$ is noetherian, this series
is finite.  Say $M = \langle 1, a, a^2, \ldots, a^m \rangle$ is the terminal module
in the series.  $a^k \in M, \forall k$.
\end{quote}
{\bf Projective and injective modules:}
If $A, B, C, A', B', C'$ are modules 
over a ring $R$ with identity and we have the diagrams
$0 \rightarrow A
{\buildrel\rm f\over \rightarrow} \; B {\buildrel\rm g\over \rightarrow} \; 
C \rightarrow 0$
and
$0 \rightarrow A'
{\buildrel\rm f'\over \rightarrow} \; B' {\buildrel\rm g'\over \rightarrow} \; 
C' \rightarrow 0$ with 
$A
{\buildrel\rm \alpha\over \rightarrow} \; 
A'$,
$B
{\buildrel\rm \beta\over \rightarrow} \; 
B'$, and
$C
{\buildrel\rm \gamma\over \rightarrow} \; 
C'$, then (1) $\beta$ is a monomorphism if $\alpha$ and $\beta$ are and (2)
$\beta$ is a epimorphism if $\alpha$ and $\beta$ are.
$P$ is projective if given $A, B, g, f$ and morphism diagrams:
$A {\buildrel\rm g\over \rightarrow} \; B \rightarrow 0$ and
$P {\buildrel\rm f\over \rightarrow} \; B$, $\exists h,
P {\buildrel\rm h\over \rightarrow} \; B$ which makes the diagram commute.
$J$ is injective if given $A, B, g, f$ and morphism diagrams:
$A {\buildrel\rm g\over \rightarrow} \; B \rightarrow 0$ and
$A {\buildrel\rm f\over \rightarrow} \; J$, $\exists h,
B {\buildrel\rm h\over \rightarrow} \; J$ which makes the diagram commute.
Every free module $F$ over $R$ with identity is projective.  
If $R$ is a ring
with identity, TFAE: (1) $P$ is projective, (2) every short exact sequence
$0 \rightarrow A
{\buildrel\rm f\over \rightarrow} \; B {\buildrel\rm g\over \rightarrow} \; 
P \rightarrow 0$ splits so $B= P \oplus A$ and (3) $\exists F$, free such that
$F=K \oplus P$.
If $R$ is a ring
with identity, TFAE: (1) $J$ is injective, (2) every short exact sequence
$0 \rightarrow A
{\buildrel\rm f\over \rightarrow} \; B {\buildrel\rm g\over \rightarrow} \; 
C \rightarrow 0$ splits so $B= J \oplus C$ and (3) $J$ is a direct summand.
$0 \rightarrow A
{\buildrel\rm \psi\over \rightarrow} \; B {\buildrel\rm \phi\over \rightarrow} \; 
C$ is exact if:
$0 \rightarrow Hom(D, A)
{\buildrel\rm \psi\over \rightarrow} \; Hom(D, B) {\buildrel\rm \phi\over \rightarrow} \; 
Hom(D, C)$ is.
$A
{\buildrel\rm \theta\over \rightarrow} \; B {\buildrel\rm \zeta\over \rightarrow} \; 
C \rightarrow 0$ 
is exact if:
$0 \rightarrow Hom(A, D)
{\buildrel\rm \psi\over \rightarrow} \; Hom(B, D) {\buildrel\rm \phi\over \rightarrow} \; 
Hom(C, D)$ is.  The full short exact sequence is split exact iff the corresponding
dual (Hom) sequence is.
\\
\\
{\bf Integral closure of  UFD's:} 
If $A$ us a unique factorization domain, $A$ is integrally closed.  
The integral closure in a number field
$K$ is called the ring of algebraic integers.  Algebraic integers form a free
${\mathbb Z}$-module of rank $[K:{\mathbb Q}]$.
\\
\\
{\bf Results on trace and norm:}
Let $[E:F]=n$ and $[F(x):F]=d$ and $x_1 , x_2 , \ldots , x_d$ be the
roots of $min_F(x)$ then 
$N_{E/F}(x)= (\prod_{i=1}^d x_i)^{\frac n d}$ and
$Tr_{E/F}(x)= {\frac n d} (\sum_{i=1}^d x_i)$.  
If $E/F$ is separable then
$N_{E/F}(x)= \prod_{i=1}^n \sigma_i(x)$ and
$Tr_{E/F}(x)= \sum{i=1}^n \sigma_i(x)$.  If $F \subseteq E \subseteq K$ then
$N_{E/F}(N_{K/E}(x))= N_{K/F}(x)$ and $Tr_{E/F}(Tr_{K/E}(x))= Tr_{K/F}(x)$.  If
$E/F$ is a finite separable extension, $\exists x \in E: Tr_{E/F}(x)= 0$ and
$(x,y) \rightarrow Tr_{E/F}(xy)$ is bilinear.
For this paragraph, $L$ be a separable extension of $K$, 
$A \subseteq K$ be a ring of integers and
$B \subseteq L$ be a ring of algebraic integers.
${\vec x}$ is a basis for $L/K$ iff $\Delta({\vec x}) \ne 0$.  If
$L=K(x)$ and $f$ is a minimal polynomial of $x$ over $K$ then
$\Delta(1,x,x^2, \ldots, x^{n-1})= disc(f)= \prod_{i<j} (x_i-x_j)=
(-1)^{n \choose 2} N_{L/K}(f'(x))$.  There is a basis for $L/K$ consisting
of elements of $B$.  If $A$ is a PID then $B$ is a free $A$-module of rank $[L:K]$.
If $a_i \in A$ then $(x_1-a_1 , x_2 - a_2 , \ldots , x_n-a_n)$ is a maximal ideal.
\\
\\
{\bf Theorems on chain conditions:}
Let $M$ be an $R$ module.  
The following are equivalent
(1) $M$ satisfies \emph{ACC (Noetherian)}, 
(2) Any non empty collection of submodules of $M$
has a maximal element.
The following are equivalent
(1) $M$ satisfies \emph{DCC (Artinian)}, (2) Any non empty collection of submodules of $M$
has a minimal element.  
$M$ is Noetherian iff every submodule is finitely generated.
$M$ is Artinian iff every submodule is finitely co-generated.  
\\
\\
{\bf Example of Noetherian ring:}
PIDs, $F[x]$.  $F[x_1 ,x_2, \ldots ]$ is \emph{neither} Noetherian nor Artinian.
If $N \subseteq M$ then $M$ is Noetherian iff $N$ and $M/N$ are.  $M$ has a
composition series iff $M$ is Noetherian and Artinian.
$L$ be a separable extension of $K$, 
$A \subseteq K$ be a ring of integers
$B \subseteq L$ be a ring of algebraic integers, if $A$ is integrally closed
in $K$ and $A$ is Noetherian, so is $B$.
Let $P$ be a prime ideal of $R$ and 
$P \supseteq I_1 I_2 \ldots I_n$ then
$\exists k: P \supseteq I_k$.
Let $I$ be a non-zero ideal of a noetherian integral domain $R$ then
$I \supseteq P_1 P_2 \ldots P_n$ for $P_i$ prime.
Let $R$ be a non-zero ideal of a noetherian integral domain and $K$ its
field of quotients, $I$ is a fractional ideal if $I$ is an $R$-module and
$\exists r \in R: rI \subseteq R$.  If $I$ is a finitely generated $R$
submodule of $K$ then $I$ is a fractional ideal.  If $R$ is Noetherian and $I$
is a fractional ideal of $R$ then $I$ is a finitely generated $R$ submodule of $K$.
\\
\\
{\bf Definition:}
A \emph{Dedekind Domain (``DD'')} is an integral domain, $R$, such that 
(1) $R$ is Noetherian,
(2) $R$ is integrally closed, and, (3) Every non-zero prime ideal of $R$ is maximal.
PIDs are DDs.  Algebraic integers in a number field is a DD.  If $P$ is a non-zero
prime ideal in a DD, $R$ and $J=\{x \in K: xI \subseteq R \}$ then
(1) $R \subseteq J$ and (2) $J$ is a fractional ideal and $PJ=R$.  If $I$
is a fractional ideal in a DD, $R$ then $I= \prod_{i=1}^N {P_i}^{n_i}$ 
($n_i \in {\mathbb Z}$ not just ${\mathbb Z}^{\ge 0}$), $n_P(I)=n_i$.  The
fractional ideals form a group.  A non-zero fractional ideal is integral iff all
$n_i$ in the forgoing representation are $\ge 0$.  $I_1 \supset I_2$ iff
$\forall P, n_P(I_1) \le n_P(I_2)$.  If $I_1 , I_2$ are integral ideals then
$I_1 \mid I_2$ if $I_2= J I_1$.  $I_1 \mid I_2$ iff $I_1 \supseteq I_2$.
$L$ be a separable extension of $K$, 
$A \subseteq K$ be a ring of integers, if $A$ is a DD, $B$ is a DD.
An abelian group with a basis of $n$ elements is a \emph{free abelian
group} of rank $n$.
\\
\\
{\bf Kummer's idea:}  For a field, $K$ let
${\mathfrak D}_K$ denote the ring of integers in $K$.
Start with $K$ and extend it to $L$ such that
${\mathfrak D}_K \subseteq {\mathfrak D}_L$.  For example, 
$K={\mathbb Q}(\sqrt {15}) \subseteq {\mathbb Q}({\sqrt {3}}, {\sqrt {5}})= L$.
$10 = {\sqrt{5}} {\sqrt{5}} ({\sqrt{5}} + {\sqrt{3}}) ({\sqrt{5}} - {\sqrt{3}})$.
Let $I= ({\sqrt{5}} + {\sqrt{3}}) \cap {\mathfrak D}_K$.  
$({\sqrt{15}} + 3) \in I$,
$({\sqrt{15}} + 5) \in I$ so $2 \in I$ and $I$ is not principal.
\\
\\
{\bf Definition:} A ${\mathfrak D}$-submodule ${\mathfrak a}$ of $K$ is a 
\emph{fractional} ideal of ${\mathfrak D}$ if $\exists c \in {\mathfrak D}:
c{\mathfrak a} \subseteq {\mathfrak D}$ and $c {\mathfrak a} = {\mathfrak b}$ is an
ideal of ${\mathfrak D}$.  An ideal ${\mathfrak p}$ is prime if
${\mathfrak p} \mid {\mathfrak a} {\mathfrak b}$ implies
${\mathfrak p} \mid {\mathfrak a}$ or ${\mathfrak p} \mid {\mathfrak b}$.
${\mathfrak a}^{-1}= \{ x \in K: x {\mathfrak a} \subseteq {\mathfrak D} \}$.
\\
\\
{\bf Theorem:}  If a domain is \emph{noetherian} the elements factor into irreducibles.
If every irreducible in a domain, $D$, is prime then $D$ is a UFD.
\begin{quote}
\emph{Proof:}  
\\
\emph{Claim 1:}
Let ${\mathfrak a} \ne 0$, $\exists
{\mathfrak p}_1 , 
{\mathfrak p}_2 , \ldots ,
{\mathfrak p}_n $ 
such that
$ {\mathfrak p}_1 \cdot {\mathfrak p}_2 \cdot \ldots \cdot{\mathfrak p}_n  \subseteq {\mathfrak a}$.
\\
\emph{Claim 2:}
${\mathfrak a}^{-1}$, exists.
\\
\emph{Claim 3:}
If ${\mathfrak a}$ is a proper ideal then
${\mathfrak a} \nsupseteq {\mathfrak D}$.
\\
\emph{Claim 4:}
If 
${\mathfrak a} S \subseteq {\mathfrak a}$ for
$S \subseteq K$ then
$S \subseteq {\mathfrak D}$.
\\
\emph{Claim 5:}
If ${\mathfrak p}$ is a maximal ideal then
${\mathfrak a} {\mathfrak a}^{-1} = {\mathfrak D}$.
\\
\emph{Claim 6:}
If
${\mathfrak a}$ is a fractional ideal, it has an inverse and
${\mathfrak a} {\mathfrak a}^{-1} = {\mathfrak D}$.
\\
\emph{Claim 7:}
Every non-zero ideal ${\mathfrak a}$ is a product of prime ideals.
\\
\emph{Claim 8:}
Prime factorization is unique.
\end{quote}
{\bf Theorem:}
The non-zero fractional ideals of ${\mathfrak D}$ form a group and the identity is
${\mathfrak D}$.  Every non-zero ideal of ${\mathfrak D}$ can be written as a product of prime
ideals uniquely up to the order of factors.
\begin{quote}
\emph{Proof:}  
Todo.
\end{quote}
{\bf Definition:}
$N({\mathfrak a})= {\mathfrak D}/ {\mathfrak a}$.
\\
\\
{\bf Theorem:}  Let $G$ be an additive subgroup of
${\mathfrak D}$, a ring of algebraic integers, of rank $n$ equal to the degree of an
algebraic number field $K$ with ${\mathbb Z}$-basis $\{ \alpha_1 , \ldots , \alpha_n \}$
then $|{\mathfrak D}/G|^2 \mid \Delta(\alpha_1, \ldots , \alpha_n)$.
\begin{quote}
\emph{Proof:}  
Todo.
\end{quote}
{\bf Chain conditions and exact sequences:}
If $0 \rightarrow A
{\buildrel\rm f\over \rightarrow} \; B {\buildrel\rm g\over \rightarrow} \; 
P \rightarrow 0$, $B$ satisfies ACC (resp DCC) iff $A$ and $C$ do.
$A$ satisfies ACC on submodules iff each submodule is finitely generated (same for
rings).  Jordan-Holder for modules (composition series have unique refinements).  $A$
has a composition series iff $A$ satisfies ACC and DCC.  If $D$ is a division ring
then $Mat_{n \times n}(D)$ is both Noetherian and Artinian.  An ideal $P (\ne R)$
in a commutative ring $R$ is prime iff $R-P$ is a multiplicative set.  If
$S$ is multiplicative and $S \cap I \ne \emptyset$, $\exists P$, prime that is maximal
with respect to the disjoint property.  $rad(I)= \{ r \in R: r^n \in I \}$.
\\
\\
{\bf Theorem:}
Every transcendental extension has a \emph{transcendence basis.}
If $\langle x_1, x_2, \ldots , x_n \rangle$
spans $E$ algebraically and $S \subseteq E$ is algebraically independent then
$|S| \le n$. (Use Steinmetz replacement.) 
\\
\\
{\bf Noetherian Normalization Lemma:}
Let $R$ be an integral domain which is a finitely generated extension of $K$ and
suppose $r$ is the transcendence degree over $K$ of the quotient field of $R$, then
$\exists t_1, \ldots, t_r$ algebraically independent elements such that
$R$ is integral over $K[t_1, \ldots, t_r]$.
\\
\\
{\bf Localization:} Let $S$ be a multiplicative subset of $R$ and $h: a \mapsto a/1$ 
be the natural map.  If $J$ is an ideal in $S^{-1}R$ then
$S^{-1}J = I$ is an ideal of $R$ and
$I \subseteq h^{-1}(S^{-1}(I)$ with equality if $I \cap S = \emptyset$.
If $I$ is a prime ideal of $R$ and $I \cap S= \emptyset$ then $S^{-1}R$ is a
prime ideal of $S^{-1}R$.  If $P$ is a prime ideal of $R$ and $S=R-P$ is a
multiplicative set, denote $S^{-1}R$ as $R_P$.  $R_P$ has a unique maximal ideal
consisting of non-units of $R_P$.  ${\sqrt I}= P_1 \cap P_2 \cap \ldots \cap P_k$ for
some prime ideals $P_i$.
\subsection{Polynomials}
{\bf Basic Symmetric polynomials:} $\sigma_1= \sum x_i$, $\sigma_2 = \sum x_i x_j$,
etc.  
Every symmetric function $f(x_1, \ldots , x_n)= (z-x_1) \ldots (z-x_n)$
can be written as a polynomial
with coefficients in the basic symmetric polynomials.  
\begin{quote}
\emph{Proof 1:}
Let $ a {x_1}^{a_1} {x_2}^{a_2} \ldots {x_n}^{a_n}$
be the leading coefficient of a symmetric form in lexicographic order,
subtracting
$a {\sigma_1}^{a_1 - a_2} {\sigma_2}^{a_2 - a_3} \ldots {\sigma_n}^{a_n}$
leaves a symmetric form with leading coefficient smaller in lexicographic
order.
\\
\emph{Proof 2:}  By induction on the weight.  True for 1.  If 
$f(x_1 , \ldots, x_n)$ is symmetric, so is
${\frac {f(x_1 , \ldots, x_{n-1},0)} z}$. 
So ${\frac {f(x_1 , \ldots, x_{n-1},0)} z}= 
\phi((\sigma_1)_0, \ldots , (\sigma_{n-1})_0)$
Set  
$f_1(x_1 , \ldots, x_n)=
f(x_1 , \ldots, x_n)-
\phi((\sigma_1)_0, \ldots , (\sigma_{n-1})_0)$.
$f_1(x_1 , \ldots, x_{n-1},0)=0$ so $x_n$ and hence $\sigma_n$ divides
$f_1$ thus $f_1 = \sigma_n g$ and $g$ is writable as a polynomial in the basic
symmetric functions by induction so
$f(x_1 , \ldots, x_n)= \sigma_n \psi(\sigma_1, \ldots , \sigma_n) 
+ \phi (\sigma_1 , \ldots , \sigma_{n-1})$.  Further, the representation is essentially
unique which you can show by proving 
$\phi(y_1 , \ldots , y_n) \ne 0 \rightarrow \phi(\sigma_1 , \ldots , \sigma_n) \ne 0 $
(Prove).
\end{quote}
{\bf Theorem:} Let $\sigma_1, \ldots , \sigma_n$ be the symmetric functions on $n$ variables.
$\varphi(\sigma_1 , \ldots , \sigma_n )=0 $ iff
$\varphi(a1 , \ldots , x_n )=0 $.
\begin{quote}
\emph{Proof:} 
Proof by induction on $n$.  Trivial for $n=1$.  
Let $\phi(y_1, \ldots , y_n)=
\phi_k(y_1, \ldots , y_{n-1}) y^m + \ldots + \phi_0(y_1 , \ldots , y_{n-1})$ be a
counterexample of minimum degree in $y_n$.
Then $\phi(\sigma_1, \ldots , \sigma_n)=
\phi_k(\sigma_1, \ldots , \sigma_{n-1}) \sigma^m + \ldots + \phi_0(\sigma_1 , \ldots , \sigma_{n-1})= 0$.
Put $x_n=0$.  Then  
$\phi_0(\sigma_1 , \ldots , \sigma_{n-1})= 0$ but
$\phi_0(y_1 , \ldots , y_{n-1}) \neq 0$ which contradicts the induction hypothesis.
\end{quote}
{\bf Resultant:}
If $f_{v}(x)= v_{n}x^{n}+ \ldots + v_{0}$ and
$g_{w}(x)= w_{m}x^{m}+ \ldots + w_{0}$,
$\exists \phi_{v,w}(x), \psi_{v,w} (x):$
$\phi_{v,w}(x) f_{v}(x) + \psi_{v,w}(x) g_{u}(x) = R(v,w)
= v_{m}^n w_{n}^m \prod_{i<j} (t_{i}-u_{j})$, where $t_i , u_j$ are roots of
$f, g$ respectively.  Resultant is 0 iff equations have common solution.  
Consider the equations written in matrix notation:
$$
\left(
\begin{array}{c}
x^{m-1}f_{v} (x) \\
x^{m-2}f_{v} (x) \\
\ldots \\
f_{v} (x) \\
x^{n-1}g_{w} (x) \\
x^{n-2}g_{w} (x) \\
\ldots \\
g_{w} (x)
\end{array}
\right) =
\left(
\begin{array}{cccccccc}
v_n & v_{n-1} & \ldots & v_0 & 0 & 0 & \ldots & 0 \\
0 & v_n & v_{n-1} & \ldots & v_0 & 0 & \ldots & 0 \\
\ldots & \ldots & \ldots & \ldots & \ldots & \ldots & \ldots & \ldots\\
0 & 0 & \ldots & 0 & v_n & v_{n-1} & \ldots & v_0 \\
w_m & w_{m-1} & \ldots & w_0 & 0 & 0 & \ldots & 0 \\
0 & w_m & w_{m-1} & \ldots & w_0 & 0 & \ldots & 0 \\
\ldots & \ldots & \ldots & \ldots & \ldots & \ldots & \ldots & \ldots\\
0 & 0 & \ldots & 0 & w_m & w_{m-1} & \ldots & w_0 \\
\end{array}
\right)
\left(
\begin{array}{c}
x^{n+m-1}\\
x^{n+m-2}\\
\ldots \\
\ldots \\
\ldots \\
\ldots \\
x \\
1
\end{array}
\right)
$$
\begin{quote}
\emph{Proof:} Let the column vectors be $C_{m+n-1}$ \ldots $C_{0}$.
$C= (x^{m-1}f_v(x), \ldots , g_w (x))^T$.
$C= C_{m+n-1} \cdot x_{m+n-1} + \ldots + 1 \cdot C_{0}$.
Now solve for 1.  $1= {\frac {det(C_{m+n-1}, \ldots C_{1}, C)}
{det(C_{m+n-1}, \ldots C_{1}, C_{0})}}$.
Get $\phi_{v,w}(x)f_{v}(x) +
\psi_{v,w}(x)g_{w}(x)= R(v,w)$.
\end{quote}
{\bf Theorem}:
Let $f_{1},\ldots, f_{s}$ be polynomials of one variable with indeterminate
coefficients. $\exists d_{1} , d_{2}, \ldots, d_{h}$ of integral polynomials
in the coefficients of $f_i$ such that if the coefficients are assigned values 
(\emph{``specialized''})
from $k$, $d_{i}=0$ iff either
the $f_{i}=0$ have a common solution or the leading coefficients vanish.
\begin{quote}
\emph{Proof:} Set $f_{u} = u_{1} f_{1} + \ldots + u_{s} f_{s}$,
$f_{v} = v_{1} f_{1} + \ldots + v_{s} f_{s}$. $(f_{u}, f_{v})=1$ iff
$(f_{1}, f_{2} \ldots, f_{s}) = 1$.  $R(f_{u},f_{v}) = 0$ iff $f_u$ and
$f_v$ have a non-trivial common factor.  But $R(f_u, f_v)$ is
a polynomial in $u_{i}, v_{j}$ with coefficients which are integral in the
coefficients of $f_{i}$.  Arrange these in the order of powers of $u_i v_j$.
These are the $d_i$.
The proof also shows that 
$d_{i}= 0 \jmod{(f_{1}, f_{2}, \ldots f_{r})}$ and
$(d_{1}, d_{2} \ldots d_{l})= 0 \jmod{(f_{1}, f_{2}, \ldots f_{r})}$.
\end{quote}
{\bf Theorem}: If
$f_{1},\ldots, f_{r} \in F[x_1, \ldots, x_n]$ has no common zeros, 
$\exists A_{1},...A_{r}$ such that
$\sum_{i} A_{i}f_{i}=1$.
\begin{quote}
\emph{Proof:} By the induction on number of variables.  True for $n=1$
by usual theory of polynomials over fields.
Assume it's true for $n-1$.  Let ${\overline f}_i(x)= f_i(x, x_2, \ldots , x_n) =
\sum_{j-0}^{n_i} g_{ij}(x_2, \ldots, x_n) x^j$.  
The ${\overline f}_i$ have no common solution or the $f_i$ would; thus
by the previous result, regarding the coefficients of $x^j$ as indeterminants,
$\exists d_{lk}$ which are not simultaneously $0$ [or again, the $f_i$ would have a common
solution], such that 
$\sum_{lk} B_{lk} d_{lk} =1$.  After substitution,
$\sum_{ij} C_{ij} g_{ij} =1$.  Further, $g_{ij}= \sum_j A_j f_j$, again by
the previous result.  After substituting again, we get
$\sum_j D_j f_j(x, x_2, \ldots, x_n)$ which is what we want.
\end{quote}
{\bf Nullstellensatz}:  
If $f(x_1, \ldots, x_n) \in F$ vanishes at all the common zeros of
$f_1(x_1, \ldots, x_n), \ldots , f_r(x_1, \ldots, x_n)$
in every extension of $F$, then
$f^k(x_1, \ldots, x_n) \in (f_1(x_1, \ldots, x_n), \ldots,
f_r(x_1, \ldots, x_n))$ for some $k$.
\begin{quote}  
To prove this, look at
$f_{1},\ldots, f_{r}, 1-zf$,
put $z= {\frac {1} {f}}$ and clear denominators.
Note that if $h_1 , \ldots, h_m$ are zero
for all common zeros of the 
$f_i$, $(h_1, \ldots , h_m)^{\rho}= 0 (f_1 , f_2 , \ldots , f_r)$.
\\
\\
Note that an algebraic condition for solvability is not always
possible:  Consider
$a_1 x_1 +a_2 x_2 +a_3=0$,
$b_1 x_1 +b_2 x_2 +b_3=0$; they have a solution in general if
$a_1b_2-b_1a_2 \ne 0$ and the $d_i$ (the resultant system)
would have to vanish for indeterminant $a, b$ and the
equation would always have a solution but it doesn't.  However, this does work
for homogeneous equations (forms).
\end{quote}
General idea of \emph{elimination} for forms relies on three lemmas:
\\
\\
\emph{Lemma 1:} We can assume $x_1$ appears with non-zero
constant coefficient.
\begin{quote}
\emph{Proof:} if not, substitute $x_1= u_1 x_1'$, 
$x_2= x_2'+u_2x_1'$, ..., $x_n= x_n'+u_nx_1'$. 
\end{quote}
\emph{Lemma 2:}  If ${\cal F}$ has a non-trivial common solution, the $d_i$ do too.
\begin{quote}
\emph{Proof:} If the
coefficients do not vanish, the $d_i$ give rise to a solution $(\xi_2, \ldots, \xi_n)$
in $(x_2, \ldots, x_n)$ which can be extended to $x_1$; if not, the $d_i$ vanish identically
and have a solution, say $(1,1,\ldots,1)$ and the $f_i$ have a solution $(1,0,\ldots,0)$
with the coefficients of the $x_1$ terms $0$.  
\end{quote}
\emph{Lemma 3:} The system ${\cal F}$ has a
resultant system of integral polynomials $b_j$ in the coefficients of the $f_i$ such that
for a specialization of the coefficients of the $f_i$, ${\cal F}$ has a 
non-trivial common solution iff the $b_j=0$; further, the $b_j$ are homogeneous in the
coefficients of the forms.
\\
\\
{\bf Elimination procedure:}  Successively eliminate $x_1 , x_2, \ldots, x_n$.  At each step,
the $d_i$ obtained by eliminating previous $x_i$ are forms, 
we can continue the elimination
procedure until only $x_n$ remains and the resultant system becomes:
${x_n}^{s_1} b_1,
{x_n}^{s_2} b_2, \ldots,
{x_n}^{s_k} b_k$ and by the above
${x_n}^{s_j} b_j= 0 \jmod{(f_1, \ldots, f_n)}$.
If elimination results in a non-zero constant, there is no
common solution and we get $1= 0 \jmod{(f_1, f_2, \ldots, f_r)}$.
\\
\\
Observe that not all solutions can be obtained by specialization.  Consider
$f_1= x_1^2 + x_1 x_2$, $f_2= x_1 x_2 + x_2^2 + x_1 +x_2$, $(x_1+x_2)$ is a common factor
so the resultant vanishes.  $\xi_1= -\xi_2$ is a solution; however, if $\xi_2= -1,
\xi_1=0$ is also solution which does not fit the specialization solution.
\\
\\
For the next few paragraphs, the system $\cal F$ consists of
$r$ forms, $f_1, ..., f_r$ in $n$ variables
with indeterminant coefficients.
The indeterminants in $f_1$ are $a_1, \ldots , a_{\omega}$,
the indeterminants in $f_2$ are $b_1, \ldots , b_{\omega}$
and the indeterminants in $f_r$ are $e_1, \ldots , e_{\omega}$.  
When $r=n$ the resultant system is generated by a single polynomial, $R$, called the
resultant.
\\
\\
Let ${\cal F}$ be a system of forms as above with
$deg(f_i)=l_1$ and $l_1=\alpha, l_2= \beta, \ldots, l_r= \epsilon$.  
By the above, $\exists T \in {\mathbb Z}[a_1, \ldots , e_{\omega}]$ such
that ${x_i}^{\tau} T = 0 \jmod{(f_1, \ldots, f_n)}$. $T$ is called an inertial form.
Set $f_1= f_1^* + a_{\omega}{x_n}^{\alpha}$,
$f_2= f_2^* + b_{\omega}{x_n}^{\beta}$, ... ,
$f_n= f_n^* + e_{\omega}{x_n}^{\epsilon}$, substituting
$a_{\omega}= -{\frac {f_1^*} {x_n^{\alpha}}}$, ...,
$e_{\omega}= -{\frac {f_r^*} {x_n^{\alpha}}}$, we get
$T(a_1 , \ldots, -{\frac {f_1^*} {x_n^{\alpha}}}, \ldots, 
-{\frac {f_r^*} {x_n^{\alpha}}}) =0$
(Condition ``A'')
and this actually holds for all $i$ if it holds for any $x_i$.  Conversely, if
Condition ``A'' is satisfied, ${x_n}^{\tau} T=0 \jmod{(f_1, \ldots, f_r)}$.  
\begin{quote}
\emph{Proof:}
We can use Condition A to rearrange $T$ in powers of
$a_{\omega}+{\frac {f^*_1} {x_n^{\alpha}}}, 
\ldots, e_{\omega}+{\frac {f^*_r} {x_n^{\epsilon}}}$ and the term independent of the
powers vanishes so 
$T= 0
\jmod {(a_{\omega}+{\frac {f^*_1} {x_n^{\alpha}}},
\ldots, e_{\omega}+{\frac {f^*_r} {x_n^{\epsilon}}})}$; multiplying through by the largest
power of $x_n$ in the denominators, we get 
$x_n^{\tau}T= 0 \jmod {(f_1, f_2 , \ldots , f_r)}$.
The inertial forms form and ideal ${\cal I}$ which is prime and a basis for ${\cal I}$
thus forms a resultant system.
\end{quote}
{\bf Theorem}:  If the number
of forms, $f_i$, is less than the number of variables, $n$, then there is no inertial form
distinct from $0$; if $r=n$, there is no inertial form independent of $e_{\omega}$ and
distinct from $0$.  
\begin{quote}
The proof uses the following 
\emph{Lemma:}  When a sequence of polynomials
$f_1, \ldots, f_s$ in indeterminants $a_1 , a_2 , \ldots, a_p, x_1, x_2, \ldots, x_q$
is algebraically dependent in $k[a_1, \ldots, a_p]$, this dependence is valid for
every specialization $a_p=\alpha$.  
\\
\\
\emph{Proof of Lemma:} Since
$F(a_1, \ldots, a_p, f_1, \ldots , f_s)=0$ and
$F(a_1, \ldots, a_p, z_1, \ldots , z_s) \ne 0$, $F(a,z)$ is not divisible by 
$(a_p-\alpha)$ or we could reduce the relations. So
$F(a_1, \ldots, a_{p-1}, \alpha, f_1, \ldots , f_s) \ne 0$.
\\
\\
\emph{Proof of theorem:}
If $r<n$, by Condition ``A'',
$-{\frac {f_1^*} {x_n^{\alpha}}}, \ldots,
-{\frac {f_r^*} {x_n^{\alpha}}}$ would be algebraically dependent relative
to $k[a_1, \ldots, a_{\omega -1}, e_1, \ldots, e_{\omega -1}]$ and this continues to be
true if $x_n = 1$.  If $r=n$ and the hypothesis is false, 
$-{\frac {f_1^*} {x_n^{\alpha}}}, \ldots,
-{\frac {f_{n-1}^*} {x_n^{\delta}}}$ would be algebraically dependent relative and
we can set $x_n=1$.  In both cases, the lemma applies and we can specialize over any of the
indeterminants without losing dependency.  Choose a specialization so
$f_1, \ldots, f_s^{\delta} \rightarrow x_1^{\alpha}, \ldots, x_s^{\delta}$.  This is a
contradiction since these terms are algebraically independent.
\end{quote}
{\bf Theorem}:  If $r=n$, there is a non-vanishing
inertial form $D_e$, homogeneous in the indeterminants and of degree
$L_n= l_1 l_2 \ldots l_{n-1}$ in the $e_j$.  
\begin{quote}
\emph{Proof:}  Put 
$l= 1+ \sum_i^n (l_i -1)$ and consider, ${\cal P}$, the monomials of degree $l$ in the $x_i$.
${\cal P}$ is a disjoint union of the following sets: 
monomials of degree $l$ containing ${x_1}^{l_1}$,
monomials of degree $l$ containing ${x_2}^{l_2}$ but not ${x_1}^{l_1}$, \ldots,
monomials of degree $l$ containing ${x_n}^{l_n}$ but not 
${x_1}^{l_1}$,
${x_2}^{l_2}$, etc.  Suppose $H^{(m)}_{l-l_1}$ are the complementary monomials of
the elements of the disjoint sets, i.e. - 
${x_1}^{l_1}H^{(m)}_{l-l_1}$ are in the disjoint sets.  $H^{(m)}_{l-l_n}$ has
$l_1 l_2 \ldots l_{n-1}$ power products ($x_1^k, 0 \le k < l_1$, etc).  Now form
$H^{(m)}_{l-l_i} f_i$.  Since there are as many of these as power products, the matrix is
square.  Denote its determinant as $D_e$ which has the value $1$ under the specialization
$f_i= {x_i}^{l_i}$.  Multiplying the equations $H^{(j)}_{l-l_i} f_i = \sum a_{mk} H^{(k)}_l$
by the subdeterminants of a column of $D_e$ and adding, the left hand side becomes
linear in the $f_i$ and the right hand side, $D_e H^{(k)}_l$.  Letting
$H^{(k)}_l= x_i^l$, we get $D_rx_i^l = 0 \jmod{(f_1, f_2, \ldots, f_r)}$ and $D_e$
is homogeneous in each form, $f_i$ and has degree $L_n$ in the coefficients of $f_n$.
\end{quote}
Now, let $f_1, f_2, \ldots, f_n$ be forms in $x_1, x_2, \ldots, x_n$ with indeterminate
coefficients and ${\cal I}$ the ideal generated by the inertial forms.  
{\bf Theorem:} If $R$ is a polynomial of minimal degree in $e_{\omega}$, every element of 
${\cal I}$ is divisible by $R$.  $R$ is the resultant.  
\begin{quote}
\emph{Proof:} Arrange $R$ in powers of
$e_{\omega}, R= S {e_{\omega}}^{\lambda} + \ldots $.  If $T$ is in ${\cal I}$, we can
get a polynomial, $T'=S^jT-QR$ of lower degree which is also in ${\cal I}$ but then $T'=0$ and
$R \mid T$.  Note if $R$ vanishes for a specialization, every element of ${\cal I}$
does also and the $f_i$ have a common 0; 
conversely, if the $f_i$ have a common zero, since
${x_i}^{\tau}R = A_1 f_1 + \ldots + A_n f_n$, substitution makes the right side of
the equation $0$ but at least one $x_i \ne 0$ so $R=0$.  We have:
{\bf Theorem}:
$R(gh,f_2, \ldots, f_n)= R(g,f_2, \ldots, f_n) R(h,f_2, \ldots, f_n)$, $R$ is
homogeneous of degree $L_1$ in the coefficients of $F_1$,
homogeneous of degree $L_2$ in the coefficients of $F_2$, ..., 
and homogeneous of degree $L_n$ in the coefficients of $F_n$,
$R= (D_a, D_b, \ldots, D_e)$ is a principal ideal
and the resultant contains a principal term
${a_1}^{L_1} \ldots {e_{\omega}}^{L_n}$.
\end{quote}
{\bf Bezout's theorem:}  If $n-1$ homogeneous equations have a finite number of solutions
then sum of the multiplicities equals 
the product of the degrees of the equations.
\begin{quote}
\emph{Proof of Bezout}:  Suppose the system ${\cal F}, r=n$ has a finite 
number of non-trivial
solutions $(\xi_1^{(\alpha)}, \ldots, \xi_n^{(\alpha)})$, $\alpha= 1,2, \ldots, q$.
Add the form $l= u_1 x_1 + \ldots + u_n x_n$ and form the resultant system $b_1(u),
b_2(u), \ldots, b_t(u)$.  The resultant system has a solution iff
$l_{\alpha}= u_1 {\xi_1}^{(\alpha)} + \ldots + u_n {\xi_n}^{(\alpha)} =0$.  By
the Nullstellensatz: 
$(b_i(u))^{\tau_i} = 0 \jmod{ (\prod_{\alpha} l_{\alpha})}$ and
$(\prod_{\alpha} l_{\alpha})^{\tau}= 0 \jmod{D(u)}$ where
$D(u) = ( b_1(u), \ldots, b_t(u))$.  So $D(u)= \prod_{\alpha} {l_{\alpha}}^{\rho_{\alpha}}$
(the $\rho_{\alpha}$'s are the multiplicities).
If we consider $n-1$ forms $f_i$ and add the form $l=u_1 x_1 + \ldots + u_n x_n$, we get
Bezout's theorem.
\end{quote}
{\bf Berlekamp polynomial factorization over $F_q$:} $f(x)$ square free so
$f(x)= p_1(x) \ldots p_r(x)$. 
Note that $v(x)^p=v(x)$ and $(v(x)-0)(v(x)-1) \ldots (v(x)-(p-1)) \jmod{f(x)}$
and $\exists s_i \in F_q: (f(x), v(x)-s_i)= p_i(x)$.
Compute $x^{iq} \jmod{f(x)}= \sum q_{ij} x^j$.
Find null space of $Q-I$ with basis $v_1 (x) , \ldots , v_r (x)$.
Compute $(f(x), v_k (x) - \alpha ), \alpha \in F_q$.
$f_{n}(x) = {\frac {(x^{n} - 1)} {\prod_{d \mid n, d<n} f_{d}(x)}}$.
For \emph{distinct factors}, note that if $q(x)$ is irreducible of degree $d$ then
$q(x) \mid x^{p^d} -x$ but $q(x) \mid x^{p^c} -x, c<d$.  To use this, rull out square free
again, set $w(x)=x$, $d=0$ and repeatedly check $g_d(x)= w(x)-x, v(x)$ and replace
$d$ by $d+1$,
$w(x)$ by $w(x)^p$, and $v(x)$ by ${\frac {v(x)} {g_d(x)}}$.
\\
\\
{\bf Submodules of finitely generated free modules over a PID:}
Let $D$ be a PID and $D^{(n)}$ be a free module of rank $n$ over $D$.  Then any
submodule, $K$ of $D^{(n)}$ is free with base $m \leq n$ elements.
\begin{quote}
\emph{Proof:}  By induction.  For $n=1$, $K$ is isomorphic to a principal ideal.
Suppose the result is true for $n-1$.  Let
$x_1, \ldots, x_n$ be a free basis for $M$.  
If $K$ is contained in any module generated by all
but one of the $x_i$, were done by induction.  Let $\pi_i$ be the projection on the
$i$-th basis element.  
$im((\pi_i)_{|K})$ is a principal ideal generated by $p_i$.  WLOG let $p=p_1$ generate the
maximal ideal among these, so $p | p_i, \forall i$. 
$\exists k \in K: k= p x_1 + p d_2 x_2 + \ldots + p d_n x_n$.
Put $y= x_1 + d_2 x_2 + \ldots + d_n x_n$.  $M= Dy \oplus D x_2 \oplus \ldots \oplus D x_n$ 
and $py \in K$.
Set $K_1= (K \cap \langle x_2, \ldots, x_n \rangle)$.
$K= (Dp) y + K_1$ and $Dp x_1 \cap K_1= \{0\}$ so $K= Dp y \oplus K_1$.
Since $K_1 \subseteq \langle x_2 , x_3 , \ldots , x_n \rangle$, the result follows now 
by applying the induction hypothesis.
\end{quote}
{\bf Multivariate division algorithm:}
Fix a monomial order ($\leq$) for terms in $x_1, x_2, \ldots x_n$. Denote leading term of
$f$ under this order as $in_{\leq}(f)$.   The division algorithm for $f$
with respect to the monomial order produces
$f(x)= a_1 (x) f_1 (x) + \ldots + a_m (x) f_m (x) + r(x) $
where $r=0$ or $r$ is a linear combination of monomials none of which are
divisible by $in_{\leq}(f_i )$.  This is written as
$r= {f^F}$.  \emph{Procedure for multi-variable division algorithm:} Set
$r \leftarrow f(x), a_i(x) \leftarrow 0$.
Pick ordering of $f_1(x), f_2(x), \ldots , f_m(x)$.
If $in_{\leq}(f_j) | in_{\leq}(r)$ for any $j$, pick first such $j$, set
$t \leftarrow {\frac {in_{\leq}(r)} {in_{\leq}(f_j)}}$,
$s \leftarrow s- t f_j(x)$,
$a_j(x) \leftarrow a_j(x) + t$; repeat this step until if condition fails.
$r \leftarrow s$.  In general, the result depends on the ordering of the
$f_j(x)$.  \\
\\
{\bf Grobner Basis}:  A finite subset
$G= \{ g_1 , g_2 , ..., g_s \}$ is a Grobner basis for an ideal $I$
with respect to the monomial order $\leq$ if
$\langle in_{\leq}(g_1 ), in_{\leq}(g_2 ), ... , in_{\leq}(g_s )  \rangle= 
\langle in_{\leq}(I) \rangle$.
Equivalently, if $f \in I$, $in_{\leq}(g_i ) | in_{\leq}(f)$ for some $i$.
If $G$ is a Grobner basis $f^G$ is independent of the order of the $f_i(x)$.
If $G$ is a Grobner basis and $I= \langle G \rangle$, $f \in I$ iff $f^G = 0$.
\\
\\
{\bf Dickson's Lemma:}  If $S \subseteq N^n$ then $\exists v_1, v_2, \ldots v_m$ such
that $S \subseteq (v_1 + N^n) \cup (v_2 + N^n) \cup \ldots \cup (v_m + N^n)$.
\\
Consequence: Every ideal has a Grobner basis.
\begin{quote}
\emph{Proof:} Let $S= \{v: x^v = in_{\leq}(f), f \in I \}$.  By Dickson,
$S \subseteq \bigcup_i (v_i + N^n), i= 1,2, \ldots m$. If
$f(x) \in I$,
$ax^w = in_{\leq} (f), w= v_i + v$ for some $i, v$ then
$x^w= x^{v_i}x^v$ hence $in_{\leq}(f_i) | in_{\leq}(f)$.
\end{quote}
{\bf Buchberger reduction:}
$f \in R$ reduces to 0 with respect to  
$f= \langle f_1, f_2, \ldots, f_m \rangle \subseteq R - \{0\}$
iff $\exists a_1 , a_2 , \ldots , a_m \in R$:
$f= a_1 f_1 + a_2 f_2 + \ldots + a_m f_m$ and $in_{\leq}(a_i f_i) \leq in_{\leq}(f)$
if $a_i f_i \ne 0$.  This is denoted by $f \rightarrow_F 0$.
Let $G= (g_1,g_2, \ldots, g_m)$, $I= \langle G \rangle$.  
If $f \rightarrow_G 0$ for all $f \in I$
then $G$ is a Grobner basis.  If $G$ is a Grobner basis for $I$,
$f^G=0$ iff $f \rightarrow_G 0, \forall f \in I$.
$S(f, g)= {\frac {x^{\gamma}} {in_{\leq}(f)} } f - {\frac {x^{\gamma}} {in_{\leq}(g)} } g$,
where $x^{\gamma}= LCM(in_{\leq}(f),in_{\leq}(g))$.
If $S(f_i, f_j) \rightarrow_F 0, \forall i,j$ then $f \rightarrow_F 0, \forall f \in I$.
$F$ is a Grobner basis iff $S(f_i,f_j) \rightarrow_F 0, \forall i,j$ iff
$S(f_i, f_j)^F = 0, \forall i,j$.
\\
\\
{\bf Buchberger Algorithm:}
Test $S(f_i, f_j)^F \ne 0$, $F= F \cup \{ S(f_i, f_j) \}$.  Do this
until all $S(f_i, f_j)^F=0$.  This procedure terminates.
\\
\\
{\bf Minimal Grobner:}
$in_{\leq}(f_i)$ does not divide
$in_{\leq}(f_j)$ and coefficients are 1.  \emph{Reduced Grobner:} Minimal Grobner where
$in_{\leq}(f_i)$ does not divide any term of $in_{\leq}(f_j)$.
\emph{Example:} $F=(x^2+y, x^2y+1)$.  $S(x^2+y, x^2y+1)= y^2-1$,
$(x^2+y, x^2y+1, y^2-1)$ is a Grobner basis.
\emph{Elimination ideals:} $I_l = I \cap k[ x_{l+1} , ..., x_{n} ]$.
\\
\\
{\bf More on Resultants.} Condition 1: 
$F_{0}(x_0 , x_1 , \ldots , x_n) = F_{1}(x_0 , x_1 , \ldots , x_n) =
\ldots = F_{n}(x_0 , x_1 , \ldots , x_n)=0 $, 
with each $F_i$ homogeneous of degree
$d_i$ in the $x_i$.  Let $F_i (x_0 , x_1 , \ldots , x_n)= \sum_{|\alpha|= d_i} 
u_{i, \alpha} x^{\alpha}$.
{\bf Theorem 1:} Fix $d_0 , d_1 , \ldots , d_n$, there is a unique polynomial
$Res \in {\mathbb Z} [ u_{i, \alpha}]$ such that if $u_{i,\alpha}$ are replaced by the 
corresponding $c_{i,\alpha} \in {\mathbb C}$
and $F_i$ is homogeneous of degree $d_i$ then (a) the equations of condition 1 have a 
non-trivial solution in ${\mathbb C}$ iff $Res(F_0 , F_1 , \ldots , F_n)=0$, (b)
$Res( x_0^{d_0}, x_1^{d_1}, \ldots , x_n^{d_n}) = 1$, (c) $Res$ is irreducible.  
Sometimes we write 
$Res_{d_0 , d_1 , \ldots , d_n}$ to emphasize degrees.
Note that $Res_{1,1, \ldots , 1}$ is just the determinant.
{\bf Theorem 2:} For fixed $j, 0 \le j \le n$, $Res$ is homogeneous in $u_{j, \alpha}$
of degree $d_0 \cdot d_1 \cdot d_{j-1} \cdot d_{j+1} \cdot d_n$; further,
$Res(F_0, \ldots , F_{j-1} , \lambda F_j , F_{j+1}, \ldots, F_n)
\lambda^{ d_0 \cdot d_1 \cdot d_{j-1} \cdot d_{j+1} \cdot d_n}
Res(F_0, F_1 , \ldots , F_n)$ and the total degree of $Res$ is $\sum_{j=0}^n
d_0 \cdot d_1 \cdot d_{j-1} \cdot d_{j+1} \cdot d_n$.  $Res$ is alternating in the
$F_i$ and 
$Res(gh , F_2, \ldots, F_n) = Res(g , F_2, \ldots, F_n) Res(h , F_2, \ldots, F_n)$.
\emph{Example:}
$Res_{2,2,2} (F_0 , F_1 , F_2)$ has $18$ variables of total degree 
$12$ and $21,894$ terms.  
If $f(x)= a_l x^l + \ldots + a_0$ and
$g(x)= b_m x^m + \ldots + b_0$ then
$Res(f,g,x)= a_l^m b_m^l \prod_{i=1}^l \prod_{j=1}^m (\xi-\eta_i)= 
a_l^m \prod_{i=1}^l g(\xi_i)=
b_m^l \prod_{i=1}^m f(\eta_i)$.  Put $A_f= k[x]/(f(x))$ and let $[h]_f$ be
the natural map from $k[x] \rightarrow A_f$, further, let
$m_g: [h]_f \mapsto [gh]_f$ then $m_g$ is a linear map and $Res(f,g,x)= det(m_g)$.
\subsection{Linear Algebra}
{\bf Homomorphisms on modules:}
Left module $M$ over $R$ with $RM \subset M$, $1 m = m, (r+s)m=rm+sm$, etc.
Notation: $End_R(X)= Hom_R(X,X)$.
$Hom_R(U,V)= \{ f, f:U \rightarrow V, 
f(r_1 u + r_2 v)= r_1 f(u) + r_2 f(v) \}$ where $r_i \in R$.
\\
\\
{\bf Definition:} 
If $V$ is a vector space (or module) then $V^*$, the set of linear functions over $V$, 
is the \emph{dual space}.  
\\
\\
{\bf Theorem:}
If $V$ is finite dimensional, $dim(V)= dim(V^*)$.  $V \approx V^{**}$.  
Observe that the solution space is the
kernel of linear map, $L$. 
$colRank + dim(ker(L))=n$.  $rowRank + dim(ker(L))=n$.
\begin{quote}
\emph{Proof:} Let
$L = 
\left(
\begin{array}{ccc}
a_{11} & \ldots & a_{1n}\\
\ldots & \ldots & \ldots\\
a_{m1} & \ldots & a_{mn}\\
\end{array}
\right)$.  $L: {\mathbb R}^n \rightarrow {\mathbb R}^m$.  The image of $L$ is the column
space of $L$ and the kernel of $L$ is the solution space to $Lx = 0$.
${\mathbb R}^n / ker(L) \cong im(L)$ so $n - dim(ker(L)) = colRank(L)$.
Now, $ker(L)$ is the space of vectors in ${\mathbb R}^n$ orthogonal to the
rows of $L$ so $dim(ker(L)) + rowRank(L) = n$. So, $rowRank(L) = colRank(L)$.
\end{quote}
{\bf Theorem:}   Suppose $r_1, r_2 , \ldots , r_m \in {\mathbb R}^n$ are linearly independent and
$S= \{ x \in {\mathbb R}^n: x \cdot r_i=0 \}$ then $dim(S)=n-m$.
\begin{quote}
\emph{Proof:}  Put $W= \{ r_1, r_2, \ldots , r_m \}$.
$r_i \notin S$ since $r_i \cdot r_i \ne 0$ so $dim(S) \le n-m$.  Define $\alpha: x \mapsto
(x \cdot r_1, \ldots , x \cdot r_m)$.  Then $dim(ker(\alpha)) = dim(S)$ and
$dim(ker(\alpha)) + dim(im(\alpha)) =n$.  Since $dim(im(\alpha)) \le m$, $dim(S) + m \ge n$ so
$dim(S) \ge n-m$.
\end{quote}
{\bf Theorem:} The row rank, $r$, equals column rank, $c$. 
\begin{quote}
\emph{Proof:} Let $A= (a_{ij})$ be an $m \times n$ matrix.  
Let $\langle S_1 , \ldots , S_r \rangle$ be a basis for the row space.
Put $S_i= (s_{ij}), 1 \leq j \leq n, 1 \leq i \leq r$. 
Let row $i$, $R_i= (a_{i1}, a_{i2}, \ldots, a_{in})$.
$R_i = \sum_{t=1}^r k_{it} S_t$.  
So $a_{ij}= \sum_{t=1}^r k_{it} s_{tj}, 1 \leq j \leq n, 1 \leq i \leq m$ and
the column vectors $(k_{1i}, k_{2i} , \ldots , k_{mi})^T, 1 \le i \le r$ 
span the column space.  Thus $c \leq r$.
The same holds for $A^T$ and so $r \leq c$ and $r=c$.
\\
\\
\emph{Artin's proof:}
\\
\emph{Lemma:} If $W \subset V$ are vector spaces over $k$ and $W^{\perp} \subset V^*$ then
$dim(W) + dim(W^{\perp}) = dim(V)$.
\\
\emph{Proof of result:} Let $T:k^n \rightarrow k^m$ be the
linear transformation represented by the matrix $M$ with rows $r_1 , r_2 , \ldots , r_m$
and columns $c_1, \ldots , c_n$ and let the row space of $M$ be $R$ and the column space,
$C$; finally, let $r=dim(R)$, $c=dim(C)$ and $W= ker(T)$.  Since $dim(Im(T))+dim(W)=n=dim(V)$,
$r=n-dim(W)$ and $dim(W) + dim(W^{\perp})=n$, it suffices to show $dim(W^{\perp})=r$.  
Note that
$r_i \cdot w= 0$ for $w \in W$ so, if $\lambda_i$ is the usual dual basis of $V^*$ with
respect to 
$\langle e_1 , e_2 , \ldots , e_n \rangle$ where
$\langle e_1 , e_2 , \ldots , e_k \rangle=W$.  
Let $\lambda_j$ be the natural dual basis and note that
$R \subseteq \langle e_{k+1} , e_{k+2}, \ldots , e_n \rangle$ 
since $r_i \cdot \lambda_j=0$ for $j \le k$.
Now let 
$b_{k+1} \lambda_{k+1} + \ldots + b_n \lambda_n = \lambda \in W^{\perp}$.
Consider
$\varphi: \lambda \mapsto b_{k+1} e_{k+1} + \ldots + b_n e_n$.  If 
$\varphi(\lambda)=0$, $\lambda=0$ so $dim(R)=dim(W^{\perp})$ and the result holds.
\end{quote}
Here's still another proof.\\
\\
{\bf Theorem:} If $A:V \rightarrow W$ is a linear transformation (i.e, $A \in Hom_{\mathbb R}(V,W)$) between two 
finite dimensional vector spaces over ${\mathbb R}$, then $V = U \oplus S$ where $U \cong im(A)$ and $S=ker(A)$.
\begin{quote}
\emph{Proof:}  Let $\langle u_1, u_2, \ldots , u_k \rangle$ be a basis for $im(A)$, so $k=dim(im(A))$.  Pick
$v_1, \ldots, v_k \in V: A(v_i)= u_i, 1 \leq i \leq k$. First observe $\langle v_1, \ldots, v_k \rangle$ are
linearly independent because $a_1 v_1 + \ldots + a_k v_k = 0$ implies $a_1 A(v_1) + \ldots + a_k A(v_k) = 0$ so
$a_1 u_1 + \ldots + a_k u_k = 0$ and all the $a_i$ are $0$ since $\langle u_1, u_2, \ldots , u_k \rangle$ is
a basis; also note that $U= span(v_1, \ldots, v_k) \cong im(A)$.
Let $S= ker(A)$.  $V= U + S$ so we need only show $U \cap S = 0$.
If $v= a_1 v_1 + \ldots + a_k v_k \in S$ then $a_1 A(v_1) + \ldots + a_k A(v_k) = 0$ but since $\langle u_1, u_2, \ldots , u_k \rangle$
is a basis, we must have $v=0$.
\end{quote}
{\bf Theorem:} Let $V= {\mathbb R}^n$ and $\langle v_1, \ldots, v_m \rangle$ be a set of linearly independent vectors in $V$.
Put $S= \{v \in V: (v_i, v)= 0, i = 1, \ldots, m\}$.  Finally, put $s= dim(S)$.  $s+m = n$.
\begin{quote}
\emph{Proof:}  
Put $W= span(v_1, \ldots, v_m)$.
First note that $S \cap W = 0$ because $v \in S \rightarrow (v,w)=0, \forall w \in W$.  If $v \in W$, this means $(v,v)=0$ so
$v=0$.  Now, $dim(W+S) = dim(W) + dim(S) - dim(S \cap W)= dim(W) + dim(S) \leq n$  so $n - m \geq s$.  
Define $\alpha: V \rightarrow {\mathbb R}^m$ by $\alpha(v) = ( (v_1, v), (v_2, v), \ldots, (v_m, v) )$. 
$\alpha \in Hom_{\mathbb R}({\mathbb R}^n, {\mathbb R}^m)$ so $im(\alpha) + ker(\alpha) = {\mathbb R}^n$.
$ker(\alpha) = S$ and $dim(im(\alpha)) \leq m$ since $im(\alpha) \subseteq {\mathbb R}^m$.  Thus
$m + s \geq n$ and $s \geq n-m$, this, along with the previous inequality gives $n-m=s$.
\end{quote}
{\bf Theorem:}  Let $A=
\left(
\begin{array}{cccc}
a_{11} & a_{12} & ... & a_{1n} \\
... & ... & ... & ... \\
a_{m1} &  a_{m2} & ... & \ a_{mn}\
\end{array}
\right)$.  $A: {\mathbb R}^n \rightarrow {\mathbb R}^m$.  Let $r$ be the row rank of $A$ and $c$ be the column rank. Then
$r=c$.
\begin{quote}
\emph{Proof:}  
Put $a_i^{R} = (a_{i1},  a_{i2} ,  \ldots a_{in}), 1 \leq i \leq m$ and 
$a_j^{C} = (a_{1j},  a_{2j} ,  \ldots a_{mj})^T, 1 \leq j \leq n$.
Put $S= \{x \in {\mathbb R}^n: (a_i^{R}, x)=0, 1 \leq i \leq m \}$ and $s= dim(S)$.  Note that $S=ker(A)$.  By the previous theorem,
$n= s + r$. The column space of $A$ is $V_{CS}= span(a_1^{C}, \ldots ,a_n^{C})$ which is just $im(A)$.  So
$n = c +s$.  Thus $r=c$.
\end{quote}
{\bf Change of basis for matrix:}
Let $[e]= \{ e_1 , \ldots , e_n \}$ be a basis for $V_n$ and let $L$ be a
linear transformation on $V_n$.  Let 
$v_{[e]} = [c_1 , c_2 , ... , c_n]^T$ denote the
coordinates of $v$ with respect to $[e]$: $v_{[e]}= c_1 e_1 + ...+ c_n e_n$.
Let $L_{[e]}$ denote the
matrix for $L$ with respect to $[e]$: $L_{[e]}: e_i \mapsto \sum_j a_{ji} e_j$. Then
$L_{[e]} v_{[e]} = (Lv)_{[e]}$.  If $f_i= \sum_j b_{ji} e_j$ is another basis,
$P= (b_{ij})$ is called the transition matrix from $[f]$ to $[e]$ and $P^{-1}$ is
the transition matrix from $[e]$ to $[f]$ (note the sum over the first index).
$Pv_{[f]}= v_{[e]}$ and
$v_{[f]}= P^{-1}v_{[e]}$.  Finally, $L_{[f]}= P^{-1} L_{[e]}P$.  The same holds over free
modules.  Alternate notation: $L: V \rightarrow W$, 
$V$ has basis ${\cal B}$ and
$W$ has basis ${\cal B'}$ with $L(w_i)= \sum_j a_{ij} v_j$ then
$M^{\cal B}_{\cal B'}(F)= A^T$.  If ${\cal B}$ and ${\cal B'}$ are over
the same space, 
$M^{\cal B'}_{\cal B'}(F)=
N^{-1}M^{\cal B}_{\cal B}(F)N$ where
$N= M^{\cal B'}_{\cal B}(id)$.
\\
\\
{\bf Theorem:}
The group of \emph{affine transformations} is isomorphic
to the subgroup of the matrices with last column $(0,0, \ldots, 0,1)$.
The translations form a normal subgroup.
\\
\\
{\bf Cayley-Hamilton Theorem:}  
Any matrix, $A$,
acting on a vector space $V$ of dimension $n$, over a field, $F$, whose 
characteristic roots lie in $F$ (For example, if $F$ is
algebraically closed field) is similar to a 
triangular one.  The minimum polynomial divides the characteristic polynomial.
\begin{quote}
\emph{Proof:}
The second statement follows from the first since the characteristic roots, $\lambda_i$
appear on the diagonal of the triangular matrix and the characterisitic polynomial is
thus $f(x)= \prod_{i=1}^n (x-\lambda_i)$.  $f(A)=0$ so the minimal polynomial divides the
characteristic polynomial.  The proof of the first statement is by induction on $n$.  It is
true for $n=1$.  Suppose it's true for $n-1$.  $A$ has an eigenvalue, say, $\lambda_1$ in
$F$ with $Av= \lambda_1 v_1, v_1 \ne 0$.  Put $W= \{ av_1, a \in F \}$. $A$ acts on
$V/W$ which has dimension $<n$ so by induction, $\exists v_2 , \ldots , v_n$ and
$a_{ij} \in F$: 
$A {\overline v}_2= a_{22} {\overline v}_2$,
$A {\overline v}_3= 
a_{32} {\overline v}_2 +
a_{33} {\overline v}_3$,
and so on.  Let $v_i, i>1$ be corresponding elements of $V$.  $Av_2-a_{22}v_2 \in W$ and
so on.  Thus $A$ is triangular in the basis $\{ v_1, \ldots, v_n \}$.
\end{quote}
{\bf Definitions:} Let $A^*$ denote the \emph{adjoint} (conjugate transpose).  
$(Ax,y)=(x,A^*y)$.
A \emph{hermitian matrix} is self adjoint over complex numbers.
A \emph{symmetric matrix} is self adjoint over reals.  
A \emph{unitary matrix} $AA^*=I$; equivalently: $A$ is length
preserving: $(Ax,Ay)=(x,y)$.  
$A$ is \emph{nilpotent} if $\exists q$: $A^q = 0$, smallest $q$ is degree of nilpotence.
$A$ is \emph{normal} if $A A^*= A^* A$.
\\
\\
{\bf Theorem:}
If $T$ is any linear transform on $V_n$, $\exists M_0 , M_1 , \ldots , M_n$:
(i) $AM_k \subseteq M_k$, (ii) $dim (M_j )= j$, (iii)
$\{0\} = M_0 \subseteq M_1 \subseteq \ldots \subseteq M_n= V_n$.
\begin{quote}
\emph{Proof:}
\end{quote}
{\bf Theorem:}
If $A$ is a linear transform on $V_n$ with proper values
$\lambda_1 , \lambda_2 ,  \ldots , \lambda_p$ having multiplicity
$m_1 , m_2 ,  \ldots , m_p$ then $V_n =  M_1 \oplus \ldots \oplus M_p$ with
$AM_j \subseteq M_j$, $dim(M_j ) = m_j$ and $A- \lambda_j I$
is nilpotent on $M_j$.
\begin{quote}
\emph{Proof:}
\end{quote}
{\bf Theorem:}
If $A$ is nilpotent of degree $q$, $\exists x$: $A^{q-1}x \ne 0$ and
$x , Ax , A^2 x, \ldots A^{q-1} x$ are linearly independent.  
Every linear transform is the direct sum of a nilpotent and an invertible transform.
\begin{quote}
\emph{Proof:} 
The existence of $x$ is guarenteed by nilpotence.
If $a_0 x +  a_1 Ax + a_2 A^2 x  + \ldots a_{q-1} A^{q-1} x = 0$, applying $A$
$q-1$ times, we get  $a_0 A^{q-1}(x)= 0$ which is a contradiction. 
\end{quote}
{\bf Theorem:}
If $A$ is symmetric and $X$ is orthogonal then $X A X^{-1}$ is symmetric.
\begin{quote}
\emph{Proof:}  Since $X$ is orthogonal, $X^{-1} = X^T$.
$X A X^{-1} = (X A X^T)^T = X A^T X^T = X A X^T = (X A X^{-1})^T$.
\end{quote}
{\bf Theorem:}
$T$ is orthogonal (unitary) iff it takes orthonormal basis into orthonormal basis
which happens iff $TT^*=I$.  
\begin{quote}
\emph{Proof:} 
If $T$ is unitary and $\langle v_1, \ldots , v_n \rangle$ is an orthonormal basis,
$(Tv_i, Tv_j)= (v_i, v_j)= \delta_{ij}$ and $\langle Tv_1 , \ldots , Tv_n \rangle$ is
an orthonormal basis.
If $T v_i = w_i$ and 
$\langle v_1 , \ldots , v_n \rangle$ and $\langle w_1 , \ldots , w_n \rangle$ 
are both orthogonal basis extending linearly we get, $(Tx, Ty)= (x,y)$ and $T$ is unitary.
\end{quote}
{\bf Theorem:}  (a) If $S^*S(v)=0$, $S(v)= 0$.
Suppose $N$ is \emph{normal}. 
(b) $N(v)= 0 \rightarrow N^*(v)= 0$.
(c) If $N(v)= \lambda v$, $N^*(v)= {\overline {\lambda}} v$. 
(d) If $N^k(v)=0$ then $N(v)=0$.
(e) If $N(v)= \lambda v, N(w)= \mu w, \lambda \ne \mu$ then $(v,w)= 0$.
\begin{quote}
\emph{Proof:} 
For (a), $(S^*S(v),v) = 0 = (S(v), S(v))$ and so $S(v)= 0$.
For (b), $(N^*(v), N^*(v))= (NN^*v,v)= (N^*Nv,v)=(Nv,Nv)=0$.
For (c), $(N-\lambda) (N^*-{\overline {\lambda}}) $.  Since 
$(N-\lambda)$ is normal and
$(N-\lambda)(v)= 0$, (b) gives
$(N-\lambda)^*(v)= 0$ and the result follows.
For (d), let $S=N^*N$ then $S^k v= (N^*)^k N^k v= 0$.  Since $S^*=S$, $S(v)= 0$ by
(a) and again by (a) $N(v)= 0$.
For (e), $\lambda (v,w)= (Nv,w)= (v, N^* w)= (v, {\overline {\mu}} w)= \mu (v,w)$ and
the result follows.
\end{quote}
{\bf Spectral Theorem:}  If $T$ is \emph{normal} ($TT^*=T^*T$), $\exists E_{1}, 
\ldots , E_{r}$ such that $T= \sum_{i}^{r} \lambda_{i} E_{i}$ with
$T= \sum_{i}^{r} E_{i}= I$, $E_{i}E_{j}= 0$ and transforming matrix, A,
unitary $({\overline A^{t}} = A^{-1})$.
\begin{quote}
\emph{Proof:} 
Let $\lambda_1, \ldots, \lambda_k$ be the distinct characteristic roots of $T$.  By
the primary decomposition theorem, $V= V_1 \oplus \ldots \oplus V_n$
and each $V_i$ is annihilated by $(T-\lambda_i)^{n_i}$.  Vectors
in different $V_i$'s are orthogonal and each $V_i$ has an orthonormal basis by Gram-Schmidt.
$T$ can be transformed into an upper triangular matrix with its eigenvalued on the diagonal,
sinfe $T$ is normal, this matrix must be diagonal.
\end{quote}
{\bf Theorem:}
If $A$ is symmetric there is a $P$ such that $P^TAP$ is diagonal.  All the 
eigenvalues are real.
\begin{quote}
\emph{Proof:} Diagonalizability follows from the Spectral theorem.  If
$Av= \lambda v$, $(AA^Tv, v)= (Av,Av)= \lambda^2>0$ so the eigenvalues are real.
\end{quote}
{\bf Sylvester's Theorem:}
Every real quadratic form is equivalent to a diagonal one
with a signature of positive and negative coefficients.   
Two forms are
equivalent iff they have the same rank and signature.
\begin{quote}
\emph{Proof:} 
Since the matrix for the form is real and symmetric it can be brought into diagonal form
by an orthogonal transformation and the eigenvalues are the diagonal elements of the
matrix.  The rank is invariant and so the number of nonzero elements is an invariant.
Because there are square roots, we can assume the elements are $\pm 1$.  If 
$r$ is the number of $-1$'s and 
$s$ is the number of $1$'s we need only show these are invariants.  Since the subspace
of vectors on which the form is positive is an invariant, we're done.
\end{quote}
{\bf Definition:}
Extreme point in {convex} set: $P$ with no $Q_1$, $Q_2$ such that $P= t Q_1 + (1-t) Q_2$,
$t>0$.
\\
\\
{\bf Krien Millman Theorem:}  
If $S$ is a closed, bounded convex set, then $S$ is the convex closure
of its extreme points.
\\
\\
{\bf Principal Axis Theorem:}
Any real quadratic form is equivalent to one with $Q(\eta)= \lambda_1{x_1}^2 +
\ldots + \lambda_n {x_n}^2$ with
$\lambda_1 \geq \lambda_2 \geq \ldots \geq \lambda_n$.
\begin{quote}
\emph{Proof:}
Find eigenvector $v$, $V= \langle v \rangle \oplus \langle v \rangle^{\perp}$.
\end{quote}
{\bf SVD:} $A= U \Sigma V^T$, $\Sigma$ diagonal and $U U^T=V V^T = 1$.
\\
\\
{\bf Theorem:}
An $n \times n$ matrix is \emph{diagonalizable} iff it has $n$ linearly independent
eigenvectors.  A matrix is diagonalizable iff its minimal polynomial is a 
product of different linear factors.
Two matrices are simultaneously diagonalizable iff
they are diagonalizable and commute.
\begin{quote}
\emph{Proof:} The first and second statements are easy.  The third statement is
proved by induction on $n$.  It is clear for $n=1$.  Let 
$ \lambda_1, \lambda_2, \ldots, \lambda_k$ be the characteristic values
of $T$ and $W_i$ be the null space of $T-\lambda_i$.  Each $W_i$ is an invariant space
and the $T_{|W_i}$ is diagonalizeable.  Since $dim(W_i)<dim(V)$ the commuting operators
restricted to these spaces can be simultaneously diagonalized.  Composing the diagonal
bases of the restricted transformations yields a basis in which all the matricies are
diagonal.
\end{quote}
{\bf Theorem:}
Let $f: A \rightarrow A'$ be surjective.  $A$, $A'$ abelian, $A'$ free.
$\exists C \subseteq A$ such that $A= ker(f) \oplus C$.
\begin{quote}
\emph{Proof:} 
Let $\langle x_i' \rangle$ be a basis of $A'$ and for each $i \in I$, let $x_i \in A$
be such that $f(x_i )= x_i'$.  Put $C= \langle x_i \rangle$ and $B=ker(f)$.  If we have
$\sum_{i \in I} n_i x_i = 0$, applying $f$ yields
$\sum_{i \in I} n_i x_i' = 0$ and $\langle x_i \rangle$ is a basis.
Similarly, if $z \in C$ and $f(z)= 0$ then $z= 0$.  Hence $B \cap C =0$.
Let $x \in A$.   Since $f(x) \in A'$, $\exists n_i, i \in I:
f(x)= \sum_{i \in I} n_i x_i'$.  Applying $f$ to $x - \sum_{i \in I} n_i x_i = b \in B$.
$x \in B + C$ and so $A= B \oplus C$.
\end{quote}
{\bf Submodules of free modules over PIDs:}  
Let $D^{(n)}$ be a free $D$-module, $D$, a PID, with basis $X= [e_1 , e_2 , \ldots , e_n]^T$.  
If $P$ is invertible, $Y= [e'_1, \ldots , e'_n]^T= P X$ is another basis for $D^{(n)}$.
Let $K$ be a submodule generated by
$U=[u_1, \ldots , u_m]^T = A X$.  $A$ is called a relations matrix.
Suppose $Q$ is invertible so that
$V= QU$ is another set of generators for $K$.
$V= QU = QAX= QAP^{-1}Y$.  $B=QAP^{-1}$ is the new relations matrix for the basis $Y$.
$A$ and $B$ are called \emph{equivalent}.
In the foregoing, $M$ is finitely generated and we note that the map
$f: e_i \mapsto u_i$ extends to a homomorphism
and $M \cong D^{(n)} / K$ where $K= ker(f)$.  $K$ is
a submodule of $D^{(n)}$ and by a previous result, is free with a base of size $m \le n$.
\\
\\
{\bf Theorem:}  If $A$ is an $m \times n$ matrix with entries from a PID,
$D$, then $A$ is equivalent to a matrix of the form
$diag(d_1, d_2, \ldots, d_r, 0, 0, \ldots, 0)$ with $d_i \ne 0$ and $d_i \mid d_{i+1}$.
The proof simply requires applying elementary row and column operations to 
$A$ as is done below.
There, let $O_{ij}= (\delta_{il} \delta_{jk})_{1 \leq l \leq n, 1 \leq k \leq n}$
and $E_{ij}(\alpha)= I+ \alpha O_{ij}$.
$E_{ij} (\alpha) A$ acts on $A$ by adding $\alpha$ times row $j$ to row  $i$.
$A E_{ij} (\alpha) A$ acts on $A$ by adding $\alpha$ times column $i$ to column $j$.
\\
\\
\emph{Example:} Suppose $D= {\mathbb Z}$ and $F= {\mathbb Z}^{(3)}$ 
is a free abelian group with basis $x_1, x_2, x_3$ and
$K$ is the submodule generated by
$u_1= 2 x_1 + 2x_2 + 8 x_3$ and $u_2= -2 x_1 + 2 x_2 + 4 x_3$ and
$F/K= 
\langle x_1 , x_2, x_3  | 2 x_1 + 2x_2 + 8 x_3 = 0$, $-2 x_1 + 2 x_2 + 4 x_3 = 0 \rangle$.
We have $ \left(
\begin{array}{ccc}
2 & 2 & 8\\
-2 & 2 & 4\\
\end{array}
\right) 
\rightarrow
\left(
\begin{array}{ccc}
2 & 0 & 0\\
0 & 4 & 0\\
\end{array}
\right)$ 
and $F/K = {\mathbb Z}_2 \oplus {\mathbb Z}_4 \oplus {\mathbb Z}$.
\\
\\
{\bf Structure theorem for finitely generated modules over principal ideal domains:} 
If $M \ne 0$ is a finitely generated module over a PID, $D$. Then $M= Dz_{1} \oplus
Dz_{2} \oplus \ldots \oplus Dz_{s}$ with
$ann(z_{1}) \supseteq ann(z_{2}) \supseteq \ldots \supseteq ann(z_{s})$, 
$z_{k} \ne 0$.  Note $D z_i \cong D/ann(z_i)$.
\begin{quote}
\emph{Proof:} 
First, $Dx \cong D/ann(x)$, always.
Let the base of $D^{(n)}$ be $e_{1}, e_{2}, \ldots , e_{n}$ and let
$\eta: D^{(n)} \rightarrow M$ be the canonical map
$\sum_i a_i e_i \mapsto \sum_i a_i x_i$ where $\langle x_i \rangle$ are generators
for $M$.  $M \cong D^{(n)}/K$. $K= ker(\eta)$ is generated by 
$f_{i}= \sum_j a_{ij} e_{j}, i= 1, 2, \ldots , n$.  
The Smith reduced canonical relations matrix is
$QAP^{-1}= diag(d_{1}, d_{2}, ..., d_{r}, 0, \ldots, 0 )$ and $d_{i} \mid d_{i+1}$,
$P=(p_{ij}), Q= (q_{ij})$.
So $f'_{i}= d_{i} e'_{i}$ where $e_i'= \sum_j p_{ij} e_j$ and 
$f_i'= \sum_j q_{ij} f_j$. 
$y_i= \sum_j p_{ij} x_j$ is another set of generators for $M$ (i.e.,
$M= \sum D y_i$).  If $\sum_i b_i y_{i}=0$ then $\sum_i b_i e_i' \in K$ so
$\sum_i b_i e_i' = \sum_i c_i f_i'= \sum_i c_i d_i e_i'$ and so $b_i = c_i d_i$ and 
$b_i y_i=0, \forall i$, since each $d_i y_i= 0$, already.
$ann(y_{i})= (d_{i})$ by construction.  If $d_{i}$ is a unit, $d_i y_i= 0 \rightarrow y_i= 0$ 
and we can drop $y_i$ from the list of generators.  Thus if the first $t$ $d_i$ are units,
putting $z_{1}= y_{t+1} , \ldots, z_{s}= y_{n}$, $s= n-t$, we get the desired result.
\end{quote}
{\bf Notation:}  Let $T$ be and endomorphism of the $D$-module, $M$ $\alpha \in M$.
$Z(\alpha, T)$ is the cyclic subspace generated by $T$.  The $T$-annihilator
of $\alpha$ is $(p(x))$ since $D$ is a PID where
$p(x)= x^k + c_{k-1} x^{k-1} + \ldots + c_0$.  
The companion matrix (matrices operating on the left) is $C_T=
\left(
\begin{array}{cccccc}
0 & 1 & 0 & \ldots & 0 & 0\\
0 & 0 & 1 & \ldots & 0 & 0\\
\ldots & \ldots & \ldots & \ldots & \ldots & \ldots \\
0 & 0 & 0 & \ldots & 0 & 1\\
-c_0 & -c_1 & -c_2 & \ldots & -c_{k-2} & -c_{k-1}\\
\end{array}
\right)$.
For a vector space of $V$ over $F$, $T: V \rightarrow V$,
the \emph{rational decomposition}
is $V= Z(\alpha_1, T) \oplus \ldots \oplus Z(\alpha_r, T)$.  Again, the PID is $D= F[x]$.
\\
\\
{\bf Application to a linear transformation:} Suppose 
$e_1, \ldots, e_n$ is a basis for $V$ over $F$ and put $D= F[\lambda]$.  Let 
$T(e_i) = \sum_j a_{ij} e_j$.  For $g(\lambda) \in D$, $g(T)$ acts on vectors in the
usual way.  As above, $M \cong {D^{(n)}}/{K}$ and $K$ has a free base over $D$
with $m \le n$ elements.\\
\\
\emph{Lemma:} $f_i = \lambda e_i - \sum_j a_{ij}e_j$ is a base for $K$ over $D$.
\begin{quote}
\emph{Proof:}
Since $Te_i= \sum_j a_{ij} e_j$,
$\lambda e_i= f_i+ \sum_j a_{ij} e_j$, and we can write any $g_i(\lambda) e_i$ as
$\sum_i g_i( \lambda ) e_i = \sum_i h_i( \lambda )f_i + \sum b_i e_i$, $b_i \in F$.
If this is in $K$, then $\sum_i b_i e_i \in K$ so $\sum_i b_i e_i = 0$.
Since the $e_i$ form a base for $V$ over $F$, $b_i= 0, \forall i$ and the element
$K$ has the form $\sum_i h_i( \lambda) f_i = 0$.  Suppose there is a non trivial relation
between the $h_i( \lambda )$, then
$\sum_{i=1}^n h_i( \lambda) \lambda e_i = \sum_{i,j=1}^n h_i( \lambda) a_{ij} \lambda e_i$
and since the $e_i$ is a base,
$h_i( \lambda) \lambda = \sum_{j=1}^n h_j( \lambda) a_{ji}$.
If any $h_i( \lambda ) \ne 0$ let $h_r( \lambda )$ be one of maximal degree.
Clearly,
$h_r( \lambda ) \lambda = \sum_j h_j( \lambda ) a_{jr}$ is impossible.  This proves
every $h_i( \lambda) = 0$ 
and so the $f_i$ form a base for $K$.
\end{quote}
After diagonalization,
$Q (\lambda I -A) P = diag(1,\ldots,1,d_1(\lambda), \ldots , d_s(\lambda))$.
As in the proof of the structure theorem,
$K$ is generated by $f'_i= d_i e'_i$.  If
$P^{-1}= (p^*_{ij})$, $v_i= \sum_j p^*_{ij} u_j$, $z_i= v_{n-s+i}$ and
$V= D z_1 \oplus \ldots \oplus D z_s$.
\\
\\
\emph{Example:}
Suppose
$T u_1= -u_1 - 2 u_2 + 6 u_3$,
$T u_2= -u_1 + 3 u_3$,
$T u_3= -u_1 - u_2 + 4 u_3$.  The matrix for $T$ in the basis
$\langle u_1 , u_2 , u_3 \rangle$ is $A$.
$$
A=
\left(
\begin{array}{ccc}
-1 &  -2 &  6 \\
-1 &  0 &  3 \\
-1 &  -1 &  4
\end{array}
\right),
Q=
\left(
\begin{array}{ccc}
0 & 1 & 0\\
0 & -1 & 1\\
1 &  2-\lambda & -3
\end{array}
\right),
P=
\left(
\begin{array}{ccc}
1 & 3 & \lambda-3\\
0 & 0 & -1\\
0 & 1 &  -1
\end{array}
\right)
$$
$$
Q(\lambda I - A) P =
\left(
\begin{array}{ccc}
1 & 0 & 0\\
0 & (\lambda -1) & 0\\
0 & 0 & (\lambda -1)^2
\end{array}
\right),
P^{-1}=
\left(
\begin{array}{ccc}
1 & \lambda & -3\\
0 & -1 & 1\\
0 & -1 & 0
\end{array}
\right)
$$
$v_1=u_1+\lambda u_2 - 3 u_3$, $v_2= -u_2 + u_3$, $v_3= -u_2$.
$z_1=v_2=-u_2 + u_3$, $z_2=v_3= -u_2$, $z_3=\lambda v_3= u_1-3u_3$. So,
$
\left(
\begin{array}{c}
1 \\
0 \\
0 
\end{array}
\right)_{z}=
\left(
\begin{array}{c}
0 \\
-1 \\
1
\end{array}
\right)_{u}
$,
$
\left(
\begin{array}{c}
0 \\
1 \\
0 
\end{array}
\right)_{z}=
\left(
\begin{array}{c}
0 \\
-1 \\
0
\end{array}
\right)_{u}
$,
$
\left(
\begin{array}{c}
0 \\
0 \\
1 
\end{array}
\right)_{z}=
\left(
\begin{array}{c}
1 \\
0 \\
-3
\end{array}
\right)_{u}
$.  Thus the transition matrix between the $u$ base and the $z$ base is
$
R=
\left(
\begin{array}{ccc}
0 & -1 & 1\\
0 & -1 & 0\\
1 & 0 & -3
\end{array}
\right)
$
and
$
R^{-1}=
\left(
\begin{array}{ccc}
3 & -3 & 1\\
0 & -1 & 0\\
1 & -1 & 0
\end{array}
\right)
$.  $R A R^{-1}= 
\left(
\begin{array}{ccc}
1 & 0 & 0\\
0 & 0 & 1\\
0 & -1 & 2
\end{array}
\right)
$, as we can verify directly by
$$
\left(
\begin{array}{ccc}
0 & -1 & 1\\
0 & -1 & 0\\
1 & 0 & -3
\end{array}
\right)
\left(
\begin{array}{ccc}
-1 &  -2 &  6 \\
-1 &  0 &  3 \\
-1 &  -1 &  4
\end{array}
\right)
\left(
\begin{array}{ccc}
0 & -1 & 1\\
0 & -1 & 0\\
1 & 0 & -3
\end{array}
\right)^{-1} =
\left(
\begin{array}{ccc}
1 & 0 & 0\\
0 & 0 & 1\\
0 & -1 & 2
\end{array}
\right).
$$
Note in the example above, the matricies are applied ``from the right.''  Conventionally,
matricies are applied from the left.  We can take tranposes of all the matricies 
in the above example to convert
the example to one where the matricies are applied conventionally.
Here is an analysis using the conventional notation:\\
Let $A$ be a matrix and $\langle e_1 , e_2 , \ldots, e_n \rangle$ the underlying basis so that
$Ae_1= (a_{11}, a_{21}, \ldots a_{n1})^T$.  Put $B(\lambda) = (\lambda I - A)$ and let $f_1, f_2,
\ldots, f_n$ be the columns of $B(\lambda)$. Suppose $D$ is a PID and
$\langle x_1 , x_2, \ldots , x_n \rangle$ be generators of the module, $M$ over $D$.
Let $\eta: D^{(n)} \rightarrow M$ be defined by $\eta(c_1 , c_2, \ldots , c_n) =
c_1 x_1 + c_2 x_2 + \ldots + c_n x_n$.  $D^{(n)}/K = M$ where $K$ is is generated by the
columns of $B(\lambda)$, namely, $f_1, f_2, \ldots, f_n$. Put $B(\lambda)$ in Smith normal
form, $PB(\lambda)Q = diag(1,1,\ldots,1, d_1(\lambda), \ldots , d_s(\lambda))$. 
$M = D/(d_1(\lambda)) \oplus \ldots \oplus D/(d_s(\lambda))$ and $d_1(\lambda) \mid d_2(\lambda) \mid
\ldots \mid d_s(\lambda)$.  Put $P= (p_{ij})$, $Q= (q_{ij})$ and $P^{-1} = ({p_{ij}}^*)$.  Then
${e_i}' = \sum_{k} {p}_{ki} e_k$,
${f_i}' = \sum_{k} {q}_{ki} f_k$.
$e_i = \sum_{k} {p_{ki}}^* {e_k}'$.
${f_i}' = \sum_{k,l,m} q_{li} b_{kl} {p_{mk}}^* {e_m}'$.  $ann({f_i}')= (d_i(\lambda))$.
$Q$ is invertible so $\langle {f_1}', {f_2}', \ldots , {f_m}' \rangle$ also generates $K$.
If $y_i = \sum_j p_{ji} x_j$, $\langle y_i \rangle$ also generates $M$ and $D y_i = D/(d_i(\lambda))$.
When computing the rational canonical form (RCF), $v_i = \sum_{k} {p_{ki}}^* e_k$ and $z_i = v_{n-s+i}$.
The remaining $z_i$ required to from the basis of the RCF by applying $A$ to the existing $z_i$, as above.
This gives $z_i = \sum_k s_{ki} e_k$.  $S$ is the transition matrix relating the original basis to the
basis in which $A$ has standard RCF. $A_{RCF}= S^{-1} A S$.  When the matricies are applied ``from the right,''
the companion matrix, for the block having minimal polynomial $f(x) = x^n + c_{n-1} x^{n-1} + \ldots + c_0$, has the form:
$$
\left(
\begin{array}{ccccc}
0 & 0 & 0 & \ldots & -c_{n-1}\\
1 & 0 & 0 & \ldots & -c_{n-2}\\
\ldots & \ldots &  \ldots & \ldots & \ldots\\
0 & 0 & \ldots & 1 & -c_{0}\\
\end{array}
\right)
$$
Here are two ways to do the computation with the current example:
$A = 
\left(
\begin{array}{ccc}
-1 & -1 & -1 \\
-2 & 0 & -1 \\
6 & 3 & 4 \\
\end{array}
\right)$,
$B(\lambda) =  (\lambda I -A) =
\left(
\begin{array}{ccc}
 \lambda + 1 & 1  & 1 \\
 1 &  \lambda & -1 \\
 -6 &  -3 &  \lambda - 4\\
\end{array}
\right)
$.
$P B(\lambda) Q =
\left(
\begin{array}{ccc}
1 & 0 & 0 \\
0 & (\lambda - 1) & 0 \\
0 & 0 & (\lambda - 1)^2 \\
\end{array}
\right)
$, where $P$ is the product of the elementary row operations:
$[R_3 \leftarrow R_3 + R_2] [R_3 \leftarrow R_3 - (\lambda - 4) R_1] [R_2 \leftarrow R_1 - R_1]$.
One way to compute $R$, the matrix with $R^{-1}AR= 
\left(
\begin{array}{ccc}
1 & 0 & 0\\
0 & 0 & -1 \\
0 & 1 & 2\\
\end{array}
\right)
$, is to compute the revised basis using the rules: (1) $R_i \leftarrow R_i + \alpha R_j$ causes a basis change of
$e_j = e_j - e_i$ and $R_i \leftrightarrow R_j$ causes a basis change of $e_j \leftrightarrow e_i$.  This results
in $(0, e_2 - e_1, e_3)$, the first entry should be zero and an additional entry can be computed by applying $A$ to
$e_3$ giving $R=
\left(
\begin{array}{ccc}
0 & 0 & -1 \\
1 & 0 & -1\\
-1 & 1 & 4\\
\end{array}
\right)
$.  We can calculate $R^{-1} =
\left(
\begin{array}{ccc}
-1 & 1 & 0 \\
3 & 1 & 1\\
-1 & 0 & 0 \\
\end{array}
\right)
$. We get
$$
\left(
\begin{array}{ccc}
1 & 0 & 0\\
0 & 0 & -1 \\
0 & 1 & 2\\
\end{array} 
\right) =
\left(
\begin{array}{ccc}
-1 & 1 & 0 \\
3 & 1 & 1\\
-1 & 0 & 0 \\
\end{array}
\right)
\left(
\begin{array}{ccc}
-1 & -1 & -1 \\
-2 & 0 & -1 \\
6 & 3 & 4 \\
\end{array}
\right)
\left(
\begin{array}{ccc}
0 & 0 & -1 \\
1 & 0 & -1\\
-1 & 1 & 4\\
\end{array}
\right)
$$
Another way, to get $R$, is to compute $P$ from the elementary row operations, so
$P = 
\left(
\begin{array}{ccc}
1 &  0 & 0\\
-1 &  1 & 0\\
(3 -\lambda) & 1  & 1 \\
\end{array}
\right)
$ and compute
$P^{-1} = 
\left(
\begin{array}{ccc}
1 & 0 & 0\\
1 & 1 & 0\\
\lambda - 4 & -1 & 1\\
\end{array}
\right)
$.   This lets us read off, from $P^{-1}$,
$R = 
\left(
\begin{array}{ccc}
0 & 0 & -1 \\
1 & 0 & -1\\
-1 & 1 & 4\\
\end{array}
\right)
$.  Again, the last column is computed as $Ae_3$ and again, we compute $R^{-1}$.
\\
\\
{\bf Observation:}
The \emph{Rational Canonical Form} and \emph{Jordan Canonical Form} are the same
over an algebraically closed field.
\\
\\
{\bf Grove's treatment:}  $R$, a pid. $f \in Hom_R(M,N)$, $M$ free of dimension $n$, $N$, free of dimension $m$. 
$M = \langle x_1, x_2, \ldots, x_n \rangle$,
$N = \langle y_1, y_2, \ldots, y_m \rangle$.
$f(x_i) = \sum_{j=1}^m a_{ji} y_j$.
$A= (a_{ij})$.
\\
\\
{\bf Theorem:} $R, M, N, f$ as above.  Put $E= Im(f)$.  Suppose that over the basis for $M$ and $N$, $f$ is represented as a matrix by:
$$
\left(
\begin{array}{ccc}
U & 0 & 0 \\
0 & D & 0 \\
0 & 0 & 0 \\
\end{array}
\right) 
$$
where
$U$ and $D$ are diagonal matricies 
$U = diag(u_1 , u_2 , \ldots , u_x \rangle$ with $u_i$ a unit and
$D = diag(d_1 , d_2 , \ldots , d_k \rangle$ with $d_i \in R$ and $d_j \mid d_{j+1}$.  Then $N/E$ is a direct sum of cyclic modules over
$R$, $N/E = \bigoplus_{i=s+1}^m R \langle y_i + E \rangle$, with invariant factors $d_1, d_2, \ldots, d_s$.  $N/E$ has rank $m-s-k$ and
$dim(E)=s+k$
\begin{quote}
\emph{Proof:} 
$f(x_i)= u_i y_i, 1 \leq u \leq s$,
$f(x_i)= d_i y_i, s+1 \leq u \leq s + k$,
$f(x_i)= 0, i geq s+k+1$.  Put $W_i = R \langle y_i + E \rangle$, $s+1 \leq i \leq m$.  $N/E = \sum_i W_i$ and $W_i \cap \sum_{i \ne j} W_j = 0$ so
the sum is direct.  Each $y_i + E$ has order $d_{i-s}$.
\end{quote}
{\bf Application to linear equations:} Consider the linear equations $\sum_{j=1}^n a_{i,j} x_j =c_i$, $i= 1, 2, \ldots, m$.  Write this as
$AX=C$.  Suppose  we have $P, Q$ so that $PAQ= diag(r_1, \ldots, r_k, 0, \ldots 0)$.  Put $X= QY$.  $AX= AQY=C$ and $PAQY=PC$. This gives
the solutions in terms of $Y$,  Now transform back to $X$.\\
\emph{Example:}
$$
A=
\left(
\begin{array}{ccc}
-33 & 42 & -20 \\
21 & -27 & 13\\
\end{array}
\right) , \;
C=
\left(
\begin{array}{c}
-26\\
16\\
\end{array}
\right)
$$
We find $P, Q$ as:
$$
P=
\left(
\begin{array}{cc}
2 & 3 \\
1 & 2 \\
\end{array}
\right) , \;
Q=
\left(
\begin{array}{ccc}
0 & 1 & 2\\
1 & 2& 3\\
2 & 3& 3\\
\end{array}
\right)
$$
which gives
$$
PAQ=
\left(
\begin{array}{ccc}
1 & 0 & 0\\
0 & 3 & 0\\
\end{array}
\right),\;
PC = 
\left(
\begin{array}{c}
-4\\
6\\
\end{array}
\right)
$$
So $y_1 = -4,3 y_2=6, y_3=a$, and finally,
$$
\left(
\begin{array}{c}
x_1\\
x_2\\
x_4\\
\end{array}
\right)
=
\left(
\begin{array}{ccc}
0 & 1 & 2\\
1 & 2& 3\\
2 & 3& 3\\
\end{array}
\right)
\left(
\begin{array}{c}
-4\\
2\\
a\\
\end{array}
\right) =
\left(
\begin{array}{c}
2\\
0\\
-2\\
\end{array}
\right) + a
\left(
\begin{array}{c}
2\\
3\\
3\\
\end{array}
\right)
$$.
\\
\\
{\bf Theorem:} $R=F[x]$.  Let $E$ be the module generated by columns of $A-xI$.  $V_T \cong N/E$.
\begin{quote}
\emph{Proof:} 
Define $\psi: N \rightarrow V_T$ by $\psi(e_i)=v_i$.  
We show $ker(\psi)=E$. 
Let $z_i = \sum_{j=1}^n a_{ji}e_j =xe_i$ then $E= \langle z_i \rangle$.
$psi(z_i) = \sum_{j=1}^n a{j,i}v_j - T(v_i)= 0$ so $E \subseteq ker(\psi)$.
Let $W = \langle e_i + E \rangle$, $1 \leq i \leq n$.  $x e_i + xE \in W$ so $W$ is a submodule of $N/E$.
Since $N= R \langle e-1 , e_2 , \ldots , e_n \rangle$, $W= N/E$.  If $u \in N$, $u = \sum_i c_i e_i + z$, $c_i \in F, x \in E$.
$\psi(u)=  \sum_i c_i \psi(e_i) + \psi(z) = \sum_i c_i v_i$ and so $u \in ker(\psi)$ iff $c_i =0$ so $u \in E$.
\end{quote}
{\bf Example:} $F = \mathbb{Q}$ and $T$ us represented over the standard basis as
$$
A =
\left(
\begin{array}{ccc}
5 & -8 & 4\\
6 & -11& 6\\
6 & -12& 7\\
\end{array}
\right)
$$
then
$$
P =
\left(
\begin{array}{ccc}
1 & 0 & 0\\
-{\frac 3 2} & 1 & 0\\
{\frac {x+5} 4} & -2& 1\\
\end{array}
\right),\;
Q =
\left(
\begin{array}{ccc}
0 & 0 & 1\\
0 & 1& {\frac 3 2}\\
1 & 2& {\frac {x+7} 4}\\
\end{array}
\right),\;
P(A-xI)Q =
\left(
\begin{array}{ccc}
4 & 0 & 0\\
0 & (1-x) & 6\\
0 & 0& {\frac {1-x^2} 4} \\
\end{array}
\right)
$$
Now compute $P^{-1}$, columns not containing $x$ and their images under $A$ will be the basis for the matrix in rational canonical form.
$$
P^{-1} =
\left(
\begin{array}{ccc}
1 & 0 & 0\\
{\frac 3 2} & 1 & 0\\
{\frac {7-x} 4}  & 2& 1\\
\end{array}
\right)
$$.
In this example, $V_t \cong R w_1 \oplus Rw_2 \oplus Rw_3$ with $w_1= e_3$, $w_2=Tw_1 = (4,6,7)^T$,
$w_3 = e2+2 e_3$ and so the basis change matrix is
$$
L =
\left(
\begin{array}{ccc}
0 & 4 & 0\\
0 & 6 & 1\\
1  & 7 & 2\\
\end{array}
\right).
$$
$$
L^{-1}AL =
\left(
\begin{array}{ccc}
0 & 1 & 0\\
1 & 0 & 0\\
0  & 0 & 1\\
\end{array}
\right)
$$.
Since $m_T(x)$ has linear factors, the Jordan canonical form is diagonal and
$$
J =
\left(
\begin{array}{ccc}
-1 & 0 & 0\\
0 & 1 & 0\\
0  & 0 & 1\\
\end{array}
\right)
$$
{\bf One more example:} $F = \mathbb{Q}$ and $T$ us represented over the standard basis as
$$
A =
\left(
\begin{array}{cc}
\lambda_1 & 0\\
0 & \lambda_2 \\
\end{array}
\right).
$$  The RCF for $A$, by inspection is 
$$
\left(
\begin{array}{cc}
x - \lambda_1 & 0\\
0 & (x - \lambda_1) (x - \lambda_2)\\
\end{array}
\right)
$$
and $m_T(X)= x^2 -(\lambda_1 + \lambda_2) x + \lambda_1 \lambda_2)$.  Further,
$$
A-xI = 
\left(
\begin{array}{cc}
x - \lambda_1 & 0\\
0 & x - \lambda_2\\
\end{array}
\right)
$$
and
$$
PAQ = 
\left(
\begin{array}{cc}
x - \lambda_1 & 0\\
0 & (x - \lambda_1) (x - \lambda_2)\\
\end{array}
\right), \;
P =
\left(
\begin{array}{cc}
1 & \lambda_1\\
1 & \lambda_2\\
\end{array}
\right), \;
$$
In RCF, the new basis are $f_1, f_2$ where
$$
\left(
\begin{array}{c}
f_1 \\
f_2\\
\end{array}
\right) =
P^T
\left(
\begin{array}{c}
e_1 \\
e_2\\
\end{array}
\right), \;
P^T =
\left(
\begin{array}{cc}
1 & 1\\
\lambda_1 & \lambda_2\\
\end{array}
\right)
$$
To switch basis, $v_{[e]}= P v_{[f]}$.
$$
P^{-1} = {\frac 1 {\lambda_1 - \lambda_2}}
\left(
\begin{array}{cc}
- \lambda_2 & \lambda_1\\
1 & -1\\
\end{array}
\right)
$$
then
$$
P^{-1} A P =
\left(
\begin{array}{cc}
0 & -\lambda_1 \lambda_2\\
1 & \lambda_1 +\lambda_2\\
\end{array}
\right)
$$ which is the RCF.
\\
\\
{\bf Principal Component Analysis:}
$P_A= A(A^TA)^{-1} A^T$ where the rank of $A$ is the number of columns, is the symmetric
projector; $P_{A^{\perp}}= I-P_A$. 
$P_A^2 = P_A$, $P_{A^{\perp}}^2 = P_{A^{\perp}}$,
$P_A^T = P_A$, $P_{A^{\perp}}^T = P_{A^{\perp}}$.  $S=A A^T$ is invertible.
$P_{\vec a} ({\vec w})$ is the projection of ${\vec w}$ along ${\vec a}$.  The
linear system $A {\vec f} = P_A {\vec b}$ has solution
${\vec f} = A^{-1} P_A {\vec b}$ the least squares approximation of data points
$(x_i,y_i)$ can be calculated from this too.  \emph{Example:} Fit $f(x)= f_0 + x f_1$ to the
data $(-1,1), (0,0), (1,2)$ by solving
$\left(
\begin{array}{cc}
1 & -1 \\
1 & 0 \\
1 & 1 
\end{array}
\right)
(f_0, f_1)^T= (1,0,2)^T
$.  In general, the least squares approximation arises from the symmetric 
projection in the sample space ${\mathbb R}^s$ where $s$ is the number of data points.
$f(A) {\vec v}= (f_0 + f_1 A + \ldots + f_n A^n) {\vec v}=
( {\vec v}, A{\vec v}, \ldots, A^n{\vec v})$.  Vandermonde determinant and Fourier
$V(x_0 , x_1 , \ldots , x_n)$ where the $x_i$ are roots of $x^{n+1}-1=0$.
$Z(x)= (x-x_j) Z_j (x)$ solves for coefficients $f_0 , f_1 , \ldots , f_n$ using
Lagrange interpolants $\Lambda_j(x) = {\frac {Z_j(x)} {Z_j(x_j)}}$.  For PCA,
$\mu_A(x)= 
(x-\lambda_1)^{m_1}
(x-\lambda_2)^{m_2}
\ldots
(x-\lambda_t)^{m_t}
$.  There are polynomials in $A$, denoted $ A_{\lambda_1} A_{\lambda_2} \ldots A_{\lambda_t} $
such that $(A-\lambda_i I)^m A_{\lambda_i} =0$ and
$A= A_{\lambda_1} A_{\lambda_2} \ldots A_{\lambda_t}$.  The $A_{\lambda_i}$ are called components.
The list of basic eigenvectors of $A$ form the columns of the diagonalizing matrix, $P$ and
$AP=PD$; $A$ is diagonalizable when $P$ is invertible.
Approximating a rank $r$ $n \times n$ matrix requires $2nr$ terms.  \emph{Mean clustering:}
replace $M$ with $D= diag(\alpha_1, \ldots, \alpha_i)$ where
$\alpha_i= {\sqrt {\frac 1 {(M M^T)_{ii}}}}$.   How closely can a scatterplot be approximated
by a line $A$ with direction ${\vec a}$?  Find the vector ${\vec a}$ that maximizes
$
|P_{\vec a} ({\vec {m_1}})|^2 +
|P_{\vec a} ({\vec {m_2}})|^2 +
|P_{\vec a} ({\vec {m_s}})|^2 =
({\vec a}^T {\vec m_1})^2 + ({\vec a}^T {\vec m_2})^2 +
\ldots ({\vec a}^T {\vec m_t})^2$.  Maximize ${\vec a}^T M M^T {\vec a},
\forall {\vec a} \in {\mathbb R}^s, |{\vec a}|=1$.  $C=M M^T$ is a correlation matrix
with
$c_{ij}$ is the correlation of $i, j$; if $u_i \perp u_j$ they are uncorrelated.
$\exists P: CP=PD$, $M M^T=C= PDP^{-1}$ and maximize ${\vec u}^T D {\vec u}, |{\vec u}|=1$,
${\vec u}= P^T {\vec a} \in {\mathbb R}^s$.  $C$ is diagonalized by $P: P P^T=I$.
\subsection {Bilinear Forms and Classical Groups}
{\bf Definition:}
A \emph{pairing}, $(W,V) \rightarrow k$ is a bilinear map.  If $V_0 \subset V$, 
$V_0^* = \{{\vec w} \in W: ({\vec w}, {\vec {v_0}})= 0, \forall {\vec {v_0}} \in V_0 \}$,
$v_0 \subset (V_0^*)^*$.  $V^*$ is called the left kernel. 
Same holds {\it mutatis mutandis}
for $W_0 \subseteq W$ provided $W^* = 0$ is the right kernel.
If $(W,V) \rightarrow k$ is a pairing with left kernel $0$ and
${\vec {w}} \in W$, define $\varphi_{\vec w} ({\vec v}) = ({\vec w}, {\vec v})$.  
$\varphi_{\vec w} \in {\hat V}$ and the map
${\vec w} \mapsto \varphi_{\vec w}$ 
is an injection from $W \rightarrow {\hat {V}}$.
Similarly,
if the right kernel is $0$, there is an injection $V \rightarrow {\hat W}$.
\\
\\
If $W_0 \subseteq W$, $codim_W(W_0)= dim(W)-dim(W_0)$.
If $W_0 \subset W$, $V_0 \subset V$ and $V^*=0$,
there are natural injective morphisms
$V/W_0^* \rightarrow {\hat {W_0}}$ and $V_0^* \rightarrow {\hat {V/V_0}}$. 
Thus, $dim(V/W_0^*) \le dim({\hat {W_0}})=dim(W_0)$ and 
$dim(W_0^{**}) \le codim(W_0^*) \le dim(W_0)$.
If $W= {\hat V}$, both kernels are $0$.
If $(W,V)$ is a pairing, (a) $dim(W/V^*) = dim(V/W^*)$, (b) if $V^* = 0$,
$dim(W_0^{**}) = codim(W_0^*) = dim(W_0)$ and if $W_0$ is
finite dimensional, $W_0^{**}= W_0$ and $W_0$ and $V/W_0^*$ are
naturally dual, (c) If $V^*=0$ and $W^*=0$, and $V$ and $W$ are
finite dimensional, $V$ and $W$ are naturally dual and there is a 1-1,
inclusion reversing correspondence of subgroups of
$V$ and $W$ under the $*$ operator: $W_0 \leftrightarrow W_0^*$.
\\
\\
Let $A=(a_{ij})$ be an $m \times n$ matrix with entries in $k$ and
${\vec x}= (x_1, \ldots , x_n)^T$.  Let
${\vec b}=(b_1 , \ldots , b_m)^T$ then $A {\vec x}= {\vec b}$ is a system
of linear equations.  Set
$x=E_1 x_1 + E_2 x_2 + \ldots + E_n x_n, E_i \in V= k^n$.  
Suppose ${\hat V}$ is dual to $V$ with basis
$\varphi_1 , \ldots , \varphi_n$: $\varphi_j E_k= \delta_{jk}$.  Let
$\psi_i (x) = (a_{i1} \varphi_1 + \ldots + a_{in} \varphi_n) (E_1 x_1 + \ldots + E_n x_n)=
a_{i1}x_1 + \ldots + a_{in}x_n$.  $W \subset {\hat V}$, $W = \langle \psi_j(x) \rangle$
and $dim(W)=$ row rank. $S_m$ is the $m-$tuple column vectors
with entries in $k$.
Note that if $A_i$ are column vectors 
forming $A$, they are in the column space of $A$ as is
${\vec b}$  and $A_1 x_1 + \ldots + A_n x_n= {\vec b}, 
{\vec b}= (\psi_1(x), \ldots, \psi_m (x))$.
If $f: V \rightarrow S_m, f(x)= (\psi_1x , \ldots , \psi_n x)$, 
$ker(f) = W^*$. If $Im(f)=U$, $U \cong V/W^*$ and $dim(U)=codim(W^*)=dim(W)$.
This shows
the row rank equals the column rank.
\\
\\
Let $B_{ij}(\lambda) = I + \lambda (\delta_{il} \delta_{jk})_{lk}$.  If
$A \in GL_n(k), A=BD(\lambda)$ where $B \in SL_n(k)$ and $D(\lambda)$ is the same as
the identity except for $\lambda$ in the lower rightmost position.
Put $Z=Z(k), S= \langle x^2, x \in k \rangle$ (as an additive group).  
If $x^2 \in Z, \forall x$  then
$k$ is commutative; further, unless $k$ is commutative and $char(k)=2$, $S=k$.
\\
\\
{\bf Definition:} $\tau \in GL_n(k)$ is a \emph{transvection} if 
$\exists H= \{h: \varphi(h) =0, \varphi \in {\hat V} \}$ with $\tau(h) =h, h \in H$ and
$\tau(x)-x \in H, \forall x \in V$.
If $\tau$ is a transvection with hyperplane
$H$, pick ${\vec b}: \varphi({\vec b})= a \ne 0$, set $\tau(x) = x-{\vec b} a^{-1} \varphi(x)$
then $\tau (t(x))=t(x)$, thus 
$\tau(x)= x+ {\vec a} \varphi(x)$ with ${\vec a}= \tau({\vec b} a^{-1})-{\vec b} a^{-1}$.
So all transvections are of this form.
\\
\\
{\bf Theorem:}
$B_{ij}(\lambda)$ is a transvection.
If $ {\vec a}, {\vec b} \in H$ then
$\tau_{\vec a} (\tau_{\vec b}(x))= \tau_{{\vec a}+{\vec b}} (x)$.  
If $\sigma \in GL_n(k)$ and $\tau$ is a transvection, so is $\tau' = \sigma \tau \sigma^{-1}$
and $\tau'(x)= x+ (\sigma(A)) \varphi(\sigma^{-1}(x))$; conversely, if
$\tau''(x)=x+{\vec {a'}} \psi(x)$ is another transvection with hyperplane $H'$,
we show $\exists \sigma$: $\sigma(H)= H'$ and $\sigma({\vec a}) = a'$ and thus that
all transvections are conjugate and hence have the same determinant.
\begin{quote}
\emph{Proof:} Pick ${\vec b}, {\vec {b'}}$ with $\varphi({\vec b})=\psi({\vec {b'}})=1$.
$\exists \sigma: \sigma({\vec a})= {\vec {a'}}, \sigma(H)= H', \sigma({\vec b})= {\vec {b'}}$.
Then $\tau''(x)=x + {\vec {a'}} \varphi(\sigma^{-1}(x))$, 
$\exists c: \phi(x)=\varphi(\sigma^{-1}(x))$, setting $x= {\vec {b'}}, \sigma^{-1}(x)={\vec b}$
we get $c=1$ and $\tau''= \tau'$. If $H$ has at least three vectors then 
$\exists {\vec a}, {\vec b}, {\vec c}$ with
${\vec c}= {\vec a} + {\vec b}$ and
$\tau_{\vec a} (\tau_{\vec b}(x))= \tau_{\vec c} (x)$ and since they 
all have the same determinant, it must be 1.
In that case, $f:GL_n(k) \rightarrow GL_n(k)/GL_n(k)'$, $f(\sigma \tau \sigma^{-1})= f(\tau)$
so all transvections have the same image under $f$ and $\tau \in GL_n(k)'=SL_n(k)$.
If $n \ge 3$ $H$ and $H'$ have independent vectors and we can choose $\sigma: det(\sigma)=1$
so the transvections are conjugate in $SL_n(k)$.  Finally, the center of $SL_n(k)$
consists of the matrices $\alpha I$ with $\alpha^n=1$.  We can conclude:
If $G$ is a normal subgroup of $GL_n(k)$ containing a transvection and
$n \ge 3$ or $n=2$ and $|k| \ge 4$ then $SL_n(k) \subseteq G$ if $G>Z(GL_n(k))$.
\end{quote}
{\bf Theorem:} If $v, w$ are linearly independent, there is a transvection, $T$: $Tv = w$.
\begin{quote}
\emph{Proof:} $W = \{x: x \cdot (v-w) = 0 \}$, $x, y \notin $. $T_{|W|} = 1$, $T(v)=w$.
\end{quote}
{\bf Theorem:} If $W_1, W_2$ are hyperplanes in $V$, and $v \in V \setminus W_1 \cup W_2$,
there is a transvection, $T$: $T(W_1)= W_2, T(v) = v$.
\begin{quote}
\emph{Proof:} $V = W_1 + W_2$.  $dim(W_1 \cap W_2)= n-2$.  $\exists x \in W_1, y \in W_2: v = x + y$.
$V= W_1 \cap W_2 + Fx + Fy$.  Define $T_{W_1 \cap W_2 + F(x + y)} = 1$ and $T(x) = y$.
\end{quote}
{\bf Theorem:} All transvections are $GL(V)$ conjugate.
If $T_1, T_2$ are transvections on $V$, $n=dim(V) \ge 3$ and $T_1$ and $T_2$ are
$GL(V)$ conjugate then they are $SL(V)$ conjugate.  
\begin{quote}
\emph{Proof:} Transvections are all of the form $T = 1 + \lambda B_{mn}$ with
$B_{mn} = (\delta_{mi} \delta_{nj})$. $B_{lm}$ is conjugate to $B_{kn}$ via permutation
matricies.  Suppose, for example that $T= 1 + \lambda_1 B_{mn} =
\left(
\begin{array}{cc}
1 & \lambda_1 \\
0 & 1
\end{array}
\right)$.
Put $S = \left(
\begin{array}{cc}
{\frac 1 {\lambda_1}} & -1 \\
0 & {\frac 1 {\lambda_2}}
\end{array}
\right)$, then 
$S^{-1} T S =
\left(
\begin{array}{cc}
1 & \lambda_2 \\
0 & 1
\end{array}
\right)$.  So all transvections are conjugate in $GL(V)$.  If $n \geq 3$, we can pick a diagonal
element so that $det(S) = 1$ and preserve the conjugacy.
\end{quote}
{\bf Theorem:} If $dim(V) \ge 3$, the transvections on $V$ generate $SL(V)$.  $SL(V)'=SL(V)$
and $PSL(V)'=PSL(V)$.
\begin{quote}
\emph{Proof:} It suffices to show there is a (non-trivial) transvection in $SL(V)'$ since they are
all conjugate.
Define $T_1(v_1) = v_1 - v_2, T_1(v_j) = v_j, j \ne 1$ and
$T_2(v_2) = v_2 - v_3, T_1(v_j) = v_j, j \ne 2$. $T_1 T_2 T_1^{-1}T_2^{-1}$ is a transvection.
\end{quote}
{\bf Theorem:} $PSL(V)$ is $2$-transitive on $P_{n-1}(V)$.
\begin{quote}
\emph{Proof:} Let $[v_1] \ne [v_2], [w_1] \ne [w_2] \in P_{n-1}(V)$.
We can choose basis
$\langle v_1, v_2, v_3, \ldots , v_n \rangle$ and
$\langle w_1, w_2, v_3, \ldots , v_n \rangle$.  Define $T_b$ by
$T_b(v_1) = b w_1$
$T_b(v_2) = w_2$, and
$T_b(v_i) = v_i, i \geq 3$.  Finally, pick $b$: $det(T_b) = 1$.
\end{quote}
{\bf Pairings and isometries:}  Let $V \times V \rightarrow k$ be a pairing with trivial
left and right kernels.   $\sigma$ is an \emph{isometry} if 
$(x,y)=(\sigma x, \sigma y), \forall x,y \in V$.  $det(\sigma)^2=1$ for all isometries;
if $det(\sigma)= 1$, $\sigma$ is a rotation,
if $det(\sigma)= -1$, $\sigma$ is a reflection.  A quadratic map, $Q$ satisfies
$Q(ax)=a^2 Q(x)$ and $(x,y)= Q(x+y)-Q(x)-Q(y)=(y,x)$ is a pairing.  If $char(F) \ne 2$,
$Q(x)= {\frac 1 2}  (x,x)$.  Pairings arising from quadratic maps are symmetric.
${\vec a} \perp {\vec b} \leftrightarrow ({\vec a}, {\vec b})=0$.  
If $\langle v_1 , v_2 , \ldots , v_n \rangle$ span $V$ and 
$({\vec {v_i}}, {\vec {v_j}})= g_{ij}$ and
if $\langle u_1 , u_2 , \ldots , u_n \rangle$ is another basis related to the original by
$u_i = \sum_j a_{ji} v_j$ then ${\overline {g_{ij}}} =A^T G A$, where $G=(g_{ij})$.
The form is symmetric if $a_{ij}= a_{ji}$, antisymmetric if $a_{ij}= - a_{ji}$.
\\
\\
{\bf Isotropic spaces}
Let $V^* = rad(V)= V \cap V^{\perp}$ and
$V= rad(V) \oplus U$, $U \cong V/rad(V)$.
Suppose $V$ is non-singular and $U \subset V$ then $U^{**}=U, dim(U) + dim(U^*) = dim(V)$ and
$rad(U) =rad(U^*)= U \cap U^*$.  The subspace $U$ is non-singular iff $U^*$ is non-singular
and then $V= U \perp U^*$.  A vector ${\vec v}$
is isotropic if $({\vec v}, {\vec v})=0$.  
$U$ is isotropic if $(u_1 , u_2)= 0, \forall u_1, u_2 \in U$.
There are two geometries for symmetric metric
spaces: (1) \emph{symplectic} if $({\vec v}, {\vec v}) = 0, \forall {\vec v} \in V$ and
$(x,y)= -(y,x)$; (2) \emph{orthogonal} if $(x,y)=(y,x), \forall x,y \in V$.
If $V$ is orthogonal and every vector is \emph{isotropic} then $V$ is isotropic.
\\
\\
{\bf Definitions:}
Suppose $dim(V)=2$ and $V$ is non-singular but has an isotropic vector, ${\vec n}$ then
$\exists {\vec m}: {\vec n}^2={\vec m}^2=0, {\vec n}{\vec m}=1$, 
$V= \langle {\vec n}, {\vec m} \rangle$. (
$V= \langle {\vec n} , {\vec a} \rangle$ for some 
${\vec a}$. Set ${\vec m}= x {\vec n} + y{\vec a}$; if
${\vec n} {\vec a}=0$, $V$ is singular so we can find $y: y{\vec n} {\vec a} = 1$.  
Can also
find $x: {\vec m}^2=0$.)  $\langle {\vec n}, {\vec m} \rangle$ is a 
\emph{hyperbolic plane}.
A non-singular space, $V$, with orthogonal geometry is an orthogonal sum of lines.
A non-singular space, $V$, with symplectic geometry is an orthogonal sum of 
hyperbolic planes.
\\
\\
{\bf Witt's Theorem:} Let $V$ and $W$ 
be isometric via $\rho$.  Let $\sigma: V_0 \rightarrow W_0$ be
an isometry for $V_0 \subset V$ and $W_0 \subset W$, then $\sigma$ can 
be extended to an isometry of $V$. $O_n$: isometries.  
\begin{quote}
\emph{Proof:}
\end{quote}
{\bf Definitions:} $O_n^+$: rotations,
$O_n^-$: reflections.
$\Omega_n= O_n'$.
If $V$ is a vector space with over ${\mathbb {R}}$ with a positive definite form 
(resp. ${\mathbb {C}}$ with a 
hermitian form) and $W$ is a subspace of $V$ then $V=W \oplus W^{\perp}$.
$V^* \otimes V \rightarrow {\cal L}(V,V)$ via $L_{\phi \otimes v } (w)= \phi(w)v$.
If $n$ is odd, $1_V = Z(O_n^+)$.  
If $n$ is even, $\pm 1_V =Z(O_n^+)$. If $n=2$ over $F_q$, the plane contains
$q+1$ lines: $\langle A+xB \rangle, \langle B \rangle$; if $V$ is isotropic, 
$\epsilon= 1$, otherwise
$V$ contains no isotropic vectors and $\epsilon = -1$.  There are $q - \epsilon$
non-isotropic lines.  $O(V)$ has $q - \epsilon$ elements.  
Let $\varphi_n$ be the number of
isotropic vectors in $V$ and $\lambda_n$ the
number of hyperbolic pairs.  If $\langle N,M \rangle$ is a hyperbolic plane, 
$\langle N,M \rangle \oplus \langle N,M \rangle^* = V$.
$\langle N^* \rangle$ contains $q \varphi_{n-2}$ isotropic vectors.  
A type I form: TBD.
Type I, II form:
$\varphi_{n}=q^{n-1}$.
Type III, IV form: $\varphi_n= q^{n-1} + c q^{\frac n 2}$, $n \ge 1$.
If $\Phi_n = |O_n^+(q)|$ or $|PSp_n(q)|$, $\Phi_n= \lambda_n \Phi_{n-2}$.
\\
\\
{\bf Classical Groups Summary.}
{\bf Theorem:}
Every isometry in ${\mathbb R}^n$ is the product of $\leq n+1$ reflections.
\begin{quote}
\emph{Proof:}
It suffices to show every isometry, $f$ with $f(0)=0$ is the product of at most $n$ reflections.
This is true for $n= 1,2$,  suppose it is true for $n-1$.
$||f(e_n)-f(0)||= ||f(e_n)||= ||e_n||= 1$.  There is a reflection, $R$ such that $g=Rf$ fixes 
$0$ and $e_n$ and hence $L= \{ t e_n \}$.  Put $W= \{ (x_1, \ldots, x_{n-1},0), x_i \in {\mathbb R} \}$.
$W \perp L$ and $g$ fixes scalar products.  If $g(x)=y$, $\pi_n(x)= \pi_n(y)$.  Regard $g$ as
a map on ${\mathbb R}^{n-1}$.  $g^*(x_1, \ldots, x_{n-1})= (y_1, \ldots , y_{n-1})$.  By induction,
$g^*= R_1^* \ldots R_{n-1}^*$ with $R_j^*(x)= x- 2(x, a_j^*) a_j^*, ||a_j^*||=1$.  Put
$R_j(x)= x- 2(x, a_j) a_j$, $a_n=0, a_j^*=a_j, j<n$.  $f= R R_1 \ldots R_{n-1}$.
\end{quote}
If $G$ is one of $SL(V)$, $Sp(V)$, $SO(V)$ or $S\Omega(V)$, $G=BWB$, where
$B$ is the \emph{Borel subgroup} (upper triangular matrices) and $W$ is the 
\emph{Weyl subgroup}
(the permutation matrices).
\subsection{Fields}
{\bf Theorem:}
If $\alpha$ is the root of an irreducible polynomial $p(x) \in F[x]$ then 
$F(\alpha)=F[\alpha]= F[x]/(p(x))$ (This is called a
\emph{field extension}).  Isomorphisms between fields can be extended
to isomorphisms of extensions over associated (under the isomorphism) polynomials.
\begin{quote}
\emph{Proof:} Suppose $\sigma: F \rightarrow K$ then the natural extension of $\sigma$ 
to $F[x]$ gives $\sigma: F[x] \rightarrow K[x]$.  If $p(x)$ is irreducible, this
isomorphism can be extended uniquely to $F[x]/(p(x)) \rightarrow K[x]/(p^{\sigma}(x))$.
\end{quote}
{\bf Theorem:} 
Any two splitting fields of the same polynomial over $F$ are isomorphic.  
\begin{quote}
\emph{Proof:}
Let $\alpha$ and $\beta$ be two roots of and irreducible polynomial which divides
a $f(x)$; let $E$ be the splitting field of $f(x)$.  There is an isomorphism from
$F(\alpha)$ into $F(\beta)$ which can be extended to an automorphism of $E$.  
\end{quote}
{\bf Definitions:}  Let $E$ be a field and $G$ be a set of automorphisms of $E$, 
$E_G = \{ x \in E: \varphi(x)=x, \forall \varphi \in G \}$.  Note that $E_G$ is a field.
A polynomial is \emph{separable} if the roots of every irreducible factor are distinct.
An extension $E/F$ is \emph{separable} if every element of $E$ is the root of a separable polynomial
in $F[x]$.
$E$ is a \emph{Galois} over $F$ if $E_G=F$.  
$E$ is \emph{normal} over $F$ if an irreducible
polynomial over $F$ with one root in $E$, \emph{splits}.
\\
\\
{\bf Artin's Lemma:} Distinct automorphisms are linearly independent.
\begin{quote}
\emph{Proof:}  Suppose not.  Let $c_1 \phi_1 (x) + c_2 \phi_2 (x) + \ldots + c_r \phi_r (x) =0$
be a minimal relation.  Since the automorphisms are distinct,
$\exists \beta: \phi_1 (\beta) \ne \phi_r (\beta)$.  Obtain two equations from the
minimal relation, the first
by substituting $\beta x$ into the equation for beta, the second by multiplying the
equation by $\phi_r(\beta)$, then subtract them.  This is a shorter relation.
\end{quote}
{\bf Theorem:} 
If $G$ is a finite set of automorphisms fixing $F$, then $r=|E:F| \geq |G|=n$.
\begin{quote}
\emph{Proof:}  Suppose not. Let $\{ \omega_1, \ldots, \omega_r\}$
be a basis for $E$ over $F$. Consider the $r$ equations:
$\phi_1 (\omega_k) x_1 + \ldots + \phi_n(\omega_k) x_n = 0$ for $k= 1,2,\ldots r$.
Since $n>r$ there is a non trivial solution $c_1, c_2, \ldots , c_n$.  Let
$x= \sum_{i=1}^r a_i \omega_i$.  Multiply the first equation by $a_1$, the second by
$a_2$ and so on then add them to get
$c_{1} \phi_{1} (x) + c_2 \phi_2 (x) +  \ldots + c_n \phi_n (x)  = 0$
for all $x$.  This contradicts the Artin's result.
\end{quote}
{\bf Theorem:} 
Let $G= \{\phi_1 , \phi_2 , \ldots, \phi_n \}$ be a finite group of
$Aut(E)$ and $F = E_G$, then $r = [E:F] = |G| = n$.
\begin{quote}
\emph{Proof:}  Suppose $r>n$. Let $\{ \omega_1, \ldots, \omega_r\}$
be a basis for $E$ over $F$. Consider the $n$ equations:
$\phi_k (\omega_1) x_1 + \ldots + \phi_k(\omega_r) x_r = 0$ for $k= 1,2,\ldots, n$.
This has a non trivial solution, $\langle c_1, c_2, \ldots , c_r \rangle$,
with $r-n$ more unknowns than equations.  
Let $\langle c_1, \ldots , c_r \rangle$ be a solution with the minimum number of non-zero
elements. We may reorder the coefficients and basis so $c_1 \neq 0$ and by dividing each of
the linear equations by $c_1$, we may assume $c_1 = 1$.  We claim each $c_i \in F, \forall i$.
If not, say $c_2 \notin F$.  Then $\exists \phi_k: \phi_k(c_2) \neq c_2$.  Thus,
$\sum_{i=1}^r c_i \phi_j(\omega_i) = 0$ and 
$\sum_{i=1}^r \phi_k(c_i \phi_j(\omega_i)) = 0$ for each $ 1 \leq j \leq n$.  Again,
reordering, 
$\sum_{i=1}^r \phi_k(c_i) \phi_j(\omega_i) = 0$ for each $ 1 \leq j \leq n$.  Subtracting
the two equations, we get a non-trivial linear relation with fewer non-zero coefficients.
So, $c_i \in F, \forall i, 1 \leq i \leq r$.  This gives a non-trivial linear dependence 
in $F$ among the $\omega_i$ which contradicts their linear independence, so
$r \leq  n$.  Now $r \geq n$ by the previous result so $n=r$.
\end{quote}
{\bf Primitive Element Theorem: }
If $E=F[\alpha_1 , \ldots , \alpha_n]$ with $\alpha_2 , \ldots , \alpha_n$ separable then
$E=F[\alpha]$, some $\alpha$.  Every separable finite extension is primitive.
\begin{quote}
\emph{Proof:}  Assume $F$ is not finite, $E=F[\alpha, \beta]$ with $f, g$ the minimal polynomials
for $\alpha= \alpha_1$ and 
$\beta=\beta_1$ respectively, $\alpha_i$  the roots of $f$ and $\beta_i$ the roots of
$g$.  Let $E$ be the splitting field of $f(x)g(x)$.  
$\alpha_i + x \beta_k= \alpha_1 + x \beta_1$ 
has one root for each $i, k$; pick $c$ such that
$\alpha_i + c \beta_k \ne \alpha_1 + c \beta_1$ and set $\theta= \alpha + c \beta$.
Claim: $E=F[\theta]$. $f(\theta-c \beta)=g(\beta)=0$ so
$(f(\theta - cx), g(x))= (x- \beta) \in F[\theta][x]$. 
\end{quote}
{\bf Theorem:} 
Let $E$ be a splitting field for $f(x)$ over $F[x]$.
If $p(x)$ is irreducible and has one zero in $E$, then $p(x)$ splits in
$E$.
\begin{quote}
\emph{Proof:} 
Let $L$ be the splitting field of $f(x)p(x)$.
Set $E= F(a_1,a_2, \ldots, a_n)$
where $a_1, a_2, \ldots , a_n$ are the roots of $f(x)$.
Suppose $p(\alpha)=0, \alpha \in E$ and $p(\beta)= 0$.
Let $\sigma: F(\alpha) \rightarrow F(\beta)$ be an isomorphism with
$\sigma(\alpha)=\beta$.
Extend $\sigma$ to $\tau: L \rightarrow L$.  $\tau$
permutes the roots of $f(x)$ so $\tau(E)= E$.
$\alpha= {\frac {m(a_1, a_2, \ldots , a_n)} {n(a_1, a_2, \ldots , a_n)}}$.
So $\beta= \tau(\alpha)= \tau({\frac {m(a_1, a_2, \ldots , a_n)}
{n(a_1, a_2, \ldots , a_n)}}) \in E$.
\end{quote}
{\bf Theorem:} 
Let $E$ be a finite extension of $F$, $char(F)=0$.  If $E$ is a splitting
field of $f(x) \in F[x]$
then $|{\cal G}(E/F)| = [E:F]$.
\begin{quote}
\emph{Proof:} $E=F(w)$, $p(w)=0$ and $p$ splits by foregoing. $deg(p)=[E:F]=|G|$.
\end{quote}
{\bf Theorem:} 
Let $F \subseteq E$, $char(F)=0$.  If $G={\cal G}(E/F)$ fixes $F$ then $E$ is
a normal extension iff $F$ is the fixed field of $G$.
\begin{quote}
\emph{Proof:} $E=F(w)$, $|G|=[E:F]$.  Let $K= \{ a: \sigma(a)=a, \forall \sigma \in G\}$.
$F \subseteq K \subseteq E$ and $E=K(w)$.  STS if $g$ is irreducible
over $F$ and $g(w)=0$ then $g$
is irreducible over $K$.  Let $p$ be an irreducible polynomial for $w$ over $K$.  Applying
elements of $G$, each root of $p$ is a root of $g$.
\end{quote}
{\bf Theorem:} 
Let $E$ be a normal extension of $F$. $E \supset K \supset F$.  If
${\cal G}(E/F)>S$ has $K$ as a fixed field then ${\cal G}(E/K)=S$.
\begin{quote}
\emph{Proof:} 
Suppose $S \subseteq {\cal G}(E/K)=T$.  By a previous result $|T|=|S|$ so $S=T$.
\end{quote}
{\bf Lemma:} Let $K$ be the splitting field of $f(x)$ over $k$
and let $p(x)$ an irreducible
factor of $f(x)$, if the roots of $p(x)$ are $\alpha_1 , \ldots , \alpha_r$,
there is a $\sigma_i \in {\cal G}(K/k)$ such that $\sigma_i(\alpha_1)= \alpha_i$.
\begin{quote}
\emph{Proof:} 
This follows from the isomorphism $F[x]/(f(x)) \rightarrow F(\alpha )$ and
$F[x]/(f(x)) \rightarrow F(\alpha_i )$.
\end{quote}
{\bf Theorem:} 
$E$ is Galois over $F$ iff (i) every irreducible polynomial in $F[x]$ with one root in $E$
splits and (ii) $E=F(\theta)$.
$GF(p^m) \subseteq GF(p^n)$ iff $m|n$.
\begin{quote}
\emph{Proof:} 
Let $\theta= \theta_1$ and $\varphi= \varphi_1$ is another root.  Let $f \in F[x]$ 
be irreducible with root $\theta$.  By the lemma, 
there is a $g \in {\cal G}(E/F)$.  $f^g(x)=f(x)$. (ii) follows from this.
$\theta^g=\varphi$ and so $\varphi$ is thus a root of $f$.
\end{quote}
{\bf Theorem:}
The following are equivalent: (1) $E$ is a splitting field over $F$
of a separable polynomial $f(x)$.  (2) $F=E_G$. (3) $E$ is finite
dimensional, normal and separable.  Moreover, if $E$ and $F$ are as in (1) and
$G= {\cal G}(E/F)$ then $F=E_G$ and if $F$ and $G$ are as in (2) then $G={\cal G}(E/F)$.
\begin{quote}
\emph{Proof of
($1 \rightarrow 2$):}
$G={\cal G}(E/F)$ and $F'=E_G$. $F \subseteq F' \subseteq E$.  $E$ is a splitting field
over $F'$ of $f(x)$ as well as over $F$ and $G={\cal G}(E/F')$. $[E:F]= |G| = [E:F']$ so
$F=F'$ and $F=E_G$
\\
\\
\emph{Proof of
($2 \rightarrow 3$):}
By Artin $[E:F] \le |G|$ so $E$ is finite dimensional over $F$.  Let $f(x) \in F[x]$
having root $r \in E$ be irreducible.  Let $\langle r=r_1, \ldots, r_m \rangle$
be an orbit of $r$ under $G$.  For $\eta \in G$, $(\eta(r_1) , \ldots , \eta(r_m))$ is
a permutation of $(r_1 , r_2, \ldots , r_m)$.  $f(r_i)=0, 1 \leq i \leq m$ and
$(x-r_i) \mid f(x)$ so $g(x)= \prod_{i=1}^m (x-r_i) \mid f(x)$.  Apply to $g(x)$
the automorphism of $E[x]$ which sends $x \mapsto x, a \mapsto \eta(a)$ for $a \in E$.
This gives $\eta(g(x))= \prod_{i=1}^m (x-\eta(r_i))= g(x)$.  Since this holds for every
$\eta \in G$, the coefficients of $g$ are $G-$invariant hence $g(x) \in F[x]$.  Since
$g(x)$ is irreducible in $F[x]$, $f(x)=g(x)= \prod_{i=1}^n (x-r_i)$ a product of linear
factors in $E[x]$.  Thus $E$ is separable and normal over $F$ and (3) holds.
\\
\\
\emph{Proof of
($3 \rightarrow 1$):}
Since $[E:F] < \infty$ so $E= F(r_1, r_2, \ldots, r_k)$ and $r_i$ is algebraic over $F$.
Let $f_i$ be the minimal polynomial for $r_i$.  By hypothesis, $f_i(x)$ is a product of
linear factors in $E[x]$.   It follows that $f(x)= \prod_{i=1}^n f_i(x)$ is separable
and $E= F(r_1, r_2, \ldots, r_k)$ is a splitting field for $E$ over $F$ and (1)
follows.
\\
\\
\emph{Proof of supplement:}
To prove the second part of the supplement, under the hypothesis of part (2) of the supplement,$[E:F] \leq G$
and since (3) holds, ${\cal G}(E/F)=[E:F]$.  Since
$G \subseteq {\cal G}(E/F)$ and 
$|G| \geq [E:F]= |{\cal G}(E/F)|$, $G= {\cal G}(E/F)$.
\end{quote}
{\bf Galois' Theorem:} Let $K$ be a normal, separable extension of $k$.  Let $G={\cal G}(K/k)$,
$H<G$, $K \supseteq F \supseteq k$.  There is a bijective pairing between $H$ and $F$, such
that
(i) $H_{1} \supseteq H_{2} \leftrightarrow K_{H_{2}} \supseteq K_{H_{1}}$;
(ii) $|H|= [K:K_H]$, $[G:H]= [K_H:k]$; and,
(iii) $H \lhd G \leftrightarrow K_H$ is normal over $k$
and ${\cal G}(K_H/k)=G/H$.
\begin{quote}
\emph{Proof:} 
\\
\\
Let  $H<G={\cal G}(K/k)$.  $k=K_G$.
Put $F = K_H$.  $k \subseteq F \subseteq K$.  By the previous result,
$|{\cal G}(K/K_H)|=|H| = [K:K_H]$.  
Applying supplementary result (2) above with $H$ in place of $G$, we get
${\cal G}(K/K_H)= H$.
Similarly,  $|H|=|{\cal G}(K/K_H)|=[K:K_H]$.
\\
\\
Now, let $F$ be any intermediate subfield between $K$ and $k$ and $H= {\cal G}(K/F)$.
$H \subseteq G={\cal G}(K/K_H)$.
$K$ is a splitting fields over $F$ of
a separable polynomial since it is a splitting field over $k$ of a separable polynomial.
The supplementary result of (1) above applied to $K$ and $F$ shows
$F=K_H= K_{{\cal G}(K/F)}$.
Thus the map between $F$ and $K_H$ are inverses.
\\
\\
If $H_1 \supseteq H_2$ then 
$K_{H_1} \subseteq K_{H_2}$.
Moreover, if $K_{H_1} \subseteq K_{H_2}$ then we also have
$H_1 = {\cal G}(K/K_{H_1}) \supseteq {\cal G}(K/K_{H_2}) = H_2$. Hence (i)
holds. 
\\
\\
The first part of (ii) follows as before.
$|G|= [K:k] = [K:K_H][K_H:k]= |H| [K_H:k]$ and $|G|= [G:H] |H|$ so
$[K_H:k]= [G:H]$.  This proves (ii).
\\
\\
If $H < G$ and $F= K_H$, the subfield corresponding 
to $\eta H \eta^{-1}$ is $\eta(F)$.
$\eta H \eta^{-1} \eta F= \eta(F)$,
so $H \lhd G$ iff $\eta(F)=F, \forall \eta \in G$.  If this holds, every
$\eta$, maps
$F$ to itself and $\eta_{|F}= {\overline {\eta}}$ is an automorphism of $F/k$.
Thus the restriction $\eta \rightarrow {\overline {\eta}}$ of
${\cal G}(K/k)$ into ${\cal G}(F/k)$ is a homomorphism.  
The image ${\overline G}$ is a group of automorphisms
of $F$ and $K_{{\overline G}}=k$.
Hence ${\overline G}= {\cal G}(F/k)$.
The kernel of the map 
$\eta \rightarrow {\overline {\eta}}$ is the set $\{ \eta \in G: \eta_{|F}=1_{F} \}$;
by the pairing, ${\cal G}(K/F)= H$.  This kernel is $H$ and
${\overline G}= {\cal G}(F/k) \approx G/H$.  Since $k= K_{\overline G}$,
$F$ is normal over $k$.
\\
\\
Conversely, suppose $F$ is normal over $k$.  Let $a \in F$ and let $f(x)$ be the
minimal polynomial for $a$ over $k$ then $f(x)= (x-a_1) (x-a_2) \ldots (x-a_n) \in k[x]$
where $a=a_1$.  If $\eta \in G$ then $f(\eta(a))=0$ which implies $\eta(a)= a_i$
for some $i$ so $\eta(a) \in F$.  
Therefore, $\eta(F) \subseteq F$.  $\eta H \eta^{-1} \subseteq H$ if $H$ is the subgroup
corresponding to $F$ in the Galois pairing. Thus $H \lhd G$, concluding the proof
of (iii).
\end{quote}
{\bf Theorem:}
If $f(x)$ is solvable by radicals, the Galois group of its splitting field
is \emph{solvable}.  Galois group of an equation is a permutation group on its roots.
Splitting field of $g(x)= 2x^5 -10 x+5$ is $S_5$.
\begin{quote}
\emph{Proof:} 
There is an element of order $5$ in $G$ since $g(x)$ is irreducible.  Complex conjugation
is an automorphism of order $2$.  These generate $S_5$.
\end{quote}
{\bf Computing the Galois group for an arbitrary polynomial:}
Suppose, 
$f(t) = t^n -s_1 t^{n-1} + \ldots + (-1)^{n}s_n$ over the field $k$.  
Let $K$ be the splitting field of $f$ over $k$ and $G = {\cal G}(K/k)$.
Assume
$f(t)$ has distinct zeros $\alpha_1, \ldots , \alpha_n$ and consider the
indeterminates $x_1, \ldots, x_n$.  The $s_k$ are elementary symmetric polynomials
in the $\alpha_k$.  Put $\beta = \sum_{j=1}^n x_j \alpha_j$.  Suppose
$\sigma \in S_n$.  We set $\sigma_x ( \beta ) = \sum_{j=1}^n x_{\sigma(j)} \alpha_j$
and $\sigma_{\alpha} ( \beta ) = \sum_{j=1}^n x_{j} \alpha_{\sigma(j)}$. Put
$q(t) = \prod_{\sigma \in S_n} (t - \sigma_x(\beta)) = 
\sum_{j=0}^{n!} g_i (s_1 , \ldots, s_n) x_1^{i_1} x_2^{i_2} \ldots x_n^{i_n} t^j$.
Now $q(t) = q_1(t) q_2(t) \ldots q_k(t)$ where each $q_k(t)$ is irreducible.
We can assume $(t - \beta) \mid q_1(t)$.
Put $G_1 = \{ \sigma \in S_n: q_1(t)^{\sigma} = q_1(t) \}$.  In fact, if we
put $S_{(1)} = \{ \sigma : (t - \sigma_x(\beta)) \mid q_1 (t) \}$, $S_{(1)} = G_1$.
Finally, define
$h(t) = \prod_{\sigma \in G} (t - \sigma_{\alpha}(\beta))$. $h(t) \mid q(t)$ so
$h(t)$ is the product of some of the irreducible factors of $q(t)$.
For each $q_k$, there is a subset $S_{(k)}$ of $S_n$ such that
$q_k(t) = \prod_{\sigma \in S_{(k)}} (t - \sigma_x(\beta))$ and $S_n$ is the disjoint union
of the $S_{(k)}$.
Since, $(t - \beta) \mid q_1(t)$, $q_1(t) \mid h(t)$.
\\
\\
{\bf Theorem:} In the above notation, $G_1 = G$.
\begin{quote}
\emph{Proof:} (1) $G_1 \subseteq G$ since $q_1(t) \mid h(t)$. (2) $G \subseteq G_1$:
If $\rho \in G$, $\rho(q_1(t)) = \prod_{\sigma \in S_1} (t- \rho(\sigma_x(\beta)))$ and
$\rho(q_1(t)) = \prod_{\sigma \in S_{(1)}} (t- \rho(\sigma_x(\beta))) = 
\prod_{\sigma \in S_{(1)}} (t- \rho_{\alpha}^{-1}(\sigma_{x}(\beta))) = \rho_{\alpha}^{-1}(q_1(t))$.
So, $\rho \in G_1$.
\end{quote}
{\bf Theorem:} Let $R$ be a UFD and $p$ a prime.  Set 
${\overline R}= R/(p)$ and let $Q_R$ and $Q_{\overline R}$ be their fields of
quotients.  Let $f(x)$ and ${\overline {f(x)}}$ be corresponding polynomials
with no double roots with corresponding splitting fields $K$ and ${\overline K}$
respectively.  Then 
${\cal G}({\overline K}/Q_{\overline R})< {\cal G}(K/Q_R)$.
\begin{quote}
\emph{Proof:}
\end{quote}
{\bf Definition:}
A \emph{valuation} is a map $\varphi: K \rightarrow {\mathbb F}^{\ge 0}$ 
where ${\mathbb F}$ is an
ordered field such that $\varphi(ab)= 
\varphi(a) \varphi(b)$, $\varphi(0)=0$, $\varphi(x)>0$ if $x \ne 0$ and 
$\varphi(a+b) \le
\varphi(a) + \varphi(b)$.  If $a= {\frac s t} p^n$, $\varphi(a)= p^{-n}$ is a valuation.
{\bf Ostowski:} A non trivial valuation of ${\mathbb Q}$ is either 
(i) $\varphi(a)= |a|^\rho, 0 < \rho \le 1$ (the Archemedean valuation) or (ii)
$\varphi(a)= \varphi_p(a)$ (the $p-$adic valuation.  $w(a)= log(\varphi(a))$ is the 
exponential valuation.  Set $\wp= \{a: w(a) >0 \}$.  Hensel:  Let $K$ be complete
in the exponential valuation $w$ and $f(x)$ a primitive polynomial in $K[x]$ with integral
coefficients. Let $g_0 , h_0$ be polynomials with integral coefficients such that
$f(x)= g_0(x) h_0(x) \; (\wp)$ then there are polynomials $f(x), h(x)$ with integral
coefficients in $K$ such that (1) $f(x)=g(x)h(x)$, 
(2) $g(x)= g_0(x) \; (\wp)$,
(3) $h(x)= h_0(x) \; (\wp)$ provided $(g_0(x), h_0(x))=1$ 
further $deg(g)= deg(g_0) \; (\wp)$.
\\
\\
{\bf Definition:}
$F$ is \emph{perfect} iff every irreducible polynomial is separable.
\\
\\
{\bf Theorem:} $F$ is perfect if
(1) $char(F)=0$, (2) $char(F)=p$ and every element is a $p$th root,
(3) $F= GF(q)$, (4) $F$ is algebraically closed, (5) every finite field is perfect.
\begin{quote}
\emph{Proof of $2$:}  Suppose $F^p$ is not contained in $F$.  Let $a \notin F^p$.
$x^p-a$ is irreducible.  Since the derivative is $0$, this is inseparable.  Hence
$F$ is not perfect.  Suppose $f(x)$ is an inseparable irreducible polynomial in
$F[x]$ then $(f,f') \ne 1$.  So $f(x)= a_0 +a_p x^p + \ldots + =
b_0 + b_p^px + b_{2p}^p x^{2p} + \ldots)^p$ contrary to irreducibility.
Hence $F \ne F^p$.
\end{quote}
{\bf Definition:}
Let $E=F[\theta]$ and $\rho= a_0 + a_1 \theta + \ldots + a_{n-1} \rho^{n-1}$. 
$T(\rho)= \sum_{g \in {\cal G}(E/F)} \rho^g$ is the \emph{trace} and
$N(\rho)= \prod_{g \in {\cal G}(E/F)} \rho^g$ is the \emph{norm}; both are in $F$.
\\
\\
{\bf Automorphisms of a finite field:}
For every $q=p^n$ there is, up to isomorphism, only one field $F=GF(q)$ and the
multiplicative group is cyclic. Consider $f(x)= x^h-1, h=q-1$ whose roots are
roots of 1.  The automorphisms of $F$ are exactly $\sigma_i : x \mapsto x^{p^i}$.
If $char(F)=p$, every irreducible polynomial $f(x)$ of degree $n$
either has distinct roots or is of
the form $\phi(x^p)$ in which case all roots have the same multiplicity $p^l$ for
some $l>0$ with $n=n' p^l$ in which case there are $n'$ relative automorphisms.  Thus
in successive extensions there are $\prod_i n_i'$ relative automorphisms which have
cardinality
$[E:F]$ if $E$ is a separable extension and $<[E:F]$ if not.
\\
\\
{\bf Theorem:}
If $G$ is solvable, $G^{(n)}=1$ for some $n$.  If $n>4$, $S_{n}^{(m)}$
contains every 3 cycle for every $m$.
\begin{quote}
\emph{Proof:} $G^{(0)} \supseteq G^{(1)} \supseteq \ldots \supseteq G^{(m)} = 1$ is
a composition series with abelian composition factors.  This can be refined into a
composition series with composition factors of prime order.  For the second part,
note that $(123) = (13)(12)(13)(12) \in S_{n}^{(1)}$.  Replacing $1$ by $a$,
$2$ by $b$ and $3$ by $c$ shows $(abc) \in S_{n}^{(1)}, \forall a, b, c$.
We show by induction that if $H$ contains all $3$-cycles, so does $H'$, if $n \ge 5$.
Since $(123)(234) = (13)(24)$, $(ab)(cd) \in H'$. $[(bc)(de),(abc)] =(abc) \in H'$ so
any $3$-cycle is in $H'$.
\end{quote}
{\bf Cyclic Extensions:}
Suppose $f \in k[x], deg(f)=n$ and let
${\cal G}_f(k)$ denote 
${\cal G}(K/k)$ where $K$ is the splitting field for $f$ over $k$. Then
${\cal G}_f(k)$ is isomorphic to some subgroup of $S_n$ and if $f$ is irreducible,
the group is transitive on $n$ symbols.  Set $\Delta= \prod_{i<j} (u_i-u_j)$ and
$Disc_k(f)= \Delta^2$, then if $f$ is irreducible, the Galois group is
$A_3$ or $S_3$ according to whether 
$Disc_k(f)= \Delta^2$ is a square in $k$.  If $f$ is a quartic with separated roots
$u_1, u_2, u_3, u_4$ and
$\alpha= u_1 u_2 + u_3 u_4$,
$\beta= u_1 u_3 + u_2 u_4$,
$\gamma= u_1 u_4 + u_2 u_3$; setting $K=k(\alpha, \beta, \gamma)$ and
$[K:k]=m$, then
${\cal G}_f(k)$ is $S_4$ if $m=6$,
${\cal G}_f(k)$ is $A_4$ if $m=3$,
${\cal G}_f(k)$ is ${\mathbb Z} \times {\mathbb Z}$ if $m=1$, and
${\cal G}_f(k)$ is ${\mathbb Z}_4$ 
or $D_4$ if $m=2$.
\\
\\
{\bf Embeddings:}
Let $k \subset K \subset {\overline k}$ and 
$\sigma_1, \sigma_2, \ldots , \sigma_r$ 
be the distinct $k$-monomorphisms from $K \rightarrow {\overline k}$, for
$u \in K$,
define 
$N_k^K(u) = (\prod_{i} \sigma_i(u))^{[K:k]_i}$ and
$Tr_k^K(u) = (K:k]_i \sum_{i} \sigma_i(u)$.  Note that distinct automorphisms are
linearly independent.  From now on, assume all extensions are separable (even Galois).
$N_k^K (uv)= N_k^K(u) N_k^K(v)$ and
$Tr_k^K (u+v)= Tr_k^K(u) + Tr_k^K(v)$; if $u \in k$,
$N_k^K (u)= u^{[K:k]}$ and
$Tr_k^K (u)= [K:k]u$; if $E$ is an intermediate field,
$N_k^K (u)= N_k^E(N_E^K(u))$ and
$Tr_k^K (u)= Tr_k^E(Tr_E^K(u))$.  If $K$ is a cyclic extension of $k$ of degree
$n$ with generator $\sigma$
then $Tr_k^K(u) = 0$ iff $\exists v \in K: u= v - \sigma(v)$ and
$N_k^K(u) = 1$ iff $\exists v \in K: u= v(\sigma(v))^{-1}$.  If $n= mp^t, (p,n)=1$
where $char(k)=p \ne 0$, there are intermediate cyclic fields, all
of which, except the last have degree $p$ and
each of which is the splitting field of $f(x)= x^p-x+a$.
If $char(k)=p \ne 0$, $K$ is a cyclic extension of degree $p$ iff
$K$ is the splitting field of an irreducible polynomial
$f(x)= x^p -x - a$ and $K=k(u), f(u)=0$.
Suppose $\zeta$ is a primitive $n$th root of unity over $k$ and $K= k(\zeta)$,
if $d \mid n$, $\zeta^{n/d}$ is a primitive $d$-th root of unity and,
$K$ is the splitting field over $k$ of an irreducible polynomial
$f(x)= x^d-a, a \in k$.
If $k$ contains a primitive  $n$-th root of unity, $\zeta$,
TFAE: (1) $K$ is cyclic of degree $d$
$d \mid n$,
(2) $K$ is the splitting field over $k$ of $f(x)= x^n-a, a \in k$,
(3) $K$ is the splitting field over $k$ of an irreducible polynomial
$f(x)= x^d-a, a \in k$.
\\
\\
{\bf Satz 90:}  Let $E/F$ be a cyclic extension with Galois group generated by
$\sigma$ then 
(a) $N_{E/F}(x)= 1$ iff $\exists y \in E: x= y/\sigma(y)$; and,
(b) $T_{E/F}(x)= 0$ iff $\exists y \in E: x= y - \sigma(y)$.
Let $(n, char(k))= 1$, and $K$ a \emph{cyclotomic extension} of $k$, then,
(1) $K= k(\zeta)$ where $\zeta$ is a primitive $n$-th root of unity;
(2) $K$ is an \emph{abelian extension} of $k$ of dimension $d, d \mid \psi(n)$;
(3) $|{\cal G}(K/k)|=d$ and is a subgroup of ${\mathbb Z}_n^*$.
\begin{quote}
\emph{Proof of additive part:}
If such an $\alpha$ exists then $Tr_k^K(\beta)=0$ as $\sigma$ permutes
the elements. Conversely,
suppose $Tr_k^K(\beta)=0$, $\exists \theta: Tr_k^K(\theta) \ne 0$.
Let $\alpha = {\frac 1 {Tr(\theta)}} [ 
\beta \theta^{\sigma} +
(\beta + \sigma \beta) \theta^{\sigma^2} + \ldots +
(\beta + \sigma \beta + \ldots + \sigma^{n-2} \beta) \theta^{\sigma^{n-1}}]$.
From this, we get $\beta= \alpha - \sigma \alpha$.
\\
\\
\emph{Proof of multiplicative part:}
Assume $\alpha$ exists.  $N(\beta)= {\frac {N(\alpha)} {N(\sigma(\alpha))}}$ 
and since elements of $G$ permute these, $N(\beta)=1$.  Now suppose
$\tau, \tau' \in G, \xi \in E: \xi^{\tau + \tau'}= 
\xi^{\tau} \xi^{\tau'} $.   By Artin's theorem on characters, the map given by
$ id+ \beta \sigma + \beta^{1+\sigma} \sigma^2+ \ldots +
\beta^{1+\sigma + \ldots + \sigma^{n-2}} \theta^{\sigma^{n-1}} \ne 0 $.
Hence $\exists \theta \in K: 
\alpha= \theta + \beta \theta^{\sigma} + \beta^{1+\sigma} \theta^{\sigma^2} + \ldots +
\beta^{1 + \sigma + \ldots + \sigma^{n-2}} \theta^{\sigma^{n-1}} \ne 0$.  
$\beta \alpha^{\sigma}= \alpha$ using the fact that $N(\beta)=1$ and so
applying $\sigma$ to the last term in the sum, we obtain $\theta$.  Dividing by
$\alpha^{\sigma}$, the proof concludes.
\end{quote}
{\bf Radical extensions:} $K= k(u_1, u_2, \ldots , u_n)$ where
$\exists n_1 : {u_1}^{n_1} \in k$ and
$\exists n_m : {u_m}^{n_m} \in k(u_1 , \ldots , u_{m-1})$.  $f$ is said to
be solvable by radicals if there is a radical extension containing the splitting
field of $f$.
If $K$ is a radical extension of $k$  and $E$ is an intermediate field
then ${\cal G}(E/k)$ is solvable.
If $E$ is a finite dimensional extension of degree $n$, $char(k) \nmid [E:k]$
and ${\cal G}(E/k)$ is solvable then there is a radical extension
$K$ of $k$ containing $E$.  If $char(k) \nmid n!$ and $f \in k[x], deg(f)=n$ then
$f(x)=0$ is solvable by radicals iff ${\cal G}_f$ is solvable.  To show this,
it suffices to consider prime exponents but we need
to prove:
\\
\\
{\bf Theorem:} If $q \nmid char(F)$ then the $q$-th root of unity are expressible as radicals, $q$,
a prime.
\\
\\
First a {\bf Lemma:} If $x^p-a$ is reducible, $a$ is a $p$-th power.
\begin{quote}
\emph{Proof:}
If $x^p-a = \psi(x) \phi(x)$, each is a product of factors of the form $(x- \zeta_{\nu} \theta)$ where
$\zeta$ is a $p$-th root of unity and $\theta^p=a$.  Say, $\psi(x)= b_k x^k + \ldots + b_0$ is 
a product of $k$ of them.  $(p,k)=1$ so $cp+dk=1$ for some $c, d \in {\mathbb Z}$.  Further,
$b= b_0= \zeta^{\nu} \theta^k$.  So $a=a^{cp} \cdot a^{dk}= b^{dp} \cdot a^{cp}$ which proves
the result.
\end{quote}
\begin{quote}
\emph{Proof of Theorem:}
By induction on $q$, clear for $q=2$.  The $q$-th roots of unity form a cyclic extension of
degree $q-1= p_1^{e_1} \ldots p_k^{e_k}$.  By induction, the successive $p_i$-th roots of unity
are radicals and adjoining them, $x^{p_k}-a$ must be irreducible by the lemma so each such
extension is a radical extension.
\end{quote}
{\bf Theorem:} Let $R_m$ be the ring of integers in ${\mathbb Q}[{\sqrt m}]$.  Suppose
$\forall x, y\in R_m$, with $y \nmid x$ and $|N(x)| \ge |N(y)|$, $\exists u, v \in R_m$ such that
$N(xu-yv) \leq |N(y)|$. Then $R_m$ is a PID.
\begin{quote}
\emph{Proof:} Let $I$ be an ideal and assume, by way of contradiction, $R_m$ is Euclidean.
Choose $y \ne 0$ so that $|N(y)|$ is minimal.  WTS $y \mid x$.  We can find
$u, v$: $N(xu-yv) \leq |N(y)|$ but this contradicts the minimality of $|N(y)|$.  Thus
$y \mid x$ and $I = (y)$.
\end{quote}
{\bf Theorem:} $R_{-19}$ is a PID.
\begin{quote}
\emph{Proof:} $R_{-19} = {\frac {a + b {\sqrt m}} {2}}$.  Suppose $y \nmid x$, $|N(x)| \geq |N(y)|$.
${\frac {x} {y}} = {\frac {a + b {\sqrt m}} {c}}, c > 1$.  Each of the cases $c=2, c=3, c=4, c \geq 5$,
yield contradictions.
\end{quote}
{\bf Theorem:} If $m \in {\mathbb Z}^{<0}$ is square-free and $m \notin \{-1, -2, -3, -7, -11 \}$,
then $R_m$ is not a Euclidean domain.
\begin{quote}
\emph{Proof:}
Suppose if is.  CHoose $b \in R_m \setminus U(R_m)$ with $d(b)$ minimal.
$\forall a, a=bq+r$ and $r =0, \pm  1$, so $b \mid 2, 3$.  However,
both $2$ and $3$ are irreducible in $R_m$.  If $m = 1 \jmod{4}$ and put $a= {\frac {1 + {\sqrt m}} {2}}$.
Neither $a, a+1, a-1$ are divisible by $2$ or $3$.  Contradiction.
\end{quote}
{\bf Theorem:} $R_{-19}$ is a PID that is not a Euclidean domain.
\begin{quote}
\emph{Proof:}  By the above, $R_{-19}$ is not a Euclidean domain and is a PID.
\end{quote}
\subsection {Computational Algebra}
{\bf Discrete Fourier Transform and FFT:} Let $c(x)=a(x)b(x)$ which
corresponds to the convolution ${\vec c} = {\vec a} * {\vec b}$.
Define the DFT as 
$F({\vec a})= A {\vec a}$, $A= \omega^{ij}$ with inverse
$A^{-1}= {\frac 1 n} \omega^{-ij}$.
Note that $F({\vec b} * {\vec c})= F({\vec b}) \cdot F({\vec c})$ (pointwise multiplication).
\emph{Tukey-Cooley Idea:} Suppose $n=pq$, set 
$j=j(j_1,j_2)= j_1 q +j_2, k=k(k_1 , k_2)= k_2p+k_1,
0 \le j_1 <p, 0 \le j_2 <q, 0 \le k_1 <p, 0 \le k_2 <q$.  Then
${\hat f}(k_1, k_2)= 
\sum_{j_2=0}^{q-1}
e^{\frac {2 \pi i j_2 (k_2p+k_1)} {n}}
\sum_{j_1=0}^{p-1}
e^{\frac {2 \pi i j_1 k_1} {p}} f(j_1 , j_2)$.  This requires $p^2q$ and $q^2p$ operations
respectively or $pq(p+q)$ rather than $(pq)^2$.  
Now do this recursively if $p, q$ factor further.
$X_n = \sum_{k=0}^{N-1} x_k e^{-ikn}$.
$x_n = {\frac 1 N} \sum_{k=0}^{N-1} X_k e^{ikn}$.
\\
\\
{\bf Strassen and FFT:}  For matrix multiply, Strassen found $7$ products that do the trick.
$m_1= (a_{12} - a_{22}) (b_{21} - b_{22})$,
$m_2= (a_{11} + a_{22}) (b_{11} + b_{22})$,
$m_3= (a_{11} - a_{21}) (b_{11} + b_{12})$,
$m_4= (a_{11} + a_{12}) b_{22}$,
$m_5= a_{11} (b_{21} - b_{22})$,
$m_6= a_{22} (b_{21} + b_{11})$,
$m_7= (a_{21} + a_{22}) b_{11}$.
$c_{11}= m_1 + m_2 -m_4 + m_6$,
$c_{12}= m_4 + m_5$,
$c_{21}= m_6 + m_7$,
$c_{22}= m_2 - m_3 + m_5 - m_7$.  $T(n)= 7 T({\frac n 2}) + 18 {\frac n 2}^2$, which is
$O(2^{lg(7)})$. $F_{i,j}= \omega^{ij}$.  $F$ evaluates, $F^{-1}$, interpolates.
$q_{l,m} = \prod_{j=l}^{l+2^m-1} (x-c_j)$ and $q_{l,m}= a_{l, m-1} q_{l+2^m,m-1}$.
What is $Rem({\frac {p(x)} {q_{l,0}(x)}}), \forall l$?  If $q= q' q''$, 
$Rem({\frac {p(x)} {q'(x)}}) =Rem({\frac {r_{l,m}(x)} {q'(x)}})$,
$q_{l,m}= x^{2^m} =\omega^{rev(l/2^m)}$.  For algorithm, crucial step is
$r_{l,m}(x) =\sum (a_j + \omega^s a_{j+2^m}) x^j$ and
$r_{l+2^m,m}(x) =\sum (a_j + \omega^{s+ {\frac n 2}} a_{j+2^m}) x^j$.
\\
\\
{\bf Hensel:}  If $I \subseteq {\mathbb R}$, $f=gh \jmod{I}$ such that the pseudo 
$GCD(g,h)=1$ then
$\exists g^*,h^*$ such that (1) $f= g^* h^* \jmod{I^2}$,
(2) $g=g^* \jmod{I}$,
(3) $h=h^* \jmod{I}$, and pseudo $GCD(g^*,h^*)=1 \jmod{I^2}$.
If $g', h'$ satisfy the conditions also, 
$g'=g^*(1+u) \jmod{i^2}$ and
$h'=h^*(1-u) \jmod{i^2}$.
\\
\\
{\bf Bivariate Factoring:}  If $|{\mathbb F}|>4d^2$, $f \in {\mathbb F}$, 
$deg_x(f) \leq d$, $\exists \in {\mathbb F}$: $f_{\beta}(x,0) \in {\mathbb F}[x]$
has no repeated factors.\\
\jt 1a Obtain square free factorization.\\
\jt 1b Find $\beta \in {\mathbb F}$ such that $f(x,\beta)$ is squarefree.\\
\jt 1c $f_{\beta}=f(x,y+\beta)$.\\
\jt 2a $f(x,y)=g(x,y)h(x,y) \jmod{y}$.\\
\jt 2b Lift $f(x,y)=g_k(x,y)h_k(x,y) \jmod{y^k}$.\\
\jt 3a Find $g''$ and $l_k$: $g''=g_k l_k \jmod{y^{2^k}}$, 
$deg_x(g'') \leq deg_x(f)$,
$deg_y(g'') \leq deg_y(f)$, $g'' \ne 0$. \\
$|Res(f,g,x)| \leq  (m+1)^{\frac n 2} (n+1)^{\frac m 2} A^{\frac m 2} B^{\frac n 2}$.
\\
\\
{\bf Extension Theorem:}
Let $I= \langle f_1 ,..., f_s \rangle \in {\mathbb C}(x_1,x_2,...,x_n)$ and $I_1$
is the first elimination ideal of $I$.  For each $1 \leq i \leq s$ write
$f_i = g(x_2,...,x_n) {x_1}^{N_i} + ...$.   Suppose $c=(c_2,...,c_n) \in V(I_1)$.
If $c \notin V(g_1,g_2,...,g_s)$, $\exists c_1$ such that $(c_1,c) \in V(I)$.  
\\
\\
{\bf Linear Programming:} $max(cx)$ subject to $Ax \leq b$, $x \geq 0$.
{\bf Quadratic Programming: }
$max(\sum \rho_{ij}\sigma_i \sigma_j x_i x_j)$, subject to $\sum x_i =1$,
$x_i \geq 0$, $\sum x_i u_i \geq R$.
\subsection{Algebraic Number Theory}
{\bf Gaussian Integers:} ${\mathbb Z}[i]$.  Let $\alpha, \beta, \gamma, \delta$ represent
gaussian integers.  $N(x+yi)= x^2 + y^2$.
$\forall \alpha, \beta, \exists \gamma, \delta$ such that
$\alpha= \beta \gamma + \delta$ with $0 \leq N(\delta) < N(\beta)$.
$\alpha$ is a unit iff $N(\alpha)=1$. Units are $1, -1, i, -i$.
Let $S= \{\alpha \eta + \beta \gamma\}$, $\phi$ with minimal norm is the
gcd.
If $\pi$ is a Gaussian integer with $N(\pi)=p$ then $\pi$ is prime.
If $\pi$ is a Gaussian prime and $\pi | \alpha \beta$ then $\pi | \alpha$ or
$\pi | \beta$.
Gaussian integers form a UFD.
Let $\pi$ be a Gaussian prime, there is one and only one $p$ such that
$\pi | p$.
Note that $\pi= x + yi$, $N(\pi)= x^2 + y^2$ divides $p$ or $p^2$ so
$x= 0, 1, 2 \jmod{4}$.
Characterization of Gaussian primes: $p=2$:  $p= -i \pi^2$.
$p=3 \jmod{4}$, $p=\pi$.
$p=1 \jmod{4}$, $p=\pi {\overline \pi}$ and $\pi$ and
${\overline \pi}$ are non-associated primes.
If $p=1 \jmod{4}$ then $p \mid (z^2+1)$.  If $\pi \mid p$,
$\pi | (z+i)(z-i)$ so $\pi | (z-i)$.
\\
\\
{\bf Definitions:}
$x$ is \emph{integral} over $A$ if $x$ is a root of a monic polynomial 
$f$ with coefficients in $A$.
If $A$ is a subring of $R$, the \emph{integral closure} of $A$ in
$R$ is the set $A_c$ of elements of $R$ 
that are integral over $A$. Note that $A \subseteq A_c$.
We say $A$ is integrally closed in $R$ if $A_c = A$. 
If $A$ is an integral domain
with quotient field $K$, and $A$ is integrally closed in $K$
we simply say
that $A$ is integrally closed without reference to $R$.
\\ 
\\
{\bf Theorem:}  Let $M$ be an $A$-module.  $M$ is faithful if 
$aM=0 \rightarrow a=0$.  Let $A \subseteq B$, $\alpha \in B$.  The following are
equivalent: (1) $\alpha$ is a root of $f(x)=x^n+a_{n-1}x^{n-1}+...+a_0$;
(2) $A[\alpha]$ is a finitely generated $A$ module; (3) $\exists$ a faithful
module, $N$, over $A[\alpha]$ which is a finitely generated $A$-module.
\begin{quote}
\emph{Proof:}
$1 \rightarrow 2$:  $1,x,\ldots, x^{n-1}$ generate $A[x]$.
$2 \rightarrow 3$:  $N=A[x]$.
$3 \rightarrow 1$:  Let $\beta_1, \ldots, \beta_n$ be the generators of $N$.
$x \beta_i = \sum_j a_{ij} \beta_j$ and $det(xI-(a_{ij}))=0$.  This gives the monic equation.
\end{quote}
{\bf More trace and norm:}
$N_{E/F}(x)= det(m(x))$, $Tr_{E/F}(x)= trace(m(x))$.
If $\alpha= x+yi$, $Tr(\alpha)= 2x$, $N(\alpha)= \alpha {\overline
\alpha}$.
$S(\alpha)= \sum_{\sigma} \alpha^{\sigma}$ is an integer, so is
$N(\alpha)= \prod_{\sigma} \alpha^{\sigma}$. $\alpha$ is a  unit iff
$|N(\alpha)| = 1$.
$\alpha$ is an integer of $Q({\sqrt d})$ iff $T(\alpha)$ and $N(\alpha)$
are integers.
\\
\\
{\bf Quadratic integers:}
$I_d= \{x+y \omega_d, x,y \in {\mathbb Z} \}$,
$\omega_d = {\sqrt d}$ if
$d= 2,3 \jmod{4}$,
${\frac { 1+{\sqrt d}} 2}$, if
$d= 1 \jmod{4}$.
Ideal Theory:
$P= (2, 1+{\sqrt {-5}})$,
$Q= (3, 1+{\sqrt {-5}})$.  $P^2= (2)$ and $Q {\overline Q}= (3)$.
Fermat analogue: $\alpha^{N(\pi)-1}= 1 \jmod{\pi}$.
\\
\\
{\bf Theorem:}
If $\theta$ is an algebraic number,
there is an integer $m$ such that $m \theta$ is an algebraic integer.
\begin{quote}
\emph{Proof:} We may assume 
$a_n \theta^n + a_{n-1} \theta^{n-1} + \ldots + a_0 =0, a_i \in {\mathbb Z}$.  Then
$(a_n \theta)^n + a_{n-1} (a_n \theta)^{n-1} + \ldots + (a_n)^{n-1}a_0 =0$.
$a_n \theta$ is an algebraic integer.
\end{quote}
{\bf Notation:} $R(\theta)$ denotes the ring of algebraic integers in ${\mathbb Q}(\theta)$.
\\
\\
{\bf Theorem:}
Every basis for $R(\theta)$ has $n$ elements, where $\theta$ is an algebraic number whose
minimal polynomial has degree $n$.
\begin{quote}
\emph{Proof:} Every integral basis is a basis and has the same number of elements.
\end{quote}
{\bf Definition:}
$\Delta(\alpha_1, \alpha_2, \ldots, \alpha_n) =
det({\alpha_i}^{\sigma_j})^2$, $\alpha_i \in R( \theta )$.
Alternatively,
$\Delta(\alpha_1, \alpha_2, \ldots, \alpha_n) =
det(T({\alpha_i}{\alpha_j}))$; this follows from the fact that 
$T(\alpha_i \alpha_j)= \sum_k \sigma_k(\alpha_i \alpha_j)$.
\\
\\
{\bf Theorem:}
$\Delta(\alpha_1, \alpha_2, \ldots, \alpha_n)$ is an integer.
\begin{quote}
\emph{Proof:} 
$\Delta(\alpha_1, \alpha_2, \ldots, \alpha_n)$ is an algebraic integer fixed by 
all $\sigma \in {\cal G}({\mathbb Q}(\theta)/{\mathbb Q}$ and so it is in ${\mathbb Q}$.
The only algebraic integers in ${\mathbb Q}$ are in ${\mathbb Z}$.
\end{quote}
{\bf Theorem:}
If $\{\alpha_i\}$ and $\{\beta_i\}$ are basis with
$\alpha_j= \sum_k a_{jk}\beta_k$ then
$\Delta(\alpha_1, \alpha_2, \ldots, \alpha_n)= det(a_{ij})^2
\Delta(\beta_1, \beta_2, \ldots, \beta_n)$.
\begin{quote}
\emph{Proof:} $T( \beta_r \beta_s)= T(\sum_{i,j} a_{r,i} a_{s,j} \alpha_i \alpha_j$,
so $(T(\beta_r \beta_s ) )= (a_{ij}) T(\alpha_i \alpha_j) (a_{ij})^T$.  Taking determinants
gives the result.
\end{quote}
{\bf Theorem:}  Suppose $F$ is separable over ${\mathbb Q}$,
$\langle \alpha_i \rangle$ is a basis iff
$\Delta(\alpha_1, \alpha_2, \ldots, \alpha_n) \ne 0$.
\begin{quote}
\emph{Proof:} If 
$\sum_j c_j \alpha_j =0$,
$\sum_j c_j \sigma_k(\alpha_j) =0, \forall k$ and $B=(\sigma_i (\alpha_j))$ has linearly dependent
columns so the determinant (discriminant) is $0$.  Suppose 
$\langle \alpha_1 , \ldots , \alpha_n \rangle$
are a basis and $\exists c_j$ not all $0$ such that $\sum_j c_j \sigma_k(\alpha_j) = 0$.
If 
$\Delta(\alpha_1, \alpha_2, \ldots, \alpha_n)= 0$, the rows of $B$ are linearly dependent so
$\sum_j c_j \sigma_k(u) = 0, \forall u$, which contradicts Artin.
\end{quote}
{\bf Theorem:}
All integral bases of $R(\theta)$ have the same discriminant.
\begin{quote}
\emph{Proof:} If 
$\langle \alpha_1 , \ldots , \alpha_n \rangle$ and
$\langle \beta_1 , \ldots , \beta_n \rangle$ are two bases
$\alpha_j= \sum_k a_{jk}\beta_k$ and
$\beta_j= \sum_k b_{jk}\alpha_k$.
$det(a_{ij})^2= det(b_{ij})^2 = 1$.
\end{quote}
{\bf Definition:}
If $\{\alpha_i\}$ is an integral basis for $R(\theta)$ then
$\Delta(\alpha_1, \alpha_2, \ldots, \alpha_n)$ is minimal,
in which case it is called the \emph{discriminant} of $R(\theta)$ and written
$Disc(R(\theta))$.
\\
\\
{\bf Theorem:} If $A$ is an ideal of $R(\theta)$ then
${\mathbb Z} \cap A \ne \emptyset$.
\begin{quote}
\emph{Proof:} 
Since $A$ is an ideal, $R(\theta) A \subseteq A$.
If $\alpha \in A$, $ \alpha^n + a_{n-1} \alpha^{n-1} + \ldots + a_0= 0, a_i \in {\mathbb Z}$.
$a_0 \in {\mathbb Z} \cap A$.
\end{quote}
{\bf Theorem:}
If $D$ is a ring of algebraic integers and $A$ is a module then $D/A$ is finite.
\begin{quote}
\emph{Proof:} $\exists a \in A \cap {\mathbb Z}$.  $(a) \subseteq A$ so $D/(a) \rightarrow D/A$ is
a homomorphism.  $D= {\mathbb Z} \omega_1 + \ldots + {\mathbb Z} \omega_n$.  Let
$S= \{ \sum_i \gamma_i \omega_i, 0 \leq \gamma_i < a \}$.  $S$ contains all coset representatives
of $D/(a)$.
\end{quote}
{\bf Theorem:} $D$ is Noetherian.
\begin{quote}
\emph{Proof:}
Since $D/A_i$ is finite there are only finitely many ideals in $A_1$.
\end{quote}
{\bf Theorem:}
Every prime ideal in $D$ is maximal.
\begin{quote}
\emph{Proof:}
$D/P$ is a finite integral domain.
\end{quote}
{\bf Theorem:}
Let $A \subseteq D$ be an ideal.  If $\beta \in F$ and $\beta A \subseteq A$ then
$\beta \in D$.
\begin{quote}
\emph{Proof:} $A$ is a finitely generated ${\mathbb Z}$-module.  $\beta$ satisfies
$a_n B^n+ \ldots + a_0 = 0, a_i \in {\mathbb Z}$.  Multiply by $a_n^{n-1}$ and $a_n \beta$ is an algebraic
integer.  $a_i a_n^{-1} \in {\mathbb Z}$.
\end{quote}
{\bf Theorem:}
If $A, B$ are ideals in $D$ and $A= AB$ then $B=D$.
\begin{quote}
\emph{Proof:} 
Let $\alpha_1, \ldots , \alpha_n$ be an integral basis for $A$.  $\exists b_i \in B:
\alpha_i = \sum_j b_{ij} \alpha_j$.  $det(b_{ij}- \delta_{ij})= 0$.  So $1 \in B$ and $D=B$.
\end{quote}
{\bf Theorem:}
If $\omega \in D$ and $( \omega ) A = B A$ then $( \omega ) = B$.
\begin{quote}
\emph{Proof:} 
$\beta \in B$ implies $(\beta / \omega ) A \subseteq A$ so $ \beta / \omega \in D$.
$A= \omega^{-1} B A$ so $\omega^{-1} B = D$.
\end{quote}
{\bf Theorem:}
Every ideal contains a basis.
\begin{quote}
\emph{Proof:} 
Let $\beta_1, \ldots , \beta_n$ be a basis of $F/{\mathbb Q}$.  $\exists a, b \in {\mathbb Z}$
such that $b \beta_1, \ldots , b\beta_n \in {\mathbb Z}$.  Choose $\alpha \in A, \alpha \neq 0$,
$b\beta_1 \alpha , \ldots , b\beta_n \alpha \in A$ are a basis.
\end{quote}
{\bf Theorem:}
$\exists M(F): \alpha , \beta \in D, \beta \ne 0, 1 \leq t \leq M$ and $\omega \in D$:
$|N( t \alpha - \omega \beta)| < |N(\beta)|$.
\begin{quote}
\emph{Proof:} 
Let $\gamma= {\frac {\alpha}{\beta}} \in F$.  It suffices to show that $\forall \gamma \in F$,
there is an $M: |N(t \alpha - \omega \beta )|< 1$ for some $1 \leq t \leq M$ and $\omega \in D$.
Let $ \omega_1 , \ldots, \omega_n $ be an integral basis for $D$.  For $\gamma \in F$,
$\gamma= \sum_{i=1}^n \gamma_i \omega_i, \gamma_i \in {\mathbb Q}$.
Notice that  $|N(\gamma)|= |\prod_j (\sum_i \gamma_i \omega_i^{(j)})| \leq C (max_i |\gamma_i|)^n$,
where $C= \prod_j ( \sum_i \omega_i^{(j)})$.  Choose $m > C^{\frac 1 n}$ and set $M= m^n$.
For $\gamma \in F, \gamma= \sum_i \gamma_i \omega_i$, 
$\gamma_i = a_i + b_i, a_i= \lceil \gamma_i \rceil$.  Put $[ \gamma_i ]= \sum_i a_i \omega_i$.
Map $F \rightarrow {\mathbb R}^n$ by $\phi(\sum_i \gamma_i \omega_i)= (\gamma_1 , \ldots , \gamma_n)$.
Partition the $n$-cube into $M=m^n$ subcubes.  
Consider the points $\phi(\{k \gamma\}), 1 \leq k \leq m^n + 1$.
At least two lie in the same subcube.  Subtracting we get $t \gamma= \omega + \delta$.
The coordinates of $\delta$ have absolute value $\leq {\frac 1 m}$.  
So $N(\delta) \leq C(1/m)^n= C/m^n<1$.
\end{quote}
{\bf Theorem:}
The class number is finite.
\begin{quote}
\emph{Proof:} 
Let $A \subseteq D$.  For $\alpha \in A$ and $\alpha \ne 0$, $|N(\alpha)| \in {\mathbb Z}$.
Choose $\beta \in A, \beta \ne 0: |N(A)|$ is minimal.  $\forall \alpha, \exists t:
|N(t \alpha - \omega \beta )| < |N(\beta)|, 1 \leq t \leq M$.  Since
$t \alpha - \omega \beta \in A$,
$t \alpha - \omega \beta = 0$.  $B = {\frac 1 {\beta}} M! A \subseteq D$,  $ (M!) \supseteq (\beta)B$ so
$(M!) \subseteq B$.  But $(M!)$ is contained in only finitely many ideals.  So $A \equiv B$, where
$B$ is one of finitely many ideals.
\end{quote}
{\bf Theorem:}
If $A, B$ are ideals in $R(\theta)$, $A|B$ iff $A=BC$ iff $B \subseteq A$.
\begin{quote}
\emph{Proof:} 
$\exists k: B^k = (\beta)$.  Since $A \subseteq B$, $B^{k-1} A \subseteq (\beta)$ so
$C= {\frac 1 {\beta}} B^{k-1}A \subseteq D$.  $BC=1$.
\end{quote}
{\bf Definition:}
If $A$ is an ideal with basis $\alpha_i= \sum_j a_{ij} \omega_j$ then
$N(A)= det (a_{ij})$.
$A \sim B$ iff $\exists \alpha, \beta$ such that $(\alpha)A=(\beta)B$.
each equivalence class is called an ideal class.
\\
\\
{\bf Theorem:} $\forall A \subseteq D, \exists k: 1 \leq k \leq h_F$ such that $A^k$ is principal.
\begin{quote}
\emph{Proof:} 
Consider $\{ a^i : 1 \leq k \leq h_F+1 \} : a^i \equiv A^j, j>i$.  $\exists \alpha , \beta \in D:
(\alpha) A^i= (\beta) A^j$.  Put $B= A^{j-i}$.  $B$ is principal since
$(\alpha) A^i = (\beta) B A^i$.  $( \alpha / \beta ) A^i \subseteq A^i$, so
$\omega= \alpha / \beta \in D$ then $(\omega) A^i = B A^i$ and thus $(\omega)= B$.
\end{quote}
{\bf Theorem:}
There are finitely many ideal classes $h$ of $R(\theta)$ and $A^h \sim
(1)$.
\begin{quote}
\emph{Proof:} 
For $K=R(\theta)$, $\exists C(K): \forall A, \exists 0 \ne \alpha \in A:
|N(\alpha)| \leq C N(A)$.  Use this to show $\exists B:N(B) \leq C$ so there are a finite
number of ideals containing $B$. $\exists \alpha: (\alpha)=AD$.  $AN \sim AD$.
\end{quote}
{\bf Theorem:}
If $A$ is an ideal in $R(\theta)$, $\exists B$ such that $AB= (a)$ for some
$a$ in $R(\theta)$.
\begin{quote}
\emph{Proof:} Todo.
\end{quote}
{\bf Theorem:}
If $A, B$ are ideals in $R(\theta)$,with  $AC=BC$ then $A=B$.
\begin{quote}
\emph{Proof:} $\exists k>0: A^k= (\alpha)$, so
$A^{k-1}AB = A^{k-1} AC$.  $(\alpha) B = (\alpha) C$ and so $B=C$.
\end{quote}
{\bf Theorem:}
If $P|AB$ and $P$ does not divide $A$ then $P|B$.
\begin{quote}
\emph{Proof:} $a \in A, b \in B$ $P \supseteq AB$ but $P \nsupseteq A$,
so $ab \in P$.  $\exists c: ac \in 1 +P$.  $ac - 1 \in P$ and
$abc - b \in P$ but $ab \in P$ so $b \in P$ and $P \supseteq B$.
\end{quote}
{\bf Theorem:}
Every ideal has finitely many distinct divisors.
\begin{quote}
\emph{Proof:} Todo.
\end{quote}
{\bf Theorem:}
Every prime ideal must divide the principal ideal of a rational prime.
\begin{quote}
\emph{Proof:} Follows from finiteness of class number.
\end{quote}
{\bf Theorem:}
Every ideal can be written as a product of prime ideals.  The factorization
is unique apart from order.
\begin{quote}
\emph{Proof:} Let $A$ be an ideal.  $D/A$ is finite so $A \subset P_1$, maximal and
$A= P_1B_1$.  If $B_1 \neq D$, $B_1$ is contained in a maximal ideal and so on
giving a chain, $A \subset B_1 \subset B_2 \subset \ldots$ which must terminate.
\end{quote}
{\bf Theorem:}
Every rational integer belongs to finitely many ideals.
\begin{quote}
\emph{Proof:} Todo.
\end{quote}
{\bf Theorem:}
Rational prime is \emph{ramified}
 if its principal ideal factors into prime ideals in which one
prime ideal is repeated.  If this happens, $p | \Delta(\alpha_1 , \ldots , \alpha_n)$.
\begin{quote}
\emph{Proof:} Todo.
\end{quote}
{\bf Theorem:}
The ring of integers ${\cal D}_K$ in the number field, $K$, has the following properties:
${\cal D}_K$ 
is a domain with field of fractions $K$.
${\cal D}_K$ is noetherian (Use the fact that
${\cal D}_K$  is a free abelian group of degree $n= K:{\mathbb Q}$.)
\begin{quote}
\emph{Proof:} Todo.
\end{quote}
{\bf Definition:}
Let ${\cal D}$ be a ring of integers, ${\mathfrak a}$
is a fractional ideal if $\exists c \in {\cal D}$: $c {\mathfrak a} \subseteq {\cal D}$.
\\
\\
{\bf Theorem:}
Every non zero prime ideal ${\mathfrak p}$ of ${\cal D}$ is maximal. (
${\cal D}/{\mathfrak a}$ is a finite integral domain.)  Fractional ideals form an abelian
group.  
\begin{quote}
\emph{Proof:} Todo.
\end{quote}
{\bf Theorem:}
Every non-zero ideal of
${\cal D}$ can be factored into prime ideals (${\cal D}$ is noetherian).
\begin{quote}
\emph{Proof:} Todo.
\end{quote}
{\bf Norm of an ideal:} 
$N({\mathfrak a})= |{\mathfrak D}/{\mathfrak a}|$; if ${\mathfrak a}= \langle a \rangle$ 
is principal
$N(a)= N({\mathfrak a})$. 
$N({\mathfrak a}{\mathfrak b})= N({\mathfrak a}) N({\mathfrak b})$.
$\Delta_{K/{\mathbb Q}}(\alpha_1 , \alpha_2, \ldots , \alpha_n )= 
[det(\sigma_i (\alpha_j))]^2$.
\\
\\
{\bf Theorem:}
Every non-zero ideal of ${\cal D}$ has a finite number
of divisors.  
\begin{quote}
\emph{Proof:} Todo.
\end{quote}
{\bf Theorem:}
A non-zero rational integer belongs to a finite number
of ideals of ${\cal D}$.  
\begin{quote}
\emph{Proof:} Todo.
\end{quote}
{\bf Theorem:}
Only a finite number of ideals of ${\cal D}$
have a given norm.  
\begin{quote}
\emph{Proof:} Todo.
\end{quote}
{\bf Theorem:}
If ${\mathfrak a} \ne {\mathfrak b}$ are ideals of ${\cal D}$ then
$\exists \alpha \in {\mathfrak a}: \alpha {\mathfrak a}^{-1} + {\mathfrak b}= {\cal D}$.
\begin{quote}
\emph{Proof:} Todo.
\end{quote}
{\bf Theorem:}
Let ${\mathfrak a} \ne 0$ be an ideal of ${\cal D}$ and $0 \ne \beta \in {\mathfrak a},
\exists \alpha \in {\mathfrak a}: {\mathfrak a}= \langle  \alpha, \beta \rangle$.
\begin{quote}
\emph{Proof:} Todo.
\end{quote}
{\bf Minkowski:}
$X$ is convex if $x, y \in X \rightarrow \lambda x + (1-\lambda)y \in X,
\forall \lambda \in [0,1]$.
$X$ is  symmetric if $x \in X \rightarrow -x \in X$.
Let $L$ be an $n$-dimensional lattice in ${\mathbb R}^n$ with
fundamental region $T$ and let $X$ be a bounded, convex, symmetric subset of
${\mathbb R}^n$; if $v(X)> 2^n v(T), \exists \alpha \in X \cap L, x \ne 0$.  Let
$L$ be a lattice then ${\mathbb R}^n/L \cong T^n$ (a torus).  Let
$T$ be a fundamental region of $L$, $\phi: T \rightarrow T^n$ then
$v(X) = v(\phi^{-1}(X))$.  If $\nu: {\mathbb R}^n \rightarrow T^n$ is the
natural homomorphism with $ker(\nu)=L$.  If $X$ is a bounded subset of
${\mathbb R}^n$, $\nu$ exists and $v(\nu(X)) \ne v(X)$ then $\nu_{|X}$ is
not injective.  \emph{Four squares:} If $p= 4k+1$ then $p=a^2 + b^2$.  
($\langle g \rangle = {\mathbb Z}_p$
is cyclic $g^k=u$ and $u^2= -1$.  Let $L= \{(a,b): b=ua \jmod{p} \}$,
${\mathbb Z}^2:L= p^2$, $vol(T_L)=p$.  $C_r: \{ x: ||x||<r \}$ and $\pi r^2>4p$,
$r^2= {\frac {3p} 2}, 0 \ne a^2 +b^2 \le r^2 <2p$.
\\
\\
\emph{Examples in algebraic fields:} In
$R= {\mathbb Z}[{\sqrt {-3}}]$, ${\frac {-1+{\sqrt {-3}}} 2}$ is a unit note that
$2 \times 2 = {-1+{\sqrt {-3}}} \times {-1-{\sqrt {-3}}}$.
In $R= {\mathbb Z}[{\sqrt {-5}}]$ ideals are not all principal; note that
$2 \times 3 = {-1+{\sqrt {-5}}} \times {-1-{\sqrt {-5}}}$.  \emph{Pell related:}
There are two equivalence classes of forms of determinant $5$:   $x^2+5y^2$ and
$2 x^2 + 2xy +3y^2$ and the class number of ${\mathbb Z} [{\sqrt {-5}}]$ is $2$.
If $p$ is a rational prime and $K/ {\mathbb Q}$ is a Galois extension 
then $G= {\cal G}(K/{\mathbb Q})$ acts transitively on the ideal divisors of $(p)$,
the exponent of the ideal divisors are called the ramification index.  The ideal generated
by a rational ideal $(p)$ factors into indecomposable factors
in an algebraic number field, $O_F$, in one of
three ways: (a) $(p)$, (b) $(p)= P \sigma(P)$ (``$p$ splits''), or (c)
$(p)= P^2$ (``$p$ ramifies'').
\\
\\
{\bf Analytic formulas:} $f*g(n)= \sum_{d|n} f(d)g({\frac n d})$.  
This is commutative, associative and has an inverse.
$\Lambda(n)= ln(n)$, if $n= p^m$,
$\Lambda(n)= 0$, otherwise.  Note: $ln(n)= \sum_{d|n} \Lambda(d)$.
$\sigma_{\alpha}(n)= \sum_{d|n} d^{\alpha}$.
$\psi(x)= \sum_{n \leq x} \Lambda(n)$,
$\vartheta(x)= \sum_{p \leq x} ln(p)$.
${\frac {\psi(x)} {x}} -
{\frac {\vartheta(x)} {x}} \leq {\frac {ln(x)^2} {2 {\sqrt x} ln(2)}}$.
$L(1,\chi)= \sum_{n=1}^{\infty} {\frac {\chi(n)} n}$, $\chi$, a
non-principal character.
{\bf Dirichlet:}  If $k>0$ and $(h,k)=1$, $\forall x>1$
$\sum_{p \leq x, p=h \jmod{k}} {\frac {ln(p)} p}=
{\frac 1 {\phi(k)}} ln(x) + O(1)$.
$\pi_a (x)= \sum_{p \leq x, p=a \; \jmod{k}} 1$.
$\pi_a (x) \approx {\frac {\pi(x)} {\phi(k)}}$, $x \rightarrow 0$,
$\forall a$, $(a, k)=1$ and $\pi_a (x) \approx \pi_b (x)$ when
$(a,k)=(b,k)=1$.
\\
\\
{\bf Definition:} A \emph{Lie algebra} is a vector space, $V$ over a field, $F$ with an operation
$[,]: V \times V \rightarrow V$ which is alternating and bilinear and which satisfies the Jacobi
identity: $[x,[y,z]] + [y,[z,x]] + [z, [x,y]]= 0$.  If $A$ is an associative algebra, there is
a corresponding Lie algebra with $[a,b]= a*b-b*a$, called an \emph{enveloping algebra}.
\emph{Example:} $n \times n$ matricies over $F$ give rise to $L_n(F)$.  ${\mathbb R}^3$ with
$[a, b]= a \times b$ is a Lie algebra.

