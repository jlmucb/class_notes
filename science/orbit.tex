\documentclass{report}
\usepackage{amsfonts}
\usepackage{amssymb}
\usepackage{multicol}
\usepackage{graphicx}
\setlength{\oddsidemargin}{0in}
\setlength{\evensidemargin}{0in}
\setlength{\topmargin}{-.5in}
\setlength{\textwidth}{6.5in}
\setlength{\textheight}{9.5in}

\begin{document}
\section{Orbits}
$r^2 \dot{\theta} =h$.
$r = {\frac {h^2}{k(1+e \cdot cos( \theta ))}}$.
$r_{min}= {\frac {h^2}{k^3}} {\frac {1} {1+e}}$,
$r_{max}= {\frac {h^2}{k^3}} {\frac {1} {1-e}}$.
$\dot{\theta} = {\frac {k^2}{h^3}} (1+ e \cdot cos(\theta ))^2$.
$\int {\frac {h^3}{k^2}} {\frac {1} {(1+e \cdot cos( \theta ))^2}}= t$.
$S$ is the satellite and $O$ is the observer.
\\
\\
1. Compute $\theta(t), r(t)$ from integral above in orbital plane.  At $t_0$,
$S$ is at ascending node position.
\\
2. Rotate orbital parameters by the inclination, $i$, in the line formed by the
ascending node and descending node.
\\
3. Rotate around $z$ by the angle of the ascending node to get to ecliptic coordinates.
$S$ is at ascending node when $t=0$.
\\
4. Get $\delta$, $RA$ of $S$.
\\
5. Get $\lambda$, $L$ of $O$.
\\
6. Find $\delta '$ and $RA '$, from $O$'s point of view.
\\
\\
$S$ is at at $( \delta , RA )$, $O$ is at $( \lambda , L)$. $\overline{x}$ means
$90-x$.
$cos ( \overline{ \delta ' }) =
cos( RA+L ) sin(\overline{\delta}) sin(\overline{\lambda}) +
cos(\overline{\delta}) cos(\overline{\lambda})$.
${\frac {\overline{\delta}} {sin(RA')}} =
{\frac {\overline{\delta'}} {sin(RA+L)}} $.  $RA$ is measured from the arc $ON$ to the arc
$OS$.
\end{document}

