\chapter{Computer Science}
\section{Basics}
{\bf Definitions:}
$f \in O(g) \leftrightarrow
g \in \Omega (f) \leftrightarrow
L_{x \rightarrow \infty} {\frac {f(x)} {g(x)}} < \infty$.
$f \in o(g) \leftrightarrow
g \in \omega (f) \leftrightarrow
L_{x \rightarrow \infty} {\frac {f(x)} {g(x)}} = 0 $.  Equivalently:
$f(n)$ is $\Theta(g(n))$ if $\exists c_1, c_2, n_0: n \geq n_0 \rightarrow 0 \leq c_1 g(n) \leq f(n) \leq c_2 g(n)$.
$f(n)$ is $O(g(n))$ if $\exists c, n_0: n \geq n_0 \rightarrow 0  \leq f(n) \leq c g(n)$.
$f(n)$ is $\Omega(g(n))$ if $\exists c, n_0: n \geq n_0 \rightarrow 0  \leq c g(n) \leq f(n)$.
$f(n)$ is $o(g(n))$ if $\forall \epsilon > 0, \exists n_0: n \geq n_0 \rightarrow 0  \leq {\frac {f(n)} {g(n)}} < \epsilon$.
$f(n)$ is $\omega(g(n))$ if $g(n)$ is $o(f(n))$.
$G_1 \subset (G,E)$ is a
\emph{strongly connected component} iff $x,y \in G_1$ means there is a directed path
$x \rightarrow y$ and a directed path $y \rightarrow x$.
\\
\\
{\bf Recurrences:}  Suppose $T(n)= aT(n/b) + f(n)$.  
If $f(n)= O(n^{log_b(a)-\epsilon})$ then $T(n)= \Theta(n^{log_b(a)})$.
If $f(n)= \Theta(n^{log_b(a)})$ then $T(n)= \Theta(n^{log_b(a)}lg(n))$.
If $f(n)= \Omega(n^{log_b(a)+\epsilon})$ and $af(n/b) \le c f(n), c<1$
then $T(n)= \Theta(f(n))$.\\
\\
{\bf Binary arithmetic:}
Adding an $m$ bit number and $n$ bit number takes $O(max(m,n))$ time and $O(m+n)$ space.
Multiplying an $m$ bit number and $n$ bit number takes $O(mn)$ time and $O(m+n)$ space.
The \emph{extended gcd} of an $m$ bit number and $n$ bit number takes 
$O(mn)$ time and $O(m+n)$ space.
$GCD(u, v)$ average running time:
$O((1+{\frac {max(u,v)} {(u,v)}}) lg(min(u, v)))$.
$A^E \jmod{M}$ where $M$ is an $m$ bit number and $E$ is an $n$ bit number 
takes $O(n m^2)$ time. For $0 \leq \lambda \leq 1$: ${n \choose {\lambda n}} \leq 2^{n H(\lambda)}$.
\\
\\
{\bf Theorem:}
Given $\epsilon>0$ there is a multiplication algorithm such that the number
of elementary operation $T(n)$ needed to multiply two $n$-bit numbers
satisfies
$T(n) <c(\epsilon)n^{1+\epsilon}$.  Strassen: $T(n)=O(n lg(n))$.
\\
\\
{\bf Floating Point Numbers:} $f \times b^{e-q}$ is represented as $(e, f)$.
\\
\\
{\bf Heapsort:} For a node, $t$, the \emph{heap property} is
$value(t) \geq value(LEFT(t))$ and $value(t) \geq value(RIGHT(t))$.  
A heap list $A[1], A[2] , \ldots, A[n]$ can represent a heap with 
$LEFT(A[i])= A[2i]$ and $RIGHT(A[i])= A[2i+1]$.  Note that a path 
from a node to an edge is linearly ordered and
so largest element is at root.  Usually the
heap is arranged so that $value(LEFT(t)) \geq value(RIGHT(t))$.
\begin{multicols} {2} {
\begin{verbatim}
heapify(A,i,j,n) 
{
    if(i is not a leaf) {
        k= child of i with largest elment;
        if(A[k]>A[i]) {
            swap (A[i], A[k]);
            heapify (A, k, j, n);
        }
    }
}
\end{verbatim}

\begin{verbatim}
buildheap(A,n) 
{
    for(i=n;i>1;i--) {
        heapify(A,i,n,n);
    }
}

heapsort(A,n) 
{
    buildheap(A,n);
    for(i=n;i>1;i--) {
        swap(A[1], A[i]); 
        heapify(&A[1],i-1, i-1);
    }
}
\end{verbatim}
}
\end{multicols}

{\bf Shortest Path:} between x and y in $G=(V, E)$ where
$l(e)>0$ is the weight of $e \in E$ is $O(e lg(n))$. d(v) contains an overestimate
of the shortest path from s to v.
$prev(v)$ contains the previous element in the
shortest path from s to v.  (Ford-Bellman version works for negative weights.)
\begin{multicols} {2} {
\begin{verbatim}
shortestpath(V,E,s)  {
    for (v in V) {
        d(v):= infinity;
        prev(v):= empty-set;
        }
    H:= empty-set; 
    d(s)= 0; 
    mark(s);
    while (H is not empty) {
        h= deletemin(H);
        for e=(v, w) in E, w unmarked) {
            if(d(w)>(d(v)+l(e))) {
                d(w)=d(v)+l(e);
                prev(w)= v;
                insert(w, H);
                }
            }
        }
    }
\end{verbatim}
}
\end{multicols}

{\bf Union-find:}  Link(x, y): make x and y kids of a common parent.  Parent node
points to itself.  $m$ UNION-FIND operations on $n$ elements is $O((m+n) lg(n))$.

\begin{multicols} {2} {
\begin{verbatim}
makeset(x) {
    p(x)= x;
    rank(x)= 0;
    } 

find(x) {
    if(x != p(x))
        p(x)=  find(p(x)); 
   return(p(x));
   }

link(x, y) {
    if(rank(x)>rank(y)) swap(x, y);
    if(rank(x)==rank(y)) rank(y)++;
    p(x)= y;
    return(y);
    }

union(x,y) {
    link(find(x), find(y));
    }
\end{verbatim}
}
\end{multicols}

{\bf Order statistics:} The algorithm below satisfies the recurrence 
$T(n) \leq T({\frac n 5}) + T({\frac {3n} 4} + cn)$.
\\
\\
Select(k,S) \\
\{\\
\jt if($k<50$)
\jt \jt sort and return k'th element;\\
\jt Partition $S$ into $5$ sequences $S_1, \ldots , S_{\lfloor {\frac{|S|} 5} \rfloor}$, $T$,
where $T$ contains the up to $4$ leftovers;\\
\jt Sort each $5$ set;\\
\jt Let $M$ be the set of the medians for the $S_i$;\\
\jt m= Select($\lceil {\frac {|M|} 2} \rceil$;\\
\jt let $S_1$, $S_2$, $S_3$ be the sets of elements $<, =, > m$ respectively;\\
\jt if($|S_2|>k$)\\
\jt \jt return Select($k, S_1$);\\
\jt if($|S_1|+|S_2|>k$) \\
\jt \jt return $m$;\\
\jt return($k-|S_1|-|S_2|$, $S_3$);\\
\}
\\
\\
{\bf $2-3$ Trees:} Interior node has smallest key of 2nd and 3rd descendant.
Insert: Do membership test stop at terminal position; id 2 kids, add one, if
not, split into two, $(n, n')$.  Add $n'$ using insert.  
Delete: If two kids left, done.
Otherwise, try to move node of a siblings under common parent; if you can't,
transfer this node to a sibling.  If this leaves a singleton, in the parent,
recurse the transfer on parent.
\\
\\
{\bf Minimal spanning trees:}  Consider a graph,
$G= (V,E)$, each edge $e \in E$ having weight $wt(e)$.  
\\
\emph{Kruskal:}
\\
 Initialize a forest of trees consisting of $V$.\\
while( there is more than one component) \\
\jt 1. Remove edge from $E$ of minimum weight.\\
\jt 2. Add it if it unites two trees.  Discard it if it creates a cycle.\\
\\
\emph{Prim:}\\
1. Choose an arbitrary vertex $S= \{x\}$, $M=\emptyset$.\\
2. while($S \ne V$) \\
\jt 3. Choose an edge $e=(x, y), x \in S$ of minimum weight.\\
\jt 4. If $y \in S$, discard.  Otherwise $M= m \cup \{e\}$
\\
\\
{\bf NP Completeness:}
$P \subseteq N$.  If $A \le B$ 
\footnote{$A \le B$ means problem $A$ can be transformed to problem $B$ in polynomial time;
this is called a reduction \emph {from} $A$ \emph{to} $B$.}
and $B \in P$ then $A \in P$.  $L \in NPC$ if and only if 
$L \in NP$, $A \in NP \rightarrow A \le L$.  Classical computation theory classifies
problems by a ``certain'' solution on all instances.  The class of problems which
can be solved in polynomial time
``up to an arbitrary error, $\epsilon$'' is called $RP$ for ``randomized
polynomial.''  $P \subseteq RP \subseteq NP$.
\\
\\
{\bf Problems in P and NP:}
P: MST.  Given a weighted graph, G, and a weight, K $\exists$ a tree,
NP: TSP.  Given a weighted graph, G, and a weight, K $\exists$ a cycle,
C, that connects all nodes  of G with weight $\leq K$.\\
P: Circuit value.  NP: Circuit SAT.\\
P: 2-SAT: Use $\phi= (a_1 \vee b_1  )  \wedge ... \wedge (a_n \vee b_n )$ to form graph with 
nodes $a_i, b_i, {\overline {a_i}}, {\overline {b_i}}$ insert
edges ${\overline {a_i}} \rightarrow b_i$
and ${\overline {b_i}} \rightarrow a_i$.  Find strongly connected components.  If no
strongly connected component contains a variable and its negation, it is satisfiable;
otherwise not.  So 2-SAT is not NP hard.  NP: 3-SAT.  Note in disjunctive normal form 
SAT is easy but translating is hard.
\\
P: matching.  NP: 3D matching.\\
P: Linear Programming.   NP: Integer Programming.\\
\\
{\bf Ford-Fulkerson:} Augmenting path p is a simple path from s to t that
increases the flow.
\begin{verbatim}
    Initialize flow, f to 0;
    while (there is an augmenting path, p)
        augment flow along p;
    return f;
\end{verbatim}
{\bf Undecidability:} Suppose $Term(P,X)$ is a boolean function which takes a program,
$P$, and an input $X$.
$Term(P,X)$ returns true iff $P$ terminates on $X$.
$Term(P,X)$ returns false iff $P$ does not terminate on $X$.
\\
\\
{\bf Theorem:}  $Term(P, X)$ does not exist.  Suppose it did. Set
\begin{verbatim}
diag(P,X) {
  if $Term(P,P)$==true 
    loop
    }
\end{verbatim}
$diag(diag)$ terminates iff it doesn't terminate.  Contradiction.\\
\\
{\bf Stable Matching} (up to $n^2$ rounds). (1) Boy goes to favorite girl
on list.  (2) Girl tells highest choice ``maybe'', tells everyone else
No.  (3) Boy crosses off girls that have said no.
(4) terminate in the round when every girl has
told one boy ``maybe'', convert ``maybe'' to yes.\\
\\
{\bf Linear Programming Standard Form:}
Maximize $x= C^{T}X$, subject to $AX= B$, $X \geq 0$.  
Problem:  There may be exponentially
many corners. (Reason: introduce to $n$ constraint inequalities $m$ slack variables;  the
corner points occur when $m$ variables are $0$.  There are ${{m+n} \choose m}$ ways to select
the variables to be set to $0$.)
Simplex idea: move along growing paths instead of trying all corners randomly.
Dual, minimize $x= B^{T}W$, subject to $A^{T}W= C$, $W \geq 0$.\\
\\
{\bf Notation:} \emph{basic variables} $\ne 0$, \emph{non basic variables} $=0$.  
A is an $m \times n$
matrix, with $m$ variables (including slack) and $m$ constraints.
Tableau has basic variables and their values in two first columns.  Top row is
all variables as labels middle is matrix (A).
Rightmost column is constants (B).  Bottom row is $C-C^{T}X$ in terms of the
non-basic variables.  Basic algorithm is:
\begin{enumerate}
\item Locate most negative coefficient in bottom row, call column containing it $x_{j}$.
\item Compute ${\frac {B_{i}} {A_{ij}}}$.  The smallest one, denoted $k$, is the pivot.
\item Convert pivot to 1 and eliminate all coefficients in the same column.
\item Replace $x_{k}$ row by $x_{j}$.
\item repeat until no negative numbers in bottom row.
\end{enumerate}
{\bf SAT/$k$-sat reduction:}
$l_1 \vee l_2 \vee \ldots \vee l_n \rightarrow
l_1 \vee l_2 \vee x_1 \wedge
{\overline {x_1}} \vee l_3 \vee x_2 \wedge \ldots
{\overline {x_{n-3}}} \vee l_{n-1} \vee \ldots \vee l_n$.
Phase transition for SAT: ${\frac {clauses} {variables}} \approx 4.3$.
3-SAT $\rightarrow$ MQ.  Replace $+$ with $\vee$, $\cdot$ with $\wedge$ , 1 with true,
0 with false.  If $c_i= x_{i_1} \vee x_{i_2} \vee x_{i_3}$ add
$x_{i_1} + x_{i_2} + x_{i_3} x_{i_4}$ and
$x_{i_1} \cdot x_{i_2} + x_{i_2} \cdot x_{i_3} + x_{i_1} \cdot x_{i_3} = x_{i_5}$
and $x_{i_4} + x_{i_5} + x_{i_4} \cdot x_{i_5} =1$.
\\
\\
{\bf Theorem:} The following problems are
NP Complete: SAT, $k$-SAT ($k>2$), k-clique, Vertex Cover, Independent set,
Subset Sum, Partition, Bin Packing, Hamilton circuit.
\emph{Clique/SAT reduction:} Each occurrence of a variable is a vertex, edges
between vertices if their occurrence in the clauses have same complementarity.
$k$ is number of clauses.
\\
\\
{\bf Hard core bit:} Let $f$ be a one-way function from
$\{0,1\}^n$ to $\{0,1\}^n$,
$x \in \{0,1\}^n$ ,
$r \in \{0,1\}^n$ ,
and let $G$ be a function that takes
$\{0,1\}^n$ to $\{0,1\}^{n+1}$ by $G(x,r)= f(x), r, <x,r>$.  Let $P$ be a
prediction function.  Goldreich-Levin: If there is an algorithm $A$ such
that $| Prob_r [A(f(x), r) = <x,r>] - {\frac {1} {2}} | \geq \epsilon$ then
there is an algorithm $I$ that produces a list $L$ of size $\leq
{\frac {1} {{\epsilon^2}}}$ with $x$ in $L$, (2) $I$ runs in time
polynomial in $n$ and ${\frac {1} {\epsilon}}$ and doesn't compute $f$.
A function is \emph{negligible:} smaller that inverse of any polynomial.
Witness: $w:\Sigma^* \rightarrow P( \Gamma^* )$.  \emph{Decision problem:}
$A_w \subseteq \Sigma^*$, $A_w= \{ x \in
\Sigma * | w(x) \ne 0 \}$.
\emph{Example:} $x \in \Sigma^*$ is an encoding of a Boolean Form.
$y \in \Gamma *$
is an encoding of a truth assignment.
$\#P$ is class of witnesses, $w$, such that: (i) there is a P-time algorithm
to decide if $x \in w(x)$ and (ii)
$\exists k \in N$ such that $\forall y \in
w(x),
|y| \leq |x|^k$.  $w \in \#P \rightarrow A_w \in NP$ and
$A \in
NP \rightarrow \exists w, A=A_w$.
\\
\\
{\bf Theorem:}
Counting perfect matchings of a bipartite graph is $\#P$ complete.
\\
\\
{\bf Finite State Machine:} Finite alphabet, $A$, finite states, $S$, two functions:
$\delta: S \times A \rightarrow S$ and
$\gamma: S \times A \rightarrow A$.\\
Finite State Automata is FSM without output.\\
\\
{\bf Regular expressions:}
Language $L$ is a subset of $A^*$.
\emph{Regular expression}, $R$ over alphabet, $A$ with letters 
$a \in A$: (1) $ \epsilon \in R$,
(2) $a \in A$, (3) $r^* \in R$ if $r \in T$,
(4) $r_1 r_2 \in R$ if $r_1 , r_2 \in R$,
(4) $r_1 \vee r_2 \in R$ if $r_1 , r_2 \in R$.
Language associated with a regular expression: (1) $ L(\epsilon)=\{ \epsilon \}$,
(2) $L(a)= \{a\}$, (3) $L(r^*) L(r)^*$,
(3) $L(r_1 r_2)= L(r_1 ) L(r_2)$,
(4) $L(r_1 \vee r_2) = L(r_1) \cup L(r_2)$.
$L$ is a \emph{regular language} if $\exists r \in R$ with $L=L(r)$.
\emph{Phrase structured Grammar}, 
$G$, consists of (1) Vocabulary $V$, (2) terminals (denoted by lower case letters) 
$T \subseteq V$, (3) variables or non-terminals $V \setminus T$ (denoted
by upper case letters), (4) a designated non-terminal $S$, called the start symbol, (5)
a finite set $P$ of productions: $\alpha \rightarrow \beta$.
$w \Rightarrow w'$ iff $\exists u,v, w=u \alpha v$ and $w'= u \beta v$.
\\
\\
{\bf Grammars:}
Grammar types are defined by production rule limitation: (1) Type 0: no limitations,
(2) Type 1: production rules of the form
$\alpha \rightarrow \beta$, $|\alpha| \leq |\beta|$ or
$\alpha \rightarrow \epsilon$,
(3) Type 2: production rules of the form $A \rightarrow \beta$,
(4) Type 3: production rules of the form
$A \rightarrow a$ or $A \rightarrow aB$,
(5) \emph{context free:} production rules of the form $A \rightarrow \beta$,
(6) \emph{context sensitive:} production rules of the form
$\alpha A \alpha' \rightarrow \alpha \beta \alpha'$,
(7) regular: production rules of the form
$A \rightarrow a$, $A \rightarrow aB$ or $S \rightarrow \epsilon$.
Backus-Naur form for type 2 context free grammar:
(i) ::= replaces $\rightarrow$,
(ii) non-terminals enclosed in brackets $<>$ and
(iii) all productions with the same non-terminal
LHS are combined into a single RHS.
\emph{Example:} 
$\langle sentence  \rangle$ ::= 
$\langle noun$ $phrase \rangle\langle verb$ $phrase \rangle$, 
$\langle noun$ $phrase \rangle$ ::= 
$\langle noun \rangle|\langle article \rangle\langle noun \rangle$,
$\langle noun \rangle$::= boy.
\\
\\
{\bf Theorem:}
A language $L$ can be generated by a type 3 (regular) grammar iff there is a finite
automaton $M$ that accepts $L$.
Pushdown automata (with infinite stack) recognize $L$ iff $L$ is context free.
$L$ is recognized by a linear bounded automata  (tape linearly bounded in length of
input) iff $L$ is context sensitive.
\\
\\
{\bf Minimizing state machines:} Two states, $s_i$, $s_j$, are $0$ equivalent if
the states have the same output for every input.  States are $k+1$ equivalent if they
have the same outputs for any input and their successor states are $k$ equivalent.
Minimization procedure:  Define $\pi_0$ as all states that are $0$ equivalent.  Do until
no further refinement happens: sub-partition $\pi_{k}$ into $\pi_{k+1}$ into subblocks are
$k+1$ equivalent.  This terminates.  When it does, merge equivalent states.
\\
\\
{\bf Pumping Lemma:}  Let $L$ be a finite state grammar accepted by a finite state
machine, $M$, with $n$ states.
If $\alpha$ is a string accepted by $M$ of length at least $n$, then
$\alpha = u||v||w$ where $u||v^i||w$ is also in $L$.\\
\\
{\bf Definition:}
Turing machines are FSMs with a bi-directionally infinite tape with a finite
number of pre-marked squares and an additional transition function
$\sigma: S \times A \rightarrow \{L, R, HALT\}$.
\\
\\
{\bf Huffman algorithm:}  Label each node with frequency.  As long as more than one
node is present, take the two nodes with the lowest frequency and combine them
into a single node with the two combinants as children.
New node has combined frequency.  Left subnode has lower of two frequencies,
right the higher.  Read code by traversing from root.  Left traversal at parent is
0, right, 1.
\\
Resulting code is prefix free.  Further $H(X) \leq l(x) \leq H(X)+1$.
\subsection{Concurrency}
\begin{verbatim}
ECMA Consistency
    1. Reads and writes cannot move before volatile read.
    2. Reads and writes cannot move after volatile write.

CompareExchange(ref int loc, int value, int comp) {
    Monitor.Enter;
    ret= loc;
    if(ret==comp) loc= value;
    Monitor.Exit;
    return ret;
}

class SpinLock {
    volatile int isEntered=0;   // 1 if lock acquired
    int Enter() {
        while(CompareExchange(isEntered,1,0)!=0);
        }
    Exit() {
        isEntered= 0;
    }
}

Memory Consistency Rules
    1. Behavior of Thread in isolation is unaffected
    2. Reads cannot move before lock
    3. Writes cannot move after lock
\end{verbatim}

\begin{multicols} {2} {
\begin{verbatim}

DPLL(C,A) {     
// C: clauses, A: literal assignments
//      Termination:
//              empty clause: unsatisfiable
//              empty set of clauses: satisfiable
    if(A is empty) 
        return SATISFIED;
    if(A has an empty clause) 
        return UNSATISFIABLE;
// unit clause is a clause with one literal
    if unit clause (l) occurs in A  
        return DPLL (assign(l,C), A + l));
    if l occurs with same polarity throughout
        return DPLL (assign(l,C), A + l));
    l= choose-literal(A);
    return DPLL (assign(l,C), A + l)) OR 
           DPLL (assign(not l,C), A + not l));
    }
Note: If A, B, C are p-free, 
(A | p) & (B|!p) &C) is inconsistent iff (A|B)&C is.

Chase(C,x) {
    set x to t;
    delete all clauses containing x from C;
    delete all occurences of !x from clauses in C;
    if (empty clause) 
        return UNSATISFIABLE;
    if (unit clause l) 
        return Chase(l,t);
    if (C is empty)  
        return SATISFIED;
\end{verbatim}
}
\end{multicols}
\begin{multicols} {2} {
\begin{verbatim}
Priority Queue (arrays start at 1 here)

ExtractMax(A) {
    if(heapsize(A)<1) 
        return error;
    max= A[1];
    A[1]= A[heapsize(A)];
    heapsize(A)=heapsize(A)-1;
    Heapify(A,1);
    return max;
}

Insert(A,k) {
    heapsize(A)=heapsize(A)+1;
    i= heapsize(A);
    while(i>1 & A[parent(i)]<k) {
        A[i]= A[parent(i)];
        i= parent(i);
        }
    A[i]= k;
    }
\end{verbatim}
}
\end{multicols}
\begin{multicols} {2} {
\begin{verbatim}
Select(A,k) {      
    // select kth element from A[1,...n-1]
    if(k==0) return min(A);
    // For randomized, choose x in A at random
    x= SideSelect(A);
    Set B= < y in A: y <=x>
    Set C= < y: y>x >
    if(k<|B|) return Select(B,k)
    return Select(C,|B|-k);
    }

SideSelect(A,k) {
    for(i<=0<=n=INT(size(A)/5))
        Sort successive 5 elements 
        // A[5i]<=A[5i+1]<=A[5i+2]<=A[5i+3]<=A[5i+4] 
    R= < A[5i+2] > , 0<=i<=n
    x= SideSelect(R,Size(R)/2);       
    // note x <= 3*INT((n-5)/10) elements.
    }
// Note E(T(n))= E(T(sn))+n,  x ~ 3/4
\end{verbatim}
}
\end{multicols}
\begin{multicols} {2} {
\begin{verbatim}
struct semaphore {
    int count;
    ProcessQueue queue;
    };

void P(semaphore s) {
    if(s.count>0) {
         (s.count)--;
    else
         s.queue.Insert(); // block
    }
void V(semaphore s) {
    if(s.queue.empty())
        (s.count)++;
    else
        s.queue.remove(); //schedule process
    }

shared semaphore s= 1;
     P(s);
     //critical section
     V(s);
\end{verbatim}
}
\end{multicols}
\begin{multicols} {2} {
\begin{verbatim}
Map()
Reduce()
Scan()      // || prefix
Scatter()
Gather()

Readers/writers
linear sweep

\end{verbatim}
}
\end{multicols}
Architecture and current PCs:
$P= C \times V^2 \times f$.  
\emph{Big endian} word: $0, 1, 2, 3$ (descending byte address).
\emph{Little endian} word: $3, 2, 1, 0$ (descending byte address).
\begin{center}
\begin{tabular} {|l|c|r|}
\hline
{\bf Optimization Level} & {\bf Description} & {\bf Level}\\
\hline
High & Procedure inlining & 3\\
Local & common subexpression & 1\\
Local & constant propagation & 1\\
Local & stack height reduction (expression tree) & 1\\
Global & global common subexpression & 2\\
Global & global constant propagation & 2\\
Global & code motion & 2\\
Global & induction variable elimination & 2\\
Global & loop unrolling & 4\\
Global & strip mining & 4\\
Arch specific & strength reduction & 1\\
Arch specific & pipeline scheduling & 1\\
Arch specific & branch offset & 1\\
\hline
\end{tabular}
\end{center}
Effect on performance of Bubblesort (100K items).  Base is 300MHz Sparc Ultra.
\begin{center}
\begin{tabular} {|r|r|r|r|r|}
\hline
{\bf Optimization level} & {\bf Relative performance} & {\bf Clocks} & {\bf Instructions} & {\bf CPI}\\
\hline
0 & 1.00 & 158,615 & 114,938 & 1.38 \\
1 & 2.37 & 66,990  & 37,470 & 1.79 \\
2 & 2.38 & 66,521 & 39,993 & 1.66 \\
3 & 2.41 & 65,747 & 44,993 & 1.46\\
\hline
\end{tabular}
\end{center}
SRAM: $.5-1 ns$, $4,000 \$/GB$.
DRAM: $50-70 ns$, $100 \$/GB$.
Disk: $10^7 ns$, $1 \$/GB$.  
Dram address setup: 1 memory cycle, access time: 15 cycles, data transfer: 1 cycle.
4-way interleave plus multiword block gets time down to 20 cycles on average.
Miss penalty to main: 500 cycles, to L2: 25 cycles.  TLB: 512 entries.  Miss: 100 cycles.
Miss percentage; .5-1.
Disk seek latency: 10 ms, rotational latency: 5 ms, transfer rate: 50 MB/s, MTTF:
$10^6$ hours.  Bus speed: system (800 MHz), NB (266 MHz), SB (33 MHz).  Bandwidth:
\begin{center}
\begin{tabular} {|r|r|}
\hline
{\bf Device} & {\bf Bandwidth} \\
\hline
Memory & 3.2GB/sec \\
Disk & 150 MB/sec\\
AGP & 2.1 GB/sec\\
PCI& 132 MB/sec\\
NIC & 20 MB/sec\\
\hline
\end{tabular}
\end{center}
{\bf Dwarves:}
Finite state machines, combinatorics, graphs, Structured/unstructured grids, dense matrix, sparse
matrix, map-reduce, backtrack/branch-and-bound, $N$-body, FFT, Graphical models.\\
\\
$LU${\bf-factorization:}  Let $A \ne 0$ be an $m \times n$ matrix.  There are permutation matrices
$P,Q$ such that $P^T A Q= LU$ where $L$ is lower triangular and $U$ is upper triangular.
$QR$-factorization: Let $X \in {\mathbb C}^{n \times p}$ have rant $p$ then
$x= QR$ where $Q$ is an orthogonal matrix and $R$ is an upper triangular matrix.  
$QR$-factorization via unitary operations is used in the least square approximation problem.
Spectral decomposition: $U^H A U= diag( \lambda_1 , \lambda_2 , \ldots , \lambda_n)$.
The eigenvalues of $X^H X$ are the sequence of singular values of $X$.
For a $p \times q$ matrix, row major storage is 
$A[1,1]= a[1]$,
$A[1,2]= a[2]$, \ldots, $A[2,1]=a[q+1]$, 
etc., 
and in general,
row major storage is $A[i,j]= a[(i-1)q+j]$,
column major storage is
$A[1,1]= a[1]$,
$A[1,2]= a[q+1]$, \ldots, $A[2,1]=a[2]$,  etc.,
and in general,
column major storage is $A[i,j]= a[(j-1)p+i]$.  One step of Gaussian Elimination:
$$
\left(
\begin{array}{cc}
\alpha_{11} & \alpha_{12}^T\\
\alpha_{21} & A{22}\\
\end{array}\right)=
\left(
\begin{array}{c}
\beta_1\\
b_2\\
\end{array}
\right)
\rightarrow
\left(
\begin{array}{cc}
\alpha_{11} & \alpha_{12}^T\\
0 & A{22}- \alpha_{11}^{-1}\alpha_{21}\alpha_{12}^T\\
\end{array}
\right)=
\left(
\begin{array}{c}
\beta_1\\
b_2- \alpha_{11}^{-1} \beta_1 \alpha_{21}\\
\end{array}
\right)
$$
{\bf Definition:}
A \emph{hard core predicate} of a one-way function, $f$ is easy to compute given $x$ but not
given $f(x)$.  If $f$ is a one-way function, $g(x,r)= (f(x),r)$ is a hard core predicate.
Let $B_x(y): \{0,1\}^n \rightarrow \{0,1\}$ be a probabilistic oracle with
$\epsilon$ advantage for $(x,y)$, that
is, $Pr(B_x(y)=(x,y))= {\frac {1+\epsilon} 2}$ and let $EQ_x(y)$ be an oracle for $x=y$.
Define 
$Gd_B= \{ y: B_x(y)= (x,y) \}$ and
$Bd_B= \{ y: B_x(y) \ne (x,y) \}$ so 
$|Gd_B|= {\frac {1 + \epsilon} 2} 2^n$ and
$|Bd_B|= {\frac {1 - \epsilon} 2} 2^n$.
\\
\\
{\bf Goldreich-Levin:}
Let  $R= \langle r_1, \ldots, r_m \rangle$ be a random selection of elements from
$[n]$, $z \in [n]$ and $b_j=(x,r_j)$.  Let $\langle S_j \rangle$ be a fixed
enumeration of the subsets of $[m]$.  $R[S]= \sum_{j \in S} r_j$.
Define the following algorithm:\\
\\
$STRONG-SC^{B_x}(z, r_1, \ldots, r_m, b_1, \ldots , b_m)$ \\
\jt $sum= 0;$ \\
\jt for($i=1; i \leq 2^m)$) \{\\
\jt \jt $R(S_i)= \sum_{j \in S_i} r_j;$ \\
\jt \jt $B[S_i]= \sum_{j \in S_i} b_j;$ \\
\jt \jt $c= B_x(z+R[S_i]) -b[S_i];$ \\
\jt \jt $sum+= c;$\\
\jt \jt \}\\
\jt return $sum > {\frac {2^m} 2}$;\\
\\
Define the following algorithm:\\
\\
$RECOVER^{B_x, EQ_x}(1^n)$ \\
\jt Pick $r_1, \ldots r_m \in [n]$ at random;\\
\jt for($i=1; i \leq 2^m)$) \{\\
\jt \jt $(b_1, \ldots, b_m)_2 = i-2;$\\
\jt \jt for($k=1; k \leq n$)\\
\jt \jt \jt $y^{(k)}= STRONG-SC^{B_x}(e_k, r_1, \ldots, r_m, b_1, \ldots, b_m);$\\
\jt \jt $y= y^{(1)}||y^{(2)} || \ldots || y^{(n)};$\\
\jt \jt if ($EQ_x(y)==1)$ \\
\jt \jt \jt return $y$;\\
\jt \jt \}\\
\\
{\bf Notation:}
Let 
$q_B$ be the number of calls to $B_x$,
$q_E$ be the number of calls to $EQ_x$,
$\epsilon$ be the advantage for $B_x$,
$n$ the length of the strings and $t$  the running time of
$RECOVER^{B_x, EQ_x}(1^n)$. We have the following:
\\
\\
{\bf Theorem:} Let $m$ be a parameter and $M=2^m$.  There is an algorithm, $A$ which
makes $q_B= nM$ calls to $B_x$, $q_E= M$ calls to $EQ_x$ and runs in time
$t=O(nM^2)$ which determines $x$ with probability 
at least $1-\delta, \delta= {\frac n {\epsilon^2 M}}$.\\
\\
First some lemmas and notation.  Let $R= \langle r_1, \ldots , r_m \rangle$ and
$X_1, X_2, \ldots , X_M: S \rightarrow R$ be a set of real valued random variables on
the sample space $S$.\\
\\
{\bf Lemma 1:} If 
$X_1, X_2, \ldots , X_M: S \rightarrow R$ are pairwise independent and
$X= X_1 + X_2 + \ldots + X_M$ then
$Var(X)= Var(X_1) + Var(X_2 )+ \ldots + Var(X_M)$.
\begin{quote}
\emph{Proof:}
$Var(X)= E(X^2)-E(X)^2= 
E(\sum_{i,j} X_i X_j) -
\sum_{i,j} E(X_i) E(X_j)
= \sum_i [E(X_i^2) - E(X_i)^2]- \sum_{i\ne j} [E(X_i X_j)- E(X_i)E(X_j)]$.  The
final bracketed sum is $0$ by pairwise independence.
\end{quote}
{\bf Lemma 2:} If 
$X_1, X_2, \ldots , X_M: S \rightarrow R$ are pairwise independent, 
$X= X_1 + X_2 + \ldots + X_M$  and $\mu= E(X)= \sum_{i=1}^M E(X_i)$ then
$Pr(|x-\mu| \geq A) \leq {\frac {\sum_{i=1}^M Var(X_i)} {A^2}}$.
\begin{quote}
\emph{Proof:}
By Chebychev, $Pr(|X-\mu| \geq A) \leq {\frac  {Var(X)} {A^2}}$ and by the previous lemma,
$Var(X)= Var(X_1) + Var(X_2) + \ldots + Var(X_M)$.
\end{quote}
{\bf Lemma 3:} Let $M=2^m$.
For any $z \in \{0, 1\}^n$,
$Pr(STRONG-SC^{B_x}(z, r_1, \ldots, r_m, b_1, \ldots , b_m) 
\ne (z,x)) \leq {\frac 1 {M \epsilon^2}}$.
\begin{quote}
\emph{Proof:} For $R= \langle r_1, r_2, \ldots, r_m \rangle$ and $S_1, S_2, \ldots,
S_{2^m}$ a fixed enumeration of $[m]$, define a random variable $X_i(R)=1$ if
$B_x(z+R[S_i])= (x,z)+ b[S_i]$ and $0$ otherwise.  $X_i(R)$ and $X_j(R)$ are pairwise
independent since there is an $r_l \in R$ that is included in \emph{either} the sum
$R[S_i]$ or $R[S_j]$ but not both.  $E(X_i)= Pr(z+R[S_i] \in Gd_B)= {\frac {1 + \epsilon} 2}$.
Since $E(X_i^2)= E(X_i)$, $Var(x_i)= E(X_1^2)=E(X_i)^2= E(X_i)-E(X_1)^2= E(X_1)(1-E(X_i))=
{\frac {1+\epsilon} 2}
{\frac {1-\epsilon} 2} =
{\frac {1-\epsilon^2} 4} $.  Note that $\mu= {\frac {M(1+\epsilon)} 2}$ and so by the
previous lemma,  $P(X < {\frac {M} {2}}) =
P(|X-\mu| > {\frac {M \epsilon} {2}})  \leq
{\frac {M(1-\epsilon^2)/4} {(M \epsilon/2)^2}} \leq {\frac 1 {M \epsilon^2}}$.
\end{quote}
{\bf Proof of Theorem:} Let $M=2^m$.
For any $z \in \{0, 1\}^n$,
$Pr(RECOVER^{B_x, EQ_x}(1^n) \ne x) \leq {\frac n {M \epsilon^2}}$.
\begin{quote}
\emph{Proof:}
The loop in $RECOVER$ calls $STRONG-SC^{B_x}$ $n$ times.  Each call is wrong with 
probability at most ${\frac 1 {M \epsilon^2}}$ so the probability that $RECOVER$ returns
the wrong answer is at most
${\frac n {M \epsilon^2}}$.
\end{quote}
Note that obtaining $x$ with $p= {\frac 1 2}$ occurs with $M= 2n\epsilon^{-2}$ and the
running time is $O(n^3 \epsilon^{-4})$ with
$q_B= O(n^2 \epsilon^{-2})$ and
$q_E= O(n \epsilon^{-2})$.
