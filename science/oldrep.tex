{\bf Representations (1):}
$M$ is a simple $R$-module if it has no non-trivial submodules.  Let $M$ be a non-zero
$R$-module.  The following are equivalent (``{\bf semisimple}''): (1) $M$ is a sum of simple
modules, (2) $M$ is a direct sum of simple modules, (3) if $N \subseteq M$ is a
submodule, there is another submodule $N'$: $M= N \oplus N'$.  
{\bf Schur:} If $f \in Hom_R(M,N)$ and $M, N$ are simple then $f=0$ or is an isomorphism.
If $M$ is simple, $A=End_R(M)$ is a division ring.
Let $M$ be a semi-simple $R$-module, $A=End_R(M)$ and $f \in End_A(M)$, 
if $m \in M, \exists r \in R: f(m)= rm$.  $End_R(V^n) \cong M_n(End_R(V))$.
{\bf Jacobson:}
Let $M$ be a semi-simple $R$-module, $A=End_R(M)$ and $f \in End_A(M)$, 
$m_1 , m_2 , \ldots , m_n \in M, \exists r \in R: f(m_i)=rm_i$.  Corollary:
Let $M$ be a faithful simple $R$-module, $D=End_R(M)$, if $M$ is finite dimensional
over $D$ then $End_D(M) \cong R$.  
If $R$ is a ring and $I$ is an ideal, say $I$ is simple if it is 
simple as a left module of $R$; say $I$ is semi simple if it is semi-simple if
it is semi-simple as a module and all simple left ideals are isomorphic.
If $R$ is a semi-simple ring then all non-zero $R$ modules are semi-simple.
Let $I$ be a simple left ideal in a semi-simple ring $R$ and let $M$
be a simple $R$-module; either $IM=M$ and $I \cong M$ or $IM=0$.
From now on
let $R$ be a semisimple ring.  $B_i= \sum_{I \subset R, I \cong I_i} I$.
If $I_j$ is not isomorphic to $I_k$ then $B_j B_k = 0$; $R= \sum B_i$ and
each $B_i$ is a two-sided ideal.  There are only finitely many isomorphism classes
of left ideals.  If $R= \bigoplus_{i=1}^t B_i$ and $1= \sum_{i=1}^t e_i$.
If $b_i \in B_i$ then $e_i b_i = b_i = b_i e_i$ and $B_i=Re_i$.  Each $B_i$ is
a simple ring.  If $M$ is a simple module it is isomorphic to some $I_k$ so there
are only finitely many isomorphism classes of simple $R$-modules.  Let $M$
be a non-zero $R$-module, define $M_i$ as the sum of all simple $R$ modules
isomorphic to $M_i$, then $\bigoplus_{i=1}^t B_i M, M_i= e_iM$.  A semi-simple ring,
$R$ is ring isomorphic to the direct product of simple rings.  Let $R$ be a simple
ring and $V$ a simple $R$-module with $D=End_D(V)$, then $V$ is a finite
dimensional vector space over $D$, $R \cong End_D(V) \cong M_n(D^o)$.
Let $B= b_1 \oplus \ldots \oplus B_n$ be a direct sum of simple algebras then
two sided ideals of $B$ are of the form $J_1 \oplus \ldots \oplus J_n$ where
the $J_i$'s are 2 sided simple ideals of the $B_i$'s.
Let $S_1 , S_2 , \ldots , S_r$ be distinct simple $A$-modules; for each
$i$, let $U_i$ be a direct sum of copies of $S_i$ and
$U= U_1 \oplus U_2 \oplus \ldots \oplus U_r$ then
$End_A(U)= End_A(U_1) \oplus End_A(U_2) \oplus \ldots \oplus End_A(U_r)$.  If
$S$ is a simple $A$-module then $End_A(nS) \cong M_n(End_A(S))$ and
if $F$ is algebraically closed then $End_A(S) \cong F$.
{\bf Wedderburn:}  Let $R$ be a semi-simple ring then 
(1) $R$ is isomorphic to the direct sum
of simple rings $B_1 , B_2 , \ldots , B_t$, (2) there are $t$ isomorphism classes
of simple$R$-modules; if $V_1 , V_2 , \ldots , V_t$ are representatives, let
$D_i=End_R(V_i)$ then $B_i \cong End_{D_i}(V_i) \cong M_n({D_i}^o)$, and
(3) $B_i V_j=0, i \ne j, B_i V_i = V_i$.  {\bf Maschke:}  If $G$ is a finite group and
$k$ is a field with $char(k) \nmid |G|$ then $kG$ is semi-simple.
Let $K=End_R(E)$ with $E$ semi-simple over $R$ and $f \in End_K(E)$; further,
let $x \in E$ then
$\exists \alpha \in R$: $f(x) = \alpha x$.  Proof: $E= Rx \oplus F$ let $\pi \in End_K(E)$
be the projection on the first factor.  $f(x)= \pi f(x) = f( \pi x) \in Rx$.
Jacobson: Let $K=End_R(E)$ with $E$ semi-simple over $R$ and $f \in End_K(E)$ let 
$x_i \in E, i= 1,2, \ldots, n$ then
$\exists \alpha \in R$: $f(x_i) = \alpha x_i$.
Let $R$ be a ring, $\psi \in End_R(R)$, $\exists \alpha \in R$: $\psi (x)= z \alpha$.
$\psi(x)= \psi(x1)= x \psi(1)$.
Rieffel:  Let $R$ be a ring without non-trivial two sided ideals.  Let $L$ be a non
zero left ideal and $R'=End_R(L)$, $R''=End_{R'}(L)$, then there is a natural map
$\lambda: R \rightarrow R''$.
Definition: $R$ is simple iff it has no non-trivial two sided ideals.
If $R$ is semi-simple, $R= R_1 \oplus R_2 \oplus \ldots R_k$ with each $R_i$
simple.  The decomposition is unique apart from order.
Proof:
Let $R_1$ be a minimal 2 sided ideal, $R= R_1 \oplus {\overline R_1}$,
$R_1= Re$, ${\overline R_2} = R(1-e)$.  Both are idempotent so sums and
products act on each summand separately.  Regularity is inherited by
the summands.  Now you can decompose ${\overline R_2}$ into a further sum.
We get $R= R_1 \oplus \ldots \oplus R_s$ and
$e= e_1 + e_2 + \ldots + e_s$, ${e_i}^2=e_1$, $e_i e_j =0$ and each $e_i$ is
in
$r_i$ and is in the center of $R_i$.
Two $FG$ modules afford equivalent representations iff they are isomorphic.  
Every irreducible ordinary representation of $G$ occurs as a component of
the regular representation $R(G)$.  The number of inequivalent irreducible
representations is the number of conjugacy classes of $G$.  If $\rho_1 ,
\rho_2 , \ldots , \rho_r$ are inequivalent representations and
$deg(\rho_i )= n_i$ then $dim(\rho_i ) = {n_i}^2$ and $\rho_i$ occurs
$n_i$ times in $R(G)$.  $|G|= \sum_{i=1}^r {n_i}^2$.
Proof:  Extend $F$ to a suitable algebraic extension so that the
center of $R_G$ is the direct sum of $r$ matrix rings: $R_1 , R_2 , \ldots
R_r$.
if $dim(r_i)= {n_i}^2$, $deg(\rho_i)= n_i$.  Since $dim(R_G)= |G|$,
$dim(R_G)= \sum_{i=1}^r {n_i}^2$.  Each $R_i$ is the direct sum of the
$n_i$ right ideals: $e_{11}R, \ldots , e_{{n_i}{n_i}}R$.  So $\rho_i$
occurs $n_i$ times in $R_G$. 
${\mathbb Z}(G)$ has an irreducible faithful representation iff it is cyclic.\\
\\
