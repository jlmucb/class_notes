{\bf Cube Attack:}
For any polynomial $P$ and term $t$, write $P= tPt+Q$, where
the variables in $P_t$ are disjoint from those in $t$
and each term in Q misses at least one variable from $t$.
$P_t$ is called the \emph{superpoly} of $t$ in $P$.  
A \emph{maxterm} of $P$ is any product $t$ of variables whose superpoly has degree $1$ 
(i.e., is a linear or affine function which is not a constant).
Example:
$P(x_1,x_2,x_3,x_4,x_5) =x_1x_2x_3+x_1x_2x_4+x_2x_4x_5+x_1x_2+x_2+x_3x_5+x_5+1 $.
Let $t=x_1x_2$, $P(x_1,x_2,x_3,x_4,x_5)= x_1x_2(x_3+x_4+1)+(x_2x_4x_5+x_3x_5+x_2+x_5+1)$,
the superpoly of $x_1x_2$ in $P$ is $(x_3+x_4+1)$.
\\
\\
{\bf Theorem:} 
$\sum_t (tP_t+Q) = P_t$
\begin{quote}
\emph{Proof:}
$\sum_t Q = 0$ since every numerical value appears an even number of times.
$\sum_t (tP_t+Q) = 
(\sum_t t) P_t$ since the only term in the sum that is non-zero is the one where all the vareables are $1$.
\end{quote}
{\bf To apply in attack:}
For each candidate maxterm $t$, choose pairs of values for all the other variables 
$X’$ and $X”$. Verify that the numerical values of the subcube sums satisfy the linearity test:  
$P_t(X’)+P_t(X”)= P_t(X’+X”)+P_t(0)$.
If the test succeeds multiple times, obtain the linear superpoly by checking the 
numeric effect of flipping each key bit $x_i$.
\\
\\

