\section{Algebraic Geometry}
\subsection{Basics}
{\bf Theorem:}
Every conic in the affine space over $R$ is equivalent under an affine transformation to
one of the following:
(1) $X^2 + Y^2 + P = 0$ (ellipse, point, empty set),
(2) $X^2 - Y^2 + P = 0$ (hyperbola, intersecting lines),
(3) $X^2 + Y + P = 0$ (parabola),
(4) $X^2 + P = 0$ (parallel lines, point empty).  In projective space (1), (2), (3) are
equivalent. In the projective space over ${\mathbb C}$, they are all projectively equivalent.
\\
\\
{\bf Definitions:}
Let $R=k[x_1, \ldots, x_n]$, $W= k^n= {\mathbb A}^n$ for the \emph{affine} setup and
$R=k[x_0, x_1, \ldots, x_n]$ (where $f \in R$ is a \emph{homogeneous} polynomial),
$W= k^{n+1} \setminus \{ {\vec 0}\}= {\mathbb P}^n$ under the usual equivalence 
for the \emph{projective} setup.
If $S \subseteq R$, $V(S)= \{ x \in W: f(x)=0, \forall f \in S \}$.  $V(S)$ is an
\emph{affine} (resp \emph{projective}) \emph{algebraic set}.  An algebraic set
is \emph{irreducible} if is cannot be expressed as the union of two non-trivial
algebraic sets.  An affine irreducible algebraic set is an \emph{algebraic variety}.
If $S= \{f\}$, $V(S)$ is called a \emph{hypersurface}.
If $V \subseteq W$, $I(V)= \{ f \in R: f(x)=0, \forall x \in V \}$.
If $\langle S \rangle= I$ as an ideal in R, $V(S)=V(I)$.
$rad(I)={\sqrt I}= \{a: a^n \in I\}$.  $I$ is a \emph{radical} ideal if $I={\sqrt I}$.
Roughly, radical ideals $\leftrightarrow$ varieties,
prime ideals $\leftrightarrow$ subvarieties,
maximal ideals $\leftrightarrow$ points.  
If $V$ is an algebraic variety, $I(V)$ is a prime ideal (proof below)
and $\Gamma[V]= R/I(V)$ is
called a \emph{coordinate ring}.
${\overline R}= k(x_1, \ldots, x_n)$ is the quotient field of $R$; members of ${\overline R}$
induce \emph{rational} maps.
${\mathcal O}_P(V)$ denotes the rational functions on $V$ defined on $P$.
The \emph{Zariski} topology on $W$ is defined by identifying the $V(S)$ with
\emph{closed sets}.
\\
\\
{\bf Theorem:}
Let $k$ be algebraically closed. 
There is a one to one correspondence between polynomial maps $\varphi: V \rightarrow W$
and the homomorphisms ${\tilde {\varphi}}: \Gamma[W] \rightarrow \Gamma[V]$.
Let ${\mathcal T}(V,k)= \{f: f:V \rightarrow W \}$.  If $\varphi: V \rightarrow W$,
$\tilde {\varphi}: {\mathcal T}(W,k) \rightarrow {\mathcal T}(V,k)$.  
\begin{quote}
\end{quote}
{\bf Theorem:}
$k \subseteq \Gamma[V)] \subseteq {\mathcal O}_P(V) \subseteq k(V)$.
\begin{quote}
\end{quote}
{\bf Definitions:}
Two affine varieties
$V, W$ are \emph{isomorphic} if $\exists \phi, \psi: 
\phi \circ \psi= id_W$. Suppose $F_i$ is a collection of functions. 
The \emph{multiplicity of a root} at a point ${\vec a}$ is
$f(t)= gcd(F_{1}({\vec a} + L(t)), \ldots , F_{m}({\vec a} + L(t)))$ where $L(t)$ is a line
through $O={\vec 0}$.
By transferring $P$ to ${\vec 0}$ and express $F= F_m + \ldots + F_n$, $F_m$ is
a form of degree $m$.
$L$ touches $X$ at $O$ if its
intersection multiplicity is greater than 1.
Locus of points touching $X$ at $x$ is the \emph{tangent space}, $\Theta_{x,X}$.
Two varieties $V, W$ are \emph{birationally equivalent} if there are \emph{rational} maps
$\varphi_1: V \rightarrow W$ and $\varphi_2: W \rightarrow V$ between them.
$f(x,y)$ is \emph{rational} if $\exists \phi, \psi$: $f(\phi(t), \psi(t))=0$.
Any conic (2nd order equation) in two variables has either infinitely many
rational solutions or none.
\\
\\
{\bf Theorem:}
Two curves are birationally equivalent iff their fields of functions are isomorphic.
\begin{quote}
\end{quote}
{\bf Theorem:}
Every irreducible curve of degree 2 is rational.
$x^{n}+y^{n}=1$ is not rational for $n>2$.
\begin{quote}
\end{quote}
{\bf Definition:} Suppose $\phi: A^n(k) \rightarrow A^m(k)$, $f \in k[y_1, \ldots, y_m]$; 
the \emph{pullback} is $\phi^*: \phi^* \circ f= f \circ \phi$.
A \emph{discrete valuation ring (``DVR'')} is a Noetherian,
local domain whose maximal ideal is principal.
If a form, F, does not vanish on an irreducible projective variety
X then $dim(X_{F})= dim(X)-1$.
\\
\\
{\bf Theorem:}
The following are equivalent: (1) The set of non-units in $R$
form an ideal; (2) $R$ has a unique maximal ideal.  
The following are equivalent and define a DVR: 
(1) $R$ is Noetherian and its maximal ideal is principal; (2) 
$\exists t \in R:
\forall  0 \ne z \in R: z=ut^n$, where $u$ is a unit.  
\begin{quote}
\emph{Proof:}
We show $1 \rightarrow 2$.  First uniqueness.  If $ut^m=vt^n$ for units $u,v$,
$ut^{m-n}=v$ but then $t^{m-n}$ is a unit.  For existance, since ${\mathfrak m}= (t)$,
for a non-unit, $z \in {\mathfrak m}$, 
$z= z_1t$ and if $z_i$ is a non-unit, $z_{i+1}= z_i t$ so
$(z_1) \subset (z_2) \subset \ldots$ and since $R$ is Noetherian eventually
$(z_n)=(z_{n+1})$.  $z_{n+1}= vz_{n}, v \in R$ $z_n= t vz_n$ so $vt=1$ but $t$ is not a unit.
\end{quote}
{\bf Theorem:} The pole set is an algebraic variety.  
$\Gamma[V]= \bigcap_{P \in V} {\mathcal O}_P(V)$.
\begin{quote}
\end{quote}
{\bf Theorem:}
If $S_1 \subseteq S_2$ then $I(S_1) \supseteq I(S_2)$.  Every algebraic set
is the intersection of hypersurfaces.
\begin{quote}
\emph{Proof:}
By the Hilbert basis theorem, $I(V)= (f_1, f_2, \ldots, f_r)$ and so
$V= V(f_1) \cup \ldots \cup V(f_r)$.
\end{quote}
{\bf Theorem:}
Every closed set is the union of finitely many irreducible ones.
Every irreducible closed set is birationally isomorphic to a hypersurface in $A^{n}$.  
\begin{quote}
\end{quote}
{\bf Theorem:}
(1) An algebraic set, $V$, is irreducible iff $I(V)$ is prime.
(2) If $k$ is algebraically closed, $I(V(f))= (f)$.
(3) If $I$ is radical, $I(V(I))= I$.
\begin{quote}
\emph{Proof:}
(3) follows from Nullstellensatz.
\end{quote}
{\bf Theorem:}
Let $R=k[x,y]$. If $f, g \in R$ have no common factor, $V(f) \cap V(g)$ is finite.
\begin{quote}
\emph{Proof:}
$k(x)[y]$ is a PID, $Af+Bg=1$ over $k(x)$; now clear denominators.
\end{quote}
{\bf Theorem:}
If $I$ is prime, $V(I)$ is irreducible.  There is a $1-1$ correspondence between
prime ideals and irreducible algebraic sets; there is a $1-1$ correspondence between
maximal ideals and points.
\begin{quote}
\end{quote}
{\bf Theorem:}
Let $V$ be an algebraic set then $V= V_1 \cup \ldots V_r$, $V_i$, irreducible,
$V_i \nsubseteq V_j, i \ne j$.
\begin{quote}
\emph{Proof:}
We use the following (follows from axiom of choice).  Let 
${\mathcal S}$ be a non-empty collection of ideals in a Noetherian ring, $R$,
then 
${\mathcal S}$ has a maximal element.  From this we conclude and collection of
algebraic sets has a minimal element. \\
Now let
${\mathcal S}= \{ V:
\textnormal{algebraic, } V
\textnormal{ is not a finite union} 
\textnormal{ of irreducible}
\textnormal{ algebraic sets} \}$ .
Let $V$ be a minimal element of
${\mathcal S}$.  $V= V_1 \cup V_2$ and this is a contradiction.  For the second condition,
throw out the $V_i$ violating it.
\end{quote}
{\bf Theorem:}
${\mathcal O}_P(V)$ is a Noetherian local domain.  Further, the maximal ideal
${\mathfrak M}_P(V)$ is generated elements of $f \in \Gamma[V]$ for which $f(P)= 0$.
\begin{quote}
\emph{Proof:}
STS $I \subseteq
{\mathcal O}(V)$ is finitely generated.
Since $\Gamma[V]$ is Noetherian, $(f_1, \ldots, f_r)= I \cap \Gamma[V]$.  Claim
$(f_1, \ldots, f_r)= I$ and an
${\mathcal O}(V)$ ideal, for if $f \in I$, $\exists b \in \Gamma[V]: b \ne 0$ and
$bf \in \Gamma[V]$ so $bf= \Gamma[V] \cap I$, $bf= \sum_i a_i f_i, a_i \in \Gamma[V]$ and
$f= \sum_i {\frac {a_i} {b_i}} f_i$.
\end{quote}
{\bf Definition:}
$I(P, F \cap G)= dim_k({\mathcal O}_P({\mathbb A}^2)/(F, G)$.
\\
\\
{\bf Theorem:}
$|V(I)| < \infty$ iff $R/I$ is finite dimensional over $k$.
\begin{quote}
\emph{Proof:}
If $P_1, \ldots, P_r \in V$, we can fine $f_i \in R: f_i(p_j)= \delta_{ij}$ so
$r \le dim_k(R/I)$.
On the other hand,
if $V(I)= \{ P_1 , P_2 , \ldots , P_r \}, P_i= (a_{i1}, a_{i2}, \ldots, a_{in})$,
define $f_j= \prod_i (x_i - a_{ij}), F_j \in I(V(I))$ so $F_j^N \in I, {\overline F}_j= 0$
and ${\overline X}_j^{rN}$ is a $k$-linear combination of
$ {\overline 1}, {\overline x}_j^{1}, \ldots, {\overline x}_j^{rN-1} $ and
$ {\overline x}_1^{m_1}, \ldots, {\overline x}_n^{m_n}; m_i < rN $ is a set of generators.
\end{quote}
{\bf Theorem:}
Let $I$ be an ideal of $R$ ($k$, algebraically closed), $V(I)= \{P_1 , \ldots , P_N
\}$ and $
{\mathcal O}_i= {\mathcal O}_{P_i}({\mathbb A}^n) $,
$\exists \varphi: R/I \rightarrow \prod_i 
{\mathcal O}_i/(I{\mathcal O}_i)$ induced
from the natural homomorphisms $\varphi_i: R \rightarrow {\mathcal O}_i/(I{\mathcal O}_i)$.
\begin{quote}
\end{quote}
{\bf Theorem:} $P$ is a simple point of $f$ iff ${\mathcal O}_P(V(f))$ is a DVR.
If $L$ is a non-tangent line $l + {\mathcal O}_P(V)$ is a uniformizing parameter.
\begin{quote}
\emph{Proof:}
WLOG, $P=(0,0), L= ax$ and $y=0$ is the tangent.
${\mathfrak M}_P(V(f))= (x,y)$ whether $P$ is simple or not.
Suppose $f= y(1+g(x,y)) - x^2 h(x)$ and $yg= x^2h \in \Gamma(f)$ so
$y= x^2 h g^{-1}$ since $g(P) \ne 0$ and 
${\mathfrak M}_P(V(f))= (x)$.
\end{quote}
{\bf Theorem:}
Suppose ${\mathfrak M}= {\mathfrak M}_P(F)$ denotes the maximal ideal of 
${\mathcal O}= {\mathcal O}_P(F)$.
$ 0 \rightarrow {\mathfrak M}^n/{\mathfrak M}^{n+1}
\rightarrow {\mathcal O}/{\mathfrak M}^{n+1}
\rightarrow {\mathcal O}/{\mathfrak M}^{n} \rightarrow 0$.
$\chi(n)= dim({\mathcal O}/{\mathfrak M}^n)=$ Hilbert polynomial.  
\\
\\
{\bf Theorem:}  Let $P$ be a point on an irreducible curve, $f$, for all sufficiently
large $n$, $m_P(f)= dim_k({\mathfrak M}_P(f)^n/{\mathfrak M}_P(f)^{n+1})$.
\begin{quote}
\end{quote}
{\bf Divisors:}
Suppose $E: y^2= x^3-x$ and $f(x,y)= {\frac x y}$.  On $E: {\frac x y}= {\frac y {x^2-1}}$.
If $u_P$ is a \emph{uniformizer}, $f= u_P^r g, r \in {\mathbb Z}, g(P) \ne 0, \infty$ and
we define $ord_P(f)=r$.  In this example, $u_{(0,0)}= y$ is a uniformizer at $(0,0)$ of
$E: y^2= x^3-x$.  Since $x= y^2({\frac 1 {x^2-1}})= y {\frac y {x^2-1}}= {\frac x y}y=x$,
$ord_{(0,0)}(x)= 2$.  In general, for $P=(x_0, y_0) \in E$, $u_P$ can be taken as any line
through $P$ not tangent to $E$, thus we can take $u_P= x-x_0$ when $y_0 \ne 0$ and
$u_p= y$ when $y_0=0$.  For example, if $E: y^2= x^3 + 72$ then $(-2, 8) \in E$.
Since $f(x,y)= x+y-6$ vanishes at $(-2,8)$ and
$f(x,y)= (x+2) + (y-6) = (x+2)( 1 + {\frac {(x+2)^2 - 6(x+2) + 12)} {y-8}})$,
$ord_P(f)=1$, $u_{\infty}= {\frac x y}$.
\\
\\
{\bf Weak Bezout:} If two curves of dimension $m$ and $n$ meet at more than $mn$
points (counting multiplicity) then they have a common component.
\begin{quote}
\emph{Proof:}
Strategy of Proof: (S-1) $\#(C_1 \cap C_2 \cap {\cal A}^2) \le
dim({\frac R {(f_1 , f_2)}}) \le n_1 n_2$, (S-2) first inequality is an equality,
(S-3) first inequality can be strengthened to
$I(C_1 \cap C_2,P) \le dim({\frac R {(f_1 , f_2)}})$, 
(S-4) inequality in 4 is an equality,
(S-5) $I$ is invariant under projective transformations --- transform so the line at
infinity does not intersect $C_1 \cap C_2$.
Notation: Let $f_1 (x, y)$, and $f_2 (x, y)$, defining curves
$C_1$ and $C_2$, have dimension $m, n$ respectively.
$R= k[x, y]$, $(f_1 (x, y), f_2 (x, y))= R f_1 + R f_2$.
\\
\\
\emph{S-1:}
$C_1 \cap C_2 \leq dim_k ({\frac {R} {(f_1 , f_2 )}}) \leq mn$. [Argument:
If $P_1, P_2, \ldots P_r$ are distinct, $\exists h_i (x,y)$ with
$h_i (P_j ) = \delta_{ij}$, so if there are $r$ common roots of $f_1$ and
$f_2$,
$\sum_{i=1}^r c_i h_i (x,y) = r_1 f_1 (x,y) + r_2 f_2 (x,y)$ implies $c_i=
0$.] \\
Let $R_d$ be polynomials of degree $\leq d$ then
$dim_k (R_d )= \phi (d) = {\frac {(d+1)(d+2)} {2}}$.
Let $W_d = R_{d-m }f_1 + R_{d-n}f_2$, for $d \geq (m+n)$.
$R_{d-m}f_1 \cap R_{d-n} f_2 = R_{d-m-n} f_1 f_2$.
$dim_k ( R_d ) - dim_k (W_d )= mn$.
$g= \sum_i^l c_j g_j$ has a non-trivial dependency for $l>mn$ with
$g \in W_d$.
\\
\emph{S-2:}
Second inequality is equality if $C_1 \cap C_2$ don't meet at infinity. 
Let $f^*$ be the homogeneous polynomial consisting of the highest degree terms
in $f$.  If $\infty \notin C_1 \cap C_2$ then $f_1^*$, $f_2^*$ have no common factor.
If $f_1^*$ and $f_2^*$ have no common factor then $(f_1 , f_2) \cap R_d= W_d$.
Under the conclusion of the previous sentence, if $d \ge n_1 + n_2$ then
$dim({\frac R {(f_1 , f_2)}}) \ge n_1 n_2$ which proves the result.
\\
Define ${\mathcal O}_P= \{ F \in K(x, y): F(P)$ exists $\}$, 
${\mathfrak M}_P= \{ f \in {\mathcal O}_P: f(P)= 0 \}$.
${\mathfrak M}_P$ is a unique maximal ideal of $O_P$.  $(f_1, f_2)_P= f_1 O_P + f_2 O_P$.
Now define $I(C_1 \cap C_2; P)= dim( {\frac {{\mathcal O}_P} {(f_1 , f_2)_P}})$.
\\
\emph{S-3:} 
${\frac {{\mathcal O}_P} {(f_1 , f_2)_P}} \le {\frac {R} {(f_1, f_2)}} < \infty$.
${\mathcal O}_P = (f_1 , f_2)_P +R$.
If $P \notin C_1 \cap C_2$ then $I(C_1 \cap C_2, P)=0$;
If $P \in C_1 \cap C_2$ then $(f_1 , f_2)_P \subset {\mathfrak M}_P$;
$I(C_1 \cap C_2; P) = 1 + dim({\frac R {(f_1 , f_2)_P}})$ iff
$(f_1 , f_2) = {\mathfrak M}_P$.
If $P \in C_1 \cap C_2$ and $r \ge dim({\frac {{\mathcal O}_P} {(f_1 , f_2)_P}})$ then
${\mathfrak M}_P^r \subset (f_1, f_2)_P$.  If 
$P, Q \in C_1 \cap C_2 \cap {\cal A}^2, \psi \in {\mathcal O}_P$
then $\exists g \in R$: 
$g= \psi \jmod{(f_1 , f_2)_P}$ and 
$g= 0 \jmod{(f_1 , f_2)_Q}$ if $P \ne Q$.
\\
\emph{S-4:} Kernel of natural map $R \rightarrow \prod_{P \in (C_1 \cap C_2 \cap {\cal A}^2)}
{\frac {{\mathcal O}_P} {(f_1 , f_2)_P}}$ is just $(f_1 , f_2)$ where the natural map is:
$f \mapsto (\ldots, f \jmod{(f_1, f_2)}, \ldots)$.  
$dim({\frac R J})
= \sum_P dim( {\frac  {{\mathcal O}_P} {(f_1 , f_2)}})
= \sum_P I(C_1 \cap C_2, P)$.  The last equality holds iff $J \subset (f_1 , f_2)$.
Define $L= \{ g \in R : gf \in (f_1 , f_2) \}$ and $1 \in L$.  $L$ is an ideal
$(f_1 , f_2) \subset L \subset R$.  
$P \in {\cal A}^2$, $\exists g \in L: g(P) =0, P \in L$.  
$\exists a \in k: 1 \notin L +R(x-a)$ and
$\exists b \in k: 1 \notin L +R(x-a) + R(y-b)$.
\\
\emph{S-5:}
Properties of intersection multiplicity.  $I((y-x^m), y; 0)= m$.  Show the definitions
make sense and that there is a line $L$ which does not contain any of the intersection
points.  The proof requires knowing there are only a finite number of points in the
intersection.
\end{quote}
{\bf Genus} for non-singular curve: $g_f = {\frac {(n-1)(n-2)} {2}} - d$.
$L(D)= \{ f: K(C)^*: div(f) \ge -D \}$.  $l(D)= dim (L(D))$.
\\
\\
{\bf Reimann Roch Theorem:} Let
$X$ be a non-singular projective plane curve.  $\exists g \ge 0: \forall D,
dim_k(L(D)) \ge deg(D)+1-g$.  The minimum such $g$ is called the genus.
\begin{quote}
\end{quote}
{\bf Resultants:} 
The $r$ forms $f_1,f_2,...,f_r$ with indeterminate coefficients possess a resultant
system of integral polynomials $b_k$ such that for special values of the 
coefficients in $K$ (algebraically closed).  
The vanishing of all resultants is a necessary and 
sufficient condition for
$f_1= f_2= ... =f_r= 0$ to have a solution $\ne 0$.  The $b_k$ are homogeneous
in the coefficients of every form $f_i$ and 
satisfy $x_k^{s_r}b_k=0 \jmod {(f_1, f_2,...,f_r)}$.  
\\
\\
{\bf Bezout's Theorem.}  If $f, g$ are two curves of degree $n,m$ respectively that
have no common component then they intersect in $mn$ points counting multiplicity.
Notes: A homogeneous system 
$f_1= f_2= ... =f_r= 0$ has solutions
$(\xi_1^{(a)}, \xi_2^{(a)},...  \xi_n^{(a)}), a= 1,2,3,...,q$.  
Set $l_x=u_1x_1+u_2x_2+...+u_nx_n$.
Form resultant system $b_1(u), ..., b_t(u)$.  The common zeros of $b_1, ...$ are
$\prod l_a$.  By Nullstellensatz, 
$(\prod_a l_a)^{\tau}=0 (b_1(u),b_2(u),...,b_t(u))$
$\rightarrow D(u)= \prod {l_a}^{\rho_a}$ and
$(b_i(u))^{r_i}=0 (\prod l_a)$
$\rightarrow D(u)= (f_1,...,f_r,l)$.  $R(u)$ is the same as the u-resultant so
$\sum \rho_a$ is the degree of $R(u)= \prod deg(f_i)$.
\\
\\
\emph{Example:} $F_1(x,y,z)=x^2+y^2-10z^2=0$,
$F_2(x,y,z)=x^2+xy+2y^2-16z^2=0$, add $F_3(x,y,z)= u_0z+u_1x+u_2y$.
$Res_{1,2,2}(F_0,F_1,F_2)=(u_0+u_1-3u_2)(u_0+2{\sqrt 2}u_1+{\sqrt 2}u_2)
(u_0-2{\sqrt 2}u_1-{\sqrt 2}u_2)$.  
Solutions are $(1,-3,1)$, $(-1,3,1)$, 
$(2{\sqrt 2}, 2{\sqrt 2}, 1)$,  
$(-2{\sqrt 2}, -2{\sqrt 2}, 1)$.
\\
\\
\emph{Example proof with generics:} $F_1(X,Y,Z)=X-Y^2$, $F_2(X,Y,Z)=XY-Z$,
$(X,Y,Z) \rightarrow (t^2,t,t^3)$ is generic because it is a solution for any
specialization of $t$ and any solution is obtainable this way.
\\
\\
Let $D$ be a domain and $\Omega=\Omega_D= {\overline {D(t_1,t_2,...)}}$ is called a universal
field.  Note that $\Omega \leftrightarrow$ {prime ideals over} $D[X_1,...]$.
\\
\\
{\bf Theorem:} $x_1, ... x_n \in \Omega$.  $I= \{ f: f(x_1,...,x_n)=0 \}$ is a prime ideal.
If $I$ is a prime ideal and $1 \notin I$ then $I$ has a generic 0.  Any extension
$K(\alpha_1 , ..., \alpha_m)$ can be embedded in $\Omega$.
\\
Hints: look at $E=D[X]/I$.  Under this homomorphism the image of $(X_1,...X_n)$ is generic.
\\
\\
{\bf Theorem:} If $\xi_1 ,..., \xi_n$ are elements of an arbitrary extension of $K$ then
If $\Re =K[X_1,...,X_n]$ and $\wp= \{ f: f(\xi_1,...,\xi_n)=0 \}$.  
$1 \notin \Re$ and $\wp$ is a prime ideal.  Every prime ideal has a generic element.
\begin{quote}
\end{quote}
{\bf Theorem:}  Any ideal $g=(f_1, ..., f_n)$ which has no zeros in $\Omega$ is the unit ideal.
\emph{Proof:} Otherwise a maximal ideal would correspond to a non-zero generic point.
\begin{quote}
\end{quote}
{\bf Extension of Nullstellensatz}:  If $p_1, ..., p_s$ all vanish at the common zeros of
$(f_1,...,f_n)$, then $\exists q$ such that powers of the $p_i$'s of degree $q$
are in $(f_1,...,f_n)$.
\begin{quote}
\emph{Proof:}  For $s=1$, this is the simple Nullstellensatz.  Let the exponent for each $i$ be
$q_i$.  Set $q= q_1 + q_2 + ...+ q_n -n +1$.
Nullstellensatz bound: $\rho \leq 13 d^n$ where $d$ is the degree and $n$ is the number of
variables.
\end{quote}
{\bf Theorem:}  
Let $N_q$ be the number of products $X_j$ of degree $q$.
Suppose $F_1, F_2,..., F_r$ are forms.  $(0,...,0)$ is the only common zero
iff all products $X_j$ can be expressed as linear combinations of the $X_{ki}F_i$ 
with coefficients in $K$.
Note:  This means they are linearly independent.  So there are other common zeros is
there are fewer than $N_q$.  Note that $X_1, ..., X_n$ satisfy the extension conditions.  
If the $X_{ki}F_i= \sum a_{kij}X_j$
are not linearly independent, the determinant families, $R_i(a)$, form a resultant set.
\begin{quote}
\end{quote}
{\bf Multivariate resultants:}  If we fix degrees 
$d_0,d_1, \ldots d_n$ 
then there is a unique
polynomial $Res \in {\mathbb Z}[u_i , \alpha]$ such that (a) if $F_0, F_1, \ldots , F_n$
are homogeneous polynomials of degrees
$d_0,d_1, \ldots d_n$ then $F_0= \ldots = F_n=0$ has a nontrivial solution over
${\mathbb C}$ iff 
$Res(F_0, \ldots , F_n)=0$, (b) 
$Res(x_0^{d_0}, \ldots , x_n^{d_n})=1$, (b) 
, (c) $Res$ is irreducible
in ${\mathbb C}[u_i, \alpha]$.  If $PP=PP(x_1, x_2, \ldots, x_n)$ is a set of
power products in the $x_i$, there are $N_m= {{m+n-1} \choose {n-1}}$  $PP$'s of degree
$m$.  \emph{Example:} 
$A_3=a_3x^2b_3y^2+c_3z^2$,
$A_2=a_2x+b_2y+c_2z$,
$A_1=a_1x+b_1y+c_1z$.  $S_i= {\frac {PP_i^d} {x_i^d}}$,
$S_1= \langle x^2, xy, xz \rangle, S_2=\langle y^2, yz \rangle, S_3= \langle z^2 \rangle$.
\\
\\
$\left(
\begin{array}{c|cccccc}
 & x^2 & xy & xz & y^2 & yz & z^2 \\
\hline
xA_1 & a_1 & b_1 & c_1 & 0 & 0 & 0 \\
yA_1 & 0 & a_1 & 0 & b_1 & c_1 & 0 \\
zA_1 & 0 & 0 & a_1&  0 & b_1 & c_1 \\
yA_2 & 0 & a_2 & 0 & b_2 & 0 & 0 \\
zA_2 & 0 & 0 & a_2 & 0 & b_2 & c_2 \\
A_3 & a_3 & 0 & 0 & b_3 & 0 & c_3 \\
\end{array}
\right)
$.
\subsection{Elliptic Curves}
{\bf Basic Definitions:}  An \emph{elliptic curve} is a smooth projective curve of
genus $1$ with a distinguished point $O$.  In the plane, (affine)
elliptic curves are described by an equation of the form
$E(F): y^2 +a_1 xy + a_3 = x^3 +a_2 x^2 + a_4 x + a_6$, $a_i \in F$ 
called the \emph{general affine Weierstrauss form (GWF)}.
An equation of the form
$E(F): y^2 = x^3 + a x + b$, $a, b \in F$, is said to be in
\emph{special affine Weierstrauss form (SWF)} and
sometimes we denote this curve by $E_{a,b}(F)$.  
If $P=(x_P, y_P)$ is a solution of
the SWF, $P$ is said to be on the elliptic curve $E_{a,b}(F)$.
\\
\\
{\bf Definitions:}
\emph{Projective coordinates:}
$(X_1 , Y_1, Z_1) \sim (X_2 , Y_2, Z_2)$, 
$X_1 = \lambda^c X_2$, $Y_1 = \lambda^d Y_2$, $Z_1 = \lambda Z_2$, 
$c,d \in {\mathbb Z}^{>0}$ and usually $c=d=1$.
\emph{Jacobian projective coordinates:} $\infty= (1:1:0)$ and $-(X:Y:Z)=(X:-Y:Z)$.
\emph{Standard projective coordinates:} $\infty= (0:1:0)$ and $-(X:Y:Z)=(X:-Y:Z)$.
\\
\\
{\bf Addition Formula for SWF:}
$Y^{2}Z= X^{3}+aXZ+bZ^{3}$, $P_{i}= (x_{i},y_{i})$, $O=(0:1:0)$.
We want to calculate $R=P_1+P_2$.  If $P_1$ or $P_2$ is $O$, result is obvious.
If $x_1=x_2$ and $y_1= - y_2$, $R=O$.
If $x_1 \ne x_2$, set
$\lambda ={\frac {y_{2}-y_{1}} {x_{2}-x_{1}}}$.  If $x_1 = x_2$ and
$y_1 \ne - y_2$, set $\lambda= (3{x_1}^2+a)(y_1+y_2)^{-1}$.
In either case, $x_3= \lambda^2 - x_1 -x_2$,
$y_3= \lambda(x_1-x_3)-y_1$ and $R= (x_3 : y_3 : 1)$.
$|\epsilon_{p}| \leq 2{\sqrt p}$.  $E_{a,b}(F)$
defined by $f(x,y)= y^2-(x^3+ax+b)$ is \emph{non-singular}
if there is no point $(x,y)\in E_{a,b}(F)$ such that
${\frac {\partial f} {\partial x}} (x,y)=0$ and
${\frac {\partial f} {\partial y}} (x,y)=0$ or equivalently
provided
$x^3_ax+b$ does not multiple roots.
$f(x,y)$ has multiple roots iff $-(4a^3 + 27 b^2)=0$.  
Usually we pick $Z$ axis tangent to
$O$ and then $(0,1,0)$ as the point at $\infty$. 
\\
\\
{\bf Theorem:}  
If $C_1$ and $C_2$ are two non cubic curves that meet
at eight points, they meet at nine points (counting multiplicity).
\begin{quote}
\emph{Proof:}
$C_1: a_1x^x+a_2y^3+ \ldots + a_{10}xyz$.
$C_2: b_1x^x+b_2y^3+ \ldots + b_{10}xyz$.  The set of cubics passing through eight
points form a two parameter family: $C= \alpha C_1 + \beta C_2$.
$C_1(P_i)=0=C_2(P_i), i \le 8 \rightarrow C(P_9)=0$.
\end{quote}
{\bf Observation:}  The ``eight point theorem'' gives a quick proof of associativity by
looking at the intersection of six lines (read down and across) in the following:
$$
\left(
\begin{array}{l | c c c}
 & l: & m: & n:\\
\hline
r: & P & X & Q+R\\
s: & Q & R & {\overline {QR}}\\
t: & {\overline {PQ}} & P+Q & O\\
\end{array}
\right)$$
Here, $X= {\overline {P(Q+R)}}= {\overline {R(P+Q)}}$ and the eight point theorem
is applied to 
$C_1: l(X,Y,Z) m(X,Y,Z) n(X,Y,Z)= 0$ and $C_2: r(X,Y,Z) s(X,Y,Z) t(X,Y,Z)= 0$.
\\
\\
{\bf Definition:}  $Z=0$ is tangent to PSWF at $[0,1,0]$ but the tangent line intersects
the curve three times. This is called a \emph{flex point}.
\\
\\
{\bf Mordell's Theorem:}  If a non-singular cubic curve over $k$ has a rational point then
the rational points are finitely generated as a $k$-module.
Use $H({\frac m n})= max(|m|, |n|)$. Let $P=(x,y)$.  Define $H(P)= H(x)$ and
$h(P)= log(H(P))$.  From now on assume $C$ is given by
$y^2= x^3 + ax^2 + bx +c$.
\begin{quote}
\emph{Proof:} To prove it, need four lemmas:
\\
\\
\emph{Lemma 1:} There are a finite number of points $P$: $h(P) < M$.
\\
\\
\emph{Lemma 2:} Fix $P_0$ on $C$, $\exists K_0(P_0 , a, b, c):  h(P+P_0) \leq
2h(P)+ k_0$.
\\
\jt Show that if $P$ is on $C(Q)$, $P=({\frac m {e^2}}, {\frac n {e^3}})$.  Then
show $n \leq KH(P)^{\frac 3 2}$.  Use this to get
$k_0$.
\\
\\
\emph{Lemma 3:} Fix  $\exists K(a, b, c):  h(2P) \geq
4h(P) - K$.
\\
\\
\emph{Lemma 4:} $|\{C(Q):2C(Q)\}| < \infty$.
\\
\\
For lemma 4, assume $y^2= x^3+ax^2+bx$ (so the curve
always has a rational point), and use $\Gamma= C(Q)$ and
$\Delta=2\Gamma$.  Define the map
$\phi(x,y)= (x+a+{\frac b x}, y{\frac {x^2 - b} {x^2}})$.
Define $\psi$ similarly.  Note that $\psi ( \phi (P))= 2P$ and
$ker(\phi) = \{0, (0,0)\}$. ${Q^*}^2$ $\alpha(x,y)= x \jmod
{{Q^*}^2}$.  $im(\phi) \subseteq ker(\alpha)$.
Let $p_i | b$, $i= 1,2, \ldots t$ then $|\Gamma: \phi(\Gamma)| \leq
2^{t+1}$.
$|\Gamma:\phi(\Gamma)| \leq 2^{t+1}$.  Use the following lemma: If $A$ and
$B$
are abelian $A \rightarrow B \rightarrow A$ and
$|B:\phi(A)| < \infty$,
$|A:\phi(B)| < \infty$, then $|A:2A| \leq
|B:\phi(A)| |A:\phi(B)|$.
\\
\\
\emph{Proof given lemmas:} Let $Q_0 , \ldots, Q_{m-1}$ be the coset
representatives.  $P-Q_{i_1}= 2P_1$ is in the subgroup, $P_1-Q_{i_2}= 2P_2$,
repeatedly doing this yields:
$P= Q_{i_1} + 2 Q{i_2} + \ldots + 2^{m-1}Q_{i_m} +
2^m P_m$, $h(P_j) \leq {\frac 3 4} h(P_{j-1})$.
Since there are a finite
number of $Q_i$ there's a $k'$ so that $h(P-Q_i) \leq 2 h(P) + k'$ for all
$P$.  Using the inequalities $h(P_j) \leq {\frac {h(P_{j-1})} 2} +
{\frac {k+k'} 4}$.
So the group is generated by the $Q_i$ and the (finite number
of) points of $ht \leq {\frac {k+k'} 4}$.
\end{quote}
{\bf Theorem:}
Let $C$ be a non-singular cubic curve $C: y^2= x^3 + ax^2 + bx +c$.  Set
$D= -4a^3 c + a^2 b^2 + 18 abc -4b^3 -27c^2$.  Let $\Phi$ be the set of
points of finite order.  Let $\phi$ be the reduction map mod $p$.  If
$(p, 2D)= 1$ then $\phi$ is an injection into $C(F_p )$.
\\
\\
{\bf Nagel-Lutz Theorem:} Same as above with $P(x,y)$ as a rational point of finite order
$y=0$ or $y|d^2$.
\begin{quote}
\end{quote}
{\bf Functions on Elliptic Curves:}
$f_1(x,y) = {\frac {s(x,y)} {t(x,y)}}$ and
$f_2(x,y) = {\frac {u(x,y)} {v(x,y)}}$ are said to be \emph{equivalent} on the
elliptic curve $E$, denoted by $f_1 \sim_E f_2$ iff $(sv-ut)= 0 \jmod {E}$.
Polynomials can be uniquely $x$ or $y$ reduced but not rational functions.
\emph{Examples:}  Let $E: y^2-(x^3+x^2+x)$ then
$ {\frac {x}{y-x}} \sim_E {\frac {y+x}{x^2+1}}$.  
Put
$F(X,Y,Z)= \{ {\frac x y} \}= [X, Y]$ then $F[0,1,0] =[0,1] \sim_E 0$ (a zero) while
if $G(X,Y,Z)= \{ {\frac y x} \}= [Y, X]$ then $G[0,1,0] =[0,1] \sim_E \infty$.
For $E: y^2= x^3-x$, $\{ {\frac y {x^2-1}} \} \sim_E \{ {\frac x y} \}$ and
$[YZ, X^2-Z^2] \sim_E [X,Y]$.  
For $E: y^2= x^3-x$, 
$\{x \}$ has a $0$ at $(0,0)$,
$\{y \}$ has a $0$ at $(-1,0), (0,0), (1,0)$,
$\{ {\frac x y} \}$ has a $0$ at $(-1,0), (1,0)$.
If $E(a,b)$ is non-singular, $E$ is irreducible and
we can embed $k[x,y]/(E)$ in the field of fractions $K(E)$ 
and we can define a map from $K(R) \rightarrow
{\overline K} \cup \{\infty\}$.
\\
\\
{\bf Definition:}
A \emph{morphism} between two elliptic curves $C_1$ and $C_2$ is a rational map
$\varphi: (C_1, I_1) \rightarrow (C_2 , I_2)$, 
$\varphi(x,y)= (f(x,y), g(x,y)), g, g \in K(C_1 )$, such that
$\varphi(P+Q)= \varphi(P)+\varphi(Q)$.  $\varphi(P)= [n]P$ is a morphism.  A
morphism is an \emph{isomorphism} if $\exists \psi$, a morphism: $\psi \circ \varphi= [1]$.
An \emph{isogony} is a surjective morphism with finite kernel preserving the base point.
If $\psi: E_1 \rightarrow E_2$ induces a $\psi^*: K(E_2) \rightarrow K(E_1)$.
$\psi^*(f)= f \circ \psi$.  $deg(\psi)= ker(\psi)$.
\\
\\
\emph{Example of dual isogony:} 
$E: y^2= x^3 + x^2 +x$,
$E': (y')^2= (x')^3 -2 (x')^2 -3 (x')$.  
$\psi(x,y)=
( {\frac {y^2} {x^2}}, {\frac {y(1-x^2)} {x^2}})$,
$\varphi(x',y')= ( {\frac {(y')^2} {4(x')^2}}, {\frac {(y')(-3-(x')^2)} {8(x')^2}},)$ 
and $\varphi \circ \psi= [2]$.
\\
\\
{\bf Definition:}
Suppose $E= E_{a,b}(K), char(K) \ne 2,3$.  Let $x_1= \mu^2x$ and $y_1= \mu^3 y$ then
$(x_1 , y_1) \in E_{a', b'}(K)$ with $a'= \mu^4 a$ and $b'= \mu^6b$.  Two 
elliptic curves related in this way are said to be \emph{isomorphic}.
\\
\\
{\bf Theorem:}
If $K= F_{p^m}, p \ne 2, 3$ and 
$E_K^{(1)}(a, b) \cong E_K^{(2)}({\overline a}, {\overline b})$ iff 
$\exists u \in K^*$ such that $u^4 {\overline a}=a$ and
$u^6 {\overline b}=b$ under the map $(x,y) \mapsto (u^2 x, u^3 y)$.
\begin{quote}
\end{quote}
{\bf Definition:}
$j(E)= 1728 {\frac {4a^3} {4a^3+27b^2}}$ is called the
\emph{$j$-invariant}. 
\\
\\
{\bf Theorem:}
$j(E)$ is invariant under 
the transformation above (i.e. - two isomorphic curves have the same
$j$-invariant) and, conversely, two curves with the same $j$ value are related in 
this way (and are
thus isomorphic in the elliptic curve defined over the algebraic closure).
\begin{quote}
\end{quote}
{\bf Theorem:}
If $j(E_1)=j(E_2)$ then
$\exists \mu \in {\overline K}, \mu \ne 0: a_2= \mu^4 a_1, b_2= \mu^6 b_2$.  
\begin{quote}
\end{quote}
{\bf Homogeneous forms:}
Let $G(u,v)$ be a homogeneous polynomial and $(u_0, v_0) \in {\mathbb P}^1_K$,
$\exists k \ge 0$ and $H(u, v)$ with $H(u_0, v_0) \ne 0: G(u,v)= (v_0 u -u_0 v)^k H(u,v)$.
Any line in ${\mathbb P}^2_k$ can be parameterized by 
$(x,y,z)= (a_0 u + b_0 v, a_1 u + b_1 v, a_2 u + b_2 v)$.
$L$ intersects $C$ to order $n$ at $P=(x_0 : y_0 : z_0)$ if ${\overline C}(u,v) =
(v_0 u - u_0 v)^n H(u,v)$ in the foregoing theorem; denote this as $ord_{L, P}(C)=n$,
$ord_{L, P}(C)= \infty$ if ${\overline C}$ is identically 0.  
If $L_1, L_2$ are lines, $ord_{L_1, P}(P)= 1$
or $\infty$.  If $C$ is a curve defined by $C(x,y,z)=0$, $C$ is non singular at $P$ if
$(C_x, C_y, C_z) \ne 0$ in which case the tangent line is $C_x X + C_y Y + C_z Z =0$.  If
$C$ is non-singular at $P$ there is a line in ${\mathbb P}^2_K$ that intersect $C$ to order at
least $2$.
\\
\\
{\bf Theorem:}
The number of equivalence classes of elliptic curves over $K$ is
$2q+6$, $2q+2$, $2q+4$, $2q$ according to $q = 1, 5, 7, 11 \jmod{12}$.
If $K= F_{2^m}$ and $E_K(a, b): y^2 + xy = x^3 + a x^2 +b$ then
$E_K^{(1)}(a, b) \cong E_K^{(2)}({\overline a}, {\overline b})$ iff 
$b= {\overline b}, Tr(a)=Tr({\overline a})$ and if so $\exists s: {\overline a}= s^2+s+a$ 
under the map $(x,y) \mapsto (x, y+sx)$.
\begin{quote}
\end{quote}
{\bf Division Polynomials:}
Let $E(K): y^2+a_1xy+a_3y= x^3 +a_2 x^2 + a_4 x + a_6$ be an elliptic curve 
$m \in {\mathbb Z}, P = (x,y)$, 
$b_2 = a_1^2 + 4a_2$,
$b_4 = a_1 a_3 + 2a_4$,
$b_6 = a_3^2 + 4a_6$,
$b_8 =  a_1^2 a_6 + 2 a_4 a_6 - a_1 a_3 a_4 + a_2 a_3^2 -a_4^2$ then
$\exists \psi_m(x,y), \omega_m(x,y), \theta_m(x,y) \in K[x,y]$ such that
$[m]P=({\frac {\theta_m(x,y)} {(\psi_m(x,y))^2}}, {\frac {\omega_m(x,y)} {(\psi_m(x,y))^3}})$.
$\psi_1 = 0$, $\psi_1 = 2$,
$\psi_2 = 2y+a_1x+a_3$, 
$\psi_3 = 3x^4+b_2x^2+3b_4x^2+3b_6+b_8$,
$\psi_4 = (2x^6+b_2x^5+5 b_4 x^4 + 10 b_8 x^2 +(b_2 b_8-b_4 b_6)x+ b_4 b_8 - b_6^2) 
\psi_2$,
$\psi_{2m+1} = \psi_{m+2} \psi_m^3 - \psi_{m-1}\psi_{m+1}^3, m \ge 2$,
$\psi_{2m} = 
{\frac {\psi_m} {\psi_2}}
(\psi_{m+2} \psi_{m-1}^2 - \psi_{m-2} \psi_{m+1}^2)$,
$\theta_m = x \psi_m^2- \psi_{m+1} \psi_{m-1}$,
$\omega_m = 
{\frac {1} {4 y}} 
(\psi_{m+2} \psi_{m-1}^2- \psi_{m-2} \psi_{m+1}^2), m>2$.
Note that $deg(\psi_m)= O(m^2)$.
\\
\\
{\bf Definition:}
An \emph{endomorphism} is a homomorphic map between and an elliptic 
curve and itself that is expressible as a \emph{rational function}.
If $\alpha$ is an endomorphism and $P=(x,y),
\alpha(X+Y) = \alpha(X) + \alpha(Y), \alpha(x,y)= (r_1(x,y), r_2(x,y))$.  Because
$y^2= x^3+ax+b$, we may assume 
$\alpha(x,y)= (r_1(x), y r_2(x))$; if $r_1(x)= {\frac {p(x)} {q(x)}}$, the degree of
endomorphism is $max(deg(p(x)), deg(q(x)))$.  This endomorphism
$\alpha$ is a \emph{separable endomorphism} if $r_1'(x) \ne 0$.
The \emph{Frobenius endomorphism} is
$\varphi(x,y)= (x^q , y^q)$. 
\\
\\
{\bf Definition:}
$E[n]= \{P \in E({\overline K}): nP= \infty \}$.
\\
\\
{\bf Theorem:} (1) If $char(K) \ne 2$ $E[2]= {\mathbb Z}_2 \oplus {\mathbb Z}_2$; 
if $char(K) = 2$ $E[2]= {\mathbb Z}_2$ or $0$.
(2) If $char(K) \nmid n$ or is $0$, $E[n]= {\mathbb Z}_n \oplus {\mathbb Z}_n$.  (3)
If $char(K)=p \mid n, n=p^r n'$ then
$E[n]= {\mathbb Z}_n \oplus {\mathbb Z}_{n'}$ or
$E[n]= {\mathbb Z}_{n'} \oplus {\mathbb Z}_{n'}$.  Proof
uses division polynomials.  Let
$E$ be an elliptic curve over $F_q$.  Then 
$E(F_q)= {\mathbb Z}_n$ or
${\mathbb Z}_{n_1} \oplus {\mathbb Z}_{n_2}$ with $n_1 \mid n_2$.  
\begin{quote}
\emph{Proof:}
Apply the above theorem.
$\phi_n$ induces a linear transformation on $E[n]$.
$E[n]= Z_{n_1} \oplus \ldots \oplus Z_{n_k}$ by ther FTOAG.  Let $l \mid n_1$ so
$l \mid n_j$. $E[l] \subseteq E[n]$, so it has order $l^k$.  But the 
endomorphism induced by multiplication by $n$ has order $n^2$, so $k=2$.  Since
this map, annihilates $Z_{n_1} \oplus Z_{n_2}$, $n_1, n_2 \mid n$, and since
$n^2= \#E[n]$, $n_1=n_2=n$.
\end{quote}
{\bf Theorem:} If $\alpha$ be a separable endomorphism of $E_{A,B}(\overline{F_q})$, 
$deg((\alpha)= |ker(\alpha)|$, otherwise $deg((\alpha)> |ker(\alpha)|$.
\begin{quote}
\emph{Proof:}
$\alpha(x,y)= (r_1(x), yr_2(x))$, $r_1(x)= {\frac {p(x)}{q(x)}}$; since $\alpha$
is separable, $p'q-q'p \neq 0$.  Let $S= \{x \in \overline{K}: (pq'-qp')q=0\}$.
There are $(a,b)$ on the curve: $a \neq 0 \neq b$, $(a,b) \neq O$.
$deg(p-aq)= deg(\alpha)$, and $a \notin r_1(S)$.  There are exactly $deg(\alpha)$
points: $\alpha(x,y)= (a,b)$.  STS there are no nultiple roots.  If $x_0$ is a multiple
root, $pq'-qp'(x_0) =0$.  If $\alpha$ is not separable, the same argument holds but
the equation has at least one solution with multiple roots so it has fewer roots than
$deg( \alpha )$.
\end{quote}
{\bf Theorem:} $\alpha: E_{A,B}(\overline{K}) \rightarrow E_{A,B}(\overline{K})$ is surjective.
\begin{quote}
\emph{Proof:}
If $p-aq$ is not constant and $x_0$ is a root, $q(x_0) \neq 0$.  Choose $y_0^2= x_0^3+Ax_0+B$.
$\alpha(x_0,y_0)= (a, b')$, $b'= \pm b$.  If $b'=b$, we're done.  If 
$\alpha(x_0 , y_0 )= (a, -b')= (a, b)$.  Since $E(\overline{K})$ is infinite and
$ker( \alpha )$ is finite, only a finite number of points map onto a points
with a single $x$ coordinate.  Therefore either $p$ or $q$ is not constant.
There is a unique $a$ such that $p-aq$ is constant.  So there are at most two
points, namely, $(a, \pm b )$ not in the image.  Pick another
$P_1$: $\alpha(P_1 )=(a_1 , b_1 )$.  $(a_1, b_1 ) + (a,b) \neq (a, \pm b )$.
There is a unique $P_2$: $\alpha (P_2 ) =  ( a_1 , b_1 ) + (a, b)$ so
$\alpha (P_2 - P_1 )= (a,b)$ and
$\alpha (P_1 - P_2 )= (a,-b)$.
\end{quote}
{\bf Theorem:} $\phi_q^n-1$ is separable and $|ker(\phi_q^n-1)|= |\#E(F_q)|$.  
\begin{quote}
\emph{Proof:} 
$\phi$ is a homomorphism.  It has degree $q$ and $(x^q)'=0$.
\end{quote}
{\bf Theorem:}  Let $E_{A, B}$ be an elliptic curve.  Fix $(u,v) \in E$.  Define
$f(x,y), g(x,y)) = (x,y) + (u,v)$. Then ${\frac {\frac {df(x,y)} {dx}} {g(x,y)}}= {\frac 1 y}$.
\begin{quote}
Use the addition formula and the fact that $2y y'= 3x^2+A$ to get
$(x-u)^3u({\frac {df(x,y)}{dx}}-g(x,y))=
v(Au+u^3+v^3-Ax-x^3-y^2) + y(-Au-u^3-v^3+Ax+x^3-y^2)$.  Now use $B=y^2-x^3-Ax$ to get the result.
\end{quote}
{\bf Theorem:}  Let $\alpha_i, i= 1,2,3$ be endomorphisms.  
Write $\alpha_i(x,y)= (R_{\alpha_i}, S_{\alpha_i}y)$.  Suppose $\exists c_{\alpha_1}, c_{\alpha_2}:
{\frac {R_{\alpha_i}'} {S_{\alpha_i}}}= c_{\alpha_i}, i=1,2$.  Then
${\frac {R_{\alpha_3}'} {S_{\alpha_3}}}= c_{\alpha_1} + c_{\alpha_2}$.
\begin{quote}
\emph{Proof:} 
${\frac {\partial x_3} {\partial x_1}} = {\frac {y_3} {y_1}}$ and
${\frac {\partial x_3} {\partial x_2}} = {\frac {y_3} {y_2}}$.
${\frac {\partial x_j} {\partial x}} = c_{\alpha_j}{\frac {y_j} {y}}$.  Now apply the
chain rule.
\end{quote}
{\bf Theorem:} Let $E_{A,B}$ be an elliptic curve, $n \in {\mathbb Z}$ and
$n(x,y)= (R_n(x), yS_n(x))$.  Then
${\frac {R_{n}'} {S_{n}'}}= n$.
\begin{quote}
\emph{Proof:} 
From the inverse formula if it holds for $n>0$, it holds for $n<0$.  It holds for $n=1$ and the
result above shows if it holds for $n$, it holds for $n+1$.
\end{quote}
{\bf Theorem:} $r, s \in \mathbb{Z}$, $r \neq 0 \neq s$ then $r\phi_n+s$ is separable iff $p \nmid s$.
$(x,y) \in E(F_q)$ iff $\phi_p(x,y)= (x,y)$.
\begin{quote}
\emph{Proof:}
$r(x,y)= (R_r(x), y S_r(x))$. 
$(R_{r \phi_q(x)}(x), y_{r\phi_q}(x))= (R_r^q, y(x^3+Ax+B)^{\frac {q-1} 2} S_r^q(x))$.
Thus $c_{r \phi_q}= R_{r \phi_q}'/S_{r \phi_q}=0$.
$R_{r \phi_q +s} /S_{r \phi_1 +s} = c_{r \phi_q} + c_s = s \neq 0 \jmod{p}$.
\end{quote}
{\bf Theorem:} $deg(a \alpha + b \beta)= a^2 deg( \alpha ) + b^2 deg( \beta ) + ab (deg (\alpha + \beta )
- deg( \alpha ) -deg( \beta )$.
\begin{quote}
\emph{Proof:}
Consider $\alpha_n$ and $\beta_n$ given by the matricies when viewed as maps in $E[n]$.  The matrix calculation
is straigntforward.
\end{quote}
{\bf Theorem:} Let $E$ be an elliptic curve over $K$ and $n$ a positive integer, $char(K) \nmid n$.  There
is a pairing $e_n : E[n] \times E[n] \rightarrow \mu_n$ (the Weil pairing) with 
(1) $e_n$ bilinear in each variable, (2) $e_n$, non-degenerate, (3) $e_n(T,T)=1$,
(4) $e_n(T,S)= e(S, T)^{-1}$, $e_n( \sigma T , \sigma S)$ for $\sigma \in Aut ( \overline{K} )$, and
(6) $e_n( \alpha (S) , \alpha (T))= e_n(s, T)^{deg( \alpha )}$, for all seperable automorphisms of $E$.
If the coefficients are in $F_q$, the statement holds for $\phi_p$.
\begin{quote}
\emph{Proof:}  The proof uses divisor theory.
\end{quote}
{\bf Theorem:} Let $\{T_1 , T_2 \}$ be a basis in $E[n]$.  Then $e_n(T_1 , T_2 )$ is a primitive $n$-th
root of unity.
\begin{quote}
\emph{Proof:}  Suppose $e_n(T_1 , T_2 )= \zeta$, $zeta^d=1$.  $e_n(T_1, d T_2)=1$ and
$e_n(T_2, dT_2)= e_n(T_2 , T_2 )^d= 1$.  Let $S= a T_1 +b T_2$.  
$e_n(T_1, dT_2)= e_n(T_1 , dT_2)^a e_n(T_t, dT_2)^b=1$.  This is true for all $S$ so $dT_2= \infty$.  But
this can hold iff $n \mid d$, so $\zeta$ is  a primitive root.
\end{quote}
{\bf Theorem:} Let $\alpha$ be and endomorphism of $E$ over $K$ and $n$, a positive integer
not divisible by $char(K)$.  Then $deg( \alpha_n )= deg( \alpha ) \jmod{n}$.
\begin{quote}
\emph{Proof:} By the previous result, $\zeta= e_n(T_1 , T_2 ) $ is primitive. So
$\zeta^{deg( \alpha )}= e_n(T_1 , T_2 )= e_n(aT_1 + c T_2, bT_1 + d T_2)= \zeta^{ad-bc}$.
\end{quote}
{\bf Theorem:} $r, s \in \mathbb{Z}$, $(r,q) \neq 1$ then $deg(r\phi_n-s)= r^2q+s^2- r s a$, where
$a= q+1-\#E(F_q)$.
\begin{quote}
\emph{Proof:} 
$deg(r \phi_q - s)= r^2 deg( \phi_q ) + s^2 deg(-1) + r( deg( \phi_q - 1) -
deg( \phi_q ) - deg(-1))$.
\end{quote}
{\bf Theorem:} $a= q+1-\#E(F_q)$.  $\phi_q^2-a\phi_q +q=0$ and $a=Tr((\phi)_q)_m \jmod{m}$.
\begin{quote}
\emph{Proof:}
Consider $\alpha= \phi_q^2 -a \phi_q +q$ as an endomophism in $E[m]$.
$|ker(\phi_q-1)|= deg(\phi_q-1)=det((\phi_q)_m-I)$.  By Cayley Hamilton,
$(\phi_q^2)_m -a \phi_q +q = 0 \jmod{m}$.  This is true for infinitely many
$m$ so $\alpha=0$.
\end{quote}
{\bf Hasse's Theorem:} Let $E_q$ be an elliptic curve then 
$ q+1- 2 {\sqrt q} \le \#E_q \le q+1+ 2 {\sqrt q}$,
$\#E_q=q+2-t$, $t$ is the Frobenius trace. 
\begin{quote}
\emph{Proof of Hasse:}
Let $\psi$ be the Frobenius map.  
$\#E_p = |ker([1]-\psi)|$.  First note that $deg([1])=1$ (in fact, $deg([n]) =n^2$).
$deg(\psi)= p$.  Also note that $deg(a+b)-deg(a)-deg(b)=B(a,b)$ is bilinear.
$0 \le deg([t] + [2] \psi)=t^2 -4p -2tB[1, -\psi]= 4p-t^2$; so
$(deg([1]-\psi) - deg([1]) -deg(\psi))^2 \le 4p$ but the first term is $\#E(F_p)$.
\end{quote}
{\bf Theorem:}
If $\alpha \ne 0$ is a separable endomorphism of $E$, 
$deg(\alpha)= \#ker(\alpha)$, otherwise
$deg(\alpha)> \#ker(\alpha)$.  
The endomorphism $[n]P \mapsto Q$ has degree $n^2$; most endomorphisms are of this form.
If $char(K) \nmid n$ then $E[n]= {\mathbb Z}_n \oplus {\mathbb Z}_n$. 
If $E[n] \subseteq E({\mathbb K})$ then $\mu_n \in K$.  Given $E_q(a,b), n \ge 1$, (1)
$ker(\phi_q^n -1)= \#E_{q^n}(a,b)$ and 
$\phi_q^n -1$ is separable
$\#E_{q^n}(a,b)= deg(\phi_q^n -1)$.
If $\alpha$ is separable, then $deg(\alpha)= \#ker(\alpha)$.  
$|E({\overline {F_p}})[m]|= m^2$ if $(m,p)=1$.
\begin{quote}
\end{quote}
{\bf Theorem:}
Let $E(F_q)$ be an elliptic curve $E(F_q) \approx {\mathbb Z}_n$ or
${\mathbb Z}_{n_1} \times {\mathbb Z}_{n_2}, n_1 \mid n_2 $.  
The Frobenius endomorphism has degree $q$ and is not separable.
\begin{quote}
\end{quote}
{\bf Definitions:}
If $E_q$ is an elliptic curve over
a finite field of characteristic $p$, $E_q$ is said to be \emph{supersingular} if
$E_q[p]= \{ \infty \}$.
(1) $char(F) \ne 2,3$, $(x,y) \mapsto ({\frac {x-3a_1^2-12 a_2} {36}}, 
{\frac {y - 3 a_1 x} {216}} - {\frac {a_1^3 + 4 a_a a_2 -12 a_3} {24}})$, 
sends the general equation to 
$E_q(a,b): y^2= x^3 + ax +b, \Delta = -16(4 a^3 + 27 b^2)$.
(2) $char(F) = 2, a_1 \ne 0$, $(x,y) \mapsto 
(a_1^2x+{\frac {a_3} {a_1}}, y+{\frac {a_1^2 a_4 - a_3^2} {a_1^2}})$,
sends the general equation to 
$E_q(a,b): y^2 + xy = x^3 + ax +b, \Delta = b$.  This is \emph{non-supersingular}.
(3) $char(F) = 2, a_1 = 0$, $(x,y) \mapsto 
(x+a_2, y)$, sends the general equation to 
$E_q(a,b): y^2 + cy = x^3 + ax +b, \Delta = c^4$.  This is supersingular.
(4) $char(F) = 3, a_1^2 \ne - a_2$, $(x,y) \mapsto 
(x+{\frac {d_4} {d_2}}, y+ a_1 x + a_1 {\frac {d_4} {d_2}} +a_3)$, 
$d_2= a_1^2+a_2, d_4= a_4-a_1 a_3$,
sends the general equation to 
$E_q(a,b): y^2= x^3 + ax +b, \Delta = - a^3 b$.  This is non-supersingular.
(5) $char(F) = 3, a_1^2 = - a_2$, $(x,y) \mapsto 
(x, y+ a_1 x + a_3)$, sends the general equation to 
$E_q(a,b): y^2= x^3 + ax +b, \Delta = - a^3 $.  This is supersingular.
\\
\\
{\bf Theorem:}  $q=p^m$, $\exists E_q: \#E_q= q+1-t$ iff
(i) $t \ne 0 (p), t^2 \le 4q$; or, (ii) $m= 1(2)$ and either (a) $t=0$ or (b)
$t^2=2q, p=2$, or (c) $t^2=3q, p=3$; or, (iii) $m=0 (2)$ and either (a)
$t^2=4q$ or (b) $t^2=q, p \ne 1 (3)$ or (c) $t=0, p \ne 1 (4)$. $E_{p^m}$ is supersingular
iff $p \mid t$.  $E_q = {\mathbb Z}_{n_1} \oplus {\mathbb Z}_{n_2}$ and 
$ n_2 \mid n_1 \mid (q-1)$.  
\begin{quote}
\end{quote}
{\bf Divisors:}
Given $E(K)$, $P \in E({\overline K})$, define $D= \sum_j a_j [P_j], a_j \in {\mathbb Z}$ and
$deg(D)= \sum_j a_j$.  $Div^0(E)$ are the \emph{divisors} of degree 0.  
If $f$ is a function on $E$,
$div(f) = \sum_P ord_P(f) [P] \in div(E)$.  Two divisors $D_1, D_2$ are said to be
\emph{equivalent} if $D_1-D_2 = (D)_E$.
\\
\\
{\bf Theorem:} If $P-Q$ is a divisor on an elliptic curve $E$, $P-Q= c$.
The theorem can be applied as follows.  Suppose
$div(\{ f \})= P_1 + P_2 + \ldots + P_n Q_1 - Q_2 - \ldots - Q_n$.  Let
$l_1$ be the line through $P_1, P_2$ and hence $-(P_1+P_2)$ while
$m_1$ is the line through $-(P_1+P_2)$ and $O$ and hence $P_1+P_2$ then
$div( \{f\} {\frac {m_1}{l_1}})= (P_1+P_2) + \ldots + P_n +O - Q_1 - \ldots - Q_n$ continuing
and doing the same for the $Q_i$, we get
$div( \{f\} \cdot
{\frac {m_1}{l_1}} \cdot
{\frac {m_2}{l_2}} \cdot \ldots
{\frac {m_{2n-2}}{l_{2n-2}}})=
(P_1+P_2 + \ldots + P_n) + (n-1) O - (Q_1 + \ldots + Q_n)
-(n-1)O =
(P_1+P_2 + \ldots + P_n) - (Q_1 + \ldots + Q_n)$ so 
$f=c ( {\frac {l_1}{m_1}} \cdot {\frac {l_2}{m_2}} \cdot \ldots
{\frac {l_{2n-2}}{m_{2n-2}}})$ by the theorem.
\begin{quote}
\end{quote}
\emph{Example of divisor calculations:}
$div(\{x \})= 2(0,0)-2 \infty$, 
$div(\{y \})= (-1,0)+(0,0)+(1,0) -3 \infty$, 
$div(\{ {\frac x y} \})= (0,0)+ \infty -(-1,0)-(1,0)$. 
Let $E: y^2= x^3 -2x -5$ and $D= (2,3)+(2,-3)+O-2(-2,1) -(29,156)$.  We
can confirm $(2,3) + (2, -3) +O= O$ and $2(-2,1)+(29, 156)= O$ are elliptic
curve divisors.  The line containing $(2,3), (2,-3),O$ is $x-2=0$.  The
line containing $2(-2,1)$ and $(25,156)$ is $y-5x-11=0$, so
$div({\frac {x-2} {y-5x-11}})= D$.
\\
\\
{\bf Theorem:}
$D= \sum_P n_P P$ is a divisor of an elliptic curve $E$ iff
$\sum_P n_P =0$ and $\sum_P [n_P]P =0$.
\begin{quote}
\end{quote}
{\bf Definition:}
$f \circ n (P)= f(nP)$.  If $T \in E[n], \exists T' \in E[n^2]: nT'=T$ and
$g=f \circ n, div(g)= \sum_{R \in E[n]} [T'+R] -[R]$.  $g(P+S)^n= g(P)^n$ so
$({\frac {g(P+S)} {g(P)}})^n=1$.  
Define the \emph{Weil pairing} as
$e_n(S,T)= {\frac {g(P+S)} {g(P)}}$.  If $\sigma$ is an automorphism, 
$e_n( \sigma S, \sigma T)= \sigma e_n(S,T)$.  If $\alpha$ is separable,
$e_n( \alpha S, \alpha T)= e_n(S,T)^{deg(\alpha)}$.  
\\
\\
{\bf Theorem (Siegal):}  An elliptic curve over ${\mathbb Q}$ has only finitely many
integer points.
\begin{quote}
\end{quote}
{\bf Theorem (Mazur):}  The set of torsion points $T_E({\mathbb Q})$ of
$E/{\mathbb Q}$ is one of 
${\mathbb Z}/n{\mathbb Z}, n= 1,2,\ldots,10,12$ or
${\mathbb Z}/n{\mathbb Z} \times
{\mathbb Z}/2n{\mathbb Z}$, $n= 1,2,3,4$.
\begin{quote}
\end{quote}
{\bf Theorem:}
For every isogony $\psi:E \rightarrow E'$ there is a \emph{dual} $\hat{\psi}:
\hat{\psi} \psi= [2]_E$.  $Frob_n: 
{\overline {F_p}} \rightarrow
{\overline {F_p}}$ and $F_{p^n}= \{ \alpha \in {\overline {F_p}}: Frob_n( \alpha )= \alpha \}$.
\begin{quote}
\end{quote}
{\bf Theorem:}
$\#E(F_p)= p+1 + \sum_{x=0}^{p-1} ({\frac {x^3+ax+b} {p}})$.
\begin{quote}
\end{quote}
{\bf Point Counting:} 
Counting points by
baby-step giant-step is ($O(q^{{\frac 1 4} + \epsilon})$). Set
$N= \#E_q$ then 
$q+1 - 2 {\sqrt q} \le N \le q+1 + 2 {\sqrt q}$; if $[m]P= \infty$ then $N=m$, probably.
Put $m=q+1- 2{\sqrt q} +k, l= \lceil {\sqrt 4 {\sqrt q}} \rceil, k= al+b$, then
$[m]P= [c]P + [a]S + [b] P, c= q+1-2{\sqrt q}, S= [l]P$ or $[c]P+ [a]S = - [b] P$.  Baby
step computes LHS and stores it.  Giant step computes RHS and does a lookup.
\\
\\
{\bf Schoof point counting:}
Let $\varphi$ be the Frobenius automorphism $\varphi(x,y)= (x^q , y^q )$.  
Schoof calculates
$t \jmod{l}$ for a set of primes $l \in {\cal P}$ with 
$\prod_{l \in {\cal P}} l > {4 {\sqrt q}}$ and then 
construct $t$ using CRT finally returning $q+1-t$.
Here's how:\\
\jt (1) For $l=2$, $t= 0 \jmod{l}$ iff 
$(x^3+ax+b,x^q-x) \ne 1$.  \\
\jt (2) if $l$ is odd, set $q_l = q \jmod{l}, |q_l| < {\frac l 2}$;
find $(x', y')= \varphi(x,y)^2 + q_l (x,y) \jmod{\psi_l (x,y))}$;
for $j= 1,2, \ldots {\frac {l-1} 2}$:\\
\jt \jt
(i) Compute $(x_j , y_j)=j(x,y)$; \\
\jt \jt
(ii) if
$x'-x_j^q = 0 \jmod{\psi_l}$, go to iii, if not, try next $j$, if all such
$j$'s have been tried, go to (iv); \\
\jt \jt
(iii) Compute $y', y_j$, if ${\frac {y'-y_j} y} = 0 \jmod{\psi_l}$ then 
$t= j \jmod{l}$ otherwise $t= -j \jmod{l}$; \\
\jt \jt
(iv) Let $w^2=q \jmod{l}$, if no such $w$ exists, 
$t=0 \jmod{l}$;\\
\jt \jt
(v) if $(x^q - x_w, \psi_l )=1$ then 
$t=0 \jmod{l}$, otherwise, set $g= numerator({\frac {y^q - y_w} y}, \psi_l)$, if
$g \ne 1$, $t= 2w \jmod{l}$ otherwise 
$t= -2w \jmod{l}$.
\\
\\
{\bf Definition of Complex Multiplication:}
$E: y^2= x^3 - x$, $\mu: E \rightarrow E$, $\mu(x,y)= (-x, iy)$, 
$\mu^2= [-1]$.
\\
\\
{\bf Definition:} 
If $D= \sum_{P \in E} n_P [P]$, $deg(D)= \sum_P n_P$ and $sum(D)= \sum_P n_P P$.
$div(f)= \sum_{P \in E({\overline K})} ord_P(f) [P] \in Div(E)$.
\\
\\
{\bf Theorem:}  Let $E$ be an elliptic curve and $f$ be a function on $E$ not $0$ then 
(1) $f$ has finitely many poles and zeros, (2) $deg(div(f))=0$ and (3) if there are no
poles or zeros, $f$ is a constant.
\begin{quote}
\end{quote}
{\bf Theorem:}  Let $E$ be an elliptic curve and $D$ a divisor on $E$ with
$deg(D)=0$ then $\exists f$ on $E$ with $div(f)=D, sum(D)= \infty$.
\begin{quote}
\end{quote}
{\bf Pairing:} $e_n: E[n] \times E[n] \rightarrow \mu_n$.  $\mu_n$ is the $n$th roots of
unity.
\\
\\
{\bf Definition:}
For simplified case of elliptic curves in Weierstrauss form ($E: y^2=x^3+Ax+B$),
all polynomials, $f(x,y)$, on $E$, denoted $f \in K[E]$, 
can reduced to normal form, 
$f(x,y)= v(x)+yw(x)$.  If
$f(x,y)= v(x)+yw(x)$, ${\overline f}= v(x)-y w(x)$ and $N(f)= f {\overline f}$.
$deg(f)= \textnormal{max}(2 deg_x(v), 3+2deg_x(w))$.
\\
\\
{\bf Theorem:} 
If $f \in K[E]$ then $deg(f)= deg_x(N(f))$ and $deg(f)$ has the usual properties.
\begin{quote}
\end{quote}
{\bf Theorem:} 
If $r \in K(E)$ and $P \in E$, $\exists u \in K(E)$ such that (1) $u(P)=0$ and
(2) if $r \in K(E)$, $\exists s \in K(E): r=u^ds$.  Further, $d$ does not
depend on the choice of $u$.
\begin{quote}
\emph{Proof:}
For the SWF curve, we can specify $u$ as follows: 
(1) For $P=(a,b) \in E, b \ne 0$, $u(x,y)= (x-a)$;
(2) For $P=(w,0) \in E$, $u(x,y)= y$; and
(3) For $P= \infty$, $u(x,y)= {\frac x y}$.
\end{quote}
{\bf Definition:}
For the notation of the previous theorem, define $ord_P(r)= d$.
\\
\\
{\bf Theorem:} 
Let $r \in K(E)$, $\sum_P ord_P(r) = 0$.  If $f \in K[E]$ then the sum of the
multiplicities of the zeros of $f$ equals the degree of $f$.
\begin{quote}
\end{quote}
{\bf Definition:}
$\Delta \in Div(E)$ is principal if $\Delta= div(r)$ for some $r \in K(E)$.  The
principal divisors are denoted $Prin(E)$ and $Pic(E)= Div(E)/Prin(E)$.  
$|\Delta|= \sum_{P \in P- \infty} |m(P)|$.
\\
\\
{\bf Theorem:} 
Let $\Delta \in Div(E), \exists \tilde{\Delta} \in Div(E)$ with
$\Delta \sim \tilde{\Delta}$ with
$deg(\Delta)=deg(\tilde{\Delta})$ and
$|\tilde{\Delta}| \le 1$.  \begin{quote}
\end{quote}
{\bf Theorem:} 
$\forall \Delta \in Div^0(E), \exists ! P \in E$: 
$\Delta \sim \langle P \rangle -\langle \infty \rangle$.
\begin{quote}
\end{quote}
{\bf Theorem:} 
Let $[n]: E \rightarrow E$ be represented by the rational function 
$[n](P)=(g_n(P), h_n(P))$ then if $(n,p)=1$, 
${\frac {g_n} x} (O) \sim {\frac 1 {n^2}}$ and
${\frac {h_n} y} (O) \sim {\frac 1 {n^3}}$.
\begin{quote}
\end{quote}
{\bf Theorem:} 
Let $P, Q \in E$ and suppose $u$ is a uniformizing parameter at $P$,
define $T_Q(u)](R)=u(R+Q)$ then $T_Q(u)$ is the uniformizing
parameter at $P-Q$.
\begin{quote}
\end{quote}
{\bf Theorem:} 
Let $E$ be an elliptic curve with $deg(D)=0$.  $\exists f \in K(E): D= div(f)$ iff
$sum(D)= \infty$.
\begin{quote}
\end{quote}
{\bf Theorem:} 
Suppose $m>n>0$ and $m, n, m-n, m+n$ are all prime to $p$, then
$div(g_m-g_n)= 
\langle E[m+n] \rangle+
\langle E[m-n] \rangle-
\langle E[m] \rangle-
\langle E[n] \rangle$.  
If $(n,p)=1$, $|E[n]|= n^2$.
\begin{quote}
\end{quote}
{\bf Theorem:} 
If $(n,p)=1$, $div(\phi_n)= \langle E[n] \rangle - n^2 \langle \infty \rangle$.
\begin{quote}
\end{quote}
{\bf Theorem:} 
Suppose $r \in K(E)$ is a non-constant function, then $r$ takes on all values
including $\infty$.
\begin{quote}
\end{quote}
{\bf Theorem:} 
Suppose $f: E \rightarrow E, K(E)$ is a non-constant then $f$ is onto.
\begin{quote}
\end{quote}
{\bf Definition:} The \emph{ramification index} of $F$ at $P$ is defined by
$e_F(P)= ord_P(u \circ F)$.
\\
\\
{\bf Theorem:} 
$ord_P(u \circ F) = ord_{F(P)}(r) \cdot e_F(P)$.
\begin{quote}
\end{quote}
{\bf Definition:} 
$F^*: Div(E) \rightarrow Div(E)$ is 
$F^*( \langle Q \rangle)= \sum_{F(P)=Q} e_F(P) \langle P \rangle$.
\\
\\
{\bf Theorem:} 
$F^*$ is 1-1, $div(r \circ F)= F^*(div(r))$ and
$ e_{F_1 \circ F_2}(P)= e_{F_1}(F_2(P)) \cdot e_{F_2}(P)$.
\begin{quote}
\end{quote}
{\bf Theorem:} 
Suppose $\alpha: E \rightarrow E$ is a non-zero endomorphism then
$e_{\alpha}(P)$ is independent of $P$.
\begin{quote}
\end{quote}
For $m \in {\mathbb Z}, r \in K(E)$ with $D$ the derivative then
$D(r \circ [m])= (m \circ D(r)) \circ [m]$.
\begin{quote}
\end{quote}
{\bf Theorem:} 
If $\varphi$ is the Frobenius homomorphism over $GF(q)$ then $e_{\varphi}= q$.
\begin{quote}
\end{quote}
{\bf Definition:} 
$e_{\alpha}=1$ means $\alpha$ is separable.
\\
\\
{\bf Theorem:} 
If $(m,p)=1$ then $[m]$ is separable.
\begin{quote}
\end{quote}
{\bf Theorem:} 
If $(m,p)=1$ and $(n,p)=1$ then $[m]+[n] \varphi$ is separable.
\begin{quote}
\end{quote}
{\bf Theorem:} 
If $P \in E[m]$ then $[m]^*( \langle T \rangle - \langle \infty \rangle)$
is principal.
\begin{quote}
\end{quote}
{\bf Definition:} 
A rational map $F: E \rightarrow E'$ that is a group homomorphism from
$E$ into $E'$ is a \emph{morphism}.
\\
\\
{\bf Theorem:} 
Let $T_1$ and $T_2$ be a basis for $E[m]$ then $e+m(T_1 , T_2 )$ is a primitive
$m$-th root of unity.
\begin{quote}
\end{quote}
{\bf Definition:} 
If $\alpha$ is an endomorphism then $\alpha(E[m]) \subseteq E[m]$
\\
\\
{\bf Definition:} 
Let $M$ be an algebraic extension of $L$ and $A_r$ is the linear map from $M$
to $M$ represented by multiplication by $r$ with respect to the basis
$r_1 , \ldots , r_n$ and $f_r(x)= det(xI-A_r)$ with constant term 
$c_n = (-1)^n det(A_r)$.  Define $N(r)= det(A_r)$.
\\
\\
{\bf Definition:} 
Suppose $\alpha: E \rightarrow E'$ be an isogony $K'= \alpha^*(K(E')) \subseteq K(E)$.
Define $N= N_{K(E)/K'}$ then $N(r)](P)= \prod_{\alpha(Q)= \alpha(P)} r(Q)^{e_{\alpha}}$.
\\
\\
{\bf Definition:} 
Let $r \in K(E)$ and $D \in Div(E)$.  Suppose $div(r)$ and $D$ have disjoint support
define $r(D)= \prod_{P \in E} f(P)^{n_p}$.  For $f, g \in K(E)$ further
define
$\langle f, g \rangle_P= (-1)^{mn} [{\frac {f^n}{g^m}}](P)$ where 
$m= ord_P(f)$ and
$n= ord_P(g)$.  $\langle f, g \rangle_P$ is called the local symbol of $f$ and $g$.
\\
\\
{\bf Weil Reciprocity Theorem:} $\prod_{P \in E} \langle f, g \rangle_P =1$.
\\
\\
{\bf Construction of the Weil Pairing:}
For $T \in E[n]$,
$\exists f: div(f)= n[T]- n[\infty]$.  Choose $T' \in E[n^2]: nT'=T$ then 
$\exists g: div(g) = \sum_{R \in E[n]} ([T'+R]-[R])$.  Put $f \circ n (P)= f(nP)$.
This gives $f \circ n(P)= g^n(P)$.  Put
$e_n(S,T)= {\frac {g(P+S)} {g(P)}} \in \mu_n,  S \in E[n], P \in E({\overline K})$ and
$e_n$ satisfies the pairing properties.
\\
\\
{\bf Example:}
If $E(F_7): y^2= x^3+2$ then $E(F_7)[3]= {\mathbb Z}_3 \oplus {\mathbb Z}_3$.  To
compute $e_3((0,3),(5,1))$, $D_{(0,3)}= [(0,3)]-[\infty]$ and
$D_{(5,1)}= [(3,6)]-[(6,1)]$, $div(y-3)= 3 D_{(0,3)}$ and 
$div({\frac {4x-y+1} {5x-y-1}})= 3 D_{(5,1)}$.
$f_{(0,3)}(D_{(5,1)})= {\frac {f_{(0,3)}(3,6)} {f_{(0,3)}(6,1)}}= 
{\frac {6-3} {1-3}}= 2 \jmod{7}$.
$f_{(5,1)}(D_{(0,3)})= 4$, $e_3((0,3),(5,1))= {\frac 4 2}= 2 \jmod{7}$.
\\
\\
{\bf Theorem:} 
If $F= {\mathbb Q}, {\mathbb R}, {\mathbb C}$
$E(F)= {\mathbb Z}/m{\mathbb Z} \times {\mathbb Z}/m{\mathbb Z}$;
if $F= {\mathbb F_p}, p \nmid m, \exists k:$
$E(F_{p^{jk}})= {\mathbb Z}/m{\mathbb Z} \times {\mathbb Z}/m{\mathbb Z}$.
\begin{quote}
\end{quote}
{\bf Note:} 
If $m[P]=O$, $m[P]-m[O]$ is a divisor of $E(F)$.
\\
\\
{\bf Theorem:}  Let $P, Q \in E[m]$, 
$div(f_P)= m[P]=m[O]$,
$div(f_Q)= m[Q]=m[O]$ then $e_m(P,Q)= 
{\frac {f_P(Q+S)}{f_P(S)}}/
{\frac {f_Q(P-S)}{f_Q(-S)}}$ for $S \in E(F)$ is bilinear.
$E[m]= \langle aP_1+bP_2 \rangle$ and suppose.
$P= a_P P_1+b_P P_2 \rangle$, $e_m(P,Q)= \zeta^{det
\left(
\begin{array}{cc}
a_P &  a_Q \\
b_P &  b_Q \\
\end{array}
\right)
}$ where
$\zeta= e_m(P_1,P_2)$.
\\
\\
{\bf Theorem (for Miller's Algorithm):}  Let $E$ be an elliptic curve and
$P= (x_P, y_P)$ and
$Q= (x_Q, y_Q)$ are non-zero points on $E$.  Let $\lambda$ the the slope of
the line betwween $P$ and $Q$ and define 
$g_{P, Q}= x-x_P$ if $\lambda= \infty$ and
$g_{P, Q}= 
{\frac {y-y_p - \lambda(x-x_P)} {x+x_P+x_Q+\lambda^2}} $, otherwise.  Then (a)
$div(g_{P,Q})= [P]+[Q] -[P+Q]-[O]$ and
(b) For $m \ge 1$ with binary representation $[m_{n-1}, \ldots, m_0]$ satisfies
$div(f_P)= m[P]-[mP]-(m-1)[O]$ where 
$g_{T,T}$ and
$g_{T,P}$ are defined by Miller's algorithm.\\
{\bf Miller's Algorithm:}\\
\jt 1. Set $T=P$ and $f=1$;\\
\jt 2. $\textnormal{for}(i=(n-2); i \ge 0; i--)$\\
\jt \jt 3. $f= f^2 \cdot g_{T,T}$;\\
\jt \jt 4. $T= 2T$;\\
\jt \jt 5. if($m_i=1$)\\
\jt \jt \jt 6. $f= f \cdot g_{T,P}$;\\
\jt \jt \jt 7. $T= T+P$;\\
\jt 8. $return(f)$.
\\
\\
{\bf Definition:}  Let 
$P, Q \in E(F_q)[l]$, choose $f_P: div(f_P)= l[P]-l[Q]$ than the \emph{Tate pairing}
is $\tau(P,Q)= {\frac {f_P(Q+S)}{f_P(S)}}$ and the \emph{modified Tate pairing} is
$\hat{\tau}(P,Q)= \tau(P, Q)^{\frac {q-1}{l}}$.
\\
\\
{\bf Definition:}
Let $E$ be an elliptic curve over $F_p, m \ge 1, p \nmid m$, the \emph{embedding degree}
of $E$ with respect to $m$ is the smallest positive $k$ such that
$E(F_{p^k})[m]= {\mathbb Z}/m{\mathbb Z} \times {\mathbb Z}/m{\mathbb Z}$.
\\
\\
{\bf Theorem:}
Let $E$ be an elliptic curve over $F_p, p \nmid m$ and suppose $E$ has an element of
prime order $l \ne p$.  The embedding degree, $k$ is one of the following:
(1) $k=1$ ($l \le {\sqrt p}+1$);
(2) $k=l$ if $p=1 \jmod{l}$;
(3) if $p \ne 1 \jmod{l}$, $k$ is the smallest positive integer such that
$E(F_{p^k})= {\mathbb Z}/m{\mathbb Z} \times {\mathbb Z}/m{\mathbb Z}$.
\begin{quote}
\end{quote}
{\bf MOV Algorithm:}  Under the same conditions as the theorem:\\
\jt 1. Compute $N= \#E(F_{p^k})$, $l \mid N$;\\
\jt 2. Choose, at random, $T: T \in E(F_{p^k}), T \notin E(F_{p})$;\\
\jt 3. Compute $T'= (N/l)T$.  If $T'=O$, repeat step 2.\\
\jt 4. $lT'= O$, compute $\alpha= e_m(P, T')$ and $\beta= e_m(Q, T')$;\\
\jt 5. Solve $\beta= \alpha^n$ in $E(F_{p^k})$;\\
\jt 6. $Q=nP$.
\\
\\
{\bf Definition:}
Let $l \ge 3$, $P \in E[l]$, $lP=O$, $\psi: E \rightarrow E$.  $\psi$ is a
\emph{distortion map} for $P$ if (a) $\psi(nP)= n \psi(P)$ and (b)
$e_l(P, \psi(P))$ is a primitive $l$-th root of unity.
\\
\\
{\bf Theorem:}  If $E[l]=
{\mathbb Z}/l{\mathbb Z} \times {\mathbb Z}/l{\mathbb Z}$, TFAE:
(a) $P, Q$ is a basis for $E[l]$;
(b) $p \ne O$ and there is no $\alpha: Q= \alpha P$;
(c) $e_l(P,Q)$ is a primitive $l$-th root of unity.
\\
\\
Note: the \emph{decision ECDLP problem} is in $NP \cap co-NP$.  
Attacks (1) Exhaustive Search - 
to avoid, make sure $\#E_q=nh$, $n$ a large prime $>2^{160}$,
$h$, small; (2) Pohlig-Hellman/Pollard-$\rho$ 
use Pohlig to reduce from $n= p_1^{e_1} ... p_t^{e_t}$ to
$p$, since this step is easy, want $p$ large, Pollard costs $O({\sqrt p})$
[For Pollard, ``random'' function is $f(X)= X+ a_j P +b_j Q \jmod{p}$.];
(3) Isomorphism attack;
(4) MOV for anomalous curves - to avoid make sure $q=p^m$ and $p \nmid \#E_q$; 
(5) Weil-Tate pairing - to avoid make sure $n \nmid (q^k-1), k \le C$ and that the
DLP problem for $F_{q^C}$ is intractable;
(6) Weil descent - to avoid, if $q=2^m$, make sure $m$ is prime.  Index calculus attack
is unlikely because the lifting required from $E_q(a,b)$ to 
$E_{\mathbb Q} ({\overline a}, {\overline b})$ is unknown and the number of points of
small height in elliptic curves over ${\mathbb Q}$ is small.
\\
\\
{\bf Lenstra Elliptic Curve Factoring Method:}
\begin{enumerate}
\item $(n,6)= 1, n \ne m^r$.
\item Choose random $b, x_1, y_1$ between $1$ and $n$.
\item $c= {y_1}^2+{x_1}^3-bx_1 \jmod{n}$.
\item $(n, 4b^3+27c^2)= 1$.
\item $k= lcm(1,2, \ldots, K)$.
\item Compute $KP= ({\frac {a_k} {{d_k}^2}}, {\frac {b_k} {{{d_k}^3}}})$
\item $D= (d_k , n)$ If $D=1$, go to 5 and bump $K$ or go to 2 and select new curve.
\end{enumerate}
\subsection {Elliptic, Weierstrauss and Fermat} 
{\bf Definition:} 
Two curves $C, D$ are projectively equivalent if there is a projective
transformation $\phi$ with $\phi(C)=D$.
Every nonsingular cubic is equivalent to a curve which in affine coordinates is
$y^2= 4 x^3-g_2x-g_3= 0$.  This is the \emph{Weierstauss normal form.}
Note: To prove show that every non-singular curve has an inflexion point (triple tangent).
Map flex to $(0,0,1)$.
\\
\\
{\bf Elliptic Functions from Trigonometry:}
$S(x)= \int {\frac {dx} {\sqrt {1- x^2}}}$.
Let ${\frac {dx} {du}}= c(u)$,
$s(u)^2 + c(u)^2= 1$.  $s'(u)= c(u)$, $c'(u)=-S(u)$, $s(-u)=-s(u)$ and $c(-u)=c(u)$.
$s(x+y)= s(x)c(y)+s(y)c(x)$ and
$c(x+y)= c(x)c(y)-s(y)s(x)$. $\Omega: R \rightarrow S^1$ ($S^1$ is the 1-sphere - circle) by
$u \mapsto (c(u),s(u))$ is a morphism: $\Omega(x+y)= \Omega (x) \oplus \Omega (y)$.
$\Omega$ has a non-trivial kernel $K$ since $S^1$ is compact but $R$ isn't.  
$K= 2 \pi {\mathbb Z}$.
These functions are periodic, satisfy the given derivatives, parameterize $S^1$ under the
indicated morphism and provide the integration property.
By analogy, set $F(k,v)= \int {\frac {dz} {\sqrt {(1-z^2 ) (1- k^2 z^2 )}}}$ and define
$sn$ by $F(k,sn(u))=u$.  
$cn(u)= {\sqrt {1 - sn^2 (u)}}$,
$dn(u)= {\sqrt {1 - k^2 sn^2 (u)}}$.  $sn, cn, dn$ are doubly periodic with
periods $\omega_1 , \omega_2$.  
\\
\\
{\bf Weierstrauss Parameterization:} $y(t)^2= x(t)^3+a x(t) + b$.
Let $\omega_1 , \omega_2 \in {\mathbb C}$ and 
$\Lambda= \{ a \omega_1 + b \omega_2: a,b \in {\mathbb Z} \}$ which
is preserved under unimodular transformations,
$\Lambda'= \lambda - \{ {\vec 0} \}$.
${\mathbb C}/\Lambda$ is an equivalence class of complex numbers equivalent
to a torus.
\\
\\
{\bf Theorem:}
Any two basis of the same discrete group of an elliptic (doubly periodic)
function are related by \emph{unimodular} transformations.
The period module of a doubly periodic function is one of the following: (1) $0$;
(2) $n \omega, n \in {\mathbb Z}$;
(3) $n \omega_1+ m \omega_2, n,m \in {\mathbb Z}$.  In the latter case, there is a 
canonical basis $(\omega_1 , \omega_2 )$ with $\tau= {\frac {\omega_2} {\omega_1}}$ such that
(i) $Im(\tau) >0$, (ii) $-{\frac 1 2} \leq Re(\tau) \leq {\frac 1 2}$, (iii) $|\tau| \geq 1$ and
$Re(\tau)>0$ if $|\tau|=1$.
\begin{quote}
\emph{Proof:}
Suppose neither (1) or (2) hold.  Let $\omega_1$ be the element of $M$ with smallest non zero modulus and
let $\omega_2$ be the element of $M \notin \{ m \omega_1 \}$ with the smallest non-zero modulus.  First,
if ${\frac {\omega_2} {\omega_1}} \in {\mathbb R}, \exists n:  n < {\frac {\omega_2}{\omega_1}} < n+1$ which
means $|n \omega_2 - \omega_1| < | \omega_1|$ which is a contradiction.  We may put
$\omega= \lambda_1 \omega_1 + \lambda_2 \omega_2, \lambda_1, \lambda_2 \in {\mathbb R} $ 
so $\exists m_1, m_2 \in {\mathbb Z}: | \lambda_1 - m_1 | \leq {\frac 1 2}, | \lambda_2 - m_2 | \leq {\frac 1 2}$.
But then $\omega'= \omega - m_1 \omega_1 - m_2 \omega_2 \in M$ and
$ |\omega'| \leq {\frac 1 2} |\omega_1|+ {\frac 1 2} |\omega_2| \leq |\omega_2| $.  To show (i, (ii), (iii) and
(iv), pick 
$\omega_1$ and
$\omega_2$ as above.  We already have, 
$ |\omega_1| \leq |\omega_2| $,
$ |\omega_2| \leq |\omega_1+\omega_2| $,
$ |\omega_2| \leq |\omega_1-\omega_2| $.  
If $Im(\tau) <0$ replace $(\omega_1 , \omega_2)$ with $(-\omega_1 , \omega_2)$.
If $Re(\tau) = - {\frac 1 2}$ replace $(\omega_1 , \omega_2)$ with $(\omega_1 , \omega_1+\omega_2)$.
If $|\tau|=1$ and $Re(\tau) < 0$ replace $(\omega_1 , \omega_2)$ with $(-\omega_2 , \omega_1)$.
\end{quote}
{\bf Theorem:}  An elliptic function without poles is constant.
\begin{quote}
\end{quote}
{\bf Theorem:}  If $f$ is elliptic, $\sum_{a \in {\cal P}} Res(f,a) = 0$, ${\cal P}$ is the set of \emph{poles}.
\begin{quote}
\end{quote}
{\bf Theorem:}  If $f$ is elliptic,  $M$ is the module of periods,
$\langle a_1 , \ldots, a_n \rangle$ are the zeros,
$\langle b_1 , \ldots, b_n \rangle$ are the poles then
$\sum_i a_i= \sum_i b_i \jmod{M}$.
\begin{quote}
\end{quote}
{\bf Elliptic functions of order $2$:} $\wp(z)= z^{-2}+ a_2x^2+a_4 z^4 + \ldots $.
$\wp(z)= {\frac 1 {z^2}} + \sum_{\omega \ne 0} {\frac 1 {(z- \omega)^2)}}- {\frac 1 {\omega)^2}}$.  If
$|\omega| \geq 2|z|$, 
$|{\frac 1 {(z- \omega)^2)}}- {\frac 1 {\omega)^2}}| \leq {\frac {10 |z|} {|\omega|^3}}$.
$\wp'(z)^2= 4 \wp(z)^3 - g_2 \wp - g_3= 4 (\wp(z)-e_1) (\wp(z)-e_2) (\wp(z)-e_3) $,
$e_1, e_2, e_3$ are distinct.  The equation can be solved by $z= \int^w {\frac {dw}
{\sqrt{4w^3 -  g_2 w - g_3}}}$.  Since $\wp(\omega_1 -z)= \wp(z)$, 
$ \wp'({\frac {\omega_1} {2}})= 0 $; similarly,
$ \wp'({\frac {\omega_2} {2}})= 0 $ and
$ \wp'({\frac {\omega_1+\omega_2} {2}})= 0 $. So
$e_1=  {\frac {\omega_1} {2}}$,
$e_2= {\frac {\omega_2} {2}}$, and
$e_3= {\frac {\omega_1+\omega_2} {2}}$.  Put $\lambda( \tau )= {\frac {e_3 - e_2} {e_1- e_2}}$.
$\lambda$ is the quotient of two analytic functions in the upper half-plane.
\\
\\
{\bf Definitions:}
If $\lambda({\frac {a \tau + b} {c \tau + d}})= \lambda(\tau)$, the linear transformation is
an \emph{automorphism} of $\wp$.  The automorphisms
$\lambda({\frac {a \tau + b} {c \tau + d}})= \lambda(\tau)$ for
$
\left(
\begin{array}{cc}
a & b\\
c & d\\
\end{array}
\right)=
\left(
\begin{array}{cc}
1 & 0\\
0 & 1\\
\end{array}
\right) \jmod{2}
$ form the \emph{modular group}.
$\lambda(\tau + 1) = {\frac {\lambda(\tau)} {\lambda(\tau) -1}}$,
$\lambda({\frac 1 \tau}) = 1- \lambda(\tau)$.
$\wp(z, \Lambda)= {\frac 1 {z^2}}+ \sum_{\omega \in \Lambda'} ({\frac 1 {(z-\omega)^2}} - 
{\frac 1 {\omega^2}})$.
\\
\\
{\bf Theorem:}  
$\wp(z, \Lambda)$ converges uniformly for ${\mathbb C}/\Lambda$, $\wp(z)= \wp(-z)$ and
$\wp$ is doubly periodic.  As $z$ ranges over a fundamental region, $\wp$ takes on every
complex value twice.
$\wp(z, \Lambda)- {\frac 1 {z^2}}= 
\sum_{\omega \in \Lambda'} ({\frac 1 {(z-\omega)^2}} - {\frac 1 {\omega^2}})=
\sum_{n=1}^{\infty} (n+1) z^n
(\sum_{\omega \in \Lambda'} 
|{\frac 1 {\omega^{n+2}}}|)$. The final summation is an Eisenstein series of weight
$n+2$, denoted $G_{n+2}$.
$\wp(z)= {\frac 1 {z^2}} +3G_4 z^2+ 5G_6z^4 + \ldots$.
$P(z)= (\wp'(z), \wp(z))$ is a point on $y^2=4x^3-60 G_4 x -140 G_6$.
\begin{quote}
\end{quote}
{\bf Reimann surfaces:}
Glue two copies of $C$ to get ${\sqrt z}$.  For $N \in {\mathbb Z}, N>0$ define
$\Gamma_0 (N) \subseteq SL_2$ with $N|c$.
$\Gamma_0 (N)$ acts on $H$ and $H/
\Gamma_0 (N) \equiv X_0 (N) \backslash K$ where $K$ are the cusps.  
$X_0 (N)$ is compact and
the members are the modular functions of level $N$.
\\
\\
{\bf Semi-Stable:} For all primes $l>3$, $l|Disc$ and only two of the roots are equal 
$\jmod{l}$.
Frey curve: $C^F_{a,b} 
{\buildrel\rm def\over =} \; 
y^2= x (x-a^p) (x-b^p)$.  
If $b$ is even and $a = -1 \jmod{4}$.  Frey curve is semi-stable.
\\
\\ 
{\bf Definitions:}
Denote $E_{A,B,C,D}({\mathbb Q}) {\buildrel\rm def\over =} \; 
y^2= Ax^3 + B x^2 + CX +D, A, B, C, D \in {\mathbb Q}$.
Define $b_p$ to be the number of solutions to $E_{A,B,C,D}({\mathbb Q})= 0$. $E$ 
is \emph{modular} if
$\exists$ eigenfunction, $f(z)= \sum_n a_n e^{2 \pi i n z}$. $E/{\mathbb Q}$
is modular if $\exists f$ and eigenfunction with $a_p=p+1-b_p$ for all but finitely many
$p$.
\\
\\
{\bf Taniyama-Shimura Conjecture:}  Every elliptic curve is modular.
Alternate T-S: $E(A,B,C,D)$.  $\exists$ modular functions $f(z), g(z)$ such that
$g(z)^2= Af(z)^3+Bf(z)^2+C f(z) +D$.
\begin{quote}
\end{quote}
Define the \emph{conductor}
$Cond_{a,b,c}= \prod_{p|abc} p$.  Two elliptic curves are isomorphic
iff their \emph{$j$-invariants} are equal. The $j$-invariant of $C^F_{a,b} =
2^8 {\frac {(a^{2p} +b^{2p} +a^p b^p)^3} {a^{2p} b^{2p} c^{2p}}}$.
If $F({\frac {az+b} {cz+d}})= (cz+d)^2 F(z)$, $F$ is a modular form of weight $2$.
\\
\\
{\bf Proof of Fermat's Last Theorem:}  Suppose it's false and that $a^p + b^p= c^p$ is
a counterexample.  Let 
$C^F_{a,b}$ be the Frey curve.  $Disc(C^F_{a,b})= a^{2p} b^{2p}c^{2p}$ so
$C^F_{a,b}$ is semi-stable.
Wiles proved every semi-stable elliptic curve is modular so
$C^F_{a,b}$ is modular and has a cusp form of weight $2$ and level $N$ where $N$
is the conductor.
If $l$ is an odd prime and $l|N$,
by Serre, we can obtain a new $F$ of
weight 2 of level $N/l$.  By induction, keep doing this until $N=2$.  The dimension
of the space of cusps is equal to the genus of compact Reimann surface $X_0(N)$.  
But $Genus(X_0(2))= 0$, so there is no such cusp forms of weight $2$, level $2$.
This contradiction establishes the theorem.  Incidentially, the restriction of semi-stability
in Wiles Theorem has been removed.
