\subsection{Group Theory}
{\bf Isomorphism Theorems:}  (1) If $\varphi: G \rightarrow H$ is a homomorphism,
$G/ker(\varphi) \cong Im(\varphi)$, 
(2)  If
$G \triangleright H$ and $G \triangleright N$ and $N \subseteq H \subseteq G$
then $G/H \cong (G/N)/(H/N)$, 
(3) If $G=HN$, $G \triangleright N$ then
$HN/N \cong H/(H \cap N)$.\\
\\
{\bf Derived series:} $G^{[0]}=G$, $G^{[i+1]}= [G^{i]}, G^{[i]}]$.  $G$ is solvable iff
derived series terminates at ${1}$.  Subnormal Series: 
$G= G_0 \rhd G_1 \rhd \ldots \rhd G_k =H$, if this happens, we say
$G \rhd \rhd H$.  {\bf Normal series:} Subnormal series where $G \rhd G_i, \forall i$.
{\bf Chief series:} Normal series with no repeated terms and no normal subgroup properly
lying between two series elements.  
{\bf Zassenhaus Butterfly Lemma:}  If
$A \triangleleft A^{*}$ and $B \triangleleft B^{*}$ then
$A(A^{*} \cap B) \triangleleft A(A^{*} \cap B^{*})$ and
$B(B^{*} \cap A) \triangleleft B(B^{*} \cap A^{*})$; further,
${\frac {A(A^{*} \cap B^{*})} {A(A^{*} \cap B) }} \cong
{\frac {B(B^{*} \cap A^{*})} {B(B^{*} \cap A) }}$.
Let $G$ be a finite group.  
The following are equivalent: (1) $G$ is solvable, (2) $G$ has a normal series 
terminating at the identity whose factor groups are abelian,
(3) $G$ has a subnormal series with cyclic quotients.\\
\\
{\bf Schreier:} Two normal series for $G$ have equivalent refinements.
Two compositions series for $G$ are equivalent.
Proof:  By induction on length ($l$) of shortest such
series.  If $l=1$, $G$ is simple.
Suppose $G=G_0 \ge G_1 \ge \ldots \ge G_r = 1$ and
$H=H_0 \ge H_1 \ge \ldots \ge H_t = 1$ and assume $l=r>t$ and that the theorem
is true for all series of length less than $l$. If $H_1=G_1$ then we are done by induction
on the shortened series.  Assume $G_1 \ne H_1$, $H_1 \lhd G, G_1 \lhd G$ then
$G_1H_1 = G$ and $G/G_1 \cong H_1/K, K= G_1 \cap H_1$. Consider the two series
$G_1 \ge G_2 \ldots \ge G_r = 1$ and
$G_1 \ge K \ge K_1 \ldots \ge K_t = 1$.  
By induction, $r-1=t+1$ and they are equivalent.
Thus,
$H_1 \ge H_2 \ldots \ge H_s = 1$ and
$H_1 \ge K \ge K_1 \ldots \ge K_{r-2} = 1$ so $r=s$ and they
are equivalent.\\
\\
$\phi$ is a {\bf normal endomorphism} iff
$\phi(a^{-1}xa)=
a^{-1} \phi(x)a$, $\forall x,a \in G$.
Lemma 1: If $G$ satisfies ACC or DCC then $G$ is the direct product of indecomposable groups.
Lemma 2: If $G$ satisfies ACC (resp. DCC) on normal subgroups 
and $f$ is a normal endomorphism
of $G$, then $f$ is an automorphism iff $f$ is an epimorphism (resp automorphism).
{\bf Lemma 3 (Fitting):}
Let $G$ satisfy both chain conditions.  If $\phi$ is a
normal endomorphism of $G$ then $G= Ker(f^n) \times Im(f^n)$, some $n \ge 1$.
If $G$ is an indecomposable group satisfying ACC and DCC on normal subgroups
and if $f$ is a normal endomorphism then $f$ is nilpotent or an automorphism.
{\bf Krull-Schmidt:}  If $G$ has both chain conditions on normal subgroups and
$G= H_1 \times \ldots \times H_s = K_1 \times \ldots \times K_t$ are
two decompositions into indecomposable factors then $s=t$ and, after
reindexing, $H_i \cong K_i$ and for each $r<t$, $G= G_1 \times G_2 \times \ldots
\times G_r \times H_{r+1} \times H_t$. Proof:  Let $P(0)$ be the statement
$G= G_1 \times G_2 \times \ldots \times G_s$ and for $1 \le r \le min(s,t)$
let $P(i)$ be the statement
$G= G_1 \times G_2 \times \ldots \times G_r \times H_{r+1} \times \ldots H_t$.
$P(0)$ is true by assumption, assume $P(r-1)$.  Let $\pi_i$ (resp $\pi_i'$ be the 
canonical epimorphisms from
$G_1 \times G_2 \times \ldots \times G_s$ (resp.
$G_1 \times G_2 \times \ldots \times G_r \times H_{r+1} \times H_t$ and $\lambda_i$
(resp $\lambda_i'$) be the inclusion maps, 
$\varphi_i= \lambda_i \pi_i$
and
$\phi_i= \lambda_i' \pi_i'$.  $\varphi_r \phi_i= 0_{|G}$ for $i<r$ and
$\varphi_1 (1_{|G})= \varphi_r \phi_1 + \ldots + \varphi_r \phi_t=
\varphi_r \phi_r + \ldots + \varphi_r \phi_t$ so $(\varphi_r \phi_j)_{|G}$ is
an automorphism of $G_r$.  $\varphi_j \phi_r$ must be an automorphism of $H_j$ and
$\phi_j:G_r \rightarrow H_j$ is and isomorphism and so is $\varphi_r: H_j \rightarrow G_r$
reindexing we have the first half of $P(r)$.  Let 
$g=g_1 g_2 \ldots g_{r-1} h_r h_{r+1} \ldots h_t$ define 
$\theta(g)=g_1 g_2 \ldots g_{r-1} \varphi(h_r) h_{r+1} \ldots h_t$.  
$G=Im(\theta)=G^*= G_1 \times G_2 \times \ldots \times G_r \times H_{r+1} \times H_t$ 
which completes the argument.\\
\\
{\bf Lower Central Series:} $L_1(G)=G$, $L_{n+1}(G)= [L_n(G),G]$.  $G$ is 
{\bf nilpotent} if
$L_n(G)=1$ for some $n$.  Note $L_n(G)/L_{n+1}(G) \subseteq Z(G/L_{n+1}(G))$.
{\bf Upper Central series:}
$Z_0(G)=1$; Let $H^*= H/Z_n(G)$, define $Z_{n+1}(G)^*= Z(G/Z_n(G))$.
Upper and Lower central series have same length.
Finite nilpotent groups are direct products of
their Sylow subgroups.\\
\\
$G$ is an {\bf extension} of $K$ by $Q$ if $G \triangleright K$ and $G/K \cong Q$.
If $1 \rightarrow N \rightarrow_{i} G \rightarrow_{\varphi} Q \rightarrow 1$, the following
are equivalent
(1) $\exists Q^* \subseteq G: Q^* \rightarrow Q$ and
(2) $\exists s:Q \rightarrow G$ such that $\varphi \cdot s = id$.
(3) $G$ is a semi-direct product of $N$ by $Q$ written $N \ltimes Q$; in this
case, we say $G$ is a split extension of $N$ by $Q$.\\
\\
$G$ is {\bf complete} if it is centerless and every automorphism is inner.
in which case $G \cong Aut(G)$. $S_n$ is complete if $n \ne 2,3$.  Proof:
Let $T_k$ be the set of $k$ disjoint transpostions so $x \in T_k \rightarrow
x^2=1$; note that if $\theta \in Aut(S_n), \theta(T_1)= T_k$ for some $k$. Also
observe that $\theta$ preserves transpositions iff $\theta \in Inn(S_n)$.
Now we can show 
$|T_1|= {\frac {n(n-1)} {2}}$ and
$|T_k|= {\frac {(n-2k+1)!} {(n-2k)! k! 2^k}}$.  Comparing the two $|T_1| = |T_k|$
is possible only if $k=2 ,3$ and in fact, only if $k=3$.   If
$\theta \in Out(S_6)$ and $\tau$ is a transposition, $\theta(\tau)$ must
be a product of three transpositions and such an automorphism exists.
If $G$ is a non-abelian simple group, then $Aut(G)$ is complete.  If
$K \lhd G$ and $K$ is complete, $G= K \times Q$.  $Hol(K) \subset S_K$
is $<K^l, Aut(K)>$, $K^l \lhd Hol(K)$, $Hol(K)/K^l \cong Aut(K)$ and
$C_{Hol(K)}(K^l)= K^r$.  If $K$ is a direct factor whenever $K$ is a normal
subgroup then $K$ is complete.\\
\\
Suppose $G$ is an extension of $N$ by $H$ and let $\phi: H \rightarrow G/N$.  Pick
$s:G \rightarrow H$ such that $s(1)=1$ and $\phi(h) = N s(h)$, then 
$\exists f: H \times H \rightarrow N: s(h_1 h_2)= f(h_1, h_2) s(h_1 h_2)$
and $f(h_1, h_2) f(h_1 h_2, h_3)= f(h_2, h_3)^{s(h_1)} f(h_1 , h_2 h_3)$.  Note
that $\theta_h: n \mapsto s(h) n s(h)^{-1}$ is in $Aut(N)$ and
$\theta_{h_1}(\theta_{h_2}(n))= \theta_{h_1 h_2}(n)^{f(h_1, h_2)}$.
Given $N,H$ with $\theta_h \in Aut(N)$ and $\theta_1 = 1$ and a map
$f: H \times H \rightarrow N$ with $f(1,h)=f(h,1)=1$ and
$f(h_1, h_2) f(h_1 h_2 , h_3)= \theta_{h_1}(f(h_2, h_3)) f(h_1, h_2 h_3)$, 
suppose $f$ is compatible in the sense that 
$\theta_{h_1}(\theta_{h_2}(n))= \theta_{h_1h_2}(n)^{f(h_1, h_2)}$ then the
operation $(n_1, h_1) \cdot (n_2, h_2) = (n_1 \theta_{h_1}(n_2) f(h_1, h_2), h_1 h_2)$
defines a group $G$ which is an extension of $N$ by $H$.\\
\\
Suppose $T$ is a subset consisting of a representative of each coset
of an $G/K$ which is called a {\bf transversal}. 
If $\pi: G \rightarrow Q$ is a surjective homomorphism with kernel
$K$, $l: Q \rightarrow G$ is a {\bf lifting} if $\pi(l(x))=x$.
$G$ realizes $(Q, K, \theta )$ with $K'=1$, $\theta:Q \rightarrow Aut(K)$ and
$l:Q \rightarrow G$ if $G$ is an extension of
$K$ by $Q$ and every transversal $l: Q \rightarrow G$ satisfies $xa = \theta_x(a)
= l(x) +a-l(x)$.  Note additive notation for non-abelian operation.
If $\pi: Q \rightarrow G$ is a surjective homomorphism with
kernel $K$ and $l:Q \rightarrow G$ is a transversal with $l(1)=0$ then
$f: Q \times Q \rightarrow K$ defined by $l(x)+l(y)= f(x,y) + l(xy)$ is
called a {\bf factor set}.
{\bf Cocycle identity:} $xf(y,z)-f(xy,z)+f(x,yz)-f(x,y)=0$.  Note
$xf(y,z)= l(x)f(y,z)l(x)^{-1}$.
Given ``data,'' $(Q, K, \theta )$, $f:Q \times Q \rightarrow K$ is a factor set
iff it satisfies the cocycle identity and $f(1,y)=0=f(x,1)$.
Proof: Let $G= \{(a, x): a \in K,
x \in Q \}$.  With $(a,x) + (b, y)= (a +xb+f(x,y), xy)$.
This is a group if the conditions hold.\\
\\
Let $G$ realize $(Q, K, \theta )$ and $l$ and $l'$ be transversals with
$l(1)=l'(1)=0$ giving rise to factor sets $f$ and $f'$ then there is an
$h:Q \rightarrow K$ with $h(1)=0$ such that 
$f'(x,y)-f(x,y)= xh(y)-h(xy)+h(x), \forall x,h \in Q$ and $g$ is called a
{\bf coboundary}.  The set of all coboundaries is 
${\bf B^2 (Q,K,\theta)}$.
${\bf Z^2 (Q, K, \theta)}$ is the set of all {\bf factor sets}.
${\bf H^2 (Q, K, \theta) \cong Z^2 (Q, K, \theta) / B^2 (Q, K, \theta)}$.
Two extensions are equivalent if the difference of
their two factor sets is in $B^2 (Q, K, \theta)$.
There is a bijection from $H^2(Q,K,\theta)$ and the set of equivalence classes
of extensions realizing $(Q, K, \theta)$ taking $0$ to the class of the
semidirect product.  See proof of Schur-Zassenhaus.\\
\\
$G$, an extension of $K$ by $Q$, is a {\bf central extension} if $K<Z(G)$.  Functorially,
a central extension $G$ is a pair $(H, \pi)$ satisfying
$\pi: H \rightarrow G, ker(\pi) \subseteq Z(H)$.  A cyclic extension $G$ of $N$ is
one where $G/N$ is cyclic.  Solvable groups are built from cyclic extensions.
$\alpha: (H_1 , \pi_1) \rightarrow (H_2, \pi_2)$ is a morphism in this category. 
If $(\tilde{G}, \tilde{\pi})$ is universal if 
$\forall (H, \sigma), \exists ! \alpha : (\tilde{G}, \tilde{\pi}) \rightarrow
(H, \sigma)$.
$G$ possesses a universal central extension iff $G$ is perfect.
If $(\tilde{G}, \pi)$ is a universal central extension then $ker(\pi)$ is the
Schur multiplier.
{\bf Homological version:} If $G>N$
and $H>K$ are normal subgroups isomorphic under $\phi$, the pullback
is $(g, h)$ where $gN= \phi(hK)$.
$(Q, K, \theta)$ is trivial iff every extension realizing $(Q, K, \theta)$
is a central extension.  There's a bijection between $H^2 (Q, K, \theta)$
and central extensions. {\bf Schur multiplier:}
$M(Q)=H^2 (Q, {\mathbb C}^{\times})$ ($\theta$ is trivial).
Here $f(1,y)=f(x,1)=1$, $f(x,y) f(xy,z)^{-1} f(x,yz) f(x,y)^{-1}=1$,
$g: Q \times Q \rightarrow {\mathbb C}^{\times}$ is a coboundary
iff $\exists h: Q \rightarrow {\mathbb C}^{\times}$ with $h(1)=1$ such that
$g(x,y)= h(y)(h(xy))^{-1}h(x)$.  Assume $G$ is perfect then a central extension
$(E, \phi)$ of $G$ is universal iff (a) $E$ is perfect and (b) all 
central extensions of $E$ are trivial. In that case,
$1 \rightarrow R \rightarrow F \rightarrow G \rightarrow 1$, $F$, free and
$E= [F,F][F,R] \rightarrow [F,F]/R=G$.\\
\\
{\bf Central Product:} $G= <G_i>$, $[G_i , G_j ]=1$ for $i \ne j$.  Equivalently,
$\rho: (x_1, x_2, \ldots , x_n) \mapsto x_1 x_2 \ldots x_n$ is a surjective homomorphism
from $(G_1 \times G_2 \times \ldots \times G_n)$ to $G$ with $\rho(D_i) = G_i$ where
$\pi_i (G_1, \ldots , G_n)= D_i$ and $ker(\rho) \cap D_i = 1$, $ker(\rho) \subseteq Z(G)$.
Let $Z<Z(A)) \cap Z(B)$, $A \times B / Z$ is a central product.
Both $D_8$ and
$Q_8$ are central products of $Z_2$ by $Z_2 \times Z_2$.
Let $G_i, 1 \le i \le n$ be a family of groups with $Z(g_1)=Z(G_i)$ and
$Aut_{G_i}(Z(G_i))=Aut(Z(G_i))$.  Then up to isomorphism there is a unique
central product with $Z(G_1)=Z(G_i)$.\\
\\
{\bf Wreath Product:} $G^*= G^{X}$ - maps from $X$ to $G$.  $fg(x)= f(x) g(x)$.
Let $H$ act on $X$: $f^h(x)= f(xh^{-1})$.  Let $\phi$ be the natural action of
$H$ induced on $G^{|H|}$, then $G \wr H = H \rtimes_{\phi} G^*$. If $G_x =
\{f: f(y)= 1 \; if \; x \ne y \}$.  $G^* = \prod_X G_x$.
Put $g_x (y)= g (y)$ if $x=y$, 1 otherwise.  Note that
${g_x}^h= g_{xh}$.
If $H$ is finite and $G/K=H$, $G$ can be embedded in the
regular wreath product $K \wr H$:
{\bf Universal Embedding Theorem:}  Let $G \rhd N$ and $K \equiv G/N$, 
$\exists \phi : G \rightarrow N \; \wr \; K$ such that $\phi$ maps $N$
onto $im(\phi) \cap \bigotimes_i N$.
$exp(G)= min \{e: x^e = 1, \forall x \in G \}$.
If $Q$ is finite then $M(Q)$ is a finite abelian group and
$exp(M(Q)) \mid |Q|$.\\
\\
{\bf Representations:}
If $V$ is a ${\mathbb C}G$-module, $g$ induces a linear map
$\rho(g): V \rightarrow V$;
if ${\cal B}$ is a basis of $V$, the matrix for the matrix representing the linear
map $\rho(g)$ is denoted by $[g]_{\cal B}$.
Two such \emph {representations} are equivalent if they are
related by a similarity.
Let $V$ be a ${\mathbb C}G$-module with basis ${\cal B}$ and $\rho: g \mapsto [g]_{\cal B}$ then
(i) if ${\cal B}'$ is a basis of $V$ and $\phi(g)= [g]_{{\cal B}'}$ is another representation,
$\rho$ is equivalent to $\phi$; and, (ii) if $\sigma$ is any equivalent representation to
$\rho$, $\exists {\cal B}': \sigma(g)= [g]_{{\cal B}'}$.
Let $V, W$ be a ${\mathbb C}G$-modules; $V \cong W$ iff there are bases 
${\cal B}_1$, ${\cal B}_2$ of $V, W$ respectively such that
$\forall g \in G: [g]_{{\cal B}_1}= [g]_{{\cal B}_2}$.
Let $V, W$ be a ${\mathbb C}G$-modules, $V \cong W$ iff there are bases 
${\cal B}_1$, ${\cal B}_2$ of $V, W$ respectively such that
$[g_{{\cal B}_1}]$ and $[g_{{\cal B}_2}]$ are equivalent.  
{\bf Maschke:}  If $V$ is a ${\mathbb C}G$-module and
$U$ is a ${\mathbb C}G$-submodule, there is a 
${\mathbb C}G$-submodule, $W$, of $V$ such that $V= U \oplus W$.
{\bf Schur:}  If $V, W$ are irreducible ${\mathbb C}G$-modules and $\theta: V \rightarrow W$ is a 
${\mathbb C}G$ homomorphism then either $\theta=0$ or $\theta$ is an isomorphism.
If $V, W$ are ${\mathbb C}G$-modules and $\theta: V \rightarrow W$ is a ${\mathbb C}G$ module
homomorphism, $\exists U$, a ${\mathbb C}G$-submodule of $V$ such that $V= ker(\theta) \oplus U$.
$Hom_{{\mathbb C}G}(V,W)$ is a vector space over ${\mathbb C}$.  
If $V,W$ are irreducible ${\mathbb C}G$ modules,
$dim_{{\mathbb C}}(Hom_{{\mathbb C}G}(V, W))$ is $1$ if $V \cong W$ and $0$ otherwise.
$dim_{{\mathbb C}}(Hom_{{\mathbb C}G}(V, W)) \ne 0$ if $V$ and $W$ have a common composition
factor.
Let $V$ be an ${\mathbb C}G$ module, 
$V= U_1 \oplus U_2 \oplus \ldots \oplus U_r$ with $U_i$ irreducible; (a) if
$W$ is an irreducible ${\mathbb C} G$ module then 
$dim_{{\mathbb C}}(Hom_{{\mathbb C}G}(V, W))= dim_{{\mathbb C}}(Hom_{{\mathbb C}G}(W, V))$ is
the number of $U_i \cong W$;
(b) each $U_i$ is a composition factor in the Jordan Holder series.
${\mathbb C}G= U_1 \oplus U_2 \oplus \ldots \oplus U_r$ with $U_i$ irreducible;
if $G$ is finite, there are finitely many irreducible ${\mathbb C}G$ modules.
$dim(Hom_{{\mathbb C}G}(V_1 \oplus \ldots \oplus U_r, W_1 \oplus \ldots \oplus W_s)=
\sum_{i=1, j=1}^{r,s} dim(Hom_{{\mathbb C}G}(V_i , W_j))$. Suppose $U$ is and irreducible
${\mathbb C}G$-module then 
$dim(Hom_{{\mathbb C}G}({\mathbb C}G, U))= dim(U)$. [Proof: 
Let $d=dim(U)$ and $u_1, u_2, \ldots, u_d$ be a basis for $U$.  Define
$r \phi_i=u_i r$.  The $\phi_i$ are a basis for $Hom_{{\mathbb C}G}({\mathbb C}G, U)$].
If $V_1, V_2, \ldots, V_r$ are a complete set of irreducible ${\mathbb C}G$-modules
then $|G|= \sum_{i=1}^r dim(V_i)^2$.  [Proof: 
${\mathbb C}G= U_1 \oplus U_2 \oplus \ldots \oplus U_k$, of these, $dim(V_i)$ are isomorphic
to $V_i$ and each of these had dimension $dim(V_i)$]. 
If $G$ is abelian, any 
${\mathbb C}G$-module has dimension $1$.  If $G \lhd N$ and $\chi$ is a character of $G/N$,
define $\tilde{\chi}(g)= \chi(gN)$.  $\tilde{\chi}$ is a character of $G$ and is irreducile iff
$\chi$ is irreducible.
\\
\\
{\bf Characters:} 
$\chi_{reg}= \chi_1(1) \chi_1(g) + \chi_2(1) \chi_2(g) + \ldots + \chi_r(1) \chi_r(g)$.
Let $U, V$ be non-isomorphic irreducible ${\mathbb C}G$ modules with 
characters $\chi, \psi$, then $< \chi, \chi>=1$ and $<\chi, \psi>=0$. 
$\chi(g)$ is real iff
$\chi(g)= \chi(g^{-1}), \forall \chi$.  $N \lhd G$ iff $\exists \chi_i, i= 1, \ldots, k$
such that $\bigcap_{i=1}^k ker(\chi_i)=N$.  $G$ is not simple iff $\exists \chi, g \ne 1:
\chi(g)= \chi(1)$.  $G$ has $|G/G'|$ linear characters.
If all irreducible representations of $G$ have dimension $1$, $G$ is abelian.
Define $(\theta , \eta)= {\frac 1 {|G|}} \sum_g \theta(g) {\overline {\eta(g)}}$.
If $U = U_{1} \otimes ... \otimes U_{s}$, the number of these similar to $U_{1}$
is ${\frac {(\theta, \eta)} {(\eta, \eta)}}$.
$(\theta, \rho_{G})= \theta(1)$,
$(\chi_{i}, \chi_{j})= \delta_{ij}$,
$\sum_{g} \chi(g) = |G| \delta_{i1}$,
$\sum_{i} \chi_{i}^{2}(1) = |G| $.
$\omega_{i}(R_{j})= |r_{j}|\chi_{i}(g)/\chi_{i}(1)$,
$\omega_{t}(R_{i}) \omega_{t}(R_{j})= \sum_{s} a_{ijs} \omega_{t}(R_{s})$.
$\sum_{t} \chi_{t}(g_{i}) {\overline \chi_{t}(g_{j})} =
{\frac {|G|}{|R_{j}|}} \delta_{ij}$.
The number of conjugacy classes = number of irreducible representations.
$\omega_{i}(R_{j})$ is an algebraic integer.
$\chi_{i} | |G|$,
$(|R|, \chi(1))= 1, \chi^{irred} \rightarrow
|\chi(g)|= 1$ or $\chi(g)= 0$.
Let $H$ be the kernel of $\theta$ then (i) $|\theta(g)| \leq \theta(1)$,
(ii) $\theta(g) = \theta(1)$, iff $g \in H$,
(iii) $|\theta(g)| = \theta(1)$, iff $gH$ is in the center if $G/H$.
{\bf Characters and group structure: }
The character table determines the normal subgroups and the nilpotent groups.
General procedure for calculating characters: (1) Derive a faithful representation,
(2) generate group elements, (3) determine conjugacy classes, (4) determine structure
constants ($|C_i||C_j|= \sum_k \alpha_{ijk} |C_k|$), (5) get characters from structure
constants.
If $G \subseteq S_n$, $\alpha: G \rightarrow {\mathbb C}$ by $\alpha(g)= |fix(g)|-1$,
then $\alpha$ is a character of $G$. Define $ker(\rho)= \{g: \chi_{\rho}(g)= \chi_{\rho}(1) \}$.
$\rho$ is faithful iff $ker(\rho)=1$.  $N= \{n: |\chi(n)|= \chi(1) \} \lhd G$.  If
$N \lhd G, \exists \chi_i: \bigcap_{i=1}^r ker(\chi_i) =N$.  
$g \sim h$ iff $\chi(g)=\chi(h), \forall  \chi$.  
Let $x \in A_n$; if there is an odd permutation that commutes with $x$,
$ccl_{A_n}(x)=ccl_{S_n}(x)$ otherwise
$ccl_{S_n}(x)$ splits into two conjugacy classes in $A_n$.
Let $C_i= \sum_{x \in ccl(y)} x$ then the $C_i$ form a basis for ${\mathbb Z}(FG)$.
There are $|G/G'|$ inequivalent linear representations (characters) of $G$.
\\
\\
{\bf Application of character theory:}
Suppose $\chi$ is a character of a ${\mathbb C}G$-module, $V$, and $g \in G$ has order
$m$ then (1) $\chi(1)=dim(V)$, (2) $\chi(g)$ is a sum of $m$-th roots of unity,
(3) $\chi(g^{-1})= {\overline {\chi(g)}}$ and (4) $\chi(g)$ is real iff $g \sim g^{-1}$.
If $\chi$ is an irreducible character, $\chi(1) \mid |G|$ 
(If $g_i$ is in the $i$th conjugacy class,
${\frac {|G|} {|C_G(g_i)|}} {\frac {\chi(g_i)} {\chi(1)}}$ and ${\overline {\chi(g)}}$
are algebraic integers so
$\sum_{i=1}^k {\frac {|G|} {|C_G(g_i)|}} {\frac {\chi(g_i)} {\chi(1)}}{\overline {\chi(g)}}
={\frac {|G|} {\chi(1)}}$ is.) {\bf Burnside's Lemma:} 
$|{\frac {\chi(g)} {\chi(1)}}| \le 1$ if
$|{\frac {\chi(g)} {\chi(1)}}| \ne 1$ it is not an algebraic integer.  
Let $p$ be a prime and $G$ a finite
group with conjugacy class of size $p^r, r \ge 1$, then $G$ is not simple.
{\bf Burnside's Theorem:} Every group of order $p^a q^b$ is solvable.
Let $\chi$ be an irreducible character and $C$ a conjugacy class.
If$(\chi(1), |C|)=1$ then either $C \subseteq Z(\chi)$ or $\chi(C)=0$.  If $G$
is a non-abelian simple group $\{1\}$ is the only class with prime power
order.  
\\
\\
{\bf Feit's moduleless treatment.}
Maschke: If $char(F)$ does not divide $|G|$, then $F$-representations of $G$
are completely reducible.  For $\phi$ irreducible,
if $\exists S: \forall g, S \phi(g) = \phi(g) S$ then S is non-singular.
If $A(g), B(g)$ are $k$-irreducible then (i) if $A$ is not similar to $B$,
and,
$\sum_{g} a_{is}(g) b_{tj}(g^{-1}) = 0$; or,
(ii) $A$, $B$ are absolutely irreducible and
$\sum_{g} a_{is}(g^{-1}) a_{tj}(g) =
{\frac {|G|} {n}} \delta_{ij} \delta_{st}$, where $n \times n$ is the
dimension of $(a_{is}(g))$.
If $A^{s}$ is absolutely irreducible then $a^{s}_{ij}(g)$ are linearly
independent and $\sum_{s=1}^k n_{s}^{2} \leq |G|$.\\
\\
{\bf Induced representations:} If $H \le G$ and $\varphi$ a class function on $H$, define
$\varphi^G(g)= {\frac 1 {|H|}} \sum_{x \in G} \varphi^* (x^{-1}gx)$. {\bf Frobenius
Reciprocity:} $(\varphi^G , \theta)= (\varphi, \theta_{|H})$.  \\
\\
{\bf Brauer's Characterization of Characters:} $p$-elementary groups are the products of
a cyclic $p'$ group and $p$ group.  Every irreducible character is an induced character
of a linear character of a $p$ elementary subgroup for some $p$.\\
\\
{\bf RSK correspondence} for representations
of the symmetric group: $\exists$ bijection between $S_n$ and the set of ordered
tableau of the same shape $g \leftrightarrow (S,T)$, further
$g^{-1} \leftrightarrow (T,S)$.
{\bf Young's diagram:} $D(\lambda )$, $n= n_1 + n_2 + \ldots + n_k$,
$n_1 \geq n_2 \geq \ldots \geq n_k$.
Number of tableaus with shape $\lambda$:
$f_{ \lambda } = \frac { n!} {\prod_{i,j \in D(\lambda )} {h(i, j)}}$, where
$h(i, j)$ = number of cells in hook $H_{i,j}$.\\
\\
{\bf Fixed point free automorphisms:} 
Let $G$ be a transitive permutation group on $X$ and $1 \ne g \in G$ fixes
no more than one element then $N= \{g: X_g= \emptyset \}$ is a normal subgroup of $G$.
Thompson showed any finite group having a fixed point free automorphism is nilpotent.\\
\\
{\bf Schrier and coset enumeration:}
Let $G= < g_1 , g_2 , \ldots , g_m >$.  Let $k_1 , k_2 , \ldots ,
k_s$ be a group of coset representatives for a subgroup $H<G$. ${\overline g}$
is the coset representative for $g$ in $G/H$ and $k_1 = 1$ then
$H= < (k_i g_j ) {\overline {(k_i g_j )^{-1}}} >$
for $i= 1, 2, 3, \ldots , s$ and
$j= 1, 2, 3, \ldots , m$.  Maintain following tables: Coset, relation table
for each relation, subset table.  Column headers are generators, rows are
right coset labels.
To calculate $|G|$, calculate orbit of point.  Calculate point stabilizer by
completing paths in Schrier tree and using the resulting relations.\\
\\
${\cal B}= < \beta_1 , ... , \beta_n>$ is a base for 
$G \le Sym(\Omega)$ if $G_{\cal B} = 1$.
If $G^{[i]} = G_{\beta_1 , ..., \beta_i}$ and
$G=G^{[1]} \ge ... \ge G^{[m+1]}=1$ then
$S$ is a {\bf strong generating set} relative to ${\cal B}$ if
it is a generating set and $S \cap G^{[i]} = G^{[i]}$.  Can use this to get
orbit sizes.
Schrier-Sims calculates base and strong generating set.\\
\\
{\bf Coxeter groups:} $M= (m_{ij}), 1 \le i,j \le n, m_{ii}=1$, 
$m_{ij} \in {\mathbb Z}, m_{ij} \ge 2$.  Associate to each
such matrix a graph with nodes $i, 1 \le i \le n$, $(i,j)$ is an edge if $m_{ij}>0$ if
$m_{ij}>2$, label it with $m_{ij}-2$.  The Coxeter group is $G$ generated by
$S= \{ s_i \} , 1 \le i \le n$ with $(s_i s_j)^{m_{ij}}=1$.  Note the $s_i$'s must be
involutions, $\theta= {\frac {\pi} {m_{ij}}}$. 
Geometrically: If $\Delta= \{ r_1 , \ldots , r_n \}, ||r_i||=1$ 
is a {\bf root system} with each
$r_i$ defining a reflection along its associated hyperplane by $S_r (x) = x-2(r,x)r$ and
$\alpha_{ij}= -cos({\frac {\pi} {p_{ij}}})= (r_i , r_j)$.  Associate a marked
graph with edges labeled by $p_{ij}$ (unmarked edged have $p_{ij}=3$) and associated
quadratic form $Q({\vec x})= \sum \alpha_{ij} x_i x_j$.  The Coxeter group is generated
by the involutions $S_r$ and $S_{r_i}S_{r_j}$ has order $p_{ij}$.  The quadratic forms
are positive definite and the associated forms are irreducible iff the graphs are
connected.  The root system is effective iff the roots generate the underlying vector
space.  Union of the fundamental region under each element of $G$ is the vector
space.\\
\\
{\bf Classical Groups:}
Every {\bf transvection} in $SL_n(F)$ is conjugate if $n>2$.  Group orders:
$PSL_n(q)= {\frac 1 {(q-1)(n,q-1)}} (q^{m} -1) (q^{m} -q) (q^{m} -q^{2}) ...  (q^{m} -q^{m-1})$,
simple if $n>2$ or $q>3$.
$PSp_{2l}(q)= {\frac 1 {(2,q-1)}} q^{l^2}(q^{2} -1) (q^{4} -1) ...  (q^{2l} -1)$,
simple unless $(2l,q)= (2,2), (2,3), (4,2)$.
$PSU_n(q^2)= {\frac 1 {(n,q+1)}} (q^{\frac {n(n-1)} 2} -1) (q^{2} -1) (q^{3} +1) 
(q^4 -1)...  (q^{n} - (-1)^n)$,
simple unless $(2l,q)= (2,4), (2,9), (3,4)$.
For next two, set $\Omega_n(q)= (O_n(q))' \subseteq SO_n(q)$.
$P\Omega_{2l+1}(q)= {\frac 1 {(n,q-1)}}
q^{l^2}(q^{2} -1) (q^{4} -1) ...  (q^{2l} -1)$,
simple if $l>1$.  Note 
$P\Omega_{2l+1}(q)$ is not isomorphic to $PSp_{2l}(q)$ despite having the same order.
For $|\epsilon|=1$,
${P\Omega^{\epsilon}}_{2l}(q)= {\frac 1 {(4,q^l - \epsilon)}}
q^{l(l-1)}(q^{2} -1) (q^{4} -1) ...  
(q^{2l-2} -1)
(q^{l} - \epsilon)$,
if $q=2^k$, simple if $l>2$.\\
\\
{\bf Finite Simple Group Families: }
${\mathbb Z}_p$, Schur Multiplier: 1.
$\Sigma_n'$ simple if $n>4$,
Schur Multiplier: 6 if $n= 6,7$, 2 if $n=5, n>7$.
$A_n (q) = PSL_{n+1}(q)$ simple if $ n \geq 1$, 
Schur Multiplier: $(n+1,q-1)$ except 
$A_1(4) [2]$,
$A_1(9) [6]$,
$A_2(4) [48]$,
$A_3(2) [2]$.
$B_n (q)= P\Omega_{2n+1}(q)$
simple if $ n \geq 1$ , Schur Multiplier: $(2,q-1)$ except
$B_2(2)$,
$B_3(2) [2]$,
$B_2(2) [6]$;
$C_n (q)= PSp_{2n} (q)$ simple if $n > 2$ , Schur Multiplier: 
$(2,q-1)$ except $C_3(2) [2]$.
$D_n (q)= P\Omega_{2n}^{+}(q)$ simple if $n \geq 4$, 
Schur Multiplier: $(2,q-1)$ except $D_4(2) [4]$.
$E_6 (q)$ of order 
${\frac 1 {(3,q-1)}} q^{36} (q^{12}-1)(q^{9}-1) (q^{8}-1) (q^{6}-1) (q^{5}-1) (q^{2}-1)$, 
Schur Multiplier: $(3,q-1)$.
$E_7 (q)$ 
of order 
${\frac 1 {(3,q-1)}} q^{63} (q^{18}-1) (q^{14}-1) (q^{12}-1) 
(q^{10}-1) (q^{8}-1) (q^{6}-1) (q^{2}-1)$, 
Schur Multiplier: $(2,q-1)$.
$E_8 (q)$ 
of order 
$q^{120} (q^{30}-1) (q^{24}-1) (q^{20}-1) 
(q^{18}-1) (q^{14}-1) (q^{12}-1) (q^{8}-1) (q^{2}-1)$, 
Schur Multiplier: $1$.
$F_4 (q)$ of order $q^{24} (q^{12}-1) (q^{8}-1) (q^{6}-1) (q^{2}-1)$, 
Schur Multiplier: $1$ except $F_4(2)[4]$.
$G_2 (q)$ simple except $G_2(2)$
of order 
$q^{6} (q^{6}-1)(q^{2}-1)$, 
Schur Multiplier: $1$ except $G_2(3) [3]$, $G_2(4) [2]$.
$^2A_n (q^2) = PSU_{n+1}(q)$ simple if $ n \geq 2$, Schur Multiplier: 
$(n+1, q+1)$ except
$^2A_3(2^2) [2]$,
$^2A_3(3^2) [36]$,
$^2A_5(2^2) [12]$ .
$^2D_n (q)= P\Omega_{2n}^{-}(q)$
simple if $ n \geq 4$, Schur Multiplier: $(4, q^n+1)$.
$^3D_4 (q^3)$ 
of order 
$q^{12} (q^{8}+q^4+1) (q^{6}-1) (q^{2}-1)$, 
Schur Multiplier: $1$ .
$^2E_6 (q)$ 
of order 
$q^{36} (q^{12}-1) (q^{9}+1) (q^{8}-1) (q^{6}-1) (q^{2}-1)$, 
Schur Multiplier: $(3,q+1)$ except
$^2E_6(2^2) [12]$.
$^2B_2 (2^{2m+1})= Sz(2^{2m+1})$
simple if $ m > 1$ 
of order 
$q^{2} (q^{2}+1) (q-1)$, 
Schur Multiplier: $1, n>2$.
$^2F_4 (2^{2m+1})$ (Ree) simple if $ m >1$
of order 
$q^{12} (q^{6}+1) (q^{4}-1) (q^{3}+1) (q-1)$, 
Schur Multiplier: $1,m>1$.
$^2G_2 (3^{2m+1})$ (Ree) simple if $ m > 1$
of order 
$q^{3} (q^{3}+1) (q-1)$, 
Schur Multiplier: $1, m>1$.\\
\\
{\bf Sporadic Groups:}
$M_{11}$ ($2^4 \cdot 3^2 \cdot 5 \cdot 11$), Schur: 1.
$M_{12}$ ($2^6 \cdot 3^3 \cdot 7 \cdot 11$), Schur: 2.
$M_{22}$ ($2^{7} \cdot 3^2 \cdot 5 \cdot 7 \cdot 11$), Schur: 12.
$M_{23}$ ($2^{7} \cdot 3^2 \cdot 5 \cdot 7 \cdot 11 \cdot 23$), Schur: 1.
$M_{24}$ ($2^{10} \cdot 3^3 \cdot 5 \cdot 7 \cdot 11 \cdot 23$), Schur: 1.
$J_1$ ($2^{3} \cdot 3 \cdot 5 \cdot 7 \cdot 11 \cdot 19$), Schur: 1.
$J_2 = HJ$ ($2^{7} \cdot 3^3 \cdot 5^2 \cdot 7$), Schur: 2.
$J_3 = HJM$ ($2^{7} \cdot 3^5 \cdot 5 \cdot 17 \cdot 19$), Schur: 3.
$J_4$ ($2^{21} \cdot 3^3 \cdot 5 \cdot 7 \cdot 11^3 \cdot 23 \cdot 29 \cdot 31 \cdot 37 \cdot 43$), Schur: 1.
$Co_1$ ($2^{21} \cdot 3^9 \cdot 5^4 \cdot 7^2 \cdot 11 \cdot 13 \cdot 23$), Schur: 2.
$Co_2$ ($2^{18} \cdot 3^6 \cdot 5^3 \cdot 7 \cdot 11 \cdot 23$), Schur: 1.
$Co_3$ ($2^{10} \cdot 3^7 \cdot 5^3 \cdot 7 \cdot 11 \cdot 23$), Schur: 1.
$HS$ ($2^{9} \cdot 3^2 \cdot 5^3 \cdot 7 \cdot 11$), Schur: 2.
$Mc$ ($2^{7} \cdot 3^6 \cdot 5^3 \cdot 7 \cdot 11$), Schur: 3.
$Sz$ ($2^{13} \cdot 3^7 \cdot 5^2 \cdot 7 \cdot 11 \cdot 13$), Schur: 1.
$Ly$ ($2^{8} \cdot 3^7 \cdot 5^6 \cdot 7 \cdot 11 \cdot 31 \cdot 37 \cdot 67$), Schur: 1.
$He$ ($2^{10} \cdot 3^3 \cdot 5^2 \cdot 7^3 \cdot 17$), Schur: 1.
$Ru$ ($2^{14} \cdot 3^3 \cdot 5^3 \cdot 7 \cdot 13 \cdot 29$), Schur: 1.
$O'N-S$ ($2^9 \cdot 3^4 \cdot 5 \cdot 7^3 \cdot 11 \cdot 19 \cdot 31$), Schur: 3.
$F_{22}$ ($2^{17} \cdot 3^9 \cdot 5^2 \cdot 7 \cdot 11 \cdot 13$), Schur: 6.
$F_{23}$ ($2^{18} \cdot 3^{13} \cdot 5^2 \cdot 7 \cdot 11 \cdot 13 \cdot 17 \cdot 23$), Schur: 1.
$F_{24}$ ($2^{21} \cdot 3^{16} \cdot 5^2 \cdot 7^3 \cdot 11 \cdot 13 \cdot 17 \cdot 23 \cdot 29$), Schur: 3.
$F_3$ (Thompson) ($2^{15} \cdot 3^{10} \cdot 5^3 \cdot 7^2 \cdot 13 \cdot 19 \cdot 31$), Schur: 2.
$F_5$ (Harada) ($2^{14} \cdot 3^{6} \cdot 5^6 \cdot 7 \cdot 11 \cdot 19$), Schur: 1.
$F_2$ (Baby Monster) ($2^{41} \cdot 3^{13} \cdot 5^6 \cdot 7^2 \cdot 11 \cdot 13 
\cdot 17 \cdot 19 \cdot 23 \cdot 31 \cdot 47$), Schur: 2.
$F_1$ (Monster) ($2^{46} \cdot 3^{20} \cdot 5^9 \cdot 7^6 
\cdot 11^2 \cdot 13^3 \cdot 17 \cdot 19 \cdot 23 \cdot 29 \cdot 31 \cdot 41 \cdot 47 \cdot 
59 \cdot 71$), Schur: 1.\\
\\
Let $\Delta$ be an orbit of $G$ and let $\delta \in \Delta$.  For each $\gamma \in \Delta$
let $v(\gamma ) \in G$ be such that $\delta \mapsto \gamma$.  Finally, suppose $S$
generates $G$.  Then $G_{\delta}=<v(\gamma)sv(\gamma^{s})^{-1} | \gamma \in \Delta,
s \in S>$.\\
\\
{\bf System of imprimitivity for permutation 
group $G$:} ${\cal B}=\{ \Delta_i\}$ $|\Delta_i|>1$ with
the property that for $\Delta \in {\cal B}, g \in G$ either 
$\Delta \cap \Delta^g = \phi$ or
$\Delta = \Delta^g$.  {\bf Primitive:} No set of imprimitivity.
$\Gamma$ is $G$ invariant if $\Gamma^G= \Gamma$ so $\Gamma$ is a union of $G$ orbits.
$G/G_{\Gamma} \equiv G^{\Gamma}$.
If $\Delta \subseteq \Gamma$ and $\alpha \in \Omega$ then 
$\psi = \bigcap_{\alpha \in \Delta^g} \Delta^g$ is a block of a transitive
group $G \subseteq Sym(\Omega)$.
A transitive group is imprimitive iff $\exists Z$: $G_{\alpha} < Z < G$.
$G$ is primitive iff $G_{\alpha}$ is maximal.
Let $G$ act transitively on $\Omega$, $H \lhd G$ then
(1) The orbits of $H$ are blocks of $G$,
(2) If $\Delta$ and $\Delta'$ are two $H$ orbits then they are permutation isomorphic,
(3) If any point lies is fixed by all elements of $H$ then $H$
lies in the kernel of the action on $\Omega$,
(4) The group $H$ has at most $|G:H|$ orbits, if finite, it divides $|G:H|$,
(5) If $G$ acts primitively on $\Omega$ then either $H$ is transitive or it
lies in the kernel of the action.\\
\\
Define ${\cal G}(G, \Omega )$ as the graph of $G$ acting on $\Omega$ as follows:
$G$ acts on $\Omega \times \Omega$.  Diagonal orbital is
$\Delta_1 = \{ (\alpha , \alpha )\}$. If $\Delta = \{ (\alpha , \beta ) \}$,
$\Delta^*= \{ (\beta, \alpha) \}$.  Self paired if $\Delta^* = \Delta$.
$\Delta(\alpha)= \{ \beta : (\alpha , \beta ) \in \Delta \}$ --- corresponds to 
orbits of $G_{\alpha}$. The {\bf rank of the permutation group} is number of orbitals.
On a self-paired orbit $\Delta$, the graph ${\cal G}=(G, X,\Delta)$ is 
symmetric and $G$ is transitive on edges.
Let $G$ be a transitive permutation group of even order and rank 3 with two
necessarily self-paired non-diagonal orbits $\Delta$ and $\Gamma$.  $G$ is primitive
iff ${\cal G}$ is connected.\\
\\
A transitive permutation group is {\bf regular} if $|X|= |G^X|$ or, equivalently 
$|G_x|=1, \forall x \in X$ and $G^X$, transitive.
Let $X$ be a faithful primitive $G-set$ with $G_x$ simple.  The either $G$ is simple or
every non-trivial normal subgroup $H$ of $G$ is a regular normal subgroup.
{\bf Iwasawa:}  Let $G=G'$ and
$X$ be a faithful primitive $G-set$.  If there is an $x \in X$ and an Abelian normal
subgroup $K \lhd G_x$ whose conjugates generate $G$ then G is simple.  {\bf Permutation
representation:} Let $H \le G$ and $Hg_1 , ..., Hg_n$ be the cosets; the map
$\pi(g): <Hg_1, ..., Hg_n> \mapsto <Hg_1 g, ..., Hg_n g>$ is a map from $G$ to $\Sigma_n$
whose kernel is the largest normal subgroup of $G$ in $H$.  Corollary:
If $H<G$ and $G$ is simple then $|G| \mid |G:H|!$. 
If $G^X$ is primitive and $1 \ne N \lhd G^X$ then $N^X$ is
transitive.  If $G^X$ is primitive and $G_x$ is simple then either (1) $G$ is simple, or
(2) $\exists N \lhd G: N^X$ is regular. If $N$ is a regular normal subgroup of
$G^X$ then $G_x$ acts on $N^\#$.  If $A$ is transitive on $H^\#$ then $H \cong (Z_p)^n$,
if 2-transitive, $H \cong (Z_2)^n$ or $Z_3$, if 3-transitive, $H \cong (Z_2)^2$.
{\bf Semi-regular action:}  $C_G(a)=1, \forall a \in A^{\#}$.  Suppose $A$ acts semi-regularly
on $G$.  Then (1) $|G| = 1 \jmod{|A|}$, (2) $A$ is semi-regular on each $A-$invariant
subgroups factor group of $G$, (3) $\forall p \in \pi(G)$, $\exists! A-$invariant
Sylow $p-$subgroup of $G$, (4) $\forall a \in A, g \mapsto [g,a]$ is a permutation of
$G$, (5) if $2 | |A|, \exists t: |t|=2, t \in A: g^t = g^{-1}, g \in G$ and
$G^{(1)}=1$.
{\bf Frobenius group:} Transitive permutation group with non-trivial stabilizers
but only the identity fixes more than one letter.  If $G$ is a Frobenius group
then the set $S$ of elements which fix no points together with $e$ form a normal
subgroup of order $|G:G_a|$; Thompson showed this normal subgroup is nilpotent.\\
\\
{\bf Metacyclic:} $\exists H \lhd G: G/H ,H$ are cyclic. 
$Core_G(H)= \bigcap_{g \in G} H^g$ (Can use this to show $|G:Core_G(H)|\le |G:H|!$).
$O^{\cal A}(G)= \bigcap_{A \lhd G, G/A \in {\cal A}} A$.
$O_{\cal A}(G)= \prod_{A \lhd G, A \in {\cal A}} A$.
{\bf Socle:} $soc(G)= <M>$ where $M$ is a non-trivial minimal normal subgroup of $G$.
$O_{\pi}(G)=$ maximal normal $\pi-$subgroup of $G$.
$O^{\pi}(G)=$ smallest normal subgroup of $G$ such that $G/O^{\pi}(G)$ is a $\pi$-group.
$G$ is $p-$closed if $O_p(G) \in S_p(G)$.
$SCN(P)=$ set of self centralizing normal subgroups of $P$.
$SCN(p)= SCN(P)$ where $P \in SCN(P)$.
${\cal N}_G(A, \pi)=$ set of all $A-$ invariant $\pi$ subgroups of $G$.
${\cal N}_G^*(A, \pi)=$  maximal subgroups in ${\cal N}_G(A, \pi)$.  For a $p-$group, $P$,
$\Omega_n(P)= <x \in P: x^{p^n}=1>$ and
$\mho_n(P)= <x^{p^n}: x \in P>$.
$H \subseteq G$ and $S$ an $H$-invariant subset of $G$, $H$ is said to control fusion
in $S$ if for $s \in S$, $s^G \cap S = s^H$.
Let $X \le H \le G$.  $X$ is {\bf weakly closed} in $H$ with respect to $G$ if
$X^g \cap H = \{X\}$.
$G$ is {\bf $p-$solvable} if it has a normal series whose factors are either
$p-$groups or $p'$-groups.  
$G$ is {\bf $p$-constrained} if 
$P \in S_p ( O_{p',p} (G))$ implies $C(P) \subseteq O_{p',p} (G)$.
$G$ is {\bf $p$-stable} if $p \ne 2$ and 
if $A \in p(N(P))$ with $[P,A,A]= 1$ implies $A C(P)/C(P) \subseteq O_p (N(P)/C(P))$.
$m_p(P)$ is the rank of the largest elementary abelian $p$-group in $P$.
$O_{\infty}(G)=$ largest solvable normal subgroup of $G$.
$F(G)$ is the unique maximal normal, nilpotent subgroup of $G$ and
$F(G)= \prod_p O_p(G)$ .
$E_{p^n}$ denotes the elementary abelian $p-$group of rank $n$.
$m_{2,p}(G) = max \{ m_p (H) \}$, where $H$ is 2-local.
$e(G) = max \{ m_{2,p} (G), p \ne 2 \}$ ($e(G)$ is a good approximation of
the Lie rank.).
$O_{p'}(G)$ is called the {\bf $p$-core} of $G$. $O_{2'}(G)$ is often called the
{\bf core} of $G$ and is sometimes denoted by $O(G)$.
Walter: Let $G$ be a group with 2 rank $\ge 5$ and $O_{2'}(G)=1$ with the property
that the centralizer of every involution is $2-$constrained then $O_{2'}(C(x))=1$ for
every involution $x$.\\
\\
{\bf Modular Property:} If $A, B, C \le G$ and $A \le C$ then $AB \cap C= A(B \cap C)$.
$[ab,c]= [a,c]^b [b,c]$ and $[a,bc]=[a,c] [a,b]^c$.  
{\bf Jacobi:} $ [x, y^{-1}, z] [y, z^{-1}, x] [z, x^{-1}, y]=1$.
If $x,y \in C(z), z=[x,y]$ then
$[x^n , y^m ]= z^{mn}$ and $(yx)^n= y^n x^n z^{\frac {n(n-1)} 2}$.  
{\bf Three Subgroups:} $A, B, C \subseteq G$ and
$N \lhd G$ with
$[A,B,C] \subseteq N$ and
$[B,C,A] \subseteq N$ then
$[C,A,B] \subseteq N$.\\
\\
Let $G$ be a group with $G/Z(G)$ finite, then $G^{(1)}$ is finite.  Proof:
Let $n= |G/Z(G)|$.  For $z \in Z(G)$ and $g,h \in G$: $[g,hz]=[g,h]=[gz,h]$ so the 
set of commutators, $\Delta$, is of order at most $n^2$.
Claim: $g \in G^{(1)}$ then $g= x_1 x_2 \ldots x_m$, $x_i \in \Delta$ and
$m \le n^3$.\\
\\
{\bf Critical subgroup of a $p$-group:}
$H \; char \; G$ with $\Phi(H) \le Z(H) \ge [G,H]$.
$C_G(H)=Z(H)$.  Every $p-$group has a critical subgroup.
A $p-$group $P$ is {\bf special} if $\Phi(G)=Z(G)=G'$ and {\bf extra-special}
if $Z(G)$ is cyclic.
Let $G$ be a non-abelian group of order $p^n$ with cyclic subgroup
$H$ of index $p$ then $G \cong <p^n>, D_{2^n}, SD_{2^n}, Q_{2^n}$.\\
\\
{\bf O-Nan-Scott:}  Let $G$ be a finite primitive permutation group of degree $n$ and
$H=soc(G)$.  Then either (1) $H$ is a regular elementary abelian $p$ group for
some $p$ and $G$ is isomorphic to a subgroup of
$AGL_m(P)$ ; or, (2) $H$ is isomorphic to $T^m$ where $T$ is a non-abelian simple group
with a bunch of conditions.\\
\\
{\bf Mathieu Groups:}
$M_{11}$: $\pi_1= (123)(456)(789), \pi_2= (147)(258)(369)$,
$<\pi_1, \pi_2>= {\mathbb Z}_3 \times {\mathbb Z}_3$, 
$\rho_1= (2437)(5698), \rho_2= (2539)(4876)$, $<\rho_1 , \rho_2> = Q \cong Q_8$.  Set
$M_9= <\pi_1 , \pi_2 , \rho_1 , \rho_2>$, $|M_9|=72$.  
Now set $\sigma= (1, 10)(4,5)(6,8)(7,9)$, $\mu= (4,7)(5,8) (6,9) (10,11)$,
$\theta= (4,9) (5,7) (6,8)(11,12)$.  
$M_{10}= M_9 \cup M_9 \sigma M_9$,  $(M_{10})_x= M_{9}$,
$M_{11}= M_{10} \cup M_{10} \mu M_{10}$,  $(M_{11})_x= M_{10}$,
$M_{12}= M_{11} \cup M_{11} \theta M_{11}$,  $(M_{12})_x= M_{11}$.  $|M_{11}|=7920$.
$|M_{24}|= 24 \cdot 23 \cdot 22 \cdot 21 \cdot 20 \cdot 48 $.  $M_{11}$ is simple:
Let $N$ be a non-trivial normal subgroup, it is regular and all Sylow 11 subgroups
are contained in it (there are 144 by sylow) and $G:N$= 5.  All Sylow 3 subgroups
of $M_{11}$ are in N and $\psi= \pi_1 \sigma \pi_2^2 \sigma^{-1}$ has order 5 which is a
contradiction.  Note symmetries of $S(4,5,11)$ also generate it.  Note that $(M_{11})_a= PSL_2(9)$
and $(M_{22})_a = PSL_3(4)$.\\
\\
{\bf Schur-Zassenhaus:}
Let $G$ be a finite group, $H \lhd G$ and $(|H|, |G:H|)=1$ and either are
solvable then $G$ splits over $H$ and $G$ is transitive on $H$ complements.
Proof of existence
by induction:  Suppose it holds for all groups of order $<G$ and that $|G|=nm;
(m,n)=1; N \lhd G; |N|= n$.  If $\exists K \le G: |K|=m$ then the theorem is true.
Let $P \in S_p(N)$.  (1) We may assume $P \lhd N$:  If not 
$G=N_G(P)N, N_N(P)=N_G(P) \cap N \lhd N_G(P)$ and 
$m=|G/N|=|N(P)N/N|=|N_G (P)/(N_G (P) \cap N|= |N_G(P)/N_N(P)|$ and $N_G(P)$ has a
normal Hall group $N_N(P)$ so by induction $\exists K \subseteq N_G(P)$ with
$|K|=m$ and $N_N(P)K=N_G(P)$, so $NK=G$. (2) We may assume $P=N$: If not,
$|(G/P)/(N/P)|=m$ so $\exists L/P: (N/P)(L/P)= G/P$ and 
$|L|=m |P|$, $|L \cap N| \mid (|L|, |N|)$;
but $(m, |N|)=1$ so $L \cap N \subset P$ and $L<G$ and $\exists K \subset L: |K|=m$.
(3) May assume $N=P$ is abelian:  If not $1 \ne Z=Z(N) \; char ;\ N \lhd G$ and
$|(G/Z)/(N/Z)|=m$ so $\exists L/Z: (L/Z)(N/Z)=(G/Z)$ and $L \cap N =Z, L < G$ and
$(|Z|, |L/Z|)=1$ and $L$ and hence $G$ has a desired subgroup $K$.
(4) So it suffices to show the theorem if $N$ is a normal abelian Hall $p-$group.
Let ${\overline H}= G/N$.  If $h \in {\overline H}$ and $t , u$ are two elements of
$h$ then $t^{-1}u \in N$ so $tnt^{-1}=unu^{-1}$.  Define $^tx=txt^{-1}, t \in h$.
$H$ acts on $N$ - i.e. $H \subset Aut(N)$.
Select a transversal $\{t_h | h \in H \}$.
$ t^{-1}_{h_1h_2}N = (t_{h_1h_2}N)^{-1}=
(h_1h_2)^{-1}= {h_1}^{-1}{h_2}^{-1}, \forall h_1 , h_2 \in H$, so 
$t_{h_1}t_{h_2} t^{-1}_{h_1 h_2} \in N$.  Define $f: H \times H \rightarrow N$ by
$f(h_1, h_2) t_{h_1 h_2} = t_{h_1} t_{h_2}$.  Since
$t_{h_1}(t_{h_2} t_{h_3}) = (t_{h_1}t_{h_2}) t_{h_3}$, we get
$ {^{h_1}} f(h_2, h_3)+ f(h_1, h_2 h_3)= f(h_1, h_2)+ f(h_1 h_2 , h_3)$.  If 
$\exists c:H \rightarrow N: f(h_1 , h_2)= c(h_1 h_2) -c(h_1) - {^{h_1}} c(h_2)$, then
$c(h_1 h_2) t_{h_1 h_2}= c(t_1)t_{h_1} c(t_2) t_{h_2}$, this would be an isomorphism
whose image would satisfy the requirements of $K$.  Define: $e: H \rightarrow N$ by
$e(h) = \sum_{k \in H} f(h , k)$.
$m f( h_1 , h_2 )=  -e(h_1 h_2) + e(h_1) + {^{h_1}}e(h_2)$.  Since $(m, |N|)=1$,
${\frac x m}$ is well defined for $ x \in N$ and $c(x)= {\frac {-1} m} e(x)$ satisfies the
desired properties.
Proof of conjugacy:
Suppose $G/N$ is solvable and $\pi$ is the set of primes dividing $m=|G:N|$ and
$H,K \le G$ and $|H|=|K|=m$, put $R=O_{\pi}(G)$ so $O_{\pi}(G/R) = 1$.  
Let $L/N$ be a minimal
normal subgroup of $G/N$ then $L/N$ is an elementary abelian $p-$group for some $p$.
$H\cap L \in S_p(L)$ and
$S=(H \cap L)=(K \cap L)^g= K^g \cap L$.  $S \lhd <H, K^g>=J$.  If $J=G$, $S \lhd J$
and $S \subseteq R=1$; thus $L$ is a $p'-$group which is a contradiction.  So
$J \ne G$ and by induction $K$, $K^g$ are $J$-conjugate.  
This concludes this case.  Suppose
$N$ is solvable and again $|H|=|K|=m=|G:N|$.  
$HN'/N' \cong KN'/N'$ so $h^g \subseteq KN'$ and
again by induction, $H^{gk}= K$.\\
\\
{\bf Philip Hall's Theorem:} 
Let $G$ be a solvable group and $\pi$ a set of primes then (i) $G$ has a $\pi$-Hall
subgroup, (ii) $G$ acts transitively on its Hall $\pi$-subgroups via conjugation, 
(3) any $\pi$ subgroup is contained in a Hall $\pi$ subgroup.
Proof: By induction on $|G|$.  Let $N$ be a minimal normal subgroup of $G$ then
$1 \ne N \lhd G$.  $N$ is elementary abelian for some $p$ and $p \mid mn$.   If
$p \mid m, |G/N|= {\frac m p}$ and $\exists L: |L/N|= {\frac m p}, |L|= m$ and we're done.
If $p \mid n, \exists H: |H/N|=m, |H|= |N|m$.  If $|H| < |G|$, we're done by induction.
Otherwise $H=G, N \lhd G, |N|= n, |G:N|=m$ and $(m,n)=1$ so by Schur Zassenhaus,
$\exists K: |K|=m$.\\
\\
{\bf Theorem: } Let $G$ be a finite group possessing a Hall $\pi'$ subgroup for each $p$, then $G$ 
is solvable. (Proof requires Burnside $p^a q^b$ theorem.)\\
\\
{\bf Frattini subgroup:} $\Phi(G)$ is the intersection of all maximal subgroups of $G$.
$\Phi(G) \; char \; G$.  If $H= <X, \Phi(H)>$ then $H= <X>$.  If $P$
is a $p-$group $P/\Phi(P)$ is elementary abelian.
Frattini Argument: $H \lhd G$, $P \in S_p(H)$ then $G=H N_G(P)$.\\
\\
If $A$ is a maximal abelian normal subgroup of $P$ and $Z=\Omega_1(A)$.  Then
(1) $(C_P(A/Z) \cap C(Z))^{(1)} \le A$, (3) if $p$ is odd $\Omega_1(C_P(Z)) \le C_P(A/Z)$.
If $p$ is odd and $Z$ is a maximal elementary abelian subgroup of $P$ then
$Z \setminus \Omega_1 (C_P(Z))$.\\
\\
{\bf Co-prime action 1:}
In this paragraph $A$ acts on $G$ and $(|A|, |G|)=1$ with either
$A$ or $G$ solvable.  If $U \leq G$ is $A-$invariant and $g$ satisfies $(Ug)^A=Ug$ then
$\exists c \in C_G(A)$: $Ug=Uc$. If $N$ is an $A-$invariant normal subgroup of $G$ then
(1) $C_{G/N}(A)= C_G(A)N/N$ (This shows $G=[G,A]C_G(A)$.) and (2) if $A$ acts
trivially on $N$ and $G/N$ then $G$ acts trivially on $G$. If $p \mid |G|$ (the analogous
results hold for $\pi$) then (1) $\exists S \in S_p(G): S^A=S$, (2) all such $A-$invariant
Sylow $p-$groups are conjugate under $C_G(A)$, (3) every $A-$invariant $p$-group of
$G$ is contained in an $A-$invariant Sylow $p-$group.  
If $T=\bigcap_{S \in S_p(G), S^A=S} S$,
the $T$ is the largest $A-$invariant $p-$subgroup of $G$ normalized by $C_G(A)$.  If
$P$ is an $A-$invariant Sylow $p-$group and $H \le G$ with $H^A=H, H^{C_G}(G)=H$ then
$P \cap H \in S_p(H)$.  If $A= P \times Q$ acts on $M$ and $P, M$ are $p-$groups and
$Q$ is a $p'-$group with $C_M(P) \le C_M(Q)$ then $[M,Q]=1$. If $A$ acts trivially on
$G/\Phi(G)$ then $A$ acts trivially on $G$ and if $\Phi(G)$ is a $p-$group then
so is $A/C_A(G)$. 
Applying $P \times Q$: If $p \in \pi(G)$ and ${\overline G}= G/O_{p'}(G)$
with $C_{\overline G}(O_p({\overline G})) \le O_p({\overline G})$ then 
$\forall P \in p(G), O_{p'}(N_G(P))=O_{p'}(G) \cap N_G(P)$.
{\bf Thompson:} Let $a$ be a $\pi'$ automorphism of a $\pi$ group $P$ and suppose
$X \lhd \lhd P$ such that $[a,X]=1=[a,C_P(X)]$ then $a=1$.
${\bf P \times Q}$ {\bf Lemma:} Let $A= P \times Q$, $P$ a $p-$group, $Q$ a $p'$-group.
Suppose $M$ is a $p$-group and $C_M(P) \le C_M(Q)$.  Then $Q$ acts trivially on $M$.\\
\\
{\bf Co-prime action 2:}
If $P$ is a $p-$group and $Q$ a $p'-$ group with $Q \mapsto Aut(P)$ then
$Q$ is faithful on $P/\Phi(P)$.
A group of automorphisms $A$ of a group $P$ stabilizes a chain
$1=P_n \subseteq P_{n-1} \subseteq \ldots \subseteq P_0 = P$ if
$[A,P_i] \subseteq P_{i+1}$.  If $P$ is a $\pi$ group stabilized by
$A$ then $A$ is a $\pi$ group.  Proof: $a \in A$ is a $\pi'$ automorphism.
$x^a =xy, y \in P_1$.  Similarly, $x^{a^{|a|}}= x y^{|a|}=x$, so $y=1$ and
$[a,P]=1$.
If $A$ is a $\pi'$ group of automorphisms on a $\pi$ group $P$ with
$[P,A,A]=1$ then $[P,A]=1$.  Proof:  $A$ stabilizes
$[P,A,A] \subseteq [P,A] \subseteq P$.
Let $A$ be a $\pi'$ group of automorphisms of a $\pi$ group $P$.  Let $Q$ be
an $A-$invariant normal subgroup of $P$.  Then $C_{P/Q}(A)= (C_P(A) Q)/Q$.  Proof
uses Schur-Zassenhaus.
$P$ is a $\pi$ group, $A$ is a $\pi'$ group.  $P= [P,A] C_P(A)$.  Proof:
$[P,A] \subseteq P$ and $A$ centralizes $P/[P,A]$.
$P$ is an abelian $\pi$ group, $A$ is a $\pi'$ group.  $P= [P,A] \oplus C_P(A)$.  Proof:
$\theta= {\frac 1 {|A|}} \sum_a a$.\\
\\
If $G$ is solvable, (1) $C(F(G)) \subseteq F(G)$, (2) if $P$ is a $p-$group
of $G$ then $O_{p'}(C(P)) \subseteq O_{p'}(G)$ and $O_{p'}(N_G(P)) \subseteq O_{p'}(G)$.
If $P \in p(G)$ with $N_G(P)$ $p-$constrained then $C_G(P)$ is also $p-$constrained.
$|A|, |H| < \infty, (|A|, |H|)=1$.  Suppose $A \rightarrow Aut(H)$ and either are
solvable then (1) $\exists A-$invariant Sylow $p-$group of $H$, 
(2) $C_H(A)$ is transitive on the $A-$invariant sylow $p-$subgroups of $G$,
(3) If $K$ is an $A-$invariant normal subgroup of $H$ and $H^*=H/K$ then
$C_{H^*}(A)= N_{H^*}(A)=(C_H(A))^*$. (5) Every $A-$invariant
$p-$subgroup of $H$ is contained in an $A-$invariant Sylow $p-$group of $H$.\\
\\
{\bf Transfer:} $|G|< \infty, H \le G$ .
$|G:H|=n$ and
$\{ l_1 , l_2 , \ldots , l_n \}$ be a left traversal and suppose
$gl_i= l_j x_i$ then $V(g)= \prod_{i=1}^n x_i H'$.
$\exists h_1 , h_2 , \ldots , h_m \in H$ and  $n_1 , n_2 , \ldots , n_m$. 
(1) $h_i \in \{ l_1 , l_2 , \ldots , l_n \}$,
(2) $h_i^{-1} g^{n_i} h_i \in H$,
(3) $\sum_{i=1}^m n_i = |G:H|$,
(4) $V(g)= \prod (h_i^{-1} g^{n_i} h_i ) H'$.
If $Q$ is an abelian subgroup of finite order $n$ in $G$ and if $Q \subseteq Z(G)$ then
$V(g)= g^n, \forall g \in G$.  Let $Q \in S_p(G)$; if $g,h \in C(Q)$ and $g$ and $H$ are
$G$ conjugate then they are $N(Q)$ conjugate.
Let $Hx_i g^j, 1 \le i \le r, 0 \le j \le n_i$, cycles of $g$ 
on $G/H$.
$X= \{ x_i g^j \}$ 
then (a) $(g^{n_i})^{x_i^{-1}} \in H$ for $1 \le i \le r$, (b) $\sum_{i=1}^r n_i= |G:H|$ and
(c) $V(g) = \prod_{i=1}^r ((g^{n_i})^{x_i^{-1}})^{\alpha}$.\\
\\
Let $G$ be a finite group $H \le G$, $(p, |G:H|)=1, K \lhd H$, $H/K$ abelian,
$g$ a $p-$element in $H \setminus K$: $g^{ma} \in g^m K, \forall m$, all $a \in G$ such that
$g^{ma} \in H$ then $g \notin G^{(1)}$.\\
\\
${\bf p}${\bf -constraint:}  If $O_{p'}(G)=1$ then $C_G(P) \subseteq O_p(G)$.
${\bf \pi}${\bf -solvable:}  Normal series consists of either $\pi'$-groups or a solvable
$\pi$-groups.  {\bf Hall-Higman 1.2.3:}  Let $G$ be $\pi$-solvable and $O_{\pi'}(G)=1$, then
$C_G(O_{\pi}(G)) \subseteq O_{\pi}(G)$.  Let $P \in S_p(G)$:
(1) $N_G(P)$ controls fusion on $C_G(P)$.  
(2) If $P \subseteq Z(N(P))$ then $P$ has a normal $p$-complement.
(3) If $p$ is the smallest prime dividing $|G|$
and $P$ is cyclic then $P$ has a normal $p$-complement.
(4) $P$ has a normal $p$-complement iff $P$ controls its own fusion in $G$.
If all sylow subgroups are cyclic, $P$ is solvable.  The following are equivalent:
(a) $G$ has a normal $p$-complement; (2) $N_G(X)$ has a normal
$p$-complement for all non-trivial $X \in p(G)$;
(c) $N_G(X)/C_G(X)$ is a $p$-group $\forall X \in p(G)$.  If $A$ is a $p'$-group,
then $A$ is failthful on $G/\Phi(G)$.  $X$ is weakly closed in $H$ with respect to $G$ if
$X^G \cap H = X^H$.  {\bf Baer:} Let $X$ be a $p$-group of $G$ then either
$X \le O_p(G)$ or $\exists g: <X, X^g>$ is not a $p$-group.
\\
\\
{\bf Fusion:}
Let $p$ be a prime, $T \in S_p(G), W \le T$ with $W$ weakly closed in $T$ with respect
to $G$ and $D=C_G (W)$.  Then $N_G(W)$ controls fusion in $D$.
$P \in S_p(G)$.  $X \in p(G)$ is a tame intersection of
$Q, R \in S_p (G)$ if $X= Q \cap R$ and $N_Q (X), N_R (X) \in S_p(N(X))$.
{\bf Alperin's Fusion Theorem:}
If $P \in S_p(G), g \in G$ and $<A, A^g> \subseteq P$.  Then 
for $1 \le i \le n$, $\exists Q_i \in S_p (G)$ and $x_i \in N(P \cap Q_i)$ such
that (1) $g= x_1 x_2 ... x_n$,
(2) $P \cap Q_i$ is a tame intersection of $P$ and $Q_i$ for each $i$,
(3) $A \subseteq P \cap Q_1$ and
$A^{x_1 x_2 ... x_i} \subseteq P \cap Q_{i+1}$.  Supporting lemmas:
$R,Q \in S_p(G)$.  Say $R \rightarrow Q$ if 
$\exists Q_i \in S_p(G), X_i \in N_G(P \cap Q_i )$
such that (1) $P \cap Q_i$ is tame, (2) $P \cap R \le P \cap Q_1$ and
$P \cap R)^{x_1 x_2 \ldots x_i} \le P \cap Q_i$ and (3)
$R^x=Q, x= x_1 x_2 \ldots x_n$.  Sometimes say $R \rightarrow_x Q$.
(1) $Q \rightarrow P, \forall Q \in S_p(G)$.  (2) $P \rightarrow P$. 
(3) $\rightarrow$ is transitive.
(4) $S \rightarrow_x P$, $Q^x \rightarrow P$ and $P \cap Q = P \cap S$
then $Q \rightarrow P$.  (5) Assume $P \cap Q$ is tame and 
$S \rightarrow P, \forall S \in S_p(G)$ with $|S \cap P| > |Q \cap P|$ and $S \rightarrow P$
then $Q \rightarrow P$.\\
\\
{\bf Gaschutz:}  Let $K$ be a normal abelian p-subgroup of a finite group $G$ and let
$P \in S_p(G)$.  Then $K$ has a complement in $G$ iff $K$ has a complement in $P$.
If $K$ is an abelian normal subgroup of $G$ with $(|K|,|g:K|)=1$
then $K$ has a complement.  Proof: Set $\sigma(x)= \sum_{y \in Q} f(x, y)$.\\
\\
{\bf Focal Subgroup Theorem:}
$S \in S_p (G)$ then $S \cap G' = <x^{-1} y | x,y \in S, x \sim _G y>$.
Suppose $P \in S_p(G)$ and $A_1, A_2 \lhd G$, if $A_1^g=A_2$, then $\exists y \in N_G(P):
A_1^g=A_2$.
{\bf Burnside Normal $p$-complement}: (proved using transfer):  
If $P \in S_p(G)$ and $P \subseteq Z(N(P))$ then $P$
has a normal $p$-complement.
If $P \in S_p (G), P'=1$ then $P \cap G' = P \cap N_G(P)'$.
{\bf Frobenius Normal $p-$complement:}  The following are equivalent:
(1) $G$ has a normal $p-$complement, 
(2) Each $p-$local subgroup of $G$ has a normal $p-$complement,
(3) $Aut_G(P)$ is a $p-$group $\forall P \in p(G)$.
If $H \le G$ and $H \cap H^g = 1, \forall g \in (G \setminus H)$ then $G=NH, N \lhd G$.
\\
\\
{\bf Thompson subgroup:} $A(P)$: abelian subgroups of $P$ of maximal order.
$J(P)= <\{ A | A \in A(P) \} >$.  If $O_p(G) \ne 1$, $G$ is $p$-stable
and $p$-constrained, $p \ne 2$.  If $P \in S_p (G)$ then
$G= O_{p'} (G) N(Z(J(P)))$.
{\bf Thompson Factorization:} Let $G$ be solvable with $F(G)=O_p(G)$, $P \in S_p(G)$,
$Z= \Omega_1 (Z(P))$, $V=<Z^G>$, $G^* \cong G/Z$.  The either (i) $G=N_G(J(P))C(Z)$; or
(ii) $p \le 3$ and $J(G)^*$ is a direct product of copies of
$SL_2(p)$ permuted by $G$ and $J(P)^* \in S_p(J(G)^*)$.  Note if
$p=3$ and $G$ has an abelian Sylow $2-$subgroup, so (i) holds.
{\bf Thompson Normal $p-$Complement:}
Let $p \ne 2$ and $P \in S_p(G)$.  Assume $N_G(J(P))$
and $C_G(\Omega_1(Z(P)))$ have a normal $p-$complement then so does $G$.
By Burnside transfer, $A \in SCN(p) \rightarrow C_G(A)= A \times Q, Q \in p'(G)$.
Property PC: If $G$ is a group in which the normalizer of every $p$ group is
$p$-constrained we say $PC(G)$.
{\bf Thompson Transitivity Theorem:}  If $PC(G$) 
and if $A \in SCN_3(p)$ then $C_G(A)$ permutes all
maximal $A$-invariant $q$ groups of $G$, $q \ne p$. Consequence:
Under the TTT conditions, if $P \in S_p(G), A \in SCN_3(P)$ and $\forall q \ne p$,
$P$ normalizes some $A-$invariant $q-$subgroup of $G$; so if $P$ normalizes no
$p'$ subgroup of $G$, neither does $A$.  Used to show the 
{\bf Maximal Subgroup Theorem:}
If $P \in S_p(G), SCN_3(P) \ne \emptyset, p \ne 2$ and every element of $N^*(P)$
is $p-$constrained and $p-$stable and $\exists 1 \ne H \lhd P$: $[Q,P]=1$ if $H \in p'(G)$
and $H^P=H$ then $N^*(P)$ has a unique maximal element.\\
\\
{\bf Thompson (from N-group paper):} $G$ is not solvable iff 
$\exists x, y, z \in G \setminus \{1\}$
with $(|x|, |y|)=(|y|,|z|)=(|x|,|z|)=1$ such that $xy=z$.  If $G$ is a non-abelian
simple group all of whose $p-$locals are solvable then $G$ is isomorphic to one
of the following: (1) $PSL_2 (q), q>3$, (2) $Sz(q), q= 2^{2m+1}, m \ge 1$ or (3)
$A_7$, $PSL(2(3)$, $U_3(3)$, or $M_{11}$.\\
\\
{\bf Quadratic action:}
If $V$ is an abelian $p-$group then $a$ acts quadratically on $V$ if $[V,a,a]=1$ or
$v^{(a-1)^2}=0$.  If $G$ acts quadratically on $V$ then (a) $[v^n,a]=[v,a^n]=[v,a]^n$,
(b) $|V| \le |C_V(a)|^2$, (c) $G/C_G(V)$ is an elementary abelian $p-$group.
If $G$ acts on an $F_q$ vector space $W \ne 0$, $q=p^m$.  Suppose $G=<a,b>$ and
$a,b$ act quadratically on $W$, $G/C_G(W)$ is not a $p-$group, $|ab|=p^ek, k \mid (p-1)$
then $\exists \varphi: G \rightarrow SL_2(q)$.  $G$ is {\bf $p-$stable} if $\forall a \in G,
[V,a,a]=1$ implies $a C_G(V) \in O_p ( G/C_G(V))$.  Let $p \ne 2$ and $G$ be faithful on $V$.
Suppose (1) $G=<a,b>$ where $a$ and $b$ act quadratically on $V$ and
(2) $G$ is not a $p-$group then (1) the Sylow $2$ subgroups of $G$ are not abelian and
(2) If $Q$ is a normal $p'$-subgroup of $G$ and $[Q,a] \ne 1$ then $p=3$ and there
is a section of $G$ isomorphic to $SL_2(3)$.  If $p \ne 2$.  Suppose the action of
$G$ on $V$ is faithful and not $p-$stable then (1) the Sylow $2$-subgroups of
$G$ are non-Abelian and (2) if $G$ is $p-$separable ($G$ is said to be 
{\bf $p$-separable} if two non conjugate elements of $G$ remain non-conjugate in 
some finite $p$-group endomorphic image of $G$.) 
then $p=3$ and there is a section of
$G$ isomorphic to $SL_2(3)$.  Suppose $G$ acts faithfully on $V$ and $E_1 , E_2$ are
two subnormal subgroups of $G$ 
such that $[V,E_1 , E_2 ]=1$ then $[E_1 , E_2 ] \le O_p (G)$.  Let $G$ be a group
and $C_G(O_p(G)) \le O_p(G)$ then $V=< \Omega(Z(S)) | S \in S_p(G)>$ is an
elementary abelian normal subgroup of $G$ and $O_p(G/C_G(V))=1$.\\
\\
$Q_8= <
\left(
\begin{array}{cc}
i & 0 \\
0 & -i \\
\end{array}
\right),
\left(
\begin{array}{cc}
0 & -1 \\
-1 & 0 \\
\end{array}
\right)>$.
Let $m= max \{ |A|, A \in {\cal E}(G) \}$, ${\cal A}(G)= \{A \in {\cal E}(G) | |A|=m \}$
and $J(G)= < A | A \in {\cal A}(G) \}$.  Let $A \in {\cal A}(G)$ acts quadratically
on $V$ and $A_0 = [V,A] C_A([V,A])$ then $A_0$ is in ${\cal A}(G)$ and acts quadratically
on $V$ and if $[V,A] \ne 1$ then $[V, A_0 ] \ne 1$.  {\bf Thompson factorizable}
with respect
to $p$ if $G=O_{p'}(G) C_G( \Omega(Z(S))) N_G(J(S))$.  Let $O_{p'}(G)=1$ and
$V= < \Omega (Z(S)) | S \in S_p(G)>$ then $G$ is Thompson factorizable iff
$J(G) \le C_G(V)$.\\
\\
{\bf Weilandt:} If 
$A \lhd \lhd G$ and $B \lhd \lhd G$ then $<A, B> \lhd \lhd G$; 
if $A \lhd \lhd <A,A^g>, \forall g \in G$ then $A \lhd \lhd G$.
{\bf Quasi-simple:} $L'=L$ and $L/Z(L)$ is simple.  $L$ is a {\bf component} of $H$ if
$L \lhd \lhd H$ and $L$ is quasi-simple.  Let $Comp(G)= \{H: H$ is a component of $G \}$.
$E(G)= <Comp(G)>$ where $H$ is a component of $G$.  
If $K \in Comp(G), U \lhd \lhd G$ then $K \subseteq U$ or $[K,U]=1$.
{\bf Generalized Fitting Subgroup:} $F^*(G)=F(G)E(G)$. $C_G(F^*) \subseteq F^*(G)$. 
Let $X/Z(X)$ be a non-abelian simple group then $X=X'Z(X)$ and $X'$ is
quasi-simple.  Let $L \in Comp(G)$, $H$ and $L-$ invariant subgroup, then
(a) $L \in Comp(H)$ or $[L,H]=1$, (b) If $H$ is solvable, $[L,H]=1$.
If $E^*=E(G)/Z(E(G))$ then (a)$Z=Z(L): L \in Comp(G)>,$ 
(b) $E^*$ is a direct product of $<L: L \in Comp(G)>$,
$E$ is a central product of its components.
$G$ is of {\bf characteristic $p-$type}
if  $F^*(H)=O_p(H)$ for every $p-$local, $H$ (Groups of Lie type over characteristic
$p$ are, for example.).
$G$ is of characteristic $p-$type if $P \in p(G), N= N_G(P) \rightarrow F^*(N)=O_p(N)$.
$PSL_n(p^m)$ is of characteristic $p-$type.  Let $G$ be a non-abelian simple group,
$G$ is of characteristic $p-$type iff $F^*(N(P))= O_p(N(P))$ for every maximal $p-$local.
If $F^*(G)$ is a $p-$group then so is $F^*(N(P)), \forall P \in p(G)$ (use $P \times Q$).
\\
\\
{\bf Amalgams:} $P_1, P_2 \le G$, $|P_i|< \infty$.  Construct a graph $\Gamma(G, P_1, P_2)=\Gamma$
as follows: 
$\Gamma$ has verticies consisting of right cosets of $P_1$ and $P_2$; the verticies
$P_i g_j$ and $P_n g_m$ are
joined by an edge if 
$P_i g_j \ne P_n g_m$ and
$P_i g_j \cap P_n g_m \ne \emptyset$.  $\Delta(\alpha)$ denotes the verticies
adjacent to $\alpha$.  $G$ act on graph by right multiplication on cosets.  
$G \rightarrow Aut(\Gamma)$.  $\Gamma$ is connected iff $G= <P_1 , P_2>$.  {\bf Theorem:}
(a) $G$ has $2$ orbits.  Every vertex stabilizer  $G_{\alpha}$ is a $G-$conjugate of
$P_1$ or $P_2$. (b) $G$ acts transitively on edges of $\Gamma$; every edge stabilizer
in $G$-conjugate of $P_1 \cap P_2$.  (c) $G$ acts transitively on $\Delta(\alpha)$.
$|\Delta(\alpha):\Delta(\alpha, \beta)|= |G_{\alpha}: G_{\alpha, \beta}|, \beta \in \Delta(\alpha)$.
(d) $(P_1  \cap P_2)_G$ (the largest normal subgroup of $G$ in $P_1 \cap P_2$)
is the kernal of the action of $G$ on $\Gamma$. {\bf Condition} ${\cal A}$:  Let $G$ be a
finite group generated by 
$P_1, P_2$,
$T= P_1 \cap P_2$ satisfying: $C_{P_i}(O_2(P_i)) \le O_2(P_i)$, $T \in S_2(P_i)$,
$T_G=1$, $P_i/O_2(P_i) \approx S_3$ and $[\Omega(Z(T)), P_i] \ne 1$.  {\bf Goldschmidt:}
If ${\cal A}$ holds either (i) $P_1 \approx P_2 \approx S_4$ or 
(ii) $P_1 \approx P_2 \approx C_2 \times S_4$.
\\
\\
{\bf Classification by 
centralizers of involutions:}
Brauer proved if $G=PSL_3(q), q = 3 \jmod{4}$ and 
$x \in Inv(G)$ then $C_G(t) \cong GL_2(q)$
and that the converse is true for $q>3$; if $q=3$ other possibilities are $PSL_3(3)$ and
$M_{11}$.  Classifications fall into two steps: (I) Given $H= C_G(t), t \in Inv(G)$, find
$|G|$ and its structure and (II) find $C(t)$ for simple groups.   Note that all
simple groups are determined by their character table. Step (I)
consists of two steps:
(A) $\forall v \in Inv(H)$, determine $C_G(v)$ and the fusion patterns of $Inv(C_G(v))$,
(B) if $G$ has more than one conjugacy class, this determines the order, if not we must
examine all if $H$ using characters.  
Let $L= SL_n(q), G=PSL_n(q) = L/Z(L)$, 
$t \in Inv(G)$ corresponds to $T \in L$ with $T^2= \lambda I_n$ putting 
$Z= \{ \lambda I_n, \lambda^n=1 \}, d= |Z|= (n,q-1)$ and $C= \{X \in L: XT= \mu TX \}$,
$C_G(t)= C/Z$.  Let $p \ne 2$ and the eigenvalues of $T$ be $\rho, -\rho$ then $T$
is conjugate to
$\left(
\begin{array}{cc}
\rho I_r & 0 \\
0 & - \rho I_s \\
\end{array}
\right)$,
or
$\left(
\begin{array}{cc}
0 & I_m \\
- \lambda I_m & 0 \\
\end{array}
\right)$,
depending on whether the minimum polynomial is $(x+ \rho)(x- \rho)$ or $(x^2- \lambda)$
which depends on whether the eigenvalue is in $GF(q)$ or $GF(q^2) \setminus GF(q)$.
Let $X \in C$ with 
$X=
\left(
\begin{array}{cc}
X_1 & X_2 \\
X_3 & X_4 \\
\end{array}
\right)$, so either 
$X_2=X_3=0$ and $det(X_1) det(X_4)=1$ or
$r=s$ and
$X_1=X_4=0$ and $det(X_2) det(-X_3)=1$; let $\delta: X \mapsto det(X_1), K= ker(\delta)$
then $K= SL_r(q) \times SL_s(q)$.  Put $E=KZ/Z$, $E \lhd C/Z$ and $E= K/(K \cap Z)$ and
$E$ is a central product.\\
\\
{\bf Centralizers of the classical groups:}
Let $G= PSL_n(q)$, $q$ odd, $t \in Inv(G)$,  
(1) if $n$ is odd $\exists N \lhd C(t)$ with
$N$ the minimal central product of $SL_r(q)$ and $SL_s(q)$, $r+s=n$ (type *) and both
$C(t)/N$ and $Z(N)$ are cyclic groups with orders dividing $q-1$;
(2) if $n$ is even there is a centralizer as above and centralizers of two additional
types: 
(A) $\exists C_0 : |C(t):C_0|=2$ and $E \lhd C(t)$ of type * with $r=s$ and
$C(t)/E$ is {\bf dihedral} and $C_0/Z$ and $Z(E)$ are cyclic --- there is an element of
order 2 outside $C_0$ that interchanges the factors of $E$,
(B) $\exists C_0 : |C(t):C_0|=2$ and $E \lhd C(t)$ of type * with $r=s$ and
$E/Z(E) \cong PSL_r(q^2)$ and $Z(E)$ is cyclic with order dividing $q+1$ and
$C(t)/E$ is dihedral of order $q+1$ or $2(q+1)$; further, there is an element of order
two in $C(t) \setminus C_0$ which transforms elements in $E/Z(E)$ like the element
of order $2$ in the Galois group of $GF(q^2)/GF(q)$.
If $G= PSp_{2m}(q)$ with $q$ odd and $t \in Inv(G)$ then either
(1) $C(t)$ is a minimal central product of 
$Sp_{2r}(q)$ and
$Sp_{2s}(q)$ with $r+s=m, r \ne s$, or
(2) $\exists C_1 \lhd C(t)$ with $C_1$ a minimal central product of two copies
of $Sp_{2l}(q), 2l=m$ and there is an element of order two in
$C(t) \setminus C_1$ that interchanges the two, or 
(3) $\exists C_1 \lhd C(t)$ with $C_1 \cong GL_m(q)/\{ \pm I \}$
and there is an element of order two in
$C(t) \setminus C_1$ that corresponds to $A \mapsto ^tA^{-1}$ and $q= 1 \jmod{4}$, or
(4) $\exists C_1 \lhd C(t)$ with $C_1 \cong U_m(q)/\{ \pm I \}$
and there is an element of order two in
$C(t) \setminus C_1$ that corresponds to $A \mapsto A^{\tau}$ and $q= 3 \jmod{4}$, 
$\tau$ the generator of the Galois group.
If $G= PSU_{n}(q)$ with $q$ odd and $t \in Inv(G)$ then either
(1) $\exists N \lhd C(t)$ with $N$ a minimal central product of 
$SU_{r}(q)$ and
$SU_{s}(q)$ with $r+s=m, r \ne s$, both $C(t)/N$ and $Z(N)$ are cyclic with orders
dividing $q+1$, 
(2) if $n$ is even there is a centralizer as above and centralizers of two additional
types: 
(A) $\exists C_0 : |C(t):C_0|=2$ and $E \lhd C(t)$ of type * with $r=s$ and
$C(t)/E$ is dihedral and $C_0/Z$ and $Z(E)$ are cyclic --- there is an element of
order 2 outside $C_0$ that interchanges the factors of $E$,
(B) $\exists C_0 : |C(t):C_0|=2$ and $E \lhd C(t)$ with $r=s$ , $Z(E)$ cyclic of order
dividing $q-1$ and
$E/Z(E) \cong PSL_r(q^2)$ 
and there is an element of order two in
$C(t) \setminus C_1$ that corresponds to $A \mapsto ^t(A^{\tau})^{-1}$ 
$\tau$ the generator of the Galois group.
If $G= P\Omega_{n}(q)$ with $q$ odd and $t \in Inv(G)$ then either
(1) $\exists E \lhd C(t)$ with $C(t)/E$ solvable, $E'=E$ and $E$ is either
$SL_{m}(q)/ \{ \pm I \}$ and
$SU_{m}(q)/ \{ \pm I \}$  ($2m=n$ in both cases) or a central product of 
$\Omega_r(q)$ and $\Omega_s(q)$.  For $G= A_n$, let $H_1= \Sigma_k, H_2= Z_2 \wr \Sigma_l$
and $C(t) = H_1 \times H_2$ with $(\sigma, \rho) \in C(t), sign(\sigma)= sign(\rho)$.\\
\\
Since $C(F^*(G)) \subseteq F^*(G)$, $G \rightarrow Aut(G)$ has kernel $Z(F^* (G))$; 
further, $F^*(G)$ is uncomplicated and its embedding in $G$ is well behaved.  
Want to study relationship
of $F^*(G)$ and its $p-$locals.  Hard when $F^*(G)$ is a $p-group$ but then we
can use Thompson factorization.  Thompson $p-$complement
$\rightarrow$ nilpotence of Frobenius kernel.\\
\\
{\bf Signalizers:} $r$, prime, $G$ finite and $A$ an abelian $r-$subgroup of $G$. An
$A-$signalizer is a map $\theta: A^{\#} \rightarrow {\cal S}$ where ${\cal S}$ is a
set of $r'$ $A-$invariant subgroups such that $a,b \in A^{\#}$ and $\theta(a) \le C_G (a)$
and
$\theta(a) \cap C(b) \le \theta(b)$.  
$\theta$ is complete if $\exists \theta(G)$ an $r'$, 
$A-$invariant subgroup such that
$\theta(a)= C_{\theta(G)}(a)$ for each $a \in A^{\#}$.  $\theta(a)=C_X(a)$ is one such
function; if $m(A)  \ge 3$ then every $A-$signalizer functor is complete.  Under these
conditions, for a solvable $A-$signalizer, ${\cal N}_{\theta}(A)$ has a unique maximal
element.  Goldschmidt proved this for solvable signalizer functors.\\
\\
Let $p,q \in \pi(A)$ then for $S \subseteq A$. (1) $p \ne 2$, $S_p(A) \rightarrow$ $S$ is
cyclic.  (2) $S \in S_2(A)$ is cyclic or quaternion.  
(3) $|S|=pq \rightarrow S$ is cyclic.
(4) $|S|= 1 \jmod{2} \rightarrow S$ is metacyclic.\\
\\
If $x,y$ are two involutions in $G$ then $<x, y>$ is dihedral of order $2|xy|$.
Let $G$ be even order with $Z(G)=1$, let $m$ be the number of involutions in $G$
and $n=|G|/m$.  Then $G$ possesses a proper group of order at most $2 n^2$.\\
\\
Let $G$ be a simple group of even order,
$t$ and involution and $n= |C_G (t)|$.  Then $|G| \le (2n^2)!$. From this we get:
{\bf Brauer-Fowler:}  Let $H$ be a finite group.   There are at most a finite number
of finite simple groups with $H \cong C_G (t)$.\\
\\
{\bf Feit-Thompson:}  The only finite simple groups or odd order are
${\mathbb Z}_p, p\ne 2$. The proof follows the CN classification.\\
\\
{\bf Thompson Order Formula:}
Assume $G$ has more than two congugacy classes of involutions
$\{ {x_i}^G \}$ and let $n_i$ be the number of ordered pairs $(u, v)$ with
$u \in {x_1}^G, v \in {x_2}^G$ and $x_i \in < uv>$ then
$|G|= |C(x_1 )| |C(x_2 )| \sum_{i=1}^k {\frac {n_i} {|C(x_i )|}}$.\\
\\
Let $\Omega$ be a collection of subgroups.  Define ${\cal D} (\Omega)$ as the graph
formed by joining $A,B \in \Omega$ if $[A,B]=1$. If $k>0$ let, 
${\cal E}^p_k (G)$
be the elementary abelian subgroups of $p$-rank at least $k$. $G$ is said to be
$k-connected$ for prime $p$ if ${\cal D} ( {\cal E}^p_k (G))$
is connected.\\
\\
If $G$ is a non-abelian finite simple group
with $m_2(G) \le 2$ then either (1) a Sylow 2-group is either dihedral, semi-dihedral
or $Z_{2^n} \; wr \; Z_2$ and
$G \cong L_2(q)$,
$G \cong L_3(q)$,
$G \cong U_3(q)$ $q, $odd, or $M_{11}$; or,  (2)
$G \cong U_3(4)$.
Note that $Q_8 \in S_2(SL_2(3))$ and 
$\left(
\begin{array}{cc}
2 &  0 \\
0 &  2 \\
\end{array}
\right)$ is the unique involution.\\
\\
If $G$ is a non-abelian finite simple group
with $m_2(G) > 2$ and assume $G$ has a proper 2-generated 2-core, then either
$G$ is a group of Lie type of characteristic 2 and Lie rank 1 or $G \cong J_1$.\\
\\
{\bf Glauberman $ZJ$:}  If $C_G(O_p(G)) \le O_p(G)$ and the action of
$G$ on its chief factors of $G$ is $p-$stable then $G=N_G(Z(J(S)))$. Every group
admitting a fixed-point-free automorphism of prime order is nilpotent.
{\bf Glauberman's $Z^*$ Theorem:}
Let $G$ be a finite group and $t$ and involution in $G$ which is weakly closed in
$C(t)$.  Then $t^* \in Z(G^*)$  where $G^*= G/O_{2'} (G)$.\\
\\
{\bf $B_p$ property:} Suppose $O_{p'}(G)=1$ and $x \in G, |x|=p$ then
$O_{p',E}(C(x))= O_{p'}(C(x))E(C(x))$.
A {\bf standard subgroup} for the prime $p$ is a group $H=C_G(x), |X|=p$ such that
$H$ has a unique component, $L$, and $C_G(L)$ has a cyclic Sylow $p-$group.
{\bf Component Theorem:}
Let $G$ be a finite group with $F^*(G)$ satisfying the $B_2$ property and with
in involution, $t$ such that $O_{2', E} (C(t)) \ne O_{2'} (C(t))$ then $G$ possesses
a standard subgroup for the prime $2$.
{\bf Standard Form} problem for $(L,r)$:   Determine all finite groups, $G$, possessing a 
standard subgroup $H$ for the prime $r$ with $E(H) \cong L$. {\bf Aschbacher's program:}
Let $G$ be a minimal counter-example to the classification theorem and assume $G$ 
is generic of even characteristic.  
Then one of the following holds: (1) $G$ possesses a standard subgroup for some
$p \in \sigma(G)$; (2) there is an involution $t \in G$ such that $F^* (C(t))$ is a 
2-group of symplectic type; or, (3) $G$ is in the uniqueness case.\\
\\
In real simple groups $O_{2'}(C(t))$ is cyclic and almost central.
{\bf Bender's Theorem:} For any group $X$, we have $C_X(F^*(X)) \le F^*(X)$ and if
$W \lhd X$ and $C_X(W) \le W$ then $E(X) \le W$.  If $O_{p'}(X)=1$ then
$F(X)=O_p(X)$ and every component of $X$ has order divisible by $p$ so
$X$ is $p$-constrained iff $E(X)=1$ or, equivalently, $C_X(O_p(X)) \le O_p(X)$.
Let ${\overline X}= E(X/O_{p'}(X))$, 
$L$ is a minimal normal subgroup subject to ${\overline L}= E({\overline X})$,
${\overline {L_i}}$ is a component of $E({\overline X})$, $L_i= O^{p'}(L_i)$,
$[L_i, L_i]= L_i$ and $[L_i, L_j] \le O_{p'}(X)$, 
$L$ is called the {\bf $p$-layer}.  $F^*(X)$ controls embedding of $X$ of
$p'$-cores and the $p$-layer of every $p$-local.  $O_{\pi}((X/O_{\pi}(X)))=1$.
If $O_{\pi}(X)=1$ then $F(X)$ is divisible by $p \in \pi$ and every component is
divisible by some $p \in \pi'$.\\
\\
Recall signalizers.  The idea is that
$A-$invariant $p'$ subgroups of $G$ can be glued into a single $p'$
subgroup $\theta(G,A)$ which is either normal or strongly $p-$embedded in $G$.
$M \subseteq G$ is {\bf strongly $p-$embedded} if $p | |M|$ but $p$ does not divide
$|M \cap M^g |$ for $g \in G-M$.  {\bf Tightly embedded:} $p=2$. If $M$ is strongly
embedded, $G$ fixes one point when acting on the cosets of $M$.
Bender identified all simple groups with strongly 2-embedded subgroups, namely,
$SL_2(2^n), SZ(2^n), PSU_3(2^n)$.  No simple group of
$p-rank \ge 3$ has a strongly $2-$embedded $2'$ local subgroup.\\
\\
{\bf Bender:} Let $G$ be a finite simple group and $S \in S_2(G)$ then one of the following holds:
(a) $S$ is dihedral, (b) $S$ is semidihedral, (c) $G$ has a strongly embedded subgroup,
(d) $S$ has a non-cyclic characteristic elementary abelian subgroup, $A$, and
$E=N_G(A)$ has conjugacy classes, $<z_i^G>$, that do not fuse in $G$ such that
$G= <E, C_G(z_i )>$.  If $G$ is a finite simple group and
$H<G$ with ${\mathbb Z}(H)$ of even order and $h \approx C_H(z)$ then $G$ is said
to be of $H$-type.  Note we can construct a faithful transitive permutation representation
of $G$ given a presentation of $H$.  A group has an $H$-satellite if there are non-isomorphic
groups of $h$-type.
A finite simpe group, $G$, is uniquely determined by $C_H(z)$ for a $2-$central
involution, $z$, if $G$ does not have any non-isomorphic $H$-satellites.
\\
\\
{\bf Netto:}  Let $x,y \in S_n$ be selected randomly.  $Pr[<x,y>=S_n]= {\frac 3 4}$.  
{\bf Irreducible
characters of the symmetric group} $S_n$: $n= n-m, \mu_1, \ldots , \mu_j$, $d_{n}(\mu)$ is the dimension of the
irreducible character determined by: $l_{j+1}= \mu_j$, $l_j= \mu_{j-1}+1$, $l_{j-1}= \mu_{j-2}+2$,
\ldots, $l_1= n-m+j$.  $d_n(\mu)= \prod_{s>r} (l_r-l_s)$.

